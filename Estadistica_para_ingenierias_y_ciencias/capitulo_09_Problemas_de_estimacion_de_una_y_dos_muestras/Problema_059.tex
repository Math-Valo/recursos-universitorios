\begin{enunciado}
 ¿Qu\'e tan grande se requiere que sea la muestra en el ejercicio 9.51 si deseamos tener una confianza de $96\%$ de que nuestra proporci\'on de la muestra estar\'a dentro del $0.02$ de la fracci\'on real de la poblaci\'on votante?
\end{enunciado}

\begin{solucion}
 Usando la notaci\'on y datos como se explic\'o en la soluci\'on del ejercicio 9.51 y a\~nadiendo el error m\'aximo en el que se debe encontrar $\hat{p}$ para estimar la proporci\'on, se tiene lo siguiente:
 \begin{itemize}
  \item $n=200$ votantes en la muestra previa.
  \item $x=114$ casos favorables en la muestra previa.
  \item $\hat{p} = \frac{57}{100} = 0.57$.
  \item $\hat{q} = \frac{43}{100} = 0.43$.
  \item $\alpha=0.04$.
  \item $z_{\alpha/2} = 2.05$, seg\'un el libro, y $z_{\alpha/2} = 2.05374891$, con la aproximaci\'on realizada en R.
  \item $e = 0.02$.
 \end{itemize}
 Entonces, como $\hat{p}$ estima a $p$ y hay una muestra previa, se usa el siguiente resultado:
 \begin{equation*}
  n = \left\lceil \frac{z_{\alpha/2}^2\hat{p}\hat{q}}{e^2} \right\rceil
 \end{equation*}
 por lo tanto, el valor pedido se puede calcular, usando la primera aproximaci\'on de $z_{\alpha/2}$, como
 \begin{equation*}
  n = \left\lceil \frac{2.05^2(0.57)(0.43)}{0.02^2} \right\rceil = \left\lceil \frac{4.2025(0.2451)}{0.0004} \right\rceil = \lceil 2575.081875 \rceil
 \end{equation*}
 Por lo tanto, el tama\~no de la muestra buscada es de $n = 2\,576$.
 \par 
 Finalente, usando R, se puede calcular el tama\~no de la muestra usando la rutina del archivo anexo \texttt{P19\_Tamanyo\_de\_muestra\_2.r}, cambiando las siguientes l\'{\i}neas de c\'odigo:
 \begin{verbatim}
error<-0.02
alfa<-0.04
inter<-'D'
previo<-TRUE
n<-200
x<-114
p<-NULL
 \end{verbatim}
 \vspace{-0.5cm}
 con lo que se obtiene el siguiente resultado:
 \begin{verbatim}
[1] 2585
 \end{verbatim}
 \vspace{-0.5cm}
 Esto quiere decir, que una mejor aproximaci\'on de $z_{\alpha/2}$ da una cantidad m\'as precisa, que es el que tomar\'a como resultado final. Por lo tanto, un resultado inicial ser\'{\i}a $n = 2\,576$, pero siendo m\'as precisos, la cantidad m\'{\i}nima de la muestra para que el error se encuentre en un margen de $0.02$, es de $n = 2\,585$, que es a lo que se quer\'{\i}a llegar.${}_{\blacksquare}$
\end{solucion}
