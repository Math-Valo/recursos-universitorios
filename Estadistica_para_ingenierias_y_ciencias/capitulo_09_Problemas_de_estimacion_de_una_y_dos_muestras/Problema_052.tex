\begin{enunciado}
 Un fabricante de reproductores de discos compactos utiliza un conjunto de pruebas amplias para evaluar la funci\'on el\'ectrica de su producto. Todos los reproductores de discos compactos deben pasar todas las pruebas antes de venderse. Una muestra aleatoria de $500$ reproductores tiene como resultado $15$ que fallan en una o m\'as de las pruebas. Encuentre un intervalo de confianza de $90\%$ para la proporci\'on de los reproductores de discos compactos de la poblaci\'on que pasan todas las pruebas.
\end{enunciado}

\begin{solucion}
 Sea $X$ la variable aleatoria de la cantidad de reproductores de discos compactos que pasa todas las pruebas de entre cada $n$ reproductores, entendiendo que $n$ es peque\~na en comparaci\'on a $N$, la poblaci\'on total de los reproductores, suponiendo que $n/N \leq 0.05$, y sea $k$ la cantidad total de reproductores de discos compactos que pasa todas las pruebas, entonces $\widehat{P} = X/n$ es un estad\'{\i}stico de una proporci\'on en un experimento binomial que aproxima el valor de $k/N$, la proporci\'on buscada, entonces, del enunciado, se tienen los siguientes datos obtenidos de una muestra:
 \begin{itemize}
  \item $n=500$.
  \item $x=15$.
  \item $\alpha=0.1$.
 \end{itemize}
 por lo que $\hat{p}$, la proporci\'on de \'exitos en esta muestra, y $\hat{q} = 1 -\hat{p}$ est\'an dados por:
 \begin{itemize}
  \item $\hat{p} = \frac{x}{n} = \frac{15}{500} = \frac{3}{100} = 0.03$; y,
  \item $\hat{q} = 1 - \hat{p} = 1 - \frac{3}{100} = 0.97$.
 \end{itemize}
 Adem\'as, como se buscar\'a un intervalo de confianza bilateral y se calcular\'a el error de $\hat{p}$ para estimar $p$, entonces se requiere del valor $z_{\alpha/2} = z_{0.05}$, el cual se calcul\'o en el ejercicio 9.30 y su aproximaci\'on es de $1.645$, aunque, en R, se puede considerar con mayor precisi\'on como $1.644853626951$.
 \par 
 Ya que se busca un intervalo de confianza para la proporci\'on en un experimento binomial en donde el tama\~no de muestra es grande y se tiene que tanto $n\hat{p}$ como $n\hat{q}$ es mayor que o igual a $5$, entonces se usar\'a la f\'ormula de intervalo siguiente:
 \begin{equation*}
  \hat{p} - z_{\alpha/2}\sqrt{\frac{\hat{p}\hat{q}}{n}} < p < \hat{p} + z_{\alpha/2} \sqrt{\frac{\hat{p}\hat{q}}{n}}
 \end{equation*}
 Por lo tanto, usando los datos obtenidos y con la primera aproximaci\'on de $z_{\alpha/2}$, se tienen los siguientes c\'alculos de los l\'{\i}mites del intervalo de confianza como sigue:
 \begin{eqnarray*}
  \hat{p} \pm z_{\alpha/2}\sqrt{\frac{\hat{p}\hat{q}}{n}} & = & 0.03 \pm(1.645)\sqrt{\frac{(0.03)(0.97)}{500}} = 0.03 \pm 1.645 \left( \frac{\sqrt{3(97)}\sqrt{5}}{100(50)} \right) \\
  & = & 0.03 \pm \frac{329\sqrt{1\,455}}{1\,000\,000} = 0.03 \pm 0.000329\sqrt{1\,455} \approx 0.03 \pm 0.012549528
 \end{eqnarray*}
 Por lo tanto, el intervalo de confianza de $90\%$ de la proporci\'on reproductores de discos compactos de la poblaci\'on que pasan todas las pruebas es aproximadamente:
 \begin{equation*}
  0.01745047192 < p < 0.042549528
 \end{equation*}
 Finalmente, usando R, se puede calcular el intervalo de confianza usando el script en el archivo anexo \texttt{P17\_Intervalo\_de\_confianza\_08.r} cambiando las siguientes l\'{\i}neas de c\'odigo:
 \begin{verbatim}
> n<-500
> x<-15
> p<-NULL
> alfa<-0.1
> inter<-'D'
 \end{verbatim}
 \vspace{-0.5cm}
 con lo que se obtiene el siguiente resultado:
 \begin{verbatim}
     LimInf Proporción    LimSup
1 0.0174516       0.03 0.0425484
 \end{verbatim}
 \vspace{-0.5cm}
 por lo que, al redondear al decimal en que coinciden los resultados anteriores, se tiene que el intervalo de confianza del $90\%$ es $0.01745 < p < 0.04255$, que es a lo que se quer\'{\i}a llegar.${}_{\blacksquare}$
\end{solucion}
