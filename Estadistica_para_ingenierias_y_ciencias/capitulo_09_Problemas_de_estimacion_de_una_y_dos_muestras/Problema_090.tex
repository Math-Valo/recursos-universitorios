\begin{enunciado}
 Se llev\'o a cabo un estudio en el Instituto Polit\'ecnico y Universidad Estatal de Virginia para determinar si el fuego se puede utilizar como una herramienta de control viable, para aumentar la cantidad de forraje disponible para los venados, durante los meses cr\'{\i}ticos a finales del invierno y principios de la primavera. El calcio es un elemento que requieren las plantas y los animales. La cantidad que la planta toma y almacena est\'a estrechamente correlacionada con la cantidad presente en el suelo. Se formul\'o la hip\'otesis de que el fuego puede cambiar los niveles de calcio presentes en el suelo y afectar as\'{\i} la cantidad disponible para los venados. Se seleccion\'o una extensi\'on grande de tierra en el Fishburn Forest para efectuar un incendio controlado. Se tomaron muestras de suelo de $12$ parcelas de igual \'area justo antes de la quema, y se analizaron para verificar el contenido de calcio. Los niveles de calcio despu\'es de la quema se analizaron en las mismas parcelas. Tales valores, en kilogramos por parcela, se presentan en la siguiente tabla:
 \begin{center}
  \begin{tabular}{ccc}
   & \multicolumn{2}{c}{\hspace{-0.5cm} \textbf{Nivel de calcio (kg/parcela)}} \\
   & \textbf{Antes de} & \textbf{Despu\'es de} \\
   \textbf{Parcela} & \textbf{la quema} & \textbf{la quema} \\
   \hline 
   $\phantom{1}1$ & $50$ &  $\phantom{1}9$ \\
   $\phantom{1}2$ & $50$ & $18$ \\
   $\phantom{1}3$ & $82$ & $45$ \\
   $\phantom{1}4$ & $64$ & $18$ \\
   $\phantom{1}5$ & $82$ & $18$ \\
   $\phantom{1}6$ & $73$ & $\phantom{1}9$ \\
   $\phantom{1}7$ & $77$ & $32$ \\
   $\phantom{1}8$ & $54$ & $\phantom{1}9$ \\
   $\phantom{1}9$ & $23$ & $18$ \\
   $10$ & $45$ & $\phantom{1}9$ \\
   $11$ & $36$ & $\phantom{1}9$ \\
   $12$ & $54$ & $\phantom{1}9$ \\
  \end{tabular}
 \end{center}
 Construya un intervalo de confianza de $95\%$ para la diferencia media en el nivel de calcio presente en el suelo antes y despu\'es del incendio controlado. Suponga que la distribuci\'on de las diferencias en los niveles de calcio es aproximadamente normal.
\end{enunciado}

\begin{solucion}
 Sean $X_1$ y $X_2$ las variables aleatorias de los niveles de calcio, medidos en kilogramos por parcela, antes y despu\'es de la quema, respectivamente, entonces, del enunciado, se tiene el siguiente resumen de datos:
 \begin{itemize}
  \item $X_i \sim n\left( \mu_i, \sigma_i \right)$, para cada $i \in \{ 1,2 \}$.
  \item $\mu_i$ y $\sigma_i$ desconocidos, para cada $i \in \{ 1, 2 \}$.
  \item $n_1 = n_2 = 12$.
  \item $\alpha = 0.05$
 \end{itemize}
 Adem\'as de los datos obtenidos en las $12$ parcelas, antes y despu\'es de la quema, cuyas diferencias en kilogramos por parcela, $D_i = X_1 - X_2$, son:
 \begin{center}
  \begin{tabular}{cccc}
   & \multicolumn{2}{c}{\hspace{-0.5cm} \textbf{Nivel de calcio (kg/parcela)}} \\
   & \textbf{Antes de} & \textbf{Despu\'es de} \\
   \textbf{Parcela} & \textbf{la quema} & \textbf{la quema} & $d_i$ \\
   \hline 
   $\phantom{1}1$ & $50$ &  $\phantom{1}9$ & $41$ \\
   $\phantom{1}2$ & $50$ & $18$ & $32$ \\
   $\phantom{1}3$ & $82$ & $45$ & $37$ \\
   $\phantom{1}4$ & $64$ & $18$ & $46$ \\
   $\phantom{1}5$ & $82$ & $18$ & $64$ \\
   $\phantom{1}6$ & $73$ & $\phantom{1}9$ & $64$ \\
   $\phantom{1}7$ & $77$ & $32$ & $45$ \\
   $\phantom{1}8$ & $54$ & $\phantom{1}9$ & $45$ \\
   $\phantom{1}9$ & $23$ & $18$ & $\phantom{4}5$ \\
   $10$ & $45$ & $\phantom{1}9$ & $36$ \\
   $11$ & $36$ & $\phantom{1}9$ & $27$ \\
   $12$ & $54$ & $\phantom{1}9$ & $45$
  \end{tabular}
 \end{center}
 A partir de estos datos, se puede calcular la media y la desviaci\'on est\'andar de las observaciones pareadas, como se muestra a continuaci\'on. La media muestral se calcula como sigue:
 \begin{equation*}
  \bar{d} = \frac{41 + 32 + 37 + 46 + 64 + 64 + 45 + 45 + 5 + 36 + 27 + 45}{12} = \frac{487}{12} = 40.58\overline{3}
 \end{equation*}
 por lo que la varianza muestral se obtiene, usando el Teorema 8.1, como sigue:
 \begin{eqnarray*}
  s_d^2 & = & \frac{1}{n(n-1)} \left[ n \sum_{i=1}^n d_i^2 - \left( \sum_{i=1}^n d_i \right)^2 \right] \\
  & = & \frac{12\left( 41^2 + 32^2 + 37^2 + 46^2 + 64^2 + 64^2 + 45^2 + 45^2 + 5^2 + 36^2 + 27^2 + 45^2 \right) - 487^2}{12(11)} \\
  & = & \frac{12\left( 1681 + 1024 + 1369 + 2116 + 4096 + 4096 + 2025 + 2025 + 25 + 1296 + 729 + 2025 \right)}{132} \\
  & & - \frac{237\,169}{132} \\
  & = & \frac{ 12(22\,507) - 237\,169}{132} = \frac{270\,084 - 237\,169}{132} = \frac{32\,915}{132} = 249.35\overline{60}
 \end{eqnarray*}
 y, por lo tanto, la desviaci\'on est\'andar muestral es:
 \begin{equation*}
  s_d = \sqrt{s_d^2} = \sqrt{\frac{32\,915}{132}} = \frac{\sqrt{32\,915}\sqrt{33}}{66} = \frac{\sqrt{1\,086\,195}}{66} \approx 15.7910120196921
 \end{equation*}
 Por otro lado, como se desea encontrar el intervalo de confianza bilateral para $\mu_D = \mu_1 - \mu_2$, para observaciones pareadas, entonces se requerir\'a el valor de $t_{\alpha/2,n-1} = t_{0.025,11}$. De la Tabla A.4, se tiene que $t_{0.025,11} = 2.201$, mientras que, usando R, se obtiene el valor con los siguientes comandos:
 \begin{verbatim}
> options(digits=22)
> qt(0.025,11,lower.tail=F)
[1] 2.200985160091639691871
 \end{verbatim}
 \vspace{-0.5cm}
 por lo que tambi\'en se puede considerar como $2.20098516$.
 \par 
 Ya que se busca un intervalo de confianza para la diferencia de las medias poblacionales pareadas usando como estimador la media muestral de las diferencias de los datos pareados en una muestra peque\~na, en donde se desconoce la desviaci\'on est\'andar poblacional pero suponiendo que la distribuci\'on de las diferencias es aproximadamente normal, entonces se usar\'a la siguiente formulaci\'on:
 \begin{equation*}
  \bar{d} - t_{\alpha/2,n-1}\frac{s_d}{\sqrt{n}} < \mu_D < \bar{d} + t_{\alpha/2,n-1}\frac{s_d}{\sqrt{n}}
 \end{equation*}
 en donde $\mu_D = \mu_1 - \mu_2$. Por lo tanto, usando los datos obtenidos, considerando el valor $t_{\alpha/2,n-1}$ del libro, se tienen los c\'alculos de los l\'{\i}mites del intervalo de confianza como siguen:
 \begin{eqnarray*}
  \bar{d} \pm t_{\alpha/2,n-1}\frac{s_d}{\sqrt{n}} & = & \frac{487}{12} \pm 2.201 \left( \frac{\sqrt{1\,086\,195}/66}{\sqrt{12}} \right) = \frac{487}{12} \pm  \frac{2\,201\sqrt{362\,065}}{132\,000} \\
  & = & 40.58\overline{3} \pm 0.01667\overline{42}\sqrt{362\,065} \approx 40.58\overline{3} \pm 10.0331980169
 \end{eqnarray*}
 Por lo tanto, el intervalo del $95\%$ de confianza de la diferencia media en el nivel de calcio presente en el suelo antes y despu\'es del incendio controlado, es de:
 \begin{equation*}
  30.5501353164 < \mu_D < 50.6165313502
 \end{equation*}
 Finalmente, en R, se puede calcular el intervalo de confianza de las observaciones pareadas cambiando los siguientes comandos en el archivo anexo \texttt{P16\_Intervalo\_de\_confianza\_07.r}, y usando la base de datos \texttt{DB11\_Problema\_90.csv}.
 \begin{verbatim}
> datos<-read.csv("DB11_Problema_90.csv",sep=";",encoding="UTF-8")
> varInteres<-c("Calcio.Kgpp")
> varSel<-list("Registro")
> alfa<-0.05
 \end{verbatim}
 \vspace{-0.5cm}
 con lo que se obtiene el siguiente resultado:
 \begin{verbatim}
     variable  LimInf    Media   LimSup
1 Calcio.Kgpp 30.5502 40.58333 50.61646
 \end{verbatim}
 \vspace{-0.5cm}
 Por lo tanto, al redondear al decimal en que coinciden los resultados anteriores, se tiene que el intervalo de $95\%$ es $30.55 < \mu_1 - \mu_2 < 50.62$, que es a lo que se quer\'{\i}a llegar.${}_{\blacksquare}$
\end{solucion}
