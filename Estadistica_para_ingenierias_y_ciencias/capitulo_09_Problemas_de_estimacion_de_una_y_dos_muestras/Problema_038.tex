\begin{enunciado}
 En un proceso qu\'{\i}mico por lotes, se comparan los efectos de dos catalizadores sobre la potencia de la reacci\'on del proceso. Se prepar\'o una muestra de $12$ lotes con el uso del catalizador 1 y se obtuvo una muestra de $10$ lotes con el catalizador 2. Los $12$ lotes para los que se utiliz\'o el catalizador 1 dieron un rendimineto promedio de $85$ con una desviaci\'on est\'andar muestral de $4$; en tanto que para la segunda muestra el promedio fue de $81$ con una desviaci\'on est\'andar muestral de $5$. Encuentre un intervalo de confianza de $90\%$ para la diferencia entre las medias poblacionales, suponiendo que las poblaciones se distribuyen de forma aproximadamente normal con varianzas iguales.
\end{enunciado}

\begin{solucion}
 Sean $X_1$ y $X_2$ las variables aleatorias del rendimiento de los dos catalizadores mencionados en el enunciado sobre  la potencia de reacci\'on del proceso qu\'{\i}mico del que se hace referencia en el enunciado, entonces se tienen los siguientes datos:
 \begin{itemize}
  \item $X_i \sim n(\mu_i, \sigma_i)$, para cada $i\in\{ 1, 2 \}$.
  \item $\mu_i$ y $\sigma_i$ desconocidos, para cada $i\in\{ 1, 2 \}$.
  \item $n_1 = 12$ lotes y $n_2 = 10$ lotes.
  \item $\bar{x}_1 = 85$ y $\bar{x}_2 = 81$.
  \item $s_1 = 4$ y $s_2 = 5$.
  \item $\alpha = 0.1$.
 \end{itemize}
 Como se desea encontrar un intervalo de confianza para $\mu_1 - \mu_2$, usando como estimador $\bar{x}_1 -\bar{x}_2$, desconociendo las varianzas poblacionales, pero suponiendo que son iguales, con muestras peque\~nas y poblaciones que se distribuyen de forma aproximadamente normales, entonces se requerir\'a el valor cr\'{\i}tico $t_{\alpha/2,n_1+n_2-2} = t_{0.05,20}$. De la Tabla A.4, se tiene que $t_{0.05,20} = 1.725$, mientras que, usando R, se obtiene el valor con los siguientes comandos.
 \begin{verbatim}
> options(digits=22)
> qt(0.05, 20, lower.tail=F)
[1] 1.724718242920787236727
 \end{verbatim}
 \vspace{-0.5cm}
 por lo que tambi\'en se puede considerar como $1.7247182$.
 \par
 Ya que se busca un intervalo de confianza para la diferencia de las medias poblacionales usando como estimador la diferencia de las medias muestrales en muestras peque\~nas, en donde se desconoce las desviaciones est\'andar poblacionales pero se suponen que las poblaciones se distribuyen aproximadamente normal y las desviaciones est\'andar poblacionales son iguales, entonces se usar\'a la siguiente formulaci\'on:
 \begin{equation*}
  \left( \bar{x}_1 - \bar{x}_2 \right) - t_{\alpha/2,n_1+n_2-2} s_p \sqrt{\frac{1}{n_1} + \frac{1}{n_2}} < \mu_1 - \mu_2 < \left( \bar{x}_1 - \bar{x}_2 \right) + t_{\alpha/2,n_1+n_2-2} s_p \sqrt{\frac{1}{n_1} + \frac{1}{n_2}}
 \end{equation*}
 en donde
 \begin{equation*}
  s_p = \sqrt{\frac{(n_1-1)s_1^2 + (n_2-1)s_2^2}{n_1+n_2-2}}
 \end{equation*}
 Por lo tanto, usando los datos obtenidos, considerando el valor $t_{\alpha/2,n_1+n_2-2}$ del libro, se tienen los c\'alculos de los l\'{\i}mites del intervalo de confianza como siguen:
 \begin{eqnarray*}
  \sqrt{\frac{(n_1-1)s_1^2 + (n_2-1)s_2^2}{n_1+n_2-2}} & = & \sqrt{\frac{(12-1)4^2 + (10-1)5^2}{12+10-2}} = \sqrt{\frac{11(16)+9(25)}{20}} = \sqrt{\frac{176+225}{20}} \\
  & = & \sqrt{\frac{401}{20}} = \frac{\sqrt{401}}{2\sqrt{5}} = \frac{\sqrt{2005}}{10}
 \end{eqnarray*}
 y
 \begin{eqnarray*}
  \left( \bar{x}_1 - \bar{x}_2 \right) \pm t_{\alpha/2,n_1+n_2-2} s_p \sqrt{\frac{1}{n_1} + \frac{1}{n_2}} & = & (85-81) \pm (1.725)\left( \frac{\sqrt{2005}}{10} \right) \sqrt{\frac{1}{12} + \frac{1}{10}} \\
  & = & 4 \pm 0.1725\sqrt{2005}\sqrt{\frac{5+6}{60}} = 4 \pm 0.1725\sqrt{2005}\sqrt{\frac{11}{60}} \\
  & = & 4 \pm 0.1725\frac{\sqrt{401}\sqrt{11}}{2\sqrt{3}} = 4\pm 0.08625\frac{\sqrt{4411}\sqrt{3}}{3} \\ 
  & = & 4 \pm 0.02875\sqrt{13233}
 \end{eqnarray*}
 Por lo tanto, el intervalo del $90\%$ de confianza de la diferencia de las medias de los rendimientos de los dos catalizadores sobre la potencia de reacci\'on en el proceso qu\'{\i}mico del que se hace referencia en el enunciado, es de:
 \begin{equation*}
  0.69275015118301 < \mu_1 - \mu_2 < 7.30724984881698951
 \end{equation*}
 Finalmente, en R se puede calcular el intervalo de confianza con las siguientes l\'{\i}neas de c\'odigo, que es una adaptaci\'on del script en el archivo anexo \texttt{P13\_Intervalo\_de\_confianza\_04.r}, y que se ha registrado en el archivo anexo \texttt{P14\_Intervalo\_de\_confianza\_05.r}. El script trabaja igual a como se ha explicado previamente, nada m\'as se a\~nade la variable \texttt{varia} que es \texttt{TRUE} si, aunque no se conozcan, se supone que las varianzas poblacionales son iguales y \texttt{FALSE} en otro caso; adem\'as, el programa no se va a detener si no se conoce las desviaciones est\'andar poblacionales y la muestra es peque\~na, sino que considera estos casos para los nuevos c\'alculos, incluyendo, como antes, la posibilidad de que sean intervalos unilaterales de confianza. Por esto, este nuevo script es una extensi\'on del anterior. Las l\'{\i}neas de c\'odigo se muestran a continuaci\'on.
 \begin{verbatim}
> n1<-12
> n2<-10
> m1<-85
> m2<-81
> desv.tipica1<-4
> desv.tipica2<-5
> alfa<-0.1
> val<-FALSE
> varia<-TRUE
> inter<-'D'
> if (n1 >= 30 & n2 >= 30){
+     val<-TRUE
+ }
> difMedias<-function(n1,n2,m1,m2,desv1,desv2,alfa=0.05,colas='D',distribucion.
+                     z=FALSE,varia=FALSE){
+     diferencia<-m1-m2
+     if(n1 < 2 | n2 < 2){
+         r<-data.frame(n1=n1,n2=n2,
+                       media1=m1,media2=m2,
+                       LimInf=NA,
+                       diferencia=diferencia,
+                       LimSup=NA)
+     }else{
+         if(distribucion.z){
+             # cálculo del intervalo, bilateral o unilateral, de confianza
+             # usando desviaciones poblacionales o si la muestra es grande, 
+             # lo cual, ya se mostró antes...
+             # ...
+         }else{
+             if(varia){
+                 Sp <- sqrt(((n1-1)*desv1^2+(n2-1)*desv2^2)/(n1+n2-2))
+                 if(colas=='D'){
+                     k<-qt(1-alfa/2,n1+n2-2)
+                     LL<-round(diferencia-k*Sp*sqrt(1/n1+1/n2),7)
+                     LU<-round(diferencia+k*Sp*sqrt(1/n1+1/n2),7)
+                     r<-data.frame(n1=n1,n2=n2,
+                                   media1=m1,media2=m2,
+                                   LimInf=LL,
+                                   diferencia=diferencia,
+                                   LimSup=LU)
+                 }else{
+                     k<-qt(1-alfa,n1+n2-2)
+                     if(colas=='I'){
+                         LL<-round(diferencia-k*Sp*sqrt(1/n1+1/n2),7)
+                         r<-data.frame(n1=n1,n2=n2,
+                                       media1=m1,media2=m2,
+                                       LimInf=LL,
+                                       diferencia=diferencia)
+                     }else{
+                         LU<-round(diferencia+k*Sp*sqrt(1/n1+1/n2),7)
+                         r<-data.frame(n1=n1,n2=n2,
+                                       media=m1,media2=m2,
+                                       diferencia=diferencia,
+                                       LimSup=LU)
+                     }
+                 }
+             }else{
+                 numerador<-(desv1^2/n1+desv2^2/n2)^2
+                 denominador<-(desv1^2/n1)^2/(n1-1)+(desv2^2/n2)^2/(n2-1)
+                 grados<-round(numerador/denominador,0)
+                 if(colas=='D'){
+                     k<-qt(1-alfa/2,grados)
+                     LL<-round(diferencia-k*sqrt(desv1^2/n1+desv2^2/n2),7)
+                     LU<-round(diferencia+k*sqrt(desv1^2/n1+desv2^2/n2),7)
+                     r<-data.frame(n1=n1,n2=n2,
+                                   media1=m1,media2=m2,
+                                   LimInf=LL,
+                                   diferencia=diferencia,
+                                   LimSup=LU)
+                 }else{
+                     k<-qt(1-alfa,grados)
+                     if(colas=='I'){
+                         LL<-round(diferencia-k*sqrt(desv1^2/n1+desv2^2/n2),7)
+                         r<-data.frame(n1=n1,n2=n2,
+                                       media1=m1,media2=m2,
+                                       LimInf=LL,
+                                       diferencia=diferencia)
+                     }else{
+                         LU<-round(diferencia+k*sqrt(desv1^2/n1+desv2^2/n2),7)
+                         r<-data.frame(n1=n1,n2=n2,
+                                       media1=m1,media2=m2,
+                                       diferencia=diferencia,
+                                       LimSup=LU)
+                     }
+                 }
+             }
+         }
+     }
+     return(r)
+ }
> rFin<-difMedias(n1,n2,m1,m2,desv.tipica1,desv.tipica2,alfa=alfa,colas=inter,
+                 distribucion.z=val,varia=varia)
> rFin
  n1 n2 media1 media2    LimInf diferencia  LimSup
1 12 10     85     81 0.6932903          4 7.30671
 \end{verbatim}
 \vspace{-0.5cm}
 por lo que, al redondear al decimal en el que coinciden los resultados anteriores, se tiene que el intervalo de confianza de $90\%$ es $0.69 < \mu_1 - \mu_2 < 7.31$, que es a lo que se quer\'{\i}a llegar.${}_{\blacksquare}$
\end{solucion}
