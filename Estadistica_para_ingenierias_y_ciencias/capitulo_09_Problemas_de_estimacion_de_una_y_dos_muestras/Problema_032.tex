\begin{enunciado}
 Rem\'{\i}tase una vez m\'as al ejercicio 9.30. Suponga que las especificaciones de un comprador del hilo son que la resistencia a la tensi\'on del material debe ser, por lo menos, de $62$ kilogramos. El fabricante est\'a satisfecho si, cuando mucho, el $5\%$ de las piezas producidas tienen una resistencia a la tensi\'on menor de $62$ kilogramos. ¿Hay alguna raz\'on para preocuparse? Esta vez utilice un l\'{\i}mite de tolerancia unilateral de $95\%$ que sea exdedido por $95\%$ de los valores de resistencia a la tensi\'on.
\end{enunciado}

\begin{solucion}
 Usando la notaci\'on y datos como en la soluci\'on del ejercicio 9.30, cambiando los valores $\alpha$ y $\gamma$ del l\'{\i}mite de tolerancia, entonces se tienen los siguientes datos:
 \begin{itemize}
  \item $X\sim n(\mu, \sigma)$.
  \item $\mu$ y $\sigma$ desconocidos.
  \item $n=50$ piezas.
  \item $\bar{x} = 78.3\,$Kg.
  \item $s=5.6\,$Kg.
  \item $\gamma = 0.01$.
  \item $\alpha = 0.05$.
 \end{itemize}
 Para este problema, el factor de tolerancia, $k$, cambia con respecto al del ejercicio 9.30, y debe volverse a hallar. De la Tabla A.7, se tiene que $k= 2.269$, mientras que, usando el software estad\'{\i}stico R, se obtiene un valor m\'as preciso con los siguientes comandos:
 \begin{verbatim}
>library(tolerance)
>options(digits=22)
>K.table(50,alpha=0.01,P=0.95,side=1,method=("WBE"))
$`50`
                        0.95
0.99 2.268897684776103318427
 \end{verbatim}
 \vspace{-0.5cm}
 por lo que tambi\'en se puede considerar con mayor precisi\'on como $2.2688976847761$.
 \par 
 Dado que se desea calcular un intervalo inferior de tolerancia, de una poblaci\'on que se supone normal, entonces se usa la siguiente formulaci\'on:
 \begin{equation*}
  \bar{x} - ks
 \end{equation*}
 Por lo tanto, usando los datos obtenidos y considerando el valor $k$ del libro, se obtienen los siguientes c\'alculos:
 \begin{equation*}
  \bar{x} - ks = 78.3 - (2.269)(5.6) = 78.3-12.7064 = 65.5936
 \end{equation*}
 Entonces, el l\'{\i}mite inferior de tolerancia de $99\%$ de confianza para la resistencia a la tensi\'on del tipo de hilo, que sea excedido por $95\%$ de los valores de resistencia a la tensi\'on de las piezas, es de $65.5936\,$Kg. El c\'alculo del l\'{\i}mite inferior de tolerancia con R se puede obtener cambiando los siguientes comandos del archivo anexo \texttt{P11\_Intervalo\_de\_tolerancia\_3.r}:
 \begin{verbatim}
>n<-50
>m<-78.3
>desv<-5.6
>gamma<-0.01
>alfa<-0.05
>inter<-'I'
 \end{verbatim}
 \vspace{-0.5cm}
 con lo que se obtiene un l\'{\i}mite inferior de $65.59417$. Por lo tanto, se puede asegurar, con un $99\%$ de confianza, que el $95\%$ de la las piezas de hilo tendr\'an una resistencia mayor  a $65.59\,$Kg, que es mayor a los $62\,$Kg m\'{\i}nimos que exige las especificaciones del material, por lo que no hay raz\'on para preocuparse, que es a lo que se quer\'{\i}a llegar.${}_{\blacksquare}$
\end{solucion}
