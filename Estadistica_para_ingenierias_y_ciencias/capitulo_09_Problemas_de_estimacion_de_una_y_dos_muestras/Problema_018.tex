\begin{enunciado}
 Las siguientes mediciones se registraron para el tiempo de secado, en horas, de cierta marca de pintura l\'atex:
 \begin{center}
  \begin{tabular}{ccccc}
   $3.4$ & $2.5$ & $4.8$ & $2.9$ & $3.6$ \\
   $2.8$ & $3.3$ & $5.6$ & $3.7$ & $2.8$ \\
   $4.4$ & $4.0$ & $5.2$ & $3.0$ & $4.8$
  \end{tabular}
 \end{center}
 Suponiendo que las mediciones representan una muestra aleatoria de una poblaci\'on normal, encuentre los l\'{\i}mites de tolerancia de $99\%$ que contendr\'an $95\%$ de los tiempos de secado.
\end{enunciado}

\begin{solucion}
 Sea $X$ la variable aleatoria del tiempo de secado, medido en horas, de la marca de pintura l\'atex del que hace referencia el enunciado, entonces se tienen los siguientes datos:
 \begin{itemize}
  \item $X\sim\text{normal}(\mu,\sigma)$.
  \item $n=15$.
  \item $\gamma=0.01$.
  \item $\alpha=0.05$.
 \end{itemize}
 adem\'as de los 15 datos obtenidos en la muestra.
 \par 
 A partir de estos datos, se puede calcular la media y desviaci\'on est\'andar muestral como se muestra a continuaci\'on. La media muestra se calcula como sigue:
 \begin{eqnarray*}
  \bar{x} & = & \frac{3.4+2.5+4.8+2.9 + 3.6+2.8+3.3+5.6+3.7+2.8+4.4 + 4.0+5.2+3.0+4.8}{15} \\
  & = & \frac{56.8}{15} = 3.78\overline{6}
 \end{eqnarray*}
 por lo que la varianza muestral se obtiene como:
 \begin{eqnarray*}
  s^2 & = & \frac{1}{14}\left[ \left(3.4 - 3.78\overline{6} \right)^2 + \left(2.5 - 3.78\overline{6} \right)^2 + \left(4.8 - 3.78\overline{6} \right)^2 + \left(2.9 - 3.78\overline{6} \right)^2 + \left(3.6 - 3.78\overline{6} \right)^2 + \right. \\
  & & \left. \left(2.8 - 3.78\overline{6} \right)^2 + \left(3.3 - 3.78\overline{6} \right)^2 + \left(5.6 - 3.78\overline{6} \right)^2 + \left(3.7 - 3.78\overline{6} \right)^2 + \left(2.8 - 3.78\overline{6} \right)^2 + \right. \\
  & & \left. \left(4.4 - 3.78\overline{6} \right)^2 + \left(4 - 3.78\overline{6} \right)^2 + \left(5.2 - 3.78\overline{6} \right)^2 + \left(3 - 3.78\overline{6} \right)^2 + \left(4.8 - 3.78\overline{6} \right)^2 \right] \\
  & = & \frac{1}{14}\left( 0.1495\overline{1} + 1.6555\overline{1} + 1.0268\overline{4} + 0.7861\overline{7} + 0.0348\overline{4} + 0.9735\overline{1} + 0.2368\overline{4} + 3.2881\overline{7} + \right. \\
  & & \left. 
  0.0075\overline{1} + 0.9735\overline{1} + 0.3761\overline{7} + 0.0455\overline{1} + 1.9975\overline{1} + 0.6188\overline{4} + 1.0268\overline{4} \right) \\
  & = & \frac{13.197\overline{3}}{14} = \frac{4949/375}{14} = \frac{707}{750} \\
  & = & 0.942\overline{6}
 \end{eqnarray*}
 y, por lo tanto, la desviaci\'on est\'andar muestral es:
 \begin{equation*}
  s = \sqrt{s^2} = \sqrt{\frac{707}{750}} = \frac{\sqrt{707}}{5\sqrt{30}} = \frac{\sqrt{21210}}{150} \approx 0.97091
 \end{equation*}
 Por otro lado, como se desea encontrar l\'{\i}mites de tolerancia, se requiere encontrar el factor de tolerancia, $k$. De la Tabla A.7 se tiene que $k=3.507$, mientras que, usando el software estad\'{\i}stico R, se obtiene un valor m\'as preciso con los siguientes comandos:
 \begin{verbatim}
>library(tolerance)
>options(digits=22)
>K.table(15,alpha=0.01,P=0.95,side=2,method=("WBE"))
$'15'
                        0.95
0.99 3.507305923979956663317
 \end{verbatim}
 \vspace{-0.5cm}
 por lo que tambi\'en se puede considerar con mayor precisi\'on como $3.5073$.
 \par 
 Dado que se desea calcular un intervalo de tolerancia de una poblaci\'on que se supone normal, entonces se usa la siguiente formulaci\'on:
 \begin{equation*}
  \bar{x} \pm ks
 \end{equation*}
 Por lo tanto, usando los datos obtenidos, y considerando el valor $k$ del libro, se obtiene los siguientes c\'alculos:
 \begin{equation*}
  \bar{x} \pm ks= 3.78\overline{6} \pm (3.507)(0.97091) = 3.78\overline{6} \pm 3.40498137
 \end{equation*}
 Por lo tanto, el intervalo de tolerancia con el $99\%$ de seguridad de que contendr\'a el $95\%$ de las horas de secado de pintura l\'atex de esta marca es desde $0.38168529\overline{6}$ hasta $7.19164803\overline{6}$ horas.
 \par 
 El c\'alculo del intervalo de tolerancia puede ser obtenido tambi\'en usando el software estad\'{\i}stico R. Se ha escrito un script que  se encuentra anexo bajo el nombre de \texttt{P07\_Intervalo\_de\_tolerancia\_2.r}, el cual ha sido creado para leer un archivo externo con extensi\'on .csv que contiene los datos de la muestra. En este caso, el archivo anexo con los datos del enunciado se llama \texttt{DB02\_Problema\_18.csv}, el cual se conforma de dos columnas con t\'{\i}tulos: la primera se llama \texttt{RegistroDeSecado}, que contiene una enumeraci\'on del $1$ al $15$, y la segunda columna se llama \texttt{Tiempo.h}, que contiene los valores medidos. El script se muestra a continuaci\'on.
 \begin{verbatim}
>library(tolerance)
>datos<-read.csv("DB02_Problema_18.csv",sep=";",encoding="UTF-8")
>varInteres<-c("Tiempo.h")
>gamma<-0.01
>alfa<-0.05
>valores<-unlist(datos[,varInteres])
>variable<-factor(rep(varInteres,each=dim(datos)[1]))
>IT<-function(x,sup=TRUE,alfa=0.05,P=0.99){
>   if (length(x[!is.na(x)])>=2){
>      pruebaT<-normtol.int(x[!is.na(x)],alpha=alfa,P=P,side=2,method="WBE")
>      v<-ifelse(sup,round(pruebaT[,5],7),round(pruebaT[,4],7))
>      return(v)
>   } else return(NA)
>}
>media<-as.data.frame.table(tapply(valores,as.list(data.frame(variable)),mean,
 na.rm=TRUE),responseName="Media")
>ITs1<-as.data.frame.table(tapply(valores,as.list(data.frame(variable)),IT,alfa
 =gamma,P=1-alfa),responseName="LimSup")
>ITs2<-as.data.frame.table(tapply(valores,as.list(data.frame(variable)),IT,sup=
 FALSE,alfa=gamma,P=1-alfa),responseName="LimInf")
>ITs<-na.omit(data.frame(ITs2,Media=media[,"Media"],LimSup=ITs1[,"LimSup"]))
>ITs
 \end{verbatim}
 \vspace{-0.5cm}
 al aplicar este script, se muestra el resultado final, describiendo primero la variable de la cual se va a sacar el intervalo de tolerancia, seguido del l\'{\i}mite inferior, la media muestral y el l\'{\i}mite superior, en ese orden, de modo que sea m\'as c\'omodo de leer. Para este problema, el resultado al usar estos comandos se muestra a continuaci\'on:
 \begin{verbatim}
  variable    LimInf    Media   LimSup
1 Tiempo.h 0.3813875 3.786667 7.191946
 \end{verbatim}
 \vspace{-0.5cm}
 que es a lo que se quer\'{\i}a llegar.${}_{\blacksquare}$
\end{solucion}
