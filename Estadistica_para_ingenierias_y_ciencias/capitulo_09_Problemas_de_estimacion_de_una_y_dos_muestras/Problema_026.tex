\begin{enunciado}
 Considere el ejercicio 9.16. Calcule el intervalo de predicci\'on de $95\%$ para el siguiente n\'umero observado de palabras por minuto tecleado por un miembro del secretariado escolar.
\end{enunciado}

\begin{solucion}
 Usando la notaci\'on y datos como en la soluci\'on del ejercicio 9.16 pero ahora con $\alpha$ representando el valor para el cual hay $(1-\alpha)100\%$ de seguridad en el intervalo de predicci\'on, se tiene lo siguiente:
 \begin{itemize}
  \item $X\sim n(\mu,\sigma)$.
  \item $\mu$ desconocida.
  \item $\sigma$ desconocida.
  \item $n=12$.
  \item $\bar{x}=79.3$.
  \item $s=7.8$.
  \item $\alpha=0.05$.
  \item $t_{\alpha/2,n-1} = t_{0.025,11} = 2.201$, seg\'un la Tabla A.4 del libro, y $t_{0.025,11} = 2.20098516$, usando el software estad\'{\i}stico R.
 \end{itemize}
 Como se est\'a buscando un intervalo de predicci\'on para una muestra peque\~na que proviene de una poblaci\'on normal con varianza desconocida, entonces se usar\'a la siguiente formulaci\'on:
 \begin{equation*}
  \bar{x} - t_{\alpha/2,n-1}s\sqrt{1+1/n} < x_0 < \bar{x} + t_{\alpha/2,n-1}s\sqrt{1+1/n}
 \end{equation*}
 Por lo tanto, con los datos obtenidos y el valor $t_{\alpha/2,n-1}$ del libro, se tienen los c\'alculos de los l\'{\i}mites del intervalo de predicci\'on como siguen:
 \begin{equation*}
  \bar{x} \pm t_{\alpha/2,n-1}s\sqrt{1+1/n} = 79.3 \pm (2.201)(7.8)\sqrt{1+\frac{1}{12}} = 79.3 \pm 17.1678\frac{\sqrt{13}}{2\sqrt{3}} = 79.3 \pm  2.8613\sqrt{39}
 \end{equation*}
 Por lo tanto, el intervalo de predicci\'on de $95\%$, para el n\'umero observado de palabras por minuto tecleado por un miembro del secretariado escolar, es de:
 \begin{equation*}
  61.4311872 < x_0 < 97.16881
 \end{equation*}
 que, redondeado al segundo decimal, se reduce a que el intervalo es $(61.43, 97.17)$. El c\'aclulo del intevalo de predicci\'on usando el valor $t_{\alpha/2,n-1}$ obtenido en R se puede obtener cambiando los siguientes comandos del archivo anexo \texttt{P08\_Intervalo\_de\_prediccion\_1.r}.
 \begin{verbatim}
>n<-12
>m<-79.3
>desv<-7.8
>alfa<-0.05
>val<-FALSE
 \end{verbatim}
 \vspace{-0.5cm}
 con lo que se obtiene que el intervalo de predicci\'on
 \begin{equation*}
  61.43131 < x_0 < 97.16869
 \end{equation*}
 que es a lo que se quer\'{\i}a llegar.${}_{\blacksquare}$
\end{solucion}
