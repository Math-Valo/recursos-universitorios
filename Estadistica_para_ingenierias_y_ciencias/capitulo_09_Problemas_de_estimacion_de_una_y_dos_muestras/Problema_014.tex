\begin{enunciado}
 Una muestra aleatoria de 10 barras de chocolate energ\'etico de cierta marca tiene, en promedio, $230$ calor\'{\i}as, con una desviaci\'on est\'andar de $15$ calor\'{\i}as. Construya un intervalo de confianza de $99\%$ para el contenido medio de calor\'{\i}as real de esta marca da barras de chocolate energ\'etico. Suponga que la distribuci\'on de las calor\'{\i}as es aproximadamente normal.
\end{enunciado}

\begin{solucion}
 Sea $X$ la variable aleatoria del n\'umero de calor\'{\i}as que hay en una barra de chocolate energ\'etico de la marca referida en el enunciado, entonces el enunciado aporta los siguientes datos:
 \begin{itemize}
  \item $X\sim\text{normal}(\mu, \sigma)$.
  \item $\mu$ desconocida.
  \item $\sigma$ desconocida.
  \item $n = 10$ barras.
  \item $\bar{x} = 230$ calor\'{\i}as.
  \item $s=15$ calor\'{\i}as.
  \item $\alpha=0.01$.
 \end{itemize}
 Dado que se desconoce la desviaci\'on est\'andar poblacional y la muestra no es lo suficientemente grande, se requerir\'a del valor $t_{\alpha/2,n-1} = t_{0.005,9}$. De la Tabla A.4 se tiene que $t_{0.005,9} = 3.25$, mientras que, usando el software estad\'{\i}stico R, se obtiene un valor m\'as preciso con los siguientes comandos:
 \begin{verbatim}
>options(digits=22)
>qt(0.005,9,lower.tail=F)
[1] 3.249835541592126286758
 \end{verbatim}
 \vspace{-0.5cm}
 por lo que tambi\'en se puede considerar con mayor precisi\'on como $3.24983554$.
 \par 
 Dado que se desea calcular un intervalo de confianza para la media poblacional usando la media muestral, sin conocer la desviaci\'on est\'andar poblacional y con una muestra peque\~na, entonces se debe de usar la siguiente formulaci\'on:
 \begin{equation*}
  \bar{x}-t_{\alpha/2,n-1}\frac{s}{\sqrt{n}} < \mu < \bar{x}+t_{\alpha/2,n-1}\frac{s}{\sqrt{n}}
 \end{equation*}
 Por lo tanto, usando los datos obtenidos, y considerando el valor $t_{\alpha/2,n-1}$ del libro, se tiene los siguientes c\'alculos de los l\'{\i}mites del intervalo de confianza como siguen:
 \begin{equation*}
  \bar{x}\pm t_{\alpha/2,n-1}\frac{s}{\sqrt{n}} = 230\pm 3.25\left( \frac{15}{\sqrt{10}} \right) = 230 \pm \frac{48.75\sqrt{10}}{10} = 230 \pm 4.875\sqrt{10}
 \end{equation*}
 Por lo tanto, el intervalo del $99\%$ de confianza de la media de calor\'{\i}as que hay en una barra de chocolate energ\'etico de la marca referenciada es de aproximadamente:
 \begin{equation*}
  214.5838964 < \mu < 245.4161
 \end{equation*}
 El c\'alculo del intervalo de confianza usando el valor $t_{\alpha/2,n-1}$ obtenido en R se puede realizar usando el archivo anexo \texttt{P04\_Intervalo\_de\_confianza\_02.r} cambiando los siguientes comandos:
 \begin{verbatim}
>n<-10
>m<-230
>s<-15
>alfa<-0.01
 \end{verbatim}
 \vspace{-0.5cm}
 con lo que se obtiene el intervalo de confianza
 \begin{equation*}
  214.5847 < \mu < 245.4153
 \end{equation*}
 que es a lo que se quer\'{\i}a llegar.${}_{\blacksquare}$
\end{solucion}
