\begin{enunciado}
 Refi\'erase al ejercicio 9.7 y construya un intervalo de tolerancia de $99\%$ que contenga $99\%$ de las millas que recorren los autom\'oviles anualmente en Virginia.
\end{enunciado}

\begin{solucion}
 \textbf{Aclaración previa:} en el problema usan millas para medir la distancia pero en realidad se quer\'{\i}a medir con la misma unidad de distancia que la del ejercicio 9.7. Se ignora si debi\'o medirse en millas en el ejercicio 9.7 \'o, en \'este, debi\'o medirse en kil\'ometros. Esto se comprueba notando que, en la secci\'on de soluciones, la soluci\'on oficial solo tiene sentido considerando la misma unidad. Pese a todo, se usar\'an millas en este ejercicio como si siempre se hubiese tratado de esta medici\'on.
 \par 
 Usando la notaci\'on y datos como se explic\'o en la soluci\'on del ejercicio 9.7, y a eso a\~nadiendo los valores $\gamma$ y $\alpha$ del l\'{\i}mite de tolerancia pedido, se tiene lo siguiente:
 \begin{itemize}
  \item $X\sim\text{normal}(\mu,\sigma)$.
  \item $n=100$.
  \item $\bar{x}=23\,500$.
  \item $s=3\,900$.
  \item $\gamma=0.01$.
  \item $\alpha=0.01$.
 \end{itemize}
 Como se desea encontrar l\'{\i}mites de tolerancia, se requiere el factor de tolerancia, $k$. De la Tabla A.7 se tiene que $k=3.096$, mientras que, usando el software estad\'{\i}stico R, se obtiene un valor m\'as preciso con los siguientes comandos:
 \begin{verbatim}
>library(tolerance)
>options(digits=22)
>K.table(100,alpha=0.01,P=0.99,side=2,method=("WBE"))
$'100'
                        0.99
0.99 3.095533997752547783477
 \end{verbatim}
 \vspace{-0.5cm}
 por lo que tambi\'en se puede considerar con mayor precisi\'on como $3.09553$.
 \par
 Dado que se desea calcular un intervalo de toleranciaa de una poblaci\'on que se supone normal, entonces se usa la siguiente formulaci\'on:
 \begin{equation*}
  \bar{x}\pm ks
 \end{equation*}
 Por lo tanto, usando los datos obtenidos y considerando el valor $k$ del libro, se obtiene los siguientes c\'alculos:
 \begin{equation*}
  23\,500 \pm (3.096)(3\,900) = 23\,500 \pm 12\,074.4
 \end{equation*}
 Por lo tanto, el intervalo de tolerancia con el $99\%$ de seguridad de que contendr\'a el $99\%$ de las millas que recorren los autom\'oviles anualmente en Virginia es de $11\,425.6$ a $35\,574.4$ millas. El c\'alculo del intervalo de tolerancia usando el valor $k$ obtenido en R se puede obtener cambiando los siguientes comando del archivo anexo \texttt{P06\_Intervalo\_de\_tolerancia\_1.r}:
 \begin{verbatim}
>n<-100
>m<-23500
>s<-3900
>gamma<-0.01
>alfa<-0.01
 \end{verbatim}
 con lo que se obtiene un intervalo de tolerancia de $11\,427.42$ a $35\,572.58$ millas, que es a lo que se quer\'{\i}a llegar.${}_{\blacksquare}$
\end{solucion}
