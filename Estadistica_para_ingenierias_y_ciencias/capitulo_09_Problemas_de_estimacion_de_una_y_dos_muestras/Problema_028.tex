\begin{enunciado}
 Considere la situaci\'on del ejercicio 9.13. Aunque la estimaci\'on del di\'ametro medio sea importante, no es ni con mucho tan importante como intentar ``determinar'' la ubicaci\'on de la mayor\'{\i}a de la distribuci\'on de los di\'ametros. Para tal fin, encuentre los l\'{\i}mites de tolerancia de $95\%$ que contengan $95\%$ de los di\'ametros.
\end{enunciado}

\begin{solucion}
 Usando la notaci\'on y datos como en la soluci\'on del ejercicio 9.13 pero ahora siendo $\alpha = 0.05$ y representando el valor para el cual la proporci\'on $1-\alpha$ de la poblaci\'on es cubierta con $(1-\gamma)100\%$ de seguridad, entonces se tienen los siguientes datos:
 \begin{itemize}
  \item $X\sim n(\mu,\sigma)$.
  \item $\mu$ desconocida.
  \item $\sigma$ desconocida.
  \item $n = 9$ piezas.
  \item $\alpha = 0.05$.
  \item $\gamma = 0.05$.
  \item $\bar{x} = 1.00\overline{5}$.
  \item $s \approx 0.0245515331$.
 \end{itemize}
 Como se desea encontrar l\'{\i}mites de tolerancia, se requiere el factor de tolerancia, $k$. De la Tabla A.7 se tiene que $k = 3.532$, mientras que, usando el software estad\'{\i}stico R, se obtiene un valor m\'as preciso con los siguientes comandos:
 \begin{verbatim}
>library(tolerance)
>options(digits=22)
>K.table(9,alpha=0.05,P=0.95,side=2,method=("WBE"))
$`9`
                        0.95
0.95 3.531736753917766424848
 \end{verbatim}
 \vspace{-0.5cm}
 por lo que tambi\'en se puede considerar con mayor precisi\'on como $3.5317367539177664$.
 \par 
 Dado que se desea calcular un intervalo de tolerancia de una poblaci\'on que se supone normal, entonces se usa la siguiente formulaci\'on:
 \begin{equation*}
  \bar{x} \pm ks
 \end{equation*}
 Por lo tanto, usando los datos obtenidos y considerando el valor $k$ del libro, se obtiene los siguientes c\'alculos:
 \begin{equation*}
  \bar{x} \pm ks = 1.00\overline{5} \pm (3.532)(0.0245515331) = 1.00\overline{5} \pm 0.0867160149092
 \end{equation*}
 Por lo tanto, el intervalo de tolerancia con el $95\%$ de seguridad de que contendr\'a el $95\%$ de los di\'ametros est\'a dado desde $0.9188395406463\overline{5}$ hasta $1.0922715704647\overline{5}$. El c\'alculo del intervalo de tolerancia usando el valor $k$ obtenido en R se puede obtener cambiando los siguientes comandos del archivo anexo \texttt{P07\_Intervalo\_de\_tolerancia\_2.r}:
 \begin{verbatim}
>datos<-read.csv("DB01_Problema_13.csv",sep=";",encoding="UTF-8")
>varInteres<-c("Diámetro.cm")
>gamma<-0.05
>alfa<-0.05
 \end{verbatim}
 \vspace{-0.5cm}
 con lo que se obtiene un intervalo de tolerancia de $0.918846$ a $1.092265$ cent\'{\i}metros, que es a lo que se quer\'{\i}a llegar.${}_{\blacksquare}$
\end{solucion}
