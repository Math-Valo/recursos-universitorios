\begin{enunciado}
 Refi\'erase al ejercicio 9.15, construya un intervalo de tolerancia de $95\%$ que contenga $90\%$ de las mediciones.
\end{enunciado}

\begin{solucion}
 Usando la notaci\'on y datos como se explic\'o en la soluci\'on del ejercicio 9.7, y a eso a\~nadiendo los valores $\gamma$ y $\alpha$ del l\'{\i}mite de tolerancia pedido, se tiene lo siguiente:
 \begin{itemize}
  \item $X\sim\text{normal}(\mu,\sigma)$.
  \item $n=12$.
  \item $\bar{x}=48.5$.
  \item $s=1.5$.
  \item $\gamma=0.05$.
  \item $\alpha=0.1$.
 \end{itemize}
 Como se desea encontrar l\'{\i}mites de tolerancia, se requiere el factor de tolerancia, $k$. De la Tabla A.7 se tiene que $k=2.655$, mientras que, usando el software estad\'{\i}stico R, se obtiene un valor m\'as preciso con los siguientes comandos:
 \begin{verbatim}
>library(tolerance)
>options(digits=22)
>K.table(12,alpha=0.05,P=0.9,side=2,method=("WBE"))
$'12'
                         0.9
0.95 2.654958380278124696616
 \end{verbatim}
 \vspace{-0.5cm}
 por lo que tambi\'en se puede considerar con mayor precisi\'on como $2.65495838$.
 \par
 Dado que se desea calcular un intervalo de toleranciaa de una poblaci\'on que se supone normal, entonces se usa la siguiente formulaci\'on:
 \begin{equation*}
  \bar{x}\pm ks
 \end{equation*}
 Por lo tanto, usando los datos obtenidos y considerando el valor $k$ del libro, se obtiene los siguientes c\'alculos:
 \begin{equation*}
  \bar{x}\pm ks = 48.5 \pm (2.655)(1.5) = 48.5 \pm 3.9825
 \end{equation*}
 Por lo tanto, el intervalo de tolerancia con el $95\%$ de seguridad de que contendr\'a el $90\%$ de las durezas de Rockwell en las cabezas de los alfileres es de $44.5175$ hasta $52.4825$. El c\'alculo del intervalo de tolerancia usando el valor $k$ obtenido en R se puede obtener cambiando los siguientes comando del archivo anexo \texttt{P06\_Intervalo\_de\_tolerancia\_1.r}:
 \begin{verbatim}
>n<-12
>m<-48.5
>s<-1.5
>gamma<-0.05
>alfa<-0.1
 \end{verbatim}
 \vspace{-0.5cm}
 con lo que se obtiene un intervalo de tolerancia de $44.51756$ hasta $52.48244$, que es a lo que se quer\'{\i}a llegar.${}_{\blacksquare}$
\end{solucion}
