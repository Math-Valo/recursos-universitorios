\begin{enunciado}
 Calcule un intervalo de confianza de $98\%$ para la proporci\'on de art\'{\i}culos defectuosos en un proceso cuando se encuentra que una muestra de tama\~no $100$ da como resultado $8$ defectuosos.
\end{enunciado}

\begin{solucion}
 Sea $X$ la variable aleatoria de la cantidad de art\'{\i}culos defectuosos en el proceso que hace referencia el enunciado entre cada $n$ art\'{\i}culos, entonces $\widehat{P} = X/n$ es un estad\'{\i}stico de una proporci\'on en un experimento binomial que aproxima el valor de $p$, la proporci\'on total de art\'{\i}culos defectuosos en el proceso, entonces, del enunciado, se tienen los siguientes datos obtenidos de una muestra:
 \begin{itemize}
  \item $n=100$.
  \item $x=8$.
  \item $\alpha=0.02$.
 \end{itemize}
 por lo que $\hat{p}$, la proporci\'on de \'exitos en la muestra, y $\hat{q} = 1 - \hat{p}$ est\'an dados por:
 \begin{itemize}
  \item $\hat{p} = \frac{x}{n} = \frac{8}{100} = \frac{2}{25} = 0.08$; y,
  \item $\hat{q} = 1-\hat{p} = 1- \frac{2}{25} = 0.92$.
 \end{itemize}
 Adem\'as, como se buscar\'a un intervalo de confianza bilateral para estimar $p$, entonces se requiere del valor $z_{\alpha/2} = z_{0.01}$, el cual se calcul\'o en el ejercicio 9.6 y su aproximaci\'on es de $2.33$, aunque, en R, se puede considerar con mayor precisi\'on como $2.326347874$.
 \par 
 Ya que se busca un intervalo de confianza para la proporci\'on de un experimento binomial en donde el tama\~no de muestra es grande y se tiene que tanto $n\hat{p}$ como $n\hat{q}$ es mayor que o igual a $5$, entonces se usar\'a la f\'ormula de intervalo siguiente:
 \begin{equation*}
  \hat{p} - z_{\alpha/2}\sqrt{\frac{\hat{p}\hat{q}}{n}} < p < \hat{p} + z_{\alpha/2}\sqrt{\frac{\hat{p}\hat{q}}{n}}
 \end{equation*}
 Por lo tanto, usando los datos obtenidos y con la primera aproximaci\'on de $z_{\alpha/2}$, se tiene los siguientes c\'alculos de los l\'{\i}mites del intervalo de confianza como sigue:
 \begin{eqnarray*}
  \hat{p} \pm z_{\alpha/2}\sqrt{\frac{\hat{p}\hat{q}}{n}} & = & 0.08 \pm 2.33\sqrt{\frac{(0.08)(0.92)}{100}} = 0.08 \pm 2.33 \left( \frac{\sqrt{46}}{10(25)}\right) \\
  & = & 0.08 \pm \frac{233\sqrt{46}}{25\,000} = 0.08 \pm 0.00932\sqrt{46} \approx 0.08 \pm 0.06321131544
 \end{eqnarray*}
 Por lo tanto, el intervalo de confianza de $98\%$ de la proporci\'on de art\'{\i}culos defectuosos en aquel proceso es aproximadamente:
 \begin{equation*}
  0.016786845572725 < p < 
  0.14321131544
 \end{equation*}
 Finalmente, usando R, se puede calcular el intervalo de confianza usando el script en el archivo anexo \texttt{P17\_Intervalo\_de\_confianza\_08.r} cambiando las siguientes l\'{\i}neas de c\'odigo:
 \begin{verbatim}
> n<-100
> x<-8
> p<-NULL
> alfa<-0.02
> inter<-'D'
 \end{verbatim}
 \vspace{-0.5cm}
 con lo que se obtiene el siguientes resultado:
 \begin{verbatim}
     LimInf Proporción    LimSup
1 0.0168878       0.08 0.1431122
 \end{verbatim}
 \vspace{-0.5cm}
 por lo que, al redondear al decimal en que coinciden los resultados anteriores, se tiene que el intervalo de confianza del $98\%$ es $0.017 < p < 0.143$, que es a lo que se quer\'{\i}a llegar.${}_{\blacksquare}$
\end{solucion}
