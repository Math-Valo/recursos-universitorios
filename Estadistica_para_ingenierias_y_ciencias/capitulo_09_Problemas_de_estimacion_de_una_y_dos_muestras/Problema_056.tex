\begin{enunciado}
 Un genetista se interesa en la proporci\'on de hombres africanos que tienen cierto trastorno sangu\'{\i}neo menor. En una muestra aleatoria de $100$ hombres africanos, se encuentra que $24$ lo padecen.
 \begin{enumerate}
  \item Calcule un intervalo de confianza de $99\%$ para la proporci\'on de hombres africanos que tienen este trastorno sangu\'{\i}neo.
  
  \item ¿Qu\'e se puede asegurar con $99\%$ de confianza acerca de la posible magnitud de nuestro error, si estimamos que la proporci\'on de hombres africanos con dicho trastorno sangu\'{\i}neo es $0.24$?
 \end{enumerate}
\end{enunciado}

\begin{solucion}
 Sea $X$ la variable aleatoria de la cantidad de hombres africanos que tienen el trastorno sangu\'{\i}neo menor del que se hace referencia en el enunciado, de entre un total de $n$, entiendiendo que $n$ es peque\~na en comparaci\'on a $N$, la poblaci\'on total de hombres africanos, suponiendo que $n/N \leq 0.05$, y sea $k$ la cantidad total de hombres africanos que tienen el transtorno sangu\'{\i}neo, entonces $\widehat{P} = X/n$ es un estad\'{\i}stico de una proporci\'on en un experimento binomial que aproxima el valor de $k/N$, la proporci\'on buscada, entonces, del enunciado, se tienen lo siguientes datos obtenidos de una muestra:
 \begin{itemize}
  \item $n = 100$.
  \item $x = 24$.
  \item $\alpha = 0.01$.
 \end{itemize}
 por lo que $\hat{p}$, la proporci\'on de \'exitos en esta muestra, y $\hat{q} = 1 - \hat{p}$ est\'an dados por:
 \begin{itemize}
  \item $\hat{p} = \frac{x}{n} = \frac{24}{100} = \frac{6}{25} = 0.24$; y,
  \item $\hat{q} = 1 - \hat{p} = 1 - \frac{6}{25} = 0.76$.
 \end{itemize}
 Adem\'as, como se buscar\'a un intervalo de confianza bilateral y se calcular\'a el error de $\hat{p}$ para estimar $p$, entonces se requiere del valor $z_{\alpha/2} = z_{0.005}$, el cual se calcul\'o en el ejercicio 9.7 y su aproximaci\'on es de $2.575$, aunque, en R, se puede considerar con mayor precisi\'on como $2.5758293$.
 \begin{enumerate}
  \item Ya que se busca un intervalo de confianza para la proporci\'on en un experimento binomial en donde el tama\~no de muestra es grande y se tiene que tanto $n\hat{p}$ como $n\hat{q}$ es mayor que o igual a $5$, entonces se usar\'a la f\'ormula de intervalo siguiente:
  \begin{equation*}
   \hat{p} - z_{\alpha/2}\sqrt{\frac{\hat{p}\hat{q}}{n}} < p < \hat{p} + z_{\alpha/2}\sqrt{\frac{\hat{p}\hat{q}}{n}}
  \end{equation*}
  Por lo tanto, usando los datos obtenidos y con la primera aproximaci\'on de $z_{\alpha/2}$, se tienen los siguientes c\'alculos de los l\'{\i}mites del intervalo de confianza como sigue:
  \begin{eqnarray*}
   \hat{p} \pm z_{\alpha/2}\sqrt{\frac{\hat{p}\hat{q}}{n}} & = & 0.24 \pm 2.575\sqrt{\frac{(0.24)(0.76)}{100}} = 0.24 \pm 2.575\left( \frac{\sqrt{114}}{10(25)} \right) \\
   & = & 0.24 \pm \frac{103\sqrt{114}}{10\,000} = 0.24 \pm 0.0103\sqrt{114} \approx 0.24 \pm 0.1099739
  \end{eqnarray*}
  Por lo tanto, el intervalo de confianza de $99\%$ de la proporci\'on de hombres africanos que tienen este trastorno sangu\'{\i}neo es aproximadamente
  \begin{equation*}
   0.130026094 < p <
   0.3499739
  \end{equation*}
  Por otro lado, usando R, se puede calcular el intervalo de confianza usando el script en el archivo anexo \texttt{P17\_Intervalo\_de\_confianza\_08.r} cambiando las siguientes l\'{\i}neas de c\'odigo:
  \begin{verbatim}
> n<-100
> x<-24
> p<-NULL
> alfa<-0.01
> inter<-'D'
  \end{verbatim}
  \vspace{-0.5cm}
  con lo que se obtiene el siguiente resultado:
  \begin{verbatim}
     LimInf Proporción    LimSup
1 0.1299907       0.24 0.3500093
  \end{verbatim}
  \vspace{-0.5cm}
  por lo que, al redondear al decimal en que coinciden los resultados anteriores, se tiene que el intervalo de confianza del $99\%$ es $0.13 < p < 0.35$.${}_{\square}$
  
  \item Dado que $\hat{p} = 0.24$ estima a la fracci\'on de hombres africanos que tienen este trastorno sangu\'{\i}neo, entonces, se sabe que el error tiene una magnitud de a lo m\'as
  \begin{equation*}
   e = z_{\alpha/2}\sqrt{\frac{\hat{p}\hat{q}}{n}}
  \end{equation*}
  entonces, por los c\'alculos previos, se sabe, con un $99\%$ de confianza, que la magnitud del error de estimaci\'on es de a lo m\'as
  \begin{equation*}
   e = z_{\alpha/2}\sqrt{\frac{\hat{p}\hat{q}}{n}} = 0.24 \pm \frac{103\sqrt{114}}{10\,000} = 0.24 \pm 0.0103\sqrt{114} \approx 0.24 \pm 0.1099739
  \end{equation*}
  Finalmente, usando R, se puede calcular la magnitud del error usando el script en el archivo anexo \texttt{P18\_Estimacion\_del\_error\_2.r} cambiando las siguientes l\'{\i}neas de c\'odigo:
  \begin{verbatim}
> n<-100
> x<-24
> p<-NULL
> alfa<-0.01
> inter<-'D'
  \end{verbatim}
  \vspace{-0.5cm}
  con lo que se obtiene el siguiente resultado:
  \begin{verbatim}
[1] 0.1100093
  \end{verbatim}
  \vspace{-0.5cm}
  Por lo que, al redondear al decimal en que coinciden los resultados anteriores, se tiene con un $99\%$ de confianza que el margen de error no supera el valor de $e = 0.11$, que es a lo que se quer\'{\i}a llegar.${}_{\blacksquare}$
 \end{enumerate}
\end{solucion}
