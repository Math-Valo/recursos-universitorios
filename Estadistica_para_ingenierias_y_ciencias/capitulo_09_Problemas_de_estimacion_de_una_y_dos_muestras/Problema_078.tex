\begin{enunciado}
 Construya un intervalo de confianza de $90\%$ para $\sigma_1^2/\sigma_2^2$ en el ejercicio 9.43 de la p\'agina 298. ¿Estamos justificados al suponer que $\sigma_1^2 \neq \sigma_2^2$ cuando construimos nuestro intervalo de confianza para $\mu_1 - \mu_2$?
\end{enunciado}

\begin{solucion}
 Usando la notaci\'on y datos como se explic\'o en la soluci\'on del ejercicio 9.43, pero cambiando el significado de $\alpha$ al nivel de significancia del intervalo de confianza para $\sigma_1^2/\sigma_2^2$, entonces se tiene los siguientes datos:
 \begin{itemize}
  \item $X_i \sim n\left( \mu_i, \sigma_i \right)$, para cada $i \in \{ 1, 2 \}$.
  \item $\sigma_i$ desconocidas, para cada $i \in \{ 1, 2 \}$.
  \item $n_1 = n_2 = 12$ neum\'aticos.
  \item $\bar{x}_1 = 36\,300$ y $\bar{x}_2 = 38\,100$.
  \item $s_1 = 5\,000$ y $s_2 = 6\,100$.
  \item $\alpha = 0.1$.
 \end{itemize}
 Por otro lado, como se desea encontrar el intervalo de confianza bilateral para la proporci\'on de varianzas de poblaciones independientes y normalmente distribuidas, entonces se requerir\'a de los valores $f_{\alpha/2}(n_1 - 1, n_2-1) = f_{0.05}(11,11)$ y $f_{\alpha/2}(n_2-1,n_1-1) = f_{0.05}(11,11)$, que en este caso coinciden. De la Tabla A.6, en el ap\'endice A del libro, se puede interpolar para obtener que $f_{0.05}(11,11) = 2.82$. Por otro lado, usando R, con los siguientes comandos, se obtiene mayor precisi\'on.
 \begin{verbatim}
> options(digits=22)
> qf(0.05,11,11,lower.tail=F)
[1] 2.817930469953086269896
 \end{verbatim}
 \vspace{-0.5cm}
 Por lo que tambi\'en se puede considerar, con mayor precisi\'on que: $f_{0.05}(11,11) = 2.81793$.
 \par 
 Ya que se busca un intervalo de confianza bilateral para la proporci\'on de varianzas de poblaciones normalmente distribuidas usando la proporci\'on de las desviaciones est\'andar muestrales como estimador, entonces se usar\'a la f\'ormula de intervalo siguiente:
 \begin{equation*}
  \frac{s_1^2}{s_2^2}\cdot\frac{1}{f_{0.05}(11,11)} < \frac{\sigma_1^2}{\sigma_2^2} < \frac{s_1^2}{s_2^2}\cdot f_{\alpha/2}(n_2-1,n_1-1)
 \end{equation*}
 Por lo tanto, usando los datos obtenidos y considerando el valor $f_{0.05}(11,11)$ del libro, se tienen los c\'alculos de los l\'{\i}mites del intervalo de confianza como sigue:
 \begin{equation*}
  \frac{s_1^2}{s_2^2}\cdot\frac{1}{f_{0.05}(11,11)} = \frac{5\,000^2}{6\,100^2} \cdot \frac{1}{2.82} = \frac{50^2}{61^2} \cdot \frac{50}{141} = \frac{2\,500(50)}{3\,721(141)} = \frac{125\,000}{524\,661} \approx 0.2382
 \end{equation*}
 y
 \begin{equation*}
  \frac{s_1^2}{s_2^2}\cdot f_{\alpha/2}(n_2-1,n_1-1) = \frac{5\,000^2}{6\,100^2} (2.82) = \frac{50^{\cancel{2}}}{61^2}\left( \frac{141}{\cancel{50}} \right) = \frac{50(141)}{3\,721} = \frac{7\,050}{3\,721} \approx 1.89465
 \end{equation*}
 Por lo tanto, el intervalo de confianza de $90\%$ para la proporci\'on de las varianzas de los kil\'ometros promedio recorridos hasta desgastarse los neum\'aticos de la marca A y de la marca B es aproximadamente:
 \begin{equation*}
  0.2382 < \frac{\sigma_1^2}{\sigma_2^2} < 1.89465
 \end{equation*}
 Por otro lado, usando R, se puede calcular el intervalo de confianza usando el script con el c\'odigo en \texttt{P23\_Intervalo\_de\_confianza\_12.r}, cambiando las siguientes l\'{\i}neas de c\'odigo:
 \begin{verbatim}
> n1<-12
> n2<-12
> var1<-NULL
> var2<-NULL
> desv.est1<-5000
> desv.est2<-6100
> alfa<-0.1
> tipoInterv<-"var"
> inter<-'D'
> val<-TRUE
 \end{verbatim}
 \vspace{-0.5cm}
 con lo que se obtiene el siguiente resultado:
 \begin{verbatim}
  Estimando n1 n2 desv.Est1 desv.Est2    LimInf     razon   LimSup    Pvalor
1       var 12 12      5000      6100 0.2384241 0.6718624 1.893261 0.5204461
          Resultado
1 Var no diferentes
 \end{verbatim}
 \vspace{-0.5cm}
 Por lo que, al redondear al decimal en que coinciden los resultados anteriores, se tiene que el intervalo de confianza del $90\%$ es $0.23 < \frac{\sigma_1^2}{\sigma_2^2} < 1.89$. Por lo tanto, no est\'a justificado suponer que $\sigma_1^2 \neq \sigma_2^2$, que es a lo que se quer\'{\i}a llegar.${}_{\blacksquare}$
\end{solucion}
