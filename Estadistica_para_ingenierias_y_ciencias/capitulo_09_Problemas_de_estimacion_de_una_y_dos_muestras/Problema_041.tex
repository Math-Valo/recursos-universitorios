\begin{enunciado}
 Los siguientes datos, registrados en d\'{\i}as, representan el tiempo de recuperaci\'on para pacientes que se tratan al azar con uno de dos medicamentos para curar infecciones de la vejiga:
 \begin{center}
  \begin{tabular}{ccc}
   \textbf{Medicamento 1} & \hspace{1cm} & \textbf{Medicamento 2} \\
   \hline 
   $n_1 = 14$ & & $n_2 = 16$ \\
   $\bar{x}_1 = 17$ & & $\bar{x}_2 = 19$ \\
   $s_1^2 = 1.5$ & & $s_2^2 = 1.8$
  \end{tabular}
 \end{center}
 Encuentre un intervalo de confianza de $99\%$ para la diferencia $\mu_2 - \mu_1$ en el tiempo medio de recuperaci\'on para los dos medicamentos. Suponga poblaciones normales con varianzas iguales.
\end{enunciado}

\begin{solucion}
 Sean $X_1$ y $X_2$ las variables aleatorias del tiempo de recuperaci\'on para pacientes que usan el medicamento 1 o 2 referenciado en el enunciado, respectivamente, para curar infecci\'on en la vejiga, medido en d\'{\i}as, y usando el resto de la notaci\'on cmo se usa en el enunciado, entonces se tiene lo siguiente:
 \begin{itemize}
  \item $X_i \sim n(\mu_i, \sigma_i)$, para cada $i \in \{ 1, 2 \}$.
  \item $\mu_i$ y $\sigma_i$ desconocidas, para cada $i \in \{ 1, 2 \}$.
  \item $\sigma_1 = \sigma_2$.
  \item $n_1 = 14$ y $n_2 = 16$.
  \item $\bar{x}_1 = 17$ y $\bar{x}_2 = 19$.
  \item $s_1^2 = 1.5$ y $s_2^2 = 1.8$.
  \item $\alpha = 0.01$.
 \end{itemize}
 Como se desea encontrar un intervalo de confianza bilateral para $\mu_2 - \mu_1$, usando como estimador $\bar{x}_2 - \bar{x}_1$, desconociendo las varianzas poblacionales aunque suponi\'endolas iguales, con muestras peque\~nas de poblaciones que se distribuyen aproximadamente normal, entonces se requerir\'a del valor de $t_{\alpha/2,n_1+n_2-2} = t_{0.005,28}$, lo cual ya se obtuvo en ejercicio 39. De la Tabla A.4, se tiene que $t_{0.005,28} = 2.763$, mientras que, usando R, se obtiene el aproximado $2.763262455461$.
 \par 
 Ya que se busca un intervalo de confianza para la diferencia de las medias poblacionales usando como estimador la diferencia de las medias muestrales en muestras peque\~nas, en donde se desconoce las desviaciones est\'andar poblacionales pero suponiendo que son iguales y donde se suponen que las poblaciones se distribuyen aproximadamente normal, entonces se usar\'a la siguiente formulaci\'on:
 \begin{equation*}
  \left( \bar{x}_1 - \bar{x}_2  \right) - t_{\alpha/2,n_1+n_2-2}s_p \sqrt{\frac{1}{n_1} + \frac{1}{n_2}} < \mu_1 - \mu_2 < \left( \bar{x}_1 - \bar{x}_2  \right) + t_{\alpha/2,n_1+n_2-2}s_p \sqrt{\frac{1}{n_1} + \frac{1}{n_2}}
 \end{equation*}
 en donde
 \begin{equation*}
  s_p = \sqrt{\frac{\left( n_1-1 \right)s_1^2 + \left( n_2 - 1 \right)s_2^2}{n_1 + n_2 - 2}}
 \end{equation*}
 El cual se modifica levemente para obtener el intervalo de confianza de $\mu_2 - \mu_1$, que es lo que se pide, usando un cambio de variables en donde $X_1$ es ahora $X_2$ y vice versa:
 \begin{equation*}
  \left( \bar{x}_2 - \bar{x}_1  \right) - t_{\alpha/2,n_1+n_2-2}s_p \sqrt{\frac{1}{n_1} + \frac{1}{n_2}} < \mu_2 - \mu_1 < \left( \bar{x}_2 - \bar{x}_1  \right) + t_{\alpha/2,n_1+n_2-2}s_p \sqrt{\frac{1}{n_1} + \frac{1}{n_2}}
 \end{equation*}
 Por lo tanto, usando los datos obtenidos, considerando el valor $t_{\alpha/2,n_1+n_2-2}$ del libro, se tienen los c\'alculos de los l\'{\i}mites del intervalo de confianza como siguen:
 \begin{eqnarray*}
  s_p & = & \sqrt{\frac{\left( n_1-1 \right)s_1^2 + \left( n_2 - 1 \right)s_2^2}{n_1 + n_2 - 2}} = \sqrt{\frac{(14-1)(1.5)+(16-1)(1.8)}{14+16-2}} = \sqrt{\frac{(13)(1.5)+(15)(1.8)}{28}} \\
  & = & \sqrt{\frac{19.5 + 27}{28}} = \sqrt{\frac{46.5}{28}} = \sqrt{\frac{93}{56}} = \frac{\sqrt{1302}}{28}
 \end{eqnarray*}
 y
 \begin{eqnarray*}
  \left( \bar{x}_2 - \bar{x}_1  \right) \pm t_{\alpha/2,n_1+n_2-2}s_p \sqrt{\frac{1}{n_1} + \frac{1}{n_2}}
  & = & (19-17) \pm (2.763) \left( \frac{\sqrt{1302}}{28} \right) \sqrt{\frac{1}{14} + \frac{1}{16}} \\
  & = & 2 \pm 0.09867\overline{857142}\sqrt{1302}\sqrt{\frac{15}{112}} \\
  & = & 2 \pm 0.09867\overline{857142} \frac{\sqrt{93}\sqrt{15}}{\sqrt{8}} \\
  & = & 2 \pm 0.09867\overline{857142}\frac{\sqrt{1395}\sqrt{2}}{4} \\
  & = & 2 \pm 0.0246696\overline{428571} \sqrt{2790}
 \end{eqnarray*}
 Por lo tanto, el intervalo del $99\%$ de confianza de la diferencia de la media del tiempo de curaci\'on de infecci\'on en la vejiga usando el medicamento 2 menos la media del tiempo de curaci\'on usando el medicamento 1, en d\'{\i}as, es de:
 \begin{equation*}
  0.6969383485152542 < \mu_1 - \mu_2 < 3.303061651484745775
 \end{equation*}
 Finalmente, en R se puede calcular el intervalo de confianza usando el script en el archivo anexo \texttt{P14\_Intervalo\_de\_confianza\_05.r} cambiando las siguientes l\'{\i}neas de c\'odigo:
 \begin{verbatim}
> n1<-16
> n2<-14
> m1<-19
> m2<-17
> desv.tipica1<-sqrt(1.8)
> desv.tipica2<-sqrt(1.5)
> alfa<-0.01
> val<-FALSE
> varia<-TRUE
> inter<-'D'
 \end{verbatim}
 \vspace{-0.5cm}
 con lo que se obtiene el siguiente resultado:
 \begin{verbatim}
  n1 n2 media1 media2    LimInf diferencia   LimSup
1 16 14     19     17 0.6968146          2 3.303185
 \end{verbatim}
 \vspace{-0.5cm}
 por lo que, al redondear al decimal en que coinciden los resultados anteriores, se tiene que el intervalo de confianza del $99\%$ es $0.697 < \mu_1 - \mu < 3.303$, que es a lo que se quer\'{\i}a llegar.${}_{\blacksquare}$
\end{solucion}
