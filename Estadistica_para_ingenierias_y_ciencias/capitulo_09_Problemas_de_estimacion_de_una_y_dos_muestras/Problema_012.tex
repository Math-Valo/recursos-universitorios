\begin{enunciado}
 El consumo regular de cereales preendulzados contribuye a la ca\'{\i}da de los dientes, a las enfermedades cardiacas y a otras enfermedades degenerativas, seg\'un estudios realizados por el doctor W. H. Bowen del Instituto Nacional de Salud y el doctor J. Yudben, profesor de nutrici\'on y diet\'etica de la Universidad de Londres.
 En una muestra aleatoria de $20$ porciones sencillas similares del cereal Alpha-Bits, el contenido promedio de az\'ucar fue de $11.3$ gramos con una desviaci\'on est\'andar de $2.45$ gramos.
 Suponiendo que el contenido de az\'ucar est\'a distribuido normalmente, construya un intervalo de confianza de $95\%$ para el contenido medio de az\'ucar para porciones sencillas de Alpha-Bits.
\end{enunciado}

\begin{solucion}
 Sea $X$ la variable aleatoria del contenido de az\'ucar en una porci\'on los cereales Alpha-Bits, medido en gramos, el enunciado del problema aporta los siguientes datos:
 \begin{itemize}
  \item $X\sim\text{normal}(\mu,\sigma)$.
  \item $\mu$ desconocida.
  \item $\sigma$ desconocida.
  \item $n=20$ porciones.
  \item $\bar{x}=11.3\,$g.
  \item $s=2.45\,$g.
  \item $\alpha=0.05$.
 \end{itemize}
 Dado que se desconoce la desviaci\'on est\'andar poblacional y la muestra no es lo suficientemente grande, se requerir\'a del valor de $t_{\alpha/2,n-1} = t_{0.025,19}$. De la Tabla A.4, en el Ap\'endice A del libro, se tiene que vale $2.093$. Por otro lado, usando el software estad\'{\i}stico R, con los siguientes comandos, se obtiene un valor m\'as preciso.
 \begin{verbatim}
>options(digits=22)
>qt(0.025,19,lower.tail=F)
[1] 2.093024054408309631015
 \end{verbatim}
 \vspace{-0.5cm}
 por lo que tambi\'en se puede considerar con mayor precisi\'on como $2.093024$.
 \par 
 Dado que se desea calcular un intervalo de confianza para la media poblacional usando la media muestral, sin conocer la desviaci\'on est\'andar poblacional y con una muestra peque\~na, entonces se debe de usar la siguiente formulaci\'on:
 \begin{equation*}
  \bar{x}-t_{\alpha/2,n-1}\frac{s}{\sqrt{n}} < \mu < \bar{x}+t_{\alpha/2,n-1}\frac{s}{\sqrt{n}}
 \end{equation*}
 Por lo tanto, usando los datos obtenidos, y considerando el valor de $t_{\alpha/2,n-1}$, se tiene los c\'alculos de los l\'{\i}mites del intervalo de confianza como siguen:
 \begin{equation*}
  \bar{x}\pm t_{\alpha/2,n-1}\frac{s}{\sqrt{n}} = 11.3\pm 2.093\left( \frac{2.45}{\sqrt{20}} \right) =11.3\pm\frac{5.12785}{2\sqrt{5}} =11.3\pm 0.512785\sqrt{5}
 \end{equation*}
 Por lo tanto, el intervalo del $95\%$ de confianza de la media de az\'ucar en los cereales Alpha-Bits, medida en gramos, es de aproximadamente
 \begin{equation*}
  10.153377882 < \mu < 12.4466221
 \end{equation*}
 El c\'alculo del intervalo de confianza usando el valor $t_{\alpha/2,n-1}$ obtenido en R se puede realizar con los siguientes comandos, el cual es una modificaci\'on al primer script de R para el c\'alculo de intervalos de confianza, al cual se le agregaron unas l\'{\i}neas para presentar el resultado en una \'unica variable que muestre el l\'{\i}mite inferior, la media muestral y el l\'{\i}mite superior del intervalo de confianza, nombrados como \texttt{LimInf}, \texttt{Media} y \texttt{LimSup}, respectivamente, y que se encuentran registrados en el programa anexo \texttt{P04\_Intervalo\_de\_confianza\_02.r}
 \begin{verbatim}
>n<-20
>m<-11.3
>s<-2.45
>alfa<-0.05
>talfa<-qt(1-alfa/2,n-1)
>LL<-round(m-talfa*s/sqrt(n),7)
>LU<-round(m+talfa*s/sqrt(n),7)
>IC<-data.frame(LL,m,LU)
>names(IC)[names(IC) == "LL"]<-"LimInf"
>names(IC)[names(IC) == "m"]<-"Media"
>names(IC)[names(IC) == "LU"]<-"LimSup"
>IC
      LimInf Media   LimSup
[1] 10.15336  11.3 12.44664
 \end{verbatim}
 \vspace{-0.5cm}
 con lo que se obtiene el intervalo de confianza m\'as preciso
 \begin{equation*}
  10.15336 < \mu < 12.44664
 \end{equation*}
 Que es a lo que se quer\'{\i}a llegar.${}_{\blacksquare}$
\end{solucion}
