\begin{enunciado}
 Una muestra aleatoria de $25$ botellas de aspirinas contiene, en promedio, $325.05\,$mg de aspirina con una desviaci\'on est\'andar de $0.5$. Encuentre los l\'{\i}mites de tolerancia de $95\%$ que contendr\'an $90\%$ del contenido de aspirina para esta marca. Suponga que el contenido de aspirina se distribuye normalmente.
\end{enunciado}

\begin{solucion}
 Sea $X$ la variable aleatoria de la cantidad de miligramos de aspirina que contiene una botella de aspirinas, el enunciado provee de los siguientes datos:
 \begin{itemize}
  \item $X\sim\text{normal}(\mu,\sigma)$.
  \item $n=25$.
  \item $\bar{x} = 325.05\,$mg.
  \item $s=0.5\,$mg.
  \item $\gamma=0.05$.
  \item $\alpha=0.1$.
 \end{itemize}
 Entonces, como se desea encontrar l\'{\i}mites de tolerancia, se requiere encontrar el factor de tolerancia, $k$. De la Tabla A.7, en el Ap\'endice A del libro, se tiene que $k = 2.208$, mientras que, usando el software estad\'{\i}stico R, y con el paquete \texttt{tolerance} (que se debe instalar previamente), se obtiene un valor m\'as preciso con los siguientes comandos:
 \begin{verbatim}
>library(tolerance)
>options(digits=22)
>K.table(25,alpha=0.05,P=0.9,side=2,method=c("WBE"))
$'25'
                         0.9
0.95 2.208321826635702755937
 \end{verbatim}
 \vspace{-0.5cm}
 por lo que tambi\'en se puede considerar con mayor precisi\'on como $2.2083218266357$.
 \par
 Dado que se desea calcular un intervalo de tolerancia de una variable que se supone normal, entonces se usa la siguiente formulaci\'on:
 \begin{equation*}
  \bar{x}\pm ks
 \end{equation*}
 Por lo tanto, usando los datos obtenidos, y considerando el valor $k$ del libro, se obtiene los siguientes c\'alculos:
 \begin{equation*}
  \bar{x}\pm ks = 325.05\pm (2.208)(0.5) = 325.05 \pm 1.104
 \end{equation*}
 Por lo tanto, el intervalo de tolerancia con el $95\%$ de seguridad de que contendr\'a el $90\%$ del contenido de las aspirinas para esta marca es de $323.946$ a $326.154$ gramos. El c\'alculo del intervalo de tolerancia usando el valor $k$ obtenido en R se puede realizar usando los siguientes comandos, registrados en el archivo anexo \texttt{P06\_Intervalo\_de\_tolerancia\_1.r}.
 \begin{verbatim}
>n<-25
>m<-325.05
>s<-0.5
>gamma<-0.05
>alfa<-0.1
>k<-K.factor(n,alpha=gamma,P=1-alfa,side=2,method=c("WBE"))
>LL<-m-k*s
>LU<-m+k*s
>LL;LU;
[1] 323.9458390866821559939
[1] 326.1541609133178667435
 \end{verbatim}
 \vspace{-0.5cm}
 que es a lo que se quer\'{\i}a llegar.${}_{\blacksquare}$
\end{solucion}
