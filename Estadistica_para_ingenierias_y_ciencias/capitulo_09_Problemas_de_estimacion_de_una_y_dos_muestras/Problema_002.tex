\begin{enunciado}
 Si $X$ es una variable aleatoria binomial, demuestre que
 \begin{enumerate}
  \item $P = X/n$ es un estimador insesgado de $p$;
  \item $P' = \frac{X + \sqrt{n}/2}{n + \sqrt{n}}$ es un estimador sesgado de $p$.
 \end{enumerate}
\end{enunciado}

\begin{solucion}
 Por resultado conocido, una variable aleatoria binomial con par\'ametros $n$ y $p$ tiene una esperanza $\mu = np$, entonces
 \begin{enumerate}
  \item Si $P=\frac{X}{n}$, se sigue que
  \begin{equation*}
   E(P) = E\left( \frac{X}{n} \right) = \frac{E(X)}{n} = \frac{np}{n} = p
  \end{equation*}
  Por lo que $P$ es un estimador insesgado de $p$.
  
  \item Si $P' = \frac{X + \sqrt{n}/2}{n + \sqrt{n}}$, entonces
  \begin{equation*}
   E(P') = E\left( \frac{X + \sqrt{n}/2}{n + \sqrt{n}} \right) = \frac{E(x) + \sqrt{n}/2}{n + \sqrt{n}} = \frac{np + \sqrt{n}/2}{n + \sqrt{n}}
  \end{equation*}
  Luego, $P'$ ser\'{\i}a un estimador insesgado si y s\'olo si
  \begin{eqnarray*}
   & & \frac{np + \sqrt{n}/2}{n + \sqrt{n}} = p \\
   \Leftrightarrow & & \frac{2np+\sqrt{n}}{2\left(n+\sqrt{n}\right)} = p \\ 
   \Leftrightarrow & & 2np + \sqrt{n} = 2p\left( n + \sqrt{n} \right) \\
   \Leftrightarrow & & 2p\left( n + \sqrt{n} \right) - 2np = \sqrt{n} \\ 
   \Leftrightarrow & & 2p\left( \cancel{n} + \sqrt{n} - \cancel{n} \right) = \sqrt{n} \\ 
   \Leftrightarrow & & p = \frac{\sqrt{n}}{2\sqrt{n}} = \frac{1}{2}
  \end{eqnarray*}
  Luego entonces, $P'$ es un estimador sesgado de $p$ salvo que $p = 0.5$, lo cual s\'olo es un resultado te\'orico, ya que en realidad no se tiene certidumbre sobre el valor de $p$.
  Por lo tanto, se considera $P'$ simplemente como un estimador sesgado de $p$, Q.E.D.${}_{\blacksquare}$
 \end{enumerate}
\end{solucion}

