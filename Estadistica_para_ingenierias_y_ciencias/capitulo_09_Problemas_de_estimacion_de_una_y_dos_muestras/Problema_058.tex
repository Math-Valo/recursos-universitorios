\begin{enunciado}
 En el art\'{\i}culo del peri\'odico al que se hace referencia en el ejercicio 9.57, $32\%$ de los $1\,600$ adultos encuestados dijeron que el programa espacial estadounidense deber\'{\i}a enfatizar la exploraci\'on cient\'{\i}fica. ¿Qu\'e tan grande se necesita que sea una muestra de adultos en la encuesta si se desea tener una confianza de $95\%$ de que el porcentaje estimado est\'e dentro de $2\%$ del porcentaje real?
\end{enunciado}

\begin{solucion}
 Usando la notaci\'on y datos como se explic\'o en la soluci\'on del ejercicio 9.57, pero cambiando ahora el significado de los casos de \'exito como los adultos estadounidense que dijeron que el programa deber\'{\i}a enfatizar la exploraci\'on cient\'{\i}fica y a\~nadiendo el error m\'aximo en el que se debe encontrar $\hat{p}$ para estimar la proporci\'on, se tiene lo siguiente:
 \begin{itemize}
  \item $n=1600$ entrevistados previamente.
  \item $\hat{p} = 32\%$, proporci\'on previa de \'exitos.
  \item $\hat{q} = 1-\hat{p} = 68\%$, proporci\'on previa de fracasos.
  \item $\alpha=0.05$.
  \item $z_{\alpha/2} = 1.96$, seg\'un el libro, y $z_{\alpha/2} = 1.95996398454$, con la aproximaci\'on realizada en R.
  \item $e=0.02$.
 \end{itemize}
 Entonces, como $\hat{p}$ estima a $p$ y hay una muestra previa, se usa el siguiente resultado:
 \begin{equation*}
  n = \left\lceil \frac{z_{\alpha/2}^2\hat{p}\hat{q}}{e^2} \right\rceil
 \end{equation*}
 por lo tanto, el valor pedido se puede calcular, usando la primera aproximaci\'on de $z_{\alpha/2}$, como
 \begin{equation*}
  n = \left\lceil \frac{1.96^2(0.32)(0.68)}{0.02^2} \right\rceil = \left\lceil \frac{(3.8416)(0.2176)}{0.0004} \right\rceil = \lceil 2\,089.8304 \rceil
 \end{equation*}
 Por lo tanto, el tama\~no de la muestra buscado es de $n = 2\,090$.
 \par 
 Finalmente, usando R, se escribe una rutina para calcular el tama\~no de muestra y se registra en el archivo anexo \texttt{P19\_Tamanyo\_de\_muestra\_2.r}, el cual considera realiza los c\'alculos seg\'un si hubo muestra previa o no, y, en dado caso de que haya habido una muestra previa, calcula el valor de la proporci\'on estimada en caso de no se d\'e el valor directamente de la proporci\'on. A continuaci\'on se muestra el c\'odigo escrito:
 \begin{verbatim}
> error<-0.02
> alfa<-0.05
> inter<-'D'
> previo<-TRUE
> n<-1600
> x<-NULL
> p<-0.32
> if(inter=='D'){
+    zalfa<-qnorm(1-alfa/2)
+ }else{
+    zalfa<-qnorm(1-alfa)
+ }
> if(previo){
+    if(is.null(p)){
+       p<-x/n
+    }
+ }else{
+    p<-1/2
+ }
> q<-1-p
> n<-ceiling((zalfa/error)^2*p*q)
> n
[1] 2090
 \end{verbatim}
 \vspace{-0.5cm}
 en donde el \'ultimo valor indica el resultado final del tama\~no de muestra calculado, el cual coincide con el valor anteriormente calculado a mano, que es a lo que se quer\'{\i}a llegar.${}_{\blacksquare}$
\end{solucion}
