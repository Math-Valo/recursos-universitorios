\begin{enunciado}
 ¿Qu\'e tan grande se necesita la muestra en el ejercicio 9.54, si deseamos tener una confianza de $98\%$ de que nuestra proporci\'on de la muestra est\'e dentro del $0.05$ de la proporci\'on real de defectuosos?
\end{enunciado}

\begin{solucion}
 Usando la notaci\'on y datos como se explic\'o en el ejercicio 9.54 y a\~nadiendo el error m\'aximo en el que se debe de encontrar $\hat{p}$ para estimar la proporci\'on, se tiene lo siguiente:
 \begin{itemize}
  \item $n = 100$ art\'{\i}culos totales de una muestra previa.
  \item $x = 8$ art\'{\i}culos defectuosos en la muestra previa.
  \item $\hat{p} = \frac{2}{25} = 0.08$.
  \item $\hat{q} = \frac{23}{25} = 0.92$.
  \item $\alpha = 0.02$
  \item $z_{\alpha/2} = 2.33$, seg\'un el libro, y $z_{\alpha/2} = 2.326347874$, con la aproximaci\'on realizada en R.
  \item $e = 0.05$.
 \end{itemize}
 Entonces, como $\hat{p}$ estima $p$ y hay una muestra previa, se usa el siguiente resultado:
 \begin{equation*}
  n = \left\lceil \frac{z_{\alpha/2}^2\hat{p}\hat{q}}{e^2} \right\rceil
 \end{equation*}
 por lo tanto, el valor pedido se puede calcular, usando la primera aproximaci\'on de $z_{\alpha/2}$, como
 \begin{equation*}
  n = \left\lceil \frac{2.33^2(0.08)(0.92)}{0.05^2} \right\rceil = \left\lceil \frac{5.4289(0.0736)}{0.0025} \right\rceil = \lceil 159.826816 \rceil
 \end{equation*}
 Por lo tanto, el tama\~no de la muestra buscada es de $n = 160$.
 \par 
 Finalmente, usando R, se puede calcular el tama\~no de la muestra usando la rutina en el archivo anexo \texttt{P19\_Tamanyo\_de\_muestra\_2.r}, cambiando las siguientes l\'{\i}neas de c\'odigo:
 \begin{verbatim}
> error<-0.05
> alfa<-0.02
> inter<-'D'
> previo<-TRUE
> n<-100
> x<-8
> p<-NULL
 \end{verbatim}
 \vspace{-0.5cm}
 con lo que se obtiene el siguiente resultado:
 \begin{verbatim}
[1] 160
 \end{verbatim}
 \vspace{-0.5cm}
 que coincide con el valor ya calculado. Por lo tanto $n=160$, que es a lo que se quer\'{\i}a llegar.${}_{\blacksquare}$
\end{solucion}
