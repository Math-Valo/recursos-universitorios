\begin{enunciado}
 Una empresa de material el\'ectrico fabrica bombillas de luz que tienen una duraci\'on aproximadamente distribuida de forma normal, con una desviaci\'on est\'andar de $40$ horas.
 Si una muestra de $30$ bombillas tiene una duraci\'on promedio de $780$ horas, encuentre un intervalo de confianza de $96\%$ para la media de la poblaci\'on de todas las bombillas que produce esta empresa.
\end{enunciado}

\begin{solucion}
 Sea $X$ la variable aleatoria de la duraci\'on de las bombilas de luz fabricadas en la empresa del enunciado, medido en horas, y $\widehat{X}$ la variable aleatoria de la muestra tomada de $n$ elementos y media muestral $\bar{x}$, y llamando $\alpha$ al nivel de confianza, se tienen los siguientes datos:
 \begin{itemize}
  \item $X \sim \text{normal}(\mu, \sigma)$.
  \item $\mu_X$ es desconocida
  \item $\sigma_X = 40\,\text{h}$.
  \item $n = 30$
  \item $\bar{x} = 780$
  \item $\alpha = 0.04$
 \end{itemize}
 Como la poblaci\'on tiene distribuci\'on normal y se conoce la varianza poblacional, se va usar $z_{\alpha/2} = z_{0.02}$. De la Tabla A.3, en el ap\'endice A del libro, \'este es un valor entre $2.05$ y $2.06$, con mayor aproximidad a $2.05$, que es el que se tomar\'a. Por otro lado, usando en el software estad\'{\i}stico R con los siguientes comandos, se obtiene mayor precisi\'on.
  \begin{verbatim}
>options(digits=22)
>qnorm(0.02,mean = 0, sd = 1, lower.tail = F)
[1] 2.053748910631822521822
  \end{verbatim}
  \vspace{-0.5cm}
  por lo que tambi\'en se puede considerar, con mayor precisi\'on, como $2.05374891$.
 \par 
 Ya que se busca un intervalo de confianza para la media de una poblaci\'on que se distribuye normalmente y se conoce la desviaci\'on est\'andar, entonces se usar\'a la f\'ormula de intervalo siguiente:
 \begin{equation*}
  \bar{x} - z_{\alpha/2}\frac{\sigma}{\sqrt{n}} < \mu < \bar{x} + z_{\alpha/2}\frac{\sigma}{\sqrt{n}}
 \end{equation*}
 Por lo tanto, usando los datos obtenidos y considerando el valor de $z_{\alpha/2}$ del libro, se tienen los c\'alculos de los l\'{\i}mites del intervalo de confianza como siguen:
 \begin{equation*}
  \bar{x} \pm z_{\alpha/2}\frac{\sigma}{\sqrt{n}} = 780 \pm 2.05\left( \frac{40}{\sqrt{30}} \right) = 780\pm \frac{82}{\sqrt{30}} = \frac{780\sqrt{30}\pm82}{\sqrt{30}}
 \end{equation*}
 Por lo tanto el intervalo de confianza de $96\%$ de la vida media de las bombillas es aproximadamente:
 \begin{equation*}
  765.02891676 < \mu < 794.9710832
 \end{equation*}
 Si se usa el valor $z_{\alpha/2}$ obtenido en R, se obtienen estos otros c\'alculos de los l\'{\i}mites del intervalo de confianza:
 \begin{equation*}
  \bar{x} \pm z_{\alpha/2}\frac{\sigma}{\sqrt{n}} = 780 \pm 2.05374891\left( \frac{40}{\sqrt{30}} \right) = 780\pm \frac{82.1499564}{\sqrt{3}} = \frac{780\sqrt{30}\pm82.1499564}{\sqrt{30}}
 \end{equation*}
 Por lo tanto el intervalo de confianza de $96\%$ de la vida media de las bombillas es, con mayor precisi\'on:
 \begin{equation*}
  765.001538594 < \mu < 794.9984614
 \end{equation*}
 Finalmente, en el software estad\'{\i}stico R se puede calcular el intervalo de confianza directamente con las siguientes l\'{\i}neas de c\'odigo, registradas en el archivo anexo \texttt{P01\_Intervalo\_de\_confianza\_01.r}:
 \begin{verbatim}
>n<-30
>m<-780
>desv.tipica<-40
>alfa<-0.04
>zalfa<-qnorm(1-alfa/2)
>LL<-round(m-zalfa*desv.tipica/sqrt(n),7)
>LU<-round(m+zalfa*desv.tipica/sqrt(n),7)
>LL;LU;
[1] 765.0015
[1] 794.9985
 \end{verbatim}
 \vspace{-0.5cm}
 donde las \'ultimas dos l\'{\i}neas indican el l\'{\i}mite inferior y el superior, respectivamente, del intervalo de confianza, que es a lo que se quer\'{\i}a llegar.${}_{\blacksquare}$
\end{solucion}

