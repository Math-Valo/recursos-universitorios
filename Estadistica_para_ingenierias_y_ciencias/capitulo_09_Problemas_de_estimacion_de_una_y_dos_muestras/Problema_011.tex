\begin{enunciado}
 Un investigador de la \texttt{UCLA} afirma que la vida de los ratones se puede extender hasta un $25\%$ cuando se reducen las calor\'{\i}as en su alimento en aproximadamente $40\%$, desde el momento en que se les desteta.
 Las dietas restringidas se enriquecen a niveles normales con vitaminas y prote\'{\i}nas.
 Suponiendo que por estudios previos se sabe que $\sigma = 5.8$ meses, ?`cu\'antos ratones se deber\'{\i}an incluir en nuestra muestra, si deseamos tener $99\%$ de confianza de que la vida media de la muestra est\'e dentro de 2 meses de la media de la poblaci\'on para todos los ratones sujetos a la dieta reducida?
\end{enunciado}

\begin{solucion}
 El problema pide conocer el tama\~no de la muestra para estimar $\mu$, medido en meses, usando $\bar{x}$ con los siguientes datos:
 \begin{itemize}
  \item $\sigma = 5.8$ meses.
  \item $\alpha=0.01$.
  \item $e=2$ meses.
 \end{itemize}
 Para conocer el tama\~no de muestra se precisa obtener el valor de $z_{\alpha/2} = z_{0.005}$, el cual, por ejercicios previos resueltos, se sabe que vale aproximadamente $2.575$, seg\'un el libro, y m\'as precisamente $2.5758293$, usando el software estad\'{\i}stico R.
 \par 
 Entonces, por el tipo de problema y el valor conocido de $\sigma$, se puede usar la siguiente formulaci\'on:
 \begin{equation*}
  n = \left( \frac{z_{\alpha/2}\sigma}{e} \right)^2
 \end{equation*}
 por lo tanto, el valor pedido se puede calcular, usando el valor $z_{\alpha/2}$ del libro, como
 \begin{equation*}
  n = \left\lceil \left( \frac{2.575\times 5.8}{2} \right)^2  \right\rceil = \left\lceil \left( 7.4675 \right)^2 \right\rceil = \left\lceil 55.76355625 \right\rceil = 56
 \end{equation*}
 Por lo tanto, el tama\~no de la muestra buscada es de $n=56$.
 \par 
 N\'otese que usando el valor $z_{\alpha/2}$ de la aproximaci\'on hecha en R da el mismo valor, esto se puede verificar usando el programa anexo \texttt{P03\_Tamanyo\_de\_muestra\_1.r} y cambiando las siguientes l\'{\i}neas de c\'odigo.
 \begin{verbatim}
>error<-2
>desv.tipica<-5.8
>alfa<-0.01
 \end{verbatim}
 \vspace{-0.5cm}
 que es a lo que se quer\'{\i}a llegar.${}_{\blacksquare}$
\end{solucion}

