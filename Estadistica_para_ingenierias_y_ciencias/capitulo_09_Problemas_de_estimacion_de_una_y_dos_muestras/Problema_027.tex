\begin{enunciado}
 Consid\'erese el ejercicio 9.18. Calcule un intervalo de predicci\'on de $95\%$ en una nueva medicci\'on observada del tiempo de secado de la pintura l\'atex.
\end{enunciado}

\begin{solucion}
 Usando la notaci\'on y datos como en la soluci\'on del ejercicio 9.18 pero ahora con $\alpha$ representando el valor para el cual hay $(1-\alpha)100\%$ de seguridad en el intervalo de predicci\'on, se tiene lo siguiente:
 \begin{itemize}
  \item $X\sim n(\mu, \sigma)$.
  \item $\mu$ desconocida.
  \item $\sigma$ desconocida.
  \item $n=15$.
  \item $\bar{x} = 3.78\overline{6}$.
  \item $s\approx 0.97091$.
  \item $\alpha = 0.05$.
 \end{itemize}
 Dado que se desconoce la desviaci\'on est\'andar poblacional y la muestra no es lo suficientemente grande, se requerir\'a del valor $t_{\alpha/2,n-1} = t_{0.025,14}$. De la Tabla A.4 se tiene que $t_{0.025,14} = 2.145$, mientras que, usando el software estad\'{\i}stico R, se obtiene un valor m\'as preciso con los siguientes comandos:
 \begin{verbatim}
>options(digits=22)
>qt(0.025,14,lower.tail=F)
[1] 2.144786687917804357539
 \end{verbatim}
 \vspace{-0.5cm}
 por lo que tambi\'en se puede considerar con mayor precisi\'on como $2.1447866879178$.
 \par 
 Como se est\'a buscando un intervalo de predicci\'on para una muestra que proviene de una poblaci\'on normal con varianza desconocida, entonces se usar\'a la siguiente formulaci\'on:
 \begin{equation*}
  \bar{x}-t_{\alpha/2,n-1}s\sqrt{1+1/n} < x_0 < \bar{x}+t_{\alpha/2,n-1}s\sqrt{1+1/n}
 \end{equation*}
 Por lo tanto, con los datos obtenidos y el valor $t_{\alpha/2,n-1}$ del libro, se tienen los c\'alculos de los l\'{\i}mites del intervalo de predicci\'on como siguen:
 \begin{eqnarray*}
  \bar{x} \pm t_{\alpha/2,n-1}s\sqrt{1+1/n} 
  & = & 3.78\overline{6} \pm (2.145)(0.97091)\sqrt{1 + \frac{1}{15}} 
  = 3.78\overline{6} \pm 2.08260195 \sqrt{\frac{16}{15}} \\
  & = & 3.78\overline{6} \pm \frac{(2.08260195)(4)}{\sqrt{15}} = 3.78\overline{6} \pm \frac{8.3304078\sqrt{15}}{15} \\
  & = & 3.78\overline{6} \pm 0.55536052\sqrt{15}
 \end{eqnarray*}
 Por lo tanto, el intervalo de predicci\'on de $95\%$ para la medici\'on pr\'oxima observada en el tiempo de secado de la pintura l\'atex es de:
 \begin{equation*}
  1.63576462 < x_0 < 5.93756871
 \end{equation*}
 que, redondeando al cuarto decimal, se reduce a que el intervalo es $(1.6358, 5.9376)$. El c\'alculo del intervalo de predicci\'on usando el valor $t_{\alpha/2,n-1}$ obtenido en R se puede realizar con los siguientes comandos. Los c\'odigos escritos se encuentran registrados en un script en el archivo anexo \texttt{P09\_Intervalo\_de\_prediccion\_2.r}. El resultado mostrado al final presenta el l\'{\i}mite inferior, la media muestral y el l\'{\i}mite superior, en ese orden, que se encuentra almacenado en la variable \texttt{IPs}. Para usar los comandos, se requiere el uso de la biblioteca \texttt{EnvStats}.
 \begin{verbatim}
>library(EnvStats)
>datos<-read.csv("DB02_Problema_18.csv",sep=";",encoding="UTF-8")
>varInteres<-c("Tiempo.h")
>alfa<-0.05
>valores<-unlist(datos[,varInteres])
variable<-factor(rep(varInteres,each=dim(datos)[1]))
>IP<-function(x,sup=TRUE,alfa=0.05){
>   if (length(x[!is.na(x)])>=2){
>      RespGen<-predIntNorm(x[!is.na(x)],conf.level=alfa,pi.type="two-side",
>                           method="Bonferroni")
>      v<-ifelse(sup,round(RespGen$interval$limits[[2]],7),
>                round(RespGen$interval$limits[[1]],7))
>      return(v)
>   } else return(NA)
>}
>media<-as.data.frame.table(tapply(valores,as.list(data.frame(variable)),
>                          mean,na.rm=TRUE),responseName="Media")
>IPs1<-as.data.frame.table(tapply(valores,as.list(data.frame(variable)),
>                          IP,alfa=1-alfa),responseName="LimSup")
>IPs2<-as.data.frame.table(tapply(valores,as.list(data.frame(variable)),
>                          IP,sup=FALSE,alfa=1-alfa),responseName="LimInf")
>IPs<-na.omit(data.frame(IPs2,Media=media[,"Media"],LimSup=IPs1[,"LimSup"]))
>IPs
  variable   LimInf    Media   LimSup
1 Tiempo.h 1.635978 3.786667 5.937355
 \end{verbatim}
 \vspace{-0.5cm}
 que es a lo que se quer\'{\i}a llegar.${}_{\blacksquare}$
\end{solucion}
