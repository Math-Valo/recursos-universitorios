\begin{enunciado}
 Con referencia al ejercicio 9.43, encuentre un intervalo de confianza de $99\%$ para $\mu_1 - \mu_2$, si se asigna al azar un neum\'atico de cada compa\~n\'{\i}a a las ruedas traseras de $8$ taxis y se registran las siguientes distancias, en kil\'ometros:
 \begin{center}
  \begin{tabular}{ccc}
   \textbf{Taxi} & \textbf{Marca \textit{A}} & \textbf{Marca \textit{B}} \\
   \hline 
   1 & $34,400$ & $36,700$ \\
   2 & $45,500$ & $46,800$ \\
   3 & $36,700$ & $37,700$ \\
   4 & $32,000$ & $31,100$ \\
   5 & $48,400$ & $47,800$ \\
   6 & $32,800$ & $36,400$ \\
   7 & $38,100$ & $38,900$ \\
   8 & $30,100$ & $31,500$
  \end{tabular}
 \end{center}
 Suponga que las diferencias de las distancias se distribuyen de forma aproximadamente normal.
\end{enunciado}

\begin{solucion}
 Sean $X_1$ y $X_2$ las variables aleatorias de las distancias recorridas, medidas en kil\'ometros, por los taxis que tienen asignados un neum\'atico en las ruedas traseras de cada marca, A y B, respectivamente, entonces, del enunciado, se tiene el siguiente resumen de datos:
 \begin{itemize}
  \item $X_i \sim n\left( \mu_i, \sigma_i \right)$, para cada $i \in \{ 1, 2 \}$.
  \item $\mu_i$ y $\sigma_i$ son desconocidos, para cada $i \in \{ 1, 2 \}$.
  \item $n_1 = n_2 = 8$.
  \item $\alpha = 0.01$.
 \end{itemize}
 adem\'as de los $8$ datos obtenidos en cada muestra, cuyas diferencias de kil\'ometros recorridos, $D_i = X_1 - X_2$, son:
 \begin{center}
  \begin{tabular}{cccc}
   \textbf{Taxi} & \textbf{Marca \textit{A}} & \textbf{Marca \textit{B}} & $d_i$ \\
   \hline 
   1 & $34\,400$ & $36\,700$ & $-2\,300$ \\
   2 & $45\,500$ & $46\,800$ & $-1\,300$ \\
   3 & $36\,700$ & $37\,700$ & $-1\,000$ \\
   4 & $32\,000$ & $31\,100$ & $\phantom{-1}\,900$ \\
   5 & $48\,400$ & $47\,800$ & $\phantom{-1}\,600$ \\
   6 & $32\,800$ & $36\,400$ & $-3\,600$ \\
   7 & $38\,100$ & $38\,900$ & $-\phantom{1}\,800$ \\
   8 & $30\,100$ & $31\,500$ & $-1\,400$
  \end{tabular}
 \end{center}
 A partir de estos datos, se puede calcular la media y desviaci\'on est\'andar de las observaciones pareadas, como se muestra a continuaci\'on. La media muestral se calcula como sigue:
 \begin{eqnarray*}
  \bar{d} & = & \frac{-2\,300  -1\,300 -1\,000 + 900 + 600 - 3\,600 - 800 - 1\,400}{8} \\
  & = & -\frac{8\,900}{8} = -1\,112.5
 \end{eqnarray*}
 por lo que la varianza muestral se obtiene, usando el Teorema 8.1, como sigue:
 \begin{eqnarray*}
  s_d^2 & = & \frac{1}{n(n-1)}\left[ n \sum_{i=1}^n D_i^2 - \left( \sum_{i=1}^n D_i \right)^2 \right] \\
  & = & \frac{1}{8(7)} \left\{ 8\left[ (-2\,300)^2 + (-1\,300)^2 + (-1\,000)^2 + (900)^2 + (600)^2 + (-3\,600)^2 + (-800)^2 + \right. \right. \\
  & & \left. \left. + (-1\,400)^2 \right] - 8\,900^2 \right\} \\
  & = & \frac{5\,290\,000 + 1\,690\,000 + 1\,000\,000 + 810\,000 + 360\,000 + 12\,960\,000 + 640\,000 + 1\,960\,000}{7} \\
  & & - \frac{79\,210\,000}{56} \\
  & = & \frac{24\,710\,000}{7} - \frac{9\,901\,250}{7} = \frac{14\,808\,750}{7} \\
  & = & 2\,115\,535.\overline{714285}
 \end{eqnarray*}
 y, por lo tanto, la desviaci\'on est\'andar muestral es:
 \begin{equation*}
  s_d = \sqrt{s_d^2} = \sqrt{\frac{14\,808\,750}{7}} = \frac{25\sqrt{23\,694}\sqrt{7}}{7} = \frac{25}{7}\sqrt{165\,858} = 1\,454.488127928761846...
 \end{equation*}
 Por otro lado, como se desea encontrar el intervalo de confianza bilateral para $\mu_D = \mu_1 - \mu_2$, para observaciones pareadas, entonces se requerir\'a del valor $t_{\alpha/2, n-1} = t_{0.005,7}$. De la Tabla A.4, se tiene que $t_{0.005,7} = 3.499$, mientras que, usando R, se obtiene el valor con los siguientes comandos:
 \begin{verbatim}
> options(digits=22)
> qt(0.005,7,lower.tail=F)
[1] 3.499483297350493682387
 \end{verbatim}
 \vspace{-0.5cm}
 por lo que tambi\'en se puede considerar como $3.49948329735$. 
 \par 
 Ya que se busca un intervalo de confianza para la diferencia de las medias poblacionales pareadas usando como estimador la media muestral de las diferencias de datos pareados en una muestra peque\~na, en donde se desconoce la desviaci\'on est\'adar poblacional pero suponiendo que las poblaciones se distribuyen de forma aproximadamente normal, entonces se usar\'a la siguiente formulaci\'on:
 \begin{equation*}
  \bar{d} - t_{\alpha/2,n-1}\frac{s_d}{\sqrt{n}} < \mu_D < \bar{d} + t_{\alpha/2,n-1}\frac{s_d}{\sqrt{n}}
 \end{equation*}
 en donde $\mu_D = \mu_1 - \mu_2$. Por lo tanto, usando los datos obtenidos, considerando el valor $t_{\alpha/2,n-1}$ del libro, se tienen los c\'alculos de los l\'{\i}mites del intervalo de confianza como siguen:
 \begin{eqnarray*}
  \bar{d} \pm t_{\alpha/2,n-1}\frac{s_d}{\sqrt{n}} & = & -1\,112.5 \pm (3.499)\left( \frac{25/7\sqrt{165\,858}}{\sqrt{8}} \right) = -1\,112.5 \pm (3.499) \left( \frac{25\sqrt{82\,929}}{7(2)} \right) \\
  & = & -1\,112.5 \pm 6.2482\overline{142857}\sqrt{82\,929}
 \end{eqnarray*}
 Por lo tanto, el intervalo del $99\%$ de confianza de la diferencia entre los kil\'ometros promedio recorridos por los neum\'aticos de la marca A y de la marca B de forma pareada, es de:
 \begin{equation*}
  -2\,911.822993... < \mu_1 - \mu_2 < 686.822993...
 \end{equation*}
 Finalmente, en R se puede calcular el intervalo de confianza con las siguientes l\'{\i}neas de c\'odigo, registradas en el archivo anexo \texttt{P16\_Intervalo\_de\_confianza\_07.r}, el cual ha sido creado a partir de leer un archivo externo con extensi\'on .csv. En este caso, el archivo anexo que contiene los datos del problema se llama \texttt{DB04\_Problema\_44.csv} que contiene dos columnas: \texttt{Marca}, conformado por la categorizaci\'on de los datos, si pertenece el dato a la neum\'atico de la marca $A$ o $B$; y \texttt{Distancia.km}, conformado por los valores medidos.
 \par 
 El c\'odigo permite aparear los datos, bajo la suposici\'on de que los primeros datos pertenecen a una categorizaci\'on y la segunda mitad de datos pertenece a la otra categorizaci\'on. Las variables \texttt{varInteres} indica el nombre de la columna con los datos y \texttt{varSel} indica el nombre de la columna con las categorizaciones; adem\'as, se cuenta con la variable \texttt{alfa}, para el nivel de confianza.
 \par 
 Una vez que los datos han sido apareados, el proceso del script se comporta de forma similar al script del archivo anexo \texttt{P05\_Intervalo\_de\_confianza\_03.r}, por lo que se est\'a suponiendo adem\'as que la muestra es peque\~na y la distribuci\'on de las poblaciones es aproximadamente normal. El script junto con el resultado se muestra a continuaci\'on:
 \begin{verbatim}
> datos<-read.csv("DB04_Problema_44.csv",sep=";",encoding="UTF-8")
> varInteres<-c("Distancia.km")
> varSel<-list("Marca")
> alfa<-0.01
> valores<-datos[0:(dim(datos)[1]/2),varInteres]-
+          datos[(dim(datos)[1]/2+1):dim(datos)[1],varInteres]
> variable<-factor(rep(varInteres,each=dim(datos)[1]/2))
> IC<-function(x,sup=TRUE,alfa=0.05){
+    if (length(x[!is.na(x)])>=2){
+       pruebaT<-t.test(x[!is.na(x)],conf.level=1-alfa)
+       v<-ifelse(sup,round(pruebaT$conf.int[2],7),round(pruebaT$conf.int[1],7))
+       return(v)
+    } else return(NA)
+ }
> media<-as.data.frame.table(tapply(valores,as.list(data.frame(variable)),
+                                   mean,na.rm=TRUE),responseName="Media")
> ICs1<-as.data.frame.table(tapply(valores,as.list(data.frame(variable)),
+                                  IC,alfa=alfa),responseName="LimSup")
> ICs2<-as.data.frame.table(tapply(valores,as.list(data.frame(variable)),
+                                  IC,sup=FALSE,alfa=alfa),responseName="LimInf")
> ICs<-na.omit(data.frame(ICs2,Media=media[,"Media"],LimSup=ICs1[,"LimSup"]))
> ICs
      variable    LimInf   Media   LimSup
1 Distancia.km -2912.072 -1112.5 687.0715
 \end{verbatim}
 \vspace{-0.5cm}
 En el resultado final es similar a como se explic\'o en el script \texttt{P05\_Intervalo\_de\_confianza\_03.r}, por lo que \'este indica que el intervalo de confianza, con una probabilidad del $95\%$ y redondeando para igualar con el resultado antes mencionado, es de:
 \begin{equation*}
  -2\,912 < \mu_1 - \mu_2 < 687
 \end{equation*}
 que es a lo que se quer\'{\i}a llegar.${}_{\blacksquare}$
\end{solucion}
