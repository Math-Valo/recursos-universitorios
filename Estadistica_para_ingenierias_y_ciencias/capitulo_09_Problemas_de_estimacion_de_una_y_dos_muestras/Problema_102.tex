\begin{enunciado}
 Considere el ejercicio de repaso 9.101. Supongamos que los datos no se han recabado a\'un. Supongamos tambi\'en que los estad\'{\i}sticos previos sugieren que $\sigma_1 = \sigma_2 = \$4000$. Los tama\~nos de las muestras en el ejercicio de repaso 9.101 son suficientes para producir un intervalo de confianza de $95\%$ en $\mu_1 - \mu_2$ que tenga un ancho de s\'olo $\$1000$? Presente el desarrollo completo.
\end{enunciado}

\begin{solucion}
 Usando la notaci\'on y datos como se explic\'o en la soluci\'on del ejercicio $9.101$, pero cambiando algunos valores, seg\'un se indica en el enunciado de este problema, y agregando el error hasta el cual se permite, se tiene los siguientes datos:
 \begin{itemize}
  \item $n_1 = n_2 = 300$.
  \item $\sigma_1 = \sigma_2 = 4\,000$.
  \item $\alpha = 0.05$.
  \item $e = 1\,000$.
 \end{itemize}
 Como se desea conocer si el tama\~no de muestra, se requerir\'a adem\'as del valor $z_{\alpha/2} = z_{0.025}$, el cual se calcul\'o en el ejercicio $9.5$ y su aproximaci\'on es de $1.96$, aunque, en R, se puede considerar con mayor precisi\'on como $1.95996398454$.
 \par 
 En el caso de los intervalos  de confianza de la diferencia de dos medias, no se ha visto hasta ahora cu\'al es el tama\~no de error, ni el n\'umero m\'{\i}nimo de muestras para obtener dicho error, por lo que se proceder\'a a obtener primero dichas f\'ormulas.
 \par 
 Siguiendo la l\'ogica antes propuesta, hay una confianza de $(1-\alpha)100\%$ de que el error no exceder\'a $K\cdot e.e.\left( \bar{x}_1 - \bar{x}_2 \right)$, donde $e.e.\left( \bar{x}_1 - \bar{x}_2 \right)$ es el error est\'andar en funci\'on del estimador puntual y $K$ es una constante (ya sea $t$ o el punto porcentual normal).
 \par 
 Por lo tanto, con una confianza de $(1-\alpha)100\%$, el valor $\bar{x}_1 - \bar{x}_2$ estima $\mu_1 - \mu_2$ con una magnitud de este error seg\'un el caso en el que uno se encuentre.
 \par 
 Si la muestra es grande o se sabe que hay una distribuci\'on normal de las poblaciones, y se conozcan las varianzas o desviaciones est\'andar poblacionales, el error est\'a dado por la f\'ormula
 \begin{equation*}
  z_{\alpha/2}\sqrt{ \frac{\sigma_1^2}{n_1} + \frac{\sigma_2^2}{n_2}}
 \end{equation*}
 En caso de que la muestra sea grande y no se conozcan las varianzas o desviaciones est\'andar poblacionales, el error est\'a dado por:
 \begin{equation*}
  z_{\alpha/2}\sqrt{\frac{s_1^2}{n_1} + \frac{s_2^2}{n_2}}
 \end{equation*}
 En el caso de que la muestra sea peque\~na y no se conozcan las varianzas o desviaciones est\'andar poblacionales, pero que se supongan iguales, el error est\'a dado por
 \begin{equation*}
  t_{\alpha/2,n_1+n_2-2}s_p\sqrt{\frac{1}{n_1} + \frac{1}{n_2}}
 \end{equation*}
 en donde $s_p$ representa el estimador de la uni\'on de las desviaciones est\'andar y su cuadrado, el estimador de la uni\'on de las varianzas, est\'a dado por:
 \begin{equation*}
  S_p^2 = \frac{\left( n_1 - 1 \right)S_1^2 + \left( n_2 - 1 \right)S_2^2}{n_1 + n_2 - 2}
 \end{equation*}
 Y, finalmente, si se tiene un caso como el anterior, pero ahora suponiendo que las varianzas o desviaciones est\'andar son distintos, entonces el error est\'a dado por:
 \begin{equation*}
  t_{\alpha/2,v}\sqrt{\frac{s_1^2}{n_1} + \frac{s_2^2}{n_2}}
 \end{equation*}
 en donde $v$ representa los grados de liberta del valor $t$ y est\'a dado por la ecuaci\'on
 \begin{equation*}
  v = \left\lceil \frac{\left(s_1^2/n_1 + s_2^2/n_2\right)^2}{\left[ \left( s_1^2/n_1 \right)^2/\left( n_1 - 1 \right) \right] + \left[ \left( s_2^2/n_2 \right)^2/\left( n_2 - 1 \right) \right]} \right\rceil
 \end{equation*}
 N\'otese que cuando las muestras son peque\~nas, se requiere el valor de $t$, cuyos grados de libertad dependen del tama\~no de la muestra, por lo tanto, si se desea calcular un tama\~no de muestra m\'{\i}nimo para asegurar que el estimador no superar\'a un valor de error dado, entonces se va a suponer que la muestra es lo suficientemente grande, m\'as a\'un, para simplificar cuentas, se va a suponer que los tama\~nos de muestras son iguales, esto es, que $n_1 = n_2 = n$.
 \par 
 Por lo tanto, despejando, se tiene con una seguridad del $(1-\alpha)100\%$ que la magnitud del error $e$ no ser\'a superado si el tama\~no de muestra es el menor entero mayor o igual que
 \begin{equation*}
  n = 2\left( \frac{z_{\alpha/2} \sigma}{e}  \right)^2
 \end{equation*}
 para el caso en que se conocen las desviaciones est\'andar iguales, y
 \begin{equation*}
  n = 2\left( \frac{z_{\alpha/2} s}{e}  \right)^2
 \end{equation*}
 para el caso de las varianzas est\'andar poblacionales iguales.
 \par 
 Por lo tanto, usando los datos obtenidos, considerando la primera aproximaci\'on de $z_{\alpha/2}$, se tiene el c\'alculo del tama\~no del error como sigue:
 \begin{equation*}
  z_{\alpha/2}\sqrt{ \frac{\sigma_1^2}{n_1} + \frac{\sigma_2^2}{n_2}} = 1.96\sqrt{\frac{4\,000^2}{300} + \frac{4\,000^2}{300}} = 1.96\left( \frac{4\,000\sqrt{6}}{30} \right) = \frac{784\sqrt{6}}{3} = 261.\overline{3}\sqrt{6} \approx 640.13
 \end{equation*}
 Por lo tanto, con el tama\~no de las muestras actuales, se puede asegurar, con $95\%$ que el ancho ser\'a de menos de $\$1\,000$ para el intervalo de confianza para $\mu_1 - \mu_2$.
 \par 
 Por otro lado, se puede calcular el tama\~no de muestra m\'{\i}nimo con el c\'alculo siguiente:
 \begin{equation*}
  n = \left\lceil 2\left( \frac{z_{\alpha/2} \sigma}{e}  \right)^2 \right\rceil = \left\lceil 2\left( \frac{1.96 (4\,000)}{1000} \right)^2 \right\rceil = \left\lceil 2\left( 7.84 \right)^2 \right\rceil = \left\lceil 2(61.4656) \right\rceil = \lceil 122.9312 \rceil = 123
 \end{equation*}
 Por lo tanto, a partir de muestras de $123$ empleados, se tiene que el tama\~no de muestra ser\'a de a lo m\'as $\$1\,000$.
 \par 
 Finalmente, se puede obtener en R estos c\'alculos. Como este tipo de ejercicios no son tan comunes, no se crear\'an otros programas, sino que se usar\'an de los que ya se tienen.
 \par 
 Para ver la magnitud del error, se usa el la rutina en el archivo \texttt{P13\_Intervalo\_de\_confianza\_04.r}, cambiando las siguientes l\'{\i}neas de c\'odigo:
 \begin{verbatim}
> n1<-300
> n2<-300
> m1<-0
> m2<-0
> desv.tipica1<-4000
> desv.tipica2<-4000
> alfa<-0.05
> val<-TRUE
> inter<-'D'
 \end{verbatim}
 \vspace{-0.5cm}
 N\'otese que como no interesa el intervalo sino \'unicamente el tama\~no, se han dado a las medias muestrales los valores de $\bar{x}_1 = \bar{x}_1 = 0$, as\'{\i}, los l\'{\i}mites del intervalo dar\'an, por s\'{\i} mismos, en valor absoluto, la magnitud del intervalo, y se obtiene lo siguiente:
 \begin{verbatim}
   n1  n2 media1 media2    LimInf diferencia   LimSup
1 300 300      0      0 -640.1216          0 640.1216
 \end{verbatim}
 \vspace{-0.5cm}
 Mientras que el tama\~no de las muestras se pueden obtener al usar el script que se encuentra guardado en el archivo \texttt{P03\_Tamanyo\_de\_muestra\_1.r}, cambiando las siguientes l\'{\i}neas de c\'odigo:
 \begin{verbatim}
> error<-1000
> desv.tipica<-4000
> alfa<-0.05
 \end{verbatim}
 \vspace{-0.5cm}
 con lo que se obtiene un valor final al ejecutar todas las l\'{\i}neas de c\'odigo, sin embargo, hay que cambiar ligeramente esta f\'ormula final a la siguiente:
 \begin{verbatim}
> n<-ceiling(2*(zalfa*desv.tipica/error)^2)
 \end{verbatim}
 \vspace{-0.5cm}
 con lo que se obtiene:
 \begin{verbatim}
[1] 123
 \end{verbatim}
 \vspace{-0.5cm}
 Por lo que, al redondear al decimal en que coinciden los resultados anteriores, se tiene, con un $95\%$ de confianza, que a partir de muestras de $123$ gerentes, se puede asegurar que el intervalo de confianza no supera los $\$1\,000$, por lo tanto, el tama\~no actual no supera una magnitud en el intervalo de los $\$1\,000$, m\'as a\'un, el tama\~no ser\'a de a lo m\'as $640.1$ para la magnitud del intervalo de confianza para $\mu_1 - \mu_2$, que es a lo que se quer\'{\i}a llegar.${}_{\blacksquare}$
\end{solucion}
