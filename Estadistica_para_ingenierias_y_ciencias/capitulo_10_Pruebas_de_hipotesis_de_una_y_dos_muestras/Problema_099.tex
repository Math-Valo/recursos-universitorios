\begin{enunciado}
 Se lleva a cabo una investigaci\'on en dos ciudades de Virginia,
 para determinar la opini\'on de los votantes hacia los candidatos
 a la gubernatura en una elecci\'on pr\'oxima.
 En cada ciudad se seleccionan $500$ votantes al azar y se registran
 los siguientes datos:
 \begin{center}
  \begin{tabular}{lcc}
   & \multicolumn{2}{c}{\textbf{Ciudad}} \\
   \cline{2-3}
   \textbf{Opini\'on del votante} & \textbf{Richmond} & \textbf{Norfolk} \\
   \hline
   Favorece a A & $204$ & $225$ \\
   Favorece a B & $211$ & $198$ \\
   Indeciso & $85$ & $77$
  \end{tabular}
 \end{center}
 Con un nivel de significancia de $0.05$, pruebe la hip\'otesis nula
 de que las proporciones de votantes que favorecen al candidato A,
 al candidato B o est\'an indecisos son las mismas para cada ciudad.
\end{enunciado}

\begin{solucion}
 \begin{datos}
  $\phantom{0}$
  \begin{itemize}
   \item Tamaño de muestra total: $1\,000$.
   \item Votantes encuestados de Richmond: $500$.
   \item Votantes encuestados de Norfolk: $500$.
   \item Votantes encuestados de Ohio: $200$.
   \item Votantes encuestados con opini\'on a favor de A: $204 + 225 = 429$.
   \item Votantes encuestados con opini\'on a favor de B: $211 + 198 = 409$.
   \item Votantes encuestados con opini\'on indecisa: $85 + 77 = 162$.
   \item Frecuencias observadas y esperadas: $o_{i,j}$
   y $e_{i,j}=\frac{R_i C_j}{n}$, respectivamente,
   donde $R_i$ y $C_j$ son los marginales del rengl\'on $i$ y la columna $j$,
   respectivamente, y $n$ es el total de toda la muestra.
   As\'{\i}, pues, redondeando a un decimal, se muestra el resumen 
   en la siguiente tabla,
   en donde aparece entre par\'entesis la frecuencia esperada
   y a la izquierda el valor observado:
   \begin{center}
    \begin{tabular}{lcc|c}
     & \multicolumn{2}{c}{\textbf{Ciudad}} \\
     \cline{2-3} & & & \textbf{Marginal} \\
     \textbf{Opini\'on del votante} & \textbf{Richmond} & \textbf{Norfolk} &
     \textbf{por opini\'on} \\
     \hline
     Favorece a A & $204 (214.5)$ & $225 (214.5)$ & $429$ \\
     Favorece a B & $211 (204.5)$ & $198 (204.5)$ & $409$ \\
     Indeciso & $85 (81)$ & $77 (81)$ & $162$ \\
     \hline 
     \textbf{Marginal} & \multirow{2}{*}{$500$} & \multirow{2}{*}{$500$} &
     \textbf{TOTAL} \\
     \textbf{por estado} & & & $n=1\,000$
    \end{tabular}
   \end{center}
   \item Tama\~no de la tabla de contingencia: $r\times c = 3\times 2$.
   \item Grados de libertad de la prueba $\chi^2$: $v = (r-1)(c-1) = 2$.
  \end{itemize}
 \end{datos}
 
 \begin{hipotesis}
  \begin{eqnarray*}
   H_0: & &
   \text{Los votantes por ciudad son homog\'eneos respecto a las opiniones.} \\
   H_1: & &
   \text{Los votantes por ciudad no son homog\'eneos respecto a las opiniones.}
  \end{eqnarray*}
 \end{hipotesis}

 \begin{significancia}
  $\alpha = 0.05$.
 \end{significancia}

 \begin{region}
  De la tabla A.5, se tiene el valor cr\'{\i}tico
  $\chi^2_{\alpha,v} = \chi^2_{0.025,4} \approx 5.991$,
  por lo que la regi\'on de rechazo est\'a dado
  para $\chi^2 > 5.991$, donde
  $\chi^2 = \sum_{i} \frac{\left( o_i - e_i \right)^2}{e_i}$.
 \end{region}

 \begin{estadistico}
  \begin{eqnarray*}
   \chi^2 & = & \sum_{i} \frac{\left( o_i - e_i \right)^2}{e_i} \\
   & = & \frac{(204 - 214.5)^2}{214.5} + \frac{(225 - 214.5)^2}{214.5} +
   \frac{(211 - 204.5)^2}{204.5} + \frac{(198 - 204.5)^2}{204.5} + \\
   & & \frac{(85 - 81)^2}{81} + \frac{(77 - 81)^2}{81} \\
   & = & \frac{220.5}{214.5} + \frac{84.5}{204.5} + \frac{32}{81}
   = \frac{2\,205}{2\,145} + \frac{845}{2\,045} + \frac{32}{81} \\
   & \approx & 1.028 + 0.4132 + 0.3951 = 1.8363
  \end{eqnarray*}
 \end{estadistico}

 \begin{decision}
  No se rechaza $H_0$.
 \end{decision}

 \begin{conclusion}
  A un nivel de significancia de $0.05$, los resultados no arrojan
  evidencia suficiente rechazar la hip\'otesis nula
  y, por lo tanto, se considera que los votantes por cada ciudad
  Virginia son homog\'eneos con respecto a sus opiniones.
 \end{conclusion}

 Finalmente, usando el archivo anexo
 \texttt{P18\_Prueba\_de\_independencia\_y\_homogeniedad\_01.r},
 que a su vez requiere los datos del archivo
 \texttt{BD35\_Problema\_099.csv}, con los siguientes cambios:
 \begin{verbatim}
> datos<-read.csv("DB35_Problema_099.csv",sep=";",encoding="UTF-8")
> varInteres<-c("Ciudad","Opinión")
> varFrecuencia<-"Frecuencia"
> pruebas<-c(1,2,3)
 \end{verbatim}
 \vspace{-0.5cm}
 el programa de R lanza el siguiente resultado:
 \begin{verbatim}
$tabla
          Opinión
Ciudad     Favorece a A Favorece a B Indeciso
  Norfolk           225          198       77
  Richmond          204          211       85

$listaPruebas
$listaPruebas[[1]]

	Pearson's Chi-squared test

data:  tbl1
X-squared = 1.8362, df = 2, p-value = 0.3993


$listaPruebas[[2]]

	Log likelihood ratio (G-test) test of independence without correction

data:  tbl1
Log likelihood ratio statistic (G) = 1.8369, X-squared df = 2, p-value =
0.3991


$listaPruebas[[3]]

	Log likelihood ratio (G-test) test of independence with Williams' correction

data:  tbl1
Log likelihood ratio statistic (G) = 1.8323, X-squared df = 2, p-value =
0.4001
 \end{verbatim}
 \vspace{-0.5cm}
 Lo cual coincide con los resultados obtenidos,
 adem\'as de brindar m\'as informaci\'on como el $P-$valor
 y las pruebas con otros estad\'{\i}sticos,
 que es lo que se quer\'{\i}a llegar.${}_{\blacksquare}$
\end{solucion}
