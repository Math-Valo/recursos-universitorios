\begin{enunciado}
 Cinco muestras de una sustancia ferrosa se usan para determinar si hay una diferencia entre un an\'alisis qu\'{\i}mico de laboratorio y un an\'alisis de fluorescencia de rayos X del contenido de hierro. Cada muestra se divide en dos submuestras y se aplican los dos tipos de an\'alisis. A continuaci\'on se presentan los datos codificados que muestran los an\'alisis de contenido de hierro:
 \begin{center}
  \begin{tabular}{l|ccccc}
   & \multicolumn{5}{c}{\textbf{Muestra}} \\
   \hline 
   \textbf{An\'alisis} & \textbf{1} & \textbf{2} & \textbf{3} & \textbf{4} & \textbf{5} \\
   \hline 
   Rayos X & $2.0$ & $2.0$ & $2.3$ & $2.1$ & $2.4$ \\
   Qu\'{\i}mica & $2.2$ & $1.9$ & $2.5$ & $2.3$ & $2.4$
  \end{tabular}
 \end{center}
 Suponiendo que las poblaciones son normales, pruebe con un nivel de significancia de $0.05$ si los dos m\'etodos de an\'alisis dan, en promedio, el mismo resultado.
\end{enunciado}

\begin{solucion}
 \begin{datos}
  Resumido, se tiene que
  \begin{itemize}
   \item $X_i \sim n\left( \mu_i, \sigma_i \right)$,
   para cada $i \in \{ 1, 2 \}$.
   \item $n_1 = n_2 = 5$.
  \end{itemize}
  Como las observaciones est\'an pareadas,
  se usar\'an las diferencias de los datos,
  que muestra la siguiente tabla:
  \begin{center}
   \begin{tabular}{l|ccccc}
    & \multicolumn{5}{c}{\textbf{Muestra}} \\
    \hline 
    \textbf{An\'alisis} & \textbf{1} & \textbf{2} & \textbf{3} &  \textbf{4} & \textbf{5} \\
    \hline 
    Rayos X & $\phantom{-}2.0$ & $2.0$ & $\phantom{-}2.3$ &
    $\phantom{-}2.1$ & $2.4$ \\
    Qu\'{\i}mica & $\phantom{-}2.2$ & $1.9$ & $\phantom{-}2.5$ &
    $\phantom{-}2.3$ & $2.4$ \\
    $\phantom{000}d_i$ & $-0.2$ & $0.1$ & $-0.2$ & $-0.2$ &
    $0\phantom{.4}$
   \end{tabular}
  \end{center}
  Para obtener la media y desviaci\'on est\'andar
  de las diferencias muestrales, se calcula lo siguiente:
  \begin{eqnarray*}
   \sum_{i=1}^5 d_i & = & -0.2 + 0.1 - 0.2 - 0.2 + 0 = -0.5 \\
   \sum_{i=1}^5 d_i^2 & = &
   (-0.2)^2 + 0.1^2 + (-0.2)^2 + (-0.2)^2 + 0^2 = 0.13
  \end{eqnarray*}
  por lo que la media de las diferencias muestrales es:
  \begin{equation*}
   \bar{d} = \frac{1}{5} \sum_{i=1}^5 d_i = \frac{-0.5}{5} 
   = -\frac{1}{10} = -0.1
  \end{equation*}
  y la varianza de las diferencias muestrales se calcula,
  usando el teorema 8.1, como sigue:
  \begin{eqnarray*}
   s_D^2 & = &
   \frac{1}{5(4)}\left[
   5\sum_{i=1}^5 d_i^2 - \left( \sum_{i=1}^5 d_i \right)^2
   \right] = \frac{5(0.13) - (-0.5)^2}{20} = \frac{0.65 - 0.25}{20}
   = \frac{0.4}{20} = \frac{2}{100} \\
   & = & \frac{1}{50} = 0.02
  \end{eqnarray*}
  por lo que la desviaci\'on est\'andar de las diferencias
  muestrales es:
  \begin{equation*}
   s_D = \sqrt{s_D^2} = \sqrt{\frac{1}{50}} = \frac{\sqrt{2}}{10}
   \approx 0.1414213562
  \end{equation*}
  Por lo tanto, se resume el resto de los datos como sigue:
  \begin{itemize}
   \item $\bar{d} = -\frac{1}{10} = -0.1$.
   \item $s_D = \frac{\sqrt{2}}{10} \approx 0.1414213562$.
  \end{itemize}
 \end{datos}

 \begin{hipotesis}
  \begin{eqnarray*}
   H_0: \mu_D = \mu_1 - \mu_2 &   =  & 0 \\
   H_1: \mu_D = \mu_1 - \mu_2 & \neq & 0
  \end{eqnarray*}
 \end{hipotesis}

 \begin{significancia}
  $\alpha = 0.05$.
 \end{significancia}

 \begin{region}
  De la tabla A.4, se tiene el valor cr\'{\i}tico
  $t_{\alpha/2,n-1} = t_{0.025,4} \approx 2.776$,
  por lo que la regi\'on de rechazo est\'a dado para $|t| > 2.776$,
  donde $t = \frac{\bar{d} - d_0}{s_D/\sqrt{n}}$.
 \end{region}

 \begin{estadistico}
  \begin{eqnarray*}
   t & = & \frac{\bar{d} - d_0}{s_D/\sqrt{n}}
   = \frac{
   -\frac{1}{\cancel{10}}
   }{
   \frac{\sqrt{2}}{\cancel{10}}/\sqrt{5}
   }
   = -\frac{\sqrt{10}}{2}
   \approx -1.58113883
  \end{eqnarray*}
 \end{estadistico}

 \begin{decision}
  No se rechaza $H_0$.
 \end{decision}

 \begin{conclusion}
  La prueba no da evidencia suficiente para afirmar
  que los promedios de los an\'alisis sean distintos,
  por lo que se concluye que no hay una diferencia significativa
  entre los resultados promedios del an\'alisis qu\'{\i}mico
  de laboratorio y el an\'alisis de fluorescencia de rayos X
  sobre el contenido de hierro en la sustancia ferrosa.
 \end{conclusion}

 Finalmente, usando el archivo anexo
 \texttt{P06\_Prueba\_de\_dos\_medias\_02.r},
 que a su vez requiere los datos del archivo
 \texttt{DB06\_Problema\_042.csv},
 con los siguientes cambios:
 \begin{verbatim}
> datos<-read.csv("DB06_Problema_042.csv",sep=";",encoding="UTF-8")
> varInteres<-c("Hierro.x")
> varSel<-c("Análisis")
> mu<-0
> desv.iguales<-NULL
> alfa<-0.05
> cola<-'D'
> par<-TRUE
 \end{verbatim}
 \vspace{-0.5cm}
 el programa de R lanza el siguiente resultado:
 \begin{verbatim}
      Var1 Freq Poblaciones H0 n diferencia  desv.par  error.est grados alpha
1 Hierro.x    5    Pareadas  0 5       -0.1 0.1414214 0.06324555      4  0.05
     PValor Estadistico RegionRechazoInfT RegionRechazoSupT RegionRechazoInfX
1 0.1890037   -1.581139         -2.776445          2.776445        -0.1755978
  RegionRechazoSupX        Resultado
1         0.1755978 No se rechaza H0
 \end{verbatim}
 \vspace{-0.5cm}
 El cual coincide con los datos obtenidos,
 que es a lo que se quer\'{\i}a llegar.${}_{\blacksquare}$
\end{solucion}
