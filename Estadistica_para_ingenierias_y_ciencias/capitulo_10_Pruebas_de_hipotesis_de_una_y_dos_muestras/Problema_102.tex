\begin{enunciado}
 Considere la situaci\'on del ejercicio 10.54 de la p\'agina 361.
 El consumo de ox\'{\i}geno en ml/kg/min tambi\'en se midi\'o
 en los nueve sujetos.
 \begin{center}
  \begin{tabular}{ccc}
   \textbf{Sujeto} & \textbf{Con CO} & \textbf{Sin Co} \\
   1 & $26.46$ & $25.41$ \\
   2 & $17.46$ & $22.53$ \\
   3 & $16.32$ & $16.32$ \\
   4 & $20.19$ & $27.48$ \\
   5 & $19.84$ & $24.97$ \\
   6 & $20.65$ & $21.77$ \\
   7 & $28.21$ & $28.17$ \\
   8 & $33.94$ & $32.02$ \\
   9 & $29.32$ & $28.96$
  \end{tabular}
 \end{center}
 Se conjetura que el consumo de ox\'{\i}geno deber\'{\i}a ser mayor en un ambiente relativamente libre de CO.
 Realice una prueba de significancia y discuta la conjetura.
\end{enunciado}

\begin{solucion}
 Como se explic\'o en la soluci\'on del ejercicio 10.54,
 las observaciones fueron sobre las mismas unidades experimentales,
 por lo que se trata de una prueba de observaciones pareadas.
 \par 
 Aunque en el ejercicio 10.54 se dio por supuesto que la frecuencia de la respiraci\'on sigue una distribuci\'on normal,
 no se tiene el mismo supuesto previo para el consumo de ox\'{\i}geno,
 por lo que se recurre al archivo anexo
 \texttt{P17\_Prueba\_de\_normalidad\_01.r}, que a su vez usa el archivo 
 \texttt{DB37\_Problema\_102.csv}, con los siguientes cambios:
 \begin{verbatim}
> datos<-read.csv("DB37_Problema_102.csv",sep=";",encoding="UTF-8")
> varInteres<-"Oxígeno.ml.kg.min"
> media<-NULL
> desv.est<-NULL
> clases<-NULL
> graficaHist<-FALSE
 \end{verbatim}
 \vspace{-0.5cm}
 con lo que el programa de R lanza el siguiente resultado:
 \begin{verbatim}
            nombres  n X de chi2 param chi2 Valor-p chi2 Z de Geary
1 Oxígeno.ml.kg.min 18 0.1912547          1    0.6618744    1.47051
  Valor-p Geary  D de K-S Valor-p K-S D de K-S Lilliefors
1     0.1414236 0.1278719   0.9112188           0.1049961
  Valor-p K-S Lilliefors W de Shapiro Valor-p Shapiro
1              0.8647186    0.9590317       0.5830456
 \end{verbatim}
 \vspace{-0.5cm}
 Lo cual indica en todas las pruebas que el ajuste de los datos a una normal
 no es mala.
 \par 
 Es importante notar que la informaci\'on ingresada corresponde a todos los datos,
 tanto con CO como sin CO, por lo que podr\'{\i}a interesar probar
 individualmente los casos.
 Esto se puede hacer, manteniendo las variables de la \'ultima ejecuci\'on
 en R, agregando las siguientes l\'{\i}neas,
 las cuales se muestran a continuaci\'on junto con los resultados
 al aplicar la prueba de Geary:
 \begin{verbatim}
> lista<-split(datos[,varInteres],datos[,"CO"])
> x<-lista$Con
> n<-length(x)
> U<-sqrt(pi/2)*mean(abs(x-mean(x)))/sqrt(var(x)*(n-1)/n)
> x.Geary.statistic<-(U-1)/(0.2661/sqrt(n))
> x.Geary.p.value<-2*pnorm(abs(x.Geary.statistic),lower.tail=FALSE)
> y<-lista$Sin
> n<-length(y)
> U<-sqrt(pi/2)*mean(abs(y-mean(y)))/sqrt(var(y)*(n-1)/n)
> y.Geary.statistic<-(U-1)/(0.2661/sqrt(n))
> y.Geary.p.value<-2*pnorm(abs(y.Geary.statistic),lower.tail=FALSE)
> print("P-Valor de datos con CO usando la prueba de Geary"); x.Geary.p.value
[1] "P-Valor de datos con CO usando la prueba de Geary"
[1] 0.09889042
> print("P-Valor de datos sin CO usando la prueba de Geary"); y.Geary.p.value
[1] "P-Valor de datos sin CO usando la prueba de Geary"
[1] 0.9215314
 \end{verbatim}
 \vspace{-0.5cm}
 Lo cual indica un buen ajuste a una normal de los datos sin CO,
 aunque de los datos con CO s\'{\i} muestra baja probabilidad,
 de 0.09889042.
 Aunque con el riesgo, se considerar\'a suficiente para la prueba
 de dos medias pareada.
 \par 
 Entonces, usando el archivo anexo
 \texttt{P06\_Prueba\_de\_dos\_medias\_02.r}, que a su vez usa
 la misma base de datos en el archivo \texttt{DB37\_Problema\_102.csv},
 con los siguientes cambios:
 \begin{verbatim}
> datos<-read.csv("DB37_Problema_102.csv",sep=";",encoding="UTF-8")
> varInteres<-c("Oxígeno.ml.kg.min")
> varSel<-c("CO")
> mu<-0
> desv.iguales<-NULL
> alfa<-0.05
> cola<-'I'
> par<-TRUE
 \end{verbatim}
 \vspace{-0.5cm}
 que corresponden a los datos iniciales, la hip\'otesis de prueba:
 \begin{eqnarray*}
  H_0: & & \mu_D = \mu_1 - \mu_2 = 0 \\
  H_1: & & \mu_D = \mu_1 - \mu_2 \leq 0
 \end{eqnarray*}
 donde $\mu_1$ corresponde a la media poblacional del consumo
 de ox\'{\i}geno con CO y $\mu_2$ la media poblacional del consumo
 de ox\'{\i}geno sin CO, con un nivel de significancia de $\alpha = 0.05$,
 entonces se obtiene el siguiente resultado:
 \begin{verbatim}
               Var1 Freq Poblaciones H0 n diferencia desv.par error.est grados
1 Oxígeno.ml.kg.min    9    Pareadas  0 9  -1.693333 3.269824  1.089941      8
  alpha     PValor Estadistico RegionRechazoInfT RegionRechazoInfX
1  0.05 0.07944246     -1.5536         -1.859548         -2.026798
         Resultado
1 No se rechaza H0
 \end{verbatim}
 \vspace{-0.5cm}
 el cual indica la regi\'on cr\'{\i}tica de $t < -1.859548$, 
 donde $t = \frac{\overline{D} - d_0}{S_d/\sqrt{n}}$,
 donde $\overline{D}$ es la media muestral de los datos pareados,
 esto es $\overline{D} = -1.69\bar{3}$, $d_0$ es el valor correspondiente
 a la hip\'otesis nula, $S_d$ es la desviaci\'on est\'andar muestral
 de los datos pareados, esto es $S_d = 3.269824$ y $n$ el tama\~no
 de las unidades experimentales, es decir $n=9$,
 entonces el valor del estad\'{\i}stico obtenido es $t = -1.5536$,
 por lo que no se rechaza la hip\'otesis nula; sin embargo, 
 los resultados tambi\'en arrojan el valor $P$ que da $0.07944246$,
 lo cual indica un riesgo de no rechazar la hip\'otesis nula.
 A\'un as\'{\i}, por la prueba de significancia,
 se concluye que no hay una diferencia significativa entre el consumo
 de ox\'{\i}geno con o sin CO,
 que es a lo que se quer\'{\i}a llegar.${}_{\blacksquare}$
\end{solucion}
