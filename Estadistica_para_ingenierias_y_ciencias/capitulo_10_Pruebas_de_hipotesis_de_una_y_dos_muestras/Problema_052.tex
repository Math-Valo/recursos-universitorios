\begin{enunciado}
 Se considera una prueba $t$ de nivel $\alpha = 0.05$ para probar
 \begin{eqnarray*}
  H_0: \mu & = & 14 \\
  H_1: \mu & \neq & 14
 \end{eqnarray*}
 ¿Qu\'e tama\~no de la muestra se necesita para que la probabilidad sea $0.1$ de no rechazar de manera err\'onea $H_0$, cuando la media poblacional real difiera de $14$ en $0.5$? A partir de una muestra preliminar estimamos que $\sigma$ es $1.25$.
\end{enunciado}

\begin{solucion}
 A partir del enunciado, se tiene los siguientes datos:
 \begin{itemize}
  \item $\alpha = 0.05$.
  \item $\beta = 0.1$.
  \item $\mu = 14$, seg\'un la hip\'otesis nula.
  \item $\delta = 0.5$, donde $\mu \pm \delta$
  es una alternativa espec\'{\i}fica.
  \item $\sigma = 1.25$.
 \end{itemize}
 Entonces, de la tabla $A.3$, se deduce adem\'as que:
 \begin{itemize}
  \item $z_{\alpha/2} = 1.96$.
  \item $z_{\beta} = 1.28$.
 \end{itemize}
 Entonces se puede calcular el tama\~no de muestra requerido
 para la calidad buscada en la prueba
 con los siguientes c\'alculos:
 \begin{equation*}
  n =
  \left\lceil
  \frac{
  \left( z_{\alpha} + z_{\beta} \right)^2 \sigma^2
  }{
  \delta^2
  }
  \right\rceil
  =
  \left\lceil \frac{(1.96+1.28)^2(1.25)^2}{0.5^2} \right\rceil
  = \left\lceil \frac{3.24^2\times 1.5625}{0.25} \right\rceil
  = \lceil 10.4976 \times 6.25 \rceil
  = \lceil 65.61 \rceil
 \end{equation*}
 Por lo tanto, el tama\~no muestral buscado es $n = 66$.
 Es decir que, a partir de un tama\~no muestral de $n = 66$,
 se puede realizar una prueba de hip\'otesis bilateral de que $\mu = 14$,
 con un nivel de significancia de $\alpha = 0.05$
 y considerando una probabilidad de cometer un error tipo II de $0.1$
 para la hip\'otesis alternativa de que hay una diferencia de $0.5$
 respecto a la hip\'otesis nula,
 esto es que $|\mu - 14| = |\delta| = 0.5$.${}_{\blacksquare}$
\end{solucion}
