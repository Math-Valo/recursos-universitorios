\begin{enunciado}
 En un estudio para estimar la proporci\'on de residentes de cierta ciudad
 y sus suburbios que est\'an a favor de la construcci\'on de una planta
 de energ\'{\i}a nuclear,
 se encuentra que $63$ de $100$ residentes urbanos favorecen la construcci\'on,
 mientras que s\'olo $59$ de $125$ residentes suburbanos la favorecen.
 ¿Hay una diferencia significativa entre la proporci\'on de residentes urbanos
 y suburbanos que favorecen la construcci\'on de la planta nuclear?
 Utilice un valor $P$.
\end{enunciado}

\begin{solucion}
 \begin{datos}
  $\phantom{0}$
  \begin{itemize}
   \item $n_1 = 100$ y $n_2 = 125$.
   \item $x_1 = 63$ y $x_2 = 59$.
  \end{itemize}
  Adem\'as, se pueden calcular las proporciones estimadas de cada muestra
  y la estimaci\'on combinada de la proporci\'on a la que se supone que son iguales
  en ambas poblaciones, como se muestra a continuaci\'on:
  \begin{itemize}
   \item $\widehat{p}_1 = \frac{63}{100} = 0.63$.
   \item $\widehat{p}_2 = \frac{59}{125} = 0.472$.
   \item $\widehat{p} = \frac{x_1+x_2}{n_1+n_2} = \frac{63+59}{100+125}
   = \frac{122}{225} = 0.54\bar{2}$.
  \end{itemize}
 \end{datos}

 \begin{hipotesis}
  \begin{eqnarray*}
   H_0: p_1 &  =   & p_2 \\
   H_1: p_1 & \neq & p_2
  \end{eqnarray*}
 \end{hipotesis}

 \begin{estadistico}
  \begin{eqnarray*}
   z & \approx &
   \frac{\widehat{p}_1 - 
   \widehat{p}_2}{\sqrt{\widehat{p}\widehat{q}\left(1/n_1+1/n_2\right)}}
   = \frac{
   \displaystyle{ \frac{63}{100} - \frac{59}{125}}
   }{
   \displaystyle{ \sqrt{
   \left(\frac{122}{225}\right)\left(\frac{103}{225}\right)
   \left(\frac{1}{100} + \frac{1}{125}\right)
   }}}
   = \frac{
   \displaystyle{ \frac{7\,875 - 5\,900}{100(125)} }
   }{
   \displaystyle{ \sqrt{
   \frac{(122)(103)(\cancel{125 + 100})}{(\cancel{225})(225)(100)(125)}
   }}} \\
   & = & \frac{
   \displaystyle{\frac{\cancelto{395}{1\,975}}{10^2\cdot 5^{\cancelto{2}{3}}}}
   }{
   \displaystyle{\sqrt{\frac{12\,566}{3^2 \cdot 5^2 \cdot 10^2 \cdot 5^2 \cdot 5}}}
   }
   = \frac{
   395\left(3\cdot\cancel{5^2}\cdot\cancel{10}\sqrt{5}\right)
   \left(\sqrt{12\,566}\right)
   }{12\,566\cdot 10^{\cancel{2}}\cdot \cancel{5^{2}}}
   = \frac{
   \cancel{5}\cdot 79 \cdot 3 \sqrt{5\cdot 12\,566}
   }{
   12\,566 \cdot \cancel{5} \cdot 2
   } \\
   & = & \frac{237\sqrt{62\,830}}{25\,132} \approx 2.3637678684
  \end{eqnarray*}
 \end{estadistico}
 
 \begin{valorp}
  De la tabla A.3, se tiene que:
  \begin{equation*}
   P(|Z| > |z|) 1 = 2P(Z > z) \approx 2\left(1 - P(Z < 2.36)\right)
   \approx 2(1-0.9909) = 2(0.0091) = 0.0182
  \end{equation*}
 \end{valorp}
 
 \begin{conclusion}
  Por lo tanto, como el valor $P$ es muy peque\~no,
  se tiene evidencia suficiente para concluir
  que la proporci\'on de residentes urbanos que favorecen
  la construcci\'on de la planta nuclear es mayor que la proporci\'on
  de residentes suburbanos que favorecen dicha construcci\'on.
 \end{conclusion}

 En el c\'odigo registrado en el archivo anexo
 \texttt{P09\_Prueba\_de\_dos\_proporciones\_01.r}, en R,
 se realiza este procedimiento.
 El c\'odigo permite modificar los valores iniciales correspondientes a:
 \texttt{n1} y \texttt{n2} para el tama\~no de las muestras;
 \texttt{x1} y \texttt{x2} para la cantidad de casos favorables o,
 en su defecto, \texttt{p1} o \texttt{p2}
 para la proporci\'on de \'exitos muestrales,
 seg\'un la muestra correspondiente;
 \texttt{alfa} para el nivel de significancia;
 y, \texttt{alternativa} para indicar el tipo de prueba seg\'un la hip\'otesis
 alternativa seg\'un si se supone proporciones distintas, con \texttt{!=},
 si se supone la proporci\'on de la primera muestra mayor a la de la segunda muestra,
 con \texttt{>}, o si se supone la proporci\'on de la primera muestra menor
 a la de la segunda muestra, con \texttt{>}.
 \par 
 El programa espera al menos los datos correspondientes a las muestras:
 el tama\~no de ambas muestras; para cada muestra, la cantidad de casos
 favorables o, en su defecto, la proporci\'on muestral;
 y, el indicador del tipo de prueba, seg\'un su alternativa.
 Por lo tanto, el valor $P$ siempre se obtiene,
 independientemente de si se desea una prueba de hip\'otesis
 fijando la probabilidad del error tipo I
 o si se realiza una prueba de significancia (aproximaci\'on al valor $P$).
 En caso de no querer fijar una probabilidad de cometer un error de tipo I,
 $\alpha$, a la variable \texttt{alfa} se le asigna el valor \texttt{NULL}.
 \par 
 La prueba de hip\'otesis usar\'a el estad\'{\i}stico $Z$
 de la estandarizaci\'on de la distribuci\'on normal obtenida por la diferencia
 de las proporciones muestrales,
 considerando la estimaci\'on combinada de la proporci\'on $p$,
 el valor al que se suponen las proporciones poblacionales son iguales,
 seg\'un la hip\'otesis nula.
 \par 
 El resultado muestra la siguiente:
 \texttt{alternativa}, que muestra la hip\'otesis alternativa en contraste
 con la hip\'otesis nula de que las proporciones poblacionales son iguales;
 \texttt{n1} y \texttt{n2} para el tama\~no de cada muestra;
 \texttt{x1} y \texttt{x2} para la cantidad de casos favorables de cada muestra;
 \texttt{p1} y \texttt{p2} para la proporci\'on muestral de cada muestra;
 \texttt{pEstimada} para la estimaci\'on combinada de la proporci\'on $p$,
 el valor al que se suponen las proporciones poblacionales son iguales,
 seg\'un la hip\'otesis nula;
 \texttt{DifProp} para la diferencia de las proporciones muestrales;
 \texttt{error.est} para el valor est\'andar en el estad\'{\i}stico,
 que est\'a dado por el valor $\sqrt{\hat{p}\hat{q}(1/n_1 + 1/n_2)}$,
 donde, a su vez, el valor $\hat{p}$ es la estimaci\'on combinada
 de la proporci\'on $p$, dado por la f\'ormula $\frac{x_1+x_2}{n_1+n_2}$,
 y $\hat{q} = 1 - \hat{p}$;
 \texttt{alpha} para el nivel de significancia dado,
 el cual muestra por defecto $0.05$ en caso de asignar \texttt{NULL} a \texttt{alfa};
 \texttt{PValor} para el valor $P$,
 la probabilidad de haber obtenido un par de muestras como se obtuvo,
 suponiendo que la hip\'otesis nula sea cierta;
 \texttt{Estadistico} para el valor resultante del estad\'{\i}stico de prueba;
 \texttt{RegionRechazoZ} que indica en d\'onde se encuentra la regi\'on de rechazo
 para los valores obtenidos por el estad\'{\i}stico de prueba,
 seg\'un el valor de $\alpha$ (posiblemente el dado por defecto);
 y, en caso de haber indicado un valor a \texttt{alfa}, \texttt{Resultado}
 para indicar si se rechaza o no la hip\'otesis nula.
 \par 
 N\'otesse que no se consider\'o esta vez
 lo que en otros c\'odigos se le llama \texttt{RegionRechazoX},
 que indica una regi\'on de rechazo en t\'erminos de las unidades originales
 para los resultados obtenidos en la o las muestras.
 Esto debido a que se debe considerar primero un \'unico par\'ametro
 para los posibles resultados de las muestras,
 es decir, en t\'erminos de $x_1$ y $x_2$ o de $p_1$ y $p_2$,
 y el candidato principal es $P_1 - P_2$. Luego, el c\'alculo
 para obtener dicha regi\'on de rechazo es un despeje de la transformaci\'on
 con la que obtuvo el estad\'{\i}stico de prueba, una vez obtenida la regi\'on
 de rechazo de dicho estad\'{\i}stico.
 El problema radica en que el estad\'{\i}stico de prueba es
 $z = \frac{\hat{p}_1 - \hat{p}_2}{\sqrt{pq(1/n_1 + 1/n_2)}}$,
 donde $p$ es el valor al que se suponene que son iguales las proporciones,
 seg\'un la hip\'otesis nula;
 sin embargo, dicho valor no se puede siquiera suponer,
 por lo que en realidad se usa la transformaci\'on
 $z = \frac{\hat{p}_1 - \hat{p}_2}{\sqrt{\hat{p}\hat{q}(1/n_1 + 1/n_2)}}$,
 donde $\hat{p}$ es la estimaci\'on combinada de esta proporci\'on
 y est\'a dado por la f\'ormula $\hat{p} = \frac{x_1 + x_2}{n_1 + n_2}$.
 Esto quiere decir que tratar de despejar el par\'ametro de los resultados,
 $\hat{p}_1 - \hat{p}_2$, al pasar multiplicando este error
 que lo divide en la estandarizaci\'on a $z$ implica llegar a un valor
 que lleva consigo resultados de la muestra, es decir de $x_1$ y $x_2$,
 por lo que no ser\'{\i}a un dato \'util.
 Por otro lado, no existe un despeje directo de dicho valor
 en la transformaci\'on $z$.
 Como resultado, se ha decidido omitir en esta ocasi\'on dicha regi\'on de rechazo.
 \par 
 El c\'odigo junto con el resultado se muestra a continuaci\'on:
 \begin{verbatim}
> n1<-100
> n2<-125
> x1<-63
> x2<-59
> p1<-NULL
> p2<-NULL
> alfa<-NULL
> alternativa<-'!='
> if(n1<30 || n2<30){
+   stop("Las muestras no deben ser pequeñas (al menos de 30 observaciones).")
+ }
> if(is.null(x1)){
+   x1<-round(p1*n1)
+ }
> if(is.null(x2)){
+   x2<-round(p2*n2)
+ }
> TestProp<-function(n1,n2,x1,x2,alfa=0.05,prueba='!='){
+   p1<-x1/n1
+   p2<-x2/n2
+   p<-(x1+x2)/(n1+n2)
+   r<-data.frame(alternativa=paste("p1",prueba,"p2"),
+                 n1=n1,n2=n2,
+                 x1=x1,x2=x2,
+                 p1=p1,p2=p2,
+                 pEstimada=p,
+                 DifProp=p1-p2,
+                 error.est=sqrt(p*(1-p)*(1/n1+1/n2)),
+                 alpha=alfa
+   )
+   estadistico<-(p1-p2)/sqrt(p*(1-p)*(1/n1+1/n2))
+   if(prueba=='!='){
+     pvalor<-round(2*pnorm(abs(estadistico),lower.tail=F),7)
+     criticoz<-round(qnorm(1-alfa/2),7)
+     r$PValor<-pvalor
+     r$Estadistico<-estadistico
+     r$RegionRechazoZ<-paste("<=",-criticoz," y >=",criticoz)
+   }else{
+     criticoz<-round(qnorm(1-alfa),7)
+     if(prueba=='<'){
+       pvalor<-round(pnorm(estadistico),7)
+       r$PValor<-pvalor
+       r$Estadistico<-estadistico
+       r$RegionRechazoZ<-paste("<=",-criticoz)
+     }else{
+       pvalor<-round(pnorm(estadistico,lower.tail=F),7)
+       r$PValor<-pvalor
+       r$Estadistico<-estadistico
+       r$RegionRechazoZ<-paste(">=",criticoz)
+     }
+   }
+   return(r)
+ }
> if(is.null(alfa)){
+   Test<-TestProp(n1,n2,x1,x2,prueba=alternativa)
+ }else{
+   Test<-TestProp(n1,n2,x1,x2,alfa,alternativa)
+   resultado<-ifelse(Test[,"PValor"]>=alfa,"No se rechaza H0","Se rechaza H0")
+   Test$Resultado<-resultado
+ }
> Test
  alternativa  n1  n2 x1 x2   p1    p2 pEstimada DifProp  error.est alpha
1    p1 != p2 100 125 63 59 0.63 0.472 0.5422222   0.158 0.06684243  0.05
     PValor Estadistico              RegionRechazoZ
1 0.0180901    2.363768 <= -1.959964  y >= 1.959964
 \end{verbatim}
 Lo cual coincide con los resultado obtenido,
 que es a lo que se quer\'{\i}a llegar.${}_{\blacksquare}$
\end{solucion}
