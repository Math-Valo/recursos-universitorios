\begin{enunciado}
 De acuerdo con un estudio diet\'etico una ingesta alta de sodio se puede relacionar con \'ulceras, c\'ancer estomacal y migra\~na. El requerimiento humano de sal es de tan s\'olo $220$ miligramos diarios, el cual se rebasa en la mayor\'{\i}a de las porciones individuales de cereales listos para comerse. Si una muestra aleatoria de $20$ porciones similares de cierto cereal tiene un contenido medio de $244$ miligramos de sodio y una desviaci\'on est\'andar de $24.5$ miligramos, ¿esto sugiere, en el nivel de significancia de $0.05$, que el contenido promedio de sodio para porciones individuales de tal cereal es mayor que $220$ miligramos? Suponga que la distribuci\'on de contenidos de sodio es normal.
\end{enunciado}

\begin{solucion}
 \begin{datos}
  $\phantom{0}$
  \begin{itemize}
   \item $X \sim n(\mu, \sigma)$.
   \item $n = 20$.
   \item $\bar{x} = 244$.
   \item $s = 24.5$.
  \end{itemize}
  Por la suposici\'on de normalidad en la distribuci\'on
  poblacional, se tiene la distribuci\'on siguiente:
  \begin{itemize}
   \item $\displaystyle{\frac{\overline{X} - \mu}{S/\sqrt{n}}  \sim t(v) }$.
   \item $v = n-1 = 19$.
  \end{itemize}
 \end{datos}

 \begin{hipotesis}
  \begin{eqnarray*}
   H_0: \mu & = & 220 \\
   H_1: \mu & > & 220
  \end{eqnarray*}
 \end{hipotesis}

 \begin{significancia}
  $a = 0.05$.
 \end{significancia}

 \begin{region}
  De la tabla A.4, se tiene el valor cr\'{\i}tico $t_{\alpha,n-1} = t_{0.05,19} = 1.729$, por lo que la regi\'on de rechazo est\'a dado para $t > 1.729$, donde $t = \frac{\bar{x} - \mu_0}{s/\sqrt{n}}$.
 \end{region}

 \begin{estadistico}
  \begin{equation*}
   t = \frac{\bar{x}-\mu_0}{s/\sqrt{n}} = \frac{244-220}{24.5/\sqrt{20}} = \frac{24(2)\left(2\sqrt{5}\right)}{49} = \frac{96\sqrt{5}}{49} \approx 4.380867874
  \end{equation*}
 \end{estadistico}

 \begin{decision}
  Se rechaza $H_0$ a favor de $H_1$.
 \end{decision}

 \begin{conclusion}
  El contenido promedio de sodio para porciones individuales del cereal es significativamente mayor a los $220$ miligramos.
 \end{conclusion}

 Finalmente, usando el archivo anexo \texttt{P03\_Prueba\_de\_una\_media\_01.r}, con los siguientes cambios:
 \vspace{-0.3cm}
 \begin{verbatim}
> n<-20
> mu<-220
> m<-244
> desv<-24.5
> pobl<-FALSE
> alfa<-0.05
> cola<-'S'
> val<-TRUE
 \end{verbatim}
 \vspace{-0.8cm}
 el programa de R lanza el siguiente resultado:
 \vspace{-0.3cm}
 \begin{verbatim}
  Prueba  H0  n MediaMuestral desv.est error.est alpha    PValor Estadistico
1      t 220 20           244     24.5  5.478367  0.05 0.0001607    4.380868
  RegionRechazoT RegionRechazoX     Resultado
1    > 1.7291328  > 229.4728233 Se rechaza H0
 \end{verbatim}
 \vspace{-0.8cm}
 El cual coincide con los resultados obtenidos, que es a lo que se quer\'{\i}a llegar.${}_{\blacksquare}$
\end{solucion}
