\begin{enunciado}
 Se sabe que las emisiones de hidrocarburos disminuyeron
 de forma dram\'atica durante la d\'ecada de 1980.
 Se realiz\'o un estudio para comparar las emisiones de hidrocarburos
 a velocidad estacionaria, en partes por mill\'on (ppm),
 para autom\'oviles de 1980 y 1990.
 Se seleccionaron al azar $20$ autom\'oviles de cada modelo y se registraron
 sus niveles de emisi\'on de hidrocarburos.
 Los datos son los siguientes:
 \begin{verbatim}
Modelos 1980:
 141 359 247 940 882 494 306 210 105 880
 200 223 188 940 241 190 300 435 241 380
Modelos 1990:
 140 160  20  20 223  60  20  95 360  70
 220 400 217  58 235 380 200 175  85  65
 \end{verbatim}
 \vspace{-0.5cm}
 Pruebe la hip\'otesis de que $\sigma_1 = \sigma_2$ contra la alternativa
 de que $\sigma_1 \neq \sigma_2$.
 Suponga que ambas poblaciones son normales. Utilice un valor $P$.
\end{enunciado}

\begin{solucion}
 \begin{datos}
  Resumido se tiene que:
  \begin{itemize}
   \item $X_i \sim n\left( \mu_i, \sigma_i \right)$, para cada $i \in \{1,2\}$.
   \item $n_1 = n_2 = 20$.
  \end{itemize}
  Para obtener las varianzas se calcula primero lo siguiente:
  \begin{eqnarray*}
   \sum_{i=1}^{20} x_{1,i} & = &
   141 + 359 + 247 + 940 + 882 + 494 + 306 + 210 + 105 + 880 + \\
   & & 200 + 223 + 188 + 940 + 241 + 190 + 300 + 435 + 241 + 380 \\
   & = & 7\,902 \\
   \sum_{i=1}^{8} x_{1,i}^2 & = & 
   141^2 + 359^2 + 247^2 + 940^2 + 882^2 + 
   494^2 + 306^2 + 210^2 + 105^2 + 880^2 + \\
   & & 200^2 + 223^2 + 188^2 + 940^2 + 241^2
   + 190^2 + 300^2 + 435^2 + 241^2 + 380^2 \\
   & = & 4\,623\,052
  \end{eqnarray*}
  \begin{eqnarray*}
   \sum_{i=1}^{20} x_{2,i} & = &
   140 + 160 + 20 + 20 + 223 + 60 + 20 + 95 + 360 + 70 + \\
   & & 220 + 400 + 217 + 58 + 235 + 380 + 200 + 175 + 85 + 65 \\
   & = & 3\,203 \\
   \sum_{i=1}^{20} x_{2,i}^2 & = &
   140^2 + 160^2 + 20^2 + 20^2 + 223^2 +
   60^2 + 20^2 + 95^2 + 360^2 + 70^2 + \\
   & & 220^2 + 400^2 + 217^2 + 58^2 + 235^2
   + 380^2 + 200^2 + 175^2 + 85^2 + 65^2 \\
   & = & 783\,807
  \end{eqnarray*}
  por lo que las varianzas muestrales se calculan, usando el teorema 8.1,
  como sigue:
  \begin{eqnarray*}
   s_1^2 & = &
   \frac{1}{20(19)}
   \left[ 20\sum_{i=1}^{20} x_{1,i}^2 -
   \left( \sum_{i=1}^{20} x_{1,i} \right)^2 \right]
   = \frac{20(4\,623\,052) - 7\,902^2}{380} \\
   & = & \frac{92\,461\,040 - 62\,441\,604}{380} = \frac{30\,019\,436}{380}
   = \frac{7\,504\,859}{95} = 78\,998.5\overline{157894736842105263} \\
   s_2^2 & = &
   \frac{1}{20(19)}
   \left[ 20\sum_{i=1}^{20} x_{2,i}^2 -
   \left( \sum_{i=1}^{20} x_{2,i} \right)^2 \right]
   = \frac{20(783\,807) - 3\,203^2}{380} \\
   & = & \frac{15\,676\,140 - 10\,259\,209}{380}
   = \frac{5\,416\,931}{380} = 14\,255.08\overline{157894736842105263}
  \end{eqnarray*}
  Por lo que el resto de los datos se resume como sigue:
  \begin{itemize}
   \item $s_1^2 = \frac{7\,504\,859}{95}
   = 78\,998.5\overline{157894736842105263}$
   y $s_2^2 = \frac{5\,416\,931}{380}
   = 14\,255.08\overline{157894736842105263}$.
  \end{itemize}
  Adem\'as, por la suposici\'on de normalidad en las distribuciones
  poblacionales, se tiene la distribuci\'on siguiente:
  \begin{itemize}
   \item $\frac{s_1^2}{s_1^2} \sim f(v_1,v_2)$.
   \item $v_1 = n_1 - 1 = 19$ y $v_2 = n_2 - 1 = 19$.
  \end{itemize}
 \end{datos}
 
 \begin{hipotesis}
  \begin{eqnarray*}
   H_0: \sigma_1 &  =   & \sigma_2 \\
   H_1: \sigma_1 & \neq & \sigma_2
  \end{eqnarray*}
 \end{hipotesis}

 \begin{estadistico}
  \begin{equation*}
   f = \frac{s_1^2}{s_2^2}
   = \frac{\displaystyle{\frac{7\,504\,859}{\cancel{95}}}}
   {\displaystyle{\frac{5\,416\,931}{\cancelto{4}{380}}}}
   = \frac{7\,504\,859(4)}{5\,416\,931}
   = \frac{30\,019\,436}{5\,416\,931} \approx 5.5417792842
  \end{equation*}
 \end{estadistico}

 \begin{valorp}
  De la tabla A.6 se observa que no se puede hallar directamente un valor
  cr\'{\i}tico con $v_1=19$, sin embargo, s\'{\i} se observa que,
  para un valor $\alpha$ espec\'{\i}fico,
  $f_{\alpha,v_1,v_2} > f_{\alpha,v_1',v_2}$, cuando $v_1 < v_2'$,
  entonces, se tiene que $3.15 = f_{0.01,15,19} > f_{0.01,19,19}$ y,
  sin embargo, el estad\'{\i}stico $f$ obtenido est\'a alejado a la derecha
  de este valor, por lo que se entiende que $P(F > f) < 0.01$.
  Por lo tanto, se tiene que:
  \begin{equation*}
   2\times P\left( F_{v_1,v_2} > f \right) < 2\times(0.01) = 0.02
  \end{equation*}
 \end{valorp}

 \begin{conclusion}
  Por lo tanto, como el valor $P$ es muy peque\~no,
  se concluye que hay evidencia suficiente para rechazar la hip\'otesis nula,
  y, por lo tanto, se afirma que la varianza en las emisiones de hidrocarburos
  no solo disminuyen sino que tambi\'en difieren entre los autom\'oviles
  entre los modelos de 1980 y 1990.
 \end{conclusion}

 Finalmente, usando el archivo anexo \texttt{P14\_Prueba\_de\_dos\_varianzas\_02.r}, con los siguientes cambios:
 \begin{verbatim}
> datos<-read.csv("DB14_Problema_078.csv",sep=";",encoding="UTF-8")
> varInteres<-c("Hidrocarburos.ppm")
> varSel<-c("Modelo")
> alfa<-NULL
> cola<-'D'
 \end{verbatim}
 \vspace{-0.5cm}
 el programa de R lanza el siguiente resultado:
 \begin{verbatim}
           variable Freq n1 n2 media1 media2 varianza1 varianza2 v1 v2 alpha
1 Hidrocarburos.ppm   40 20 20  395.1 160.15  78998.52  14255.08 19 19  0.05
   PValor Estadistico             RegionRechazo
1 0.00048    5.541779 < 0.3958122 y > 2.5264509
 \end{verbatim}
 \vspace{-0.5cm}
 El cual coincide con los resultados obtenidos,
 adem\'as de ofrecer un valor m\'as preciso para el valor $P$,
 que es a lo que se quer\'{\i}a llegar.${}_{\blacksquare}$
\end{solucion}
