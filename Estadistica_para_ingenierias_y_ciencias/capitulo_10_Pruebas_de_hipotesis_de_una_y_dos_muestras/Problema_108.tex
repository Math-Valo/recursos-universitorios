\begin{enunciado}
 \textbf{Valor} $\mathbf{z}$ \textbf{para probar} $\mathbf{p_1 - p_2=d_0}$.
 Para probar la hip\'otesis nula $H_0$ de que $p_1 - p_2 = d_0$,
 donde $d_0 \neq 0$, basamos nuestra decisi\'on en
 \begin{equation*}
  z =
  \frac{\widehat{p}_1-\widehat{p}_2-d_0}{\sqrt{
  \widehat{p}_1\widehat{q}_1/n_1 + \widehat{p}_2\widehat{q}_2/n_2}}
 \end{equation*}
 que es un valor de una variable aleatoria,
 cuya distribuci\'on aproxima la distribuci\'on normal est\'andar,
 en tanto que $n_1$ y $n_2$ sean grandes.
 Con referencia al ejemplo 10.12 de la p\'agina 365,
 pruebe la hip\'otesis de que el porcentaje de votantes de la ciudad que favorecen la construcci\'on de la planta qu\'{\i}mica no exceder\'a el porcentaje de votantes del condado en m\'as de $3\%$.
 Utilice un valor $P$ en su conclusi\'on.
\end{enunciado}

\begin{solucion}
 \begin{datos}
  $\phantom{0}$
  \begin{enumerate}
   \item $n_1 = 200$ y $n_2 = 500$.
   \item $x_1 = 120$ y $x_2 = 240$.
   \item $\bar{p}_1 = \frac{x_1}{n_2} = \frac{120}{200}=\frac{3}{50} = 0.6$
   y $\bar{p}_1 = \frac{x_2}{n_2}=\frac{240}{500} = \frac{12}{25} = 0.48$.
  \end{enumerate}
 \end{datos}

 \begin{hipotesis}
  \begin{eqnarray*}
   H_0: & & p_1 - p_2 \leq 0.03 \\
   H_1: & & p_1 - p_2   >  0.03
  \end{eqnarray*}
 \end{hipotesis}

 \begin{estadistico}
  \begin{eqnarray*}
   z & = & \frac{\widehat{p}_1-\widehat{p}_2-d_0}{\sqrt{
  \widehat{p}_1\widehat{q}_1/n_1 + \widehat{p}_2\widehat{q}_2/n_2}}
  = \frac{0.6 - 0.48 - 0.03}{\sqrt{
  (0.6)(0.4)(1/200) + (0.48)(0.52)(1/500)}} \\
  & = & \frac{0.09}{\sqrt{0.24/200 + 0.2496/500}}
  = \frac{\displaystyle{\frac{9}{100}}}{\displaystyle{\sqrt{
  \frac{3}{2\,500} + \frac{39}{78\,125}}}}
  = \frac{9}{\displaystyle{100\sqrt{ 
  \frac{531}{312\,500}}}} \\
  & = & \frac{\cancelto{3}{9}}{\displaystyle{
  \frac{\cancelto{2}{100}\times \cancel{3}}{\cancelto{5}{250}}\sqrt{\frac{59}{5}}}}
  = \frac{15\sqrt{5 \times 59}}{2\times 59}
  = \frac{15\sqrt{295}}{118} \approx 2.18333441
  \end{eqnarray*}
 \end{estadistico}

 \begin{valorp}
  De la tabla A.3, es tiene que:
  \begin{equation*}
   P(Z > 2.18) = 1 - P(Z \leq 2.18) \approx 1 - 0.9854 = 0.0146
  \end{equation*}
 \end{valorp}

 \begin{conclusion}
  La muestra arroja evidencia suficiente para rechazar la hip\'otesis nula
  y, por lo tanto, concluir
  que el porcentaje de votantes de la ciudad que est\'a a favor
  de la propuesta de construir la planta qu\'{\i}mica
  dentro de los l\'{\i}mites de la ciudad es mayor, por m\'as del 3\%,
  al porcentaje de votante residentes del condado circundante
  que apoya esta misma propuesta.
 \end{conclusion}

 En el c\'odigo registrado en el archivo anexo
 \texttt{P19\_Prueba\_de\_dos\_proporciones\_02.r},
 escrito en el lenguaje de programaci\'on estad\'{\i}stica R,
 se realiza este procedimiento.
 El c\'odigo no es m\'as que una ligera modificaci\'on del script
 previamente presentado
 \texttt{P09\_Prueba\_de\_dos\_proporciones\_01.r},
 cuyas \'unicas diferencias consisten en la variable,
 dentro de la entrada de datos, \texttt{d0},
 para el valor al que es igual la diferencia de las proporciones,
 y el c\'alculo del estad\'{\i}stico que ahora sigue la f\'ormula
 $z = \frac{\widehat{p}_1-\widehat{p}_2-d_0}{\sqrt{
 \widehat{p}_1\widehat{q}_1/n_1 + \widehat{p}_2\widehat{q}_2/n_2}}$,
 que implica a su vez la eliminaci\'on del c\'alculo del estimador
 de proporciones dado por la f\'ormula
 $\widehat{p} = \frac{x_1+x_2}{n_1+n_2}$, que se daba
 cuando se supon\'{\i}a en la hip\'otesis nula que las proporciones fuesen
 iguales, que ya no se usa para este tipo de prueba.
 Por lo dem\'as, todo permanece tal cual en el c\'odigo,
 tal cual como se explic\'o en la soluci\'on del ejercicio 63.
 \par 
 El c\'odigo junto con el resultado se muestra a continuaci\'on:
 \begin{verbatim}
> n1<-200
> n2<-500
> x1<-120
> x2<-240
> p1<-NULL
> p2<-NULL
> d0<-0.03
> alfa<-NULL
> alternativa<-'>'
> if(n1<30 || n2<30){
+   stop("Las muestras no deben ser pequeñas (al menos de 30 observaciones).")
+ }
> if(is.null(x1)){
+   x1<-round(p1*n1)
+ }
> if(is.null(x2)){
+   x2<-round(p2*n2)
+ }
> 
> TestProp<-function(n1,n2,x1,x2,d0,alfa=0.05,prueba='!='){
+   p1<-x1/n1
+   p2<-x2/n2
+   estadistico<-(p1-p2-d0)/sqrt((p1*(1-p1)/n1+p2*(1-p2)/n2))
+   r<-data.frame(alternativa=paste("p1",prueba,"p2"),
+                 n1=n1,n2=n2,
+                 x1=x1,x2=x2,
+                 p1=p1,p2=p2,
+                 DifProp=p1-p2,
+                 error.est=sqrt((p1*(1-p1)/n1+p2*(1-p2)/n2)),
+                 alpha=alfa
+   )
+   if(prueba=='!='){
+     pvalor<-round(2*pnorm(abs(estadistico),lower.tail=F),7)
+     criticoz<-round(qnorm(1-alfa/2),7)
+     r$PValor<-pvalor
+     r$Estadistico<-estadistico
+     r$RegionRechazoZ<-paste("<=",-criticoz," y >=",criticoz)
+   }else{
+     criticoz<-round(qnorm(1-alfa),7)
+     if(prueba=='<'){
+       pvalor<-round(pnorm(estadistico),7)
+       r$PValor<-pvalor
+       r$Estadistico<-estadistico
+       r$RegionRechazoZ<-paste("<=",-criticoz)
+     }else{
+       pvalor<-round(pnorm(estadistico,lower.tail=F),7)
+       r$PValor<-pvalor
+       r$Estadistico<-estadistico
+       r$RegionRechazoZ<-paste(">=",criticoz)
+     }
+   }
+   return(r)
+ }
> if(is.null(alfa)){
+   Test<-TestProp(n1,n2,x1,x2,d0,prueba=alternativa)
+ }else{
+   Test<-TestProp(n1,n2,x1,x2,d0,alfa,alternativa)
+   resultado<-ifelse(Test[,"PValor"]>=alfa,"No se rechaza H0","Se rechaza H0")
+   Test$Resultado<-resultado
+ }
> Test
  alternativa  n1  n2  x1  x2  p1   p2 DifProp  error.est alpha    PValor
1     p1 > p2 200 500 120 240 0.6 0.48    0.12 0.04122135  0.05 0.0145056
  Estadistico RegionRechazoZ
1    2.183334   >= 1.6448536
 \end{verbatim}
 \vspace{-0.5cm}
 Lo cual coincide con los resultados obtenidos,
 que es a lo que se quer\'{\i}a llegar.${}_{\blacksquare}$
\end{solucion}
