\begin{enunciado}
 Suponga que un alerg\'ologo desea probar la hip\'otesis de que al menos $30\%$ del p\'ublico es al\'ergico a algunos productos de queso. Explique c\'omo el alerg\'ologo podr\'{\i}a cometer
 \begin{enumerate}
  \item un error tipo I;
  \item un error tipo II.
 \end{enumerate}
\end{enunciado}

\begin{solucion}
 $\phantom{0}$
 \begin{enumerate}
  \item Concluir que menos del $30\%$ del p\'ublico es al\'ergico a algunos productos de queso, cuando en realidad s\'{\i} es lo que \'el quer\'{\i}a probar, es decir que al menos $30\%$ del p\'ublico es al\'ergico a algunos productos de queso.
  
  \item Concluir lo que se quiere probar, es decir que al menos $30\%$ del p\'ublico es al\'ergico a algunos productos de queso, cuando en realidad es falso, es decir en realidad es menos del $30\%$ del p\'ublico el que es al\'ergico a algunos productos de queso, que a lo que se quer\'{\i}a llegar.${}_{\blacksquare}$
 \end{enumerate}
\end{solucion}
