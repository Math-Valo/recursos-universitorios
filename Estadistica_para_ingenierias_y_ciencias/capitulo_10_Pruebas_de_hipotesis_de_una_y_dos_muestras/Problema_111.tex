\begin{enunciado}
 Se lleva a cabo un estudio en el Departamento de Saludo y Educaci\'on
 F\'{\i}sica del Instituto Polit\'ecnico y Universidad Estatal
 de Virginia, para determinar
 si 8 semanas de entrenamiento realmente reducen
 los niveles de colesterol en los participantes.
 A un grupo de tratamiento que consiste en $15$ personas se les dan
 conferencias dos veces a la semana
 de c\'omo reducir su nivel de colesterol.
 Otro grupo de $18$ personas de edad similar se selecciona al azar
 como grupo de control.
 Se registran los niveles de colesterol de todos los participantes
 al final del programa de 8 semanas y se listan a continuaci\'on:
 \begin{verbatim}
  Grupo de tratamiento:
      129 131 154 172 115 126 175 191
      122 238 159 156 176 175 126
  Grupo de control:
      151 132 196 195 188 198 187 168 115
      165 137 208 133 217 191 193 140 146
 \end{verbatim}
 ¿Podemos concluir, con un nivel de significancia de $5\%$,
 que el colesterol promedio se reduce como consecuencia del programa?
 Haga una prueba adecuada en las medias.
\end{enunciado}

\begin{solucion}
 Para realizar una prueba de diferencia de promedios con pocos datos
 (menos de 30), el estad\'{\i}stico usado requiere una condici\'on inicial
 que es la suposici\'on de que las muestras provengan de poblaciones
 con distribuci\'on normal.
 Entonces se realizar\'an dos pruebas de bondad de ajuste
 para verificar que, en efecto, las muestras provienen de poblaciones
 normales. Entonces, la hip\'otesis nula para estas pruebas se pueden
 escribir de la forma
 $X_i$ sigue una distribuci\'on normal, para cada $i \in \{1,2\}$,
 donde $X_1$ y $X_2$ representan las poblaciones de las muestras
 del grupo de tratamiento y del grupo sin el tratamiento,
 respectivamente, contra la hip\'otesis alternativa
 de que $X_i$, para cada $i \in \{1,2\}$ no sigue una distribuci\'on
 normal.
 \par 
 Debido a que la muestra es peque\~na y no se consideran par\'ametros
 en la suposici\'on de las normalidades,
 la prueba con $\chi^2$ crea insuficientes clases, en cada muestra,
 que al calcular los grados de libertad junto con los grados que se restan
 por no considerar par\'ametros previos, entonces se obtiene
 un valor negativo, lo cual no puede corresponder a grados de libertad.
 Por lo tanto, se proceder\'a a realizar las pruebas de Geary,
 simult\'aneamente, con las siguientes l\'{\i}neas de c\'odigo
 escritas en R, que usa la base de datos
 \texttt{DB42\_Problema\_111.csv},
 que se presenta a continuaci\'on junto con los resultados:
 \begin{verbatim}
> datos<-read.csv("DB42_Problema_111.csv",sep=";",encoding="UTF-8")
> varInteres<-"Colesterol.nivel"; distincion<-"Grupo"
> lista<-split(datos[,varInteres],datos[,distincion])
> x<-lista$`Grupo de tratamiento`; y<-lista$`Grupo de control`
> n1<-length(x); n2<-length(y)
> U1<-sqrt(pi/2)*mean(abs(x-mean(x)))/sqrt(var(x)*(n1-1)/n1)
> U2<-sqrt(pi/2)*mean(abs(y-mean(y)))/sqrt(var(y)*(n2-1)/n2)
> x.Geary.statistic<-(U1-1)/(0.2661/sqrt(n1))
> y.Geary.statistic<-(U2-1)/(0.2661/sqrt(n2))
> x.Geary.p.value<-2*pnorm(abs(x.Geary.statistic),lower.tail=FALSE)
> y.Geary.p.value<-2*pnorm(abs(y.Geary.statistic),lower.tail=FALSE)
> print("P-valor con datos del grupo de tratamiento usando la prueba de Geary")
[1] "P-valor con datos del grupo de tratamiento usando la prueba de Geary"
> x.Geary.p.value
[1] 0.9778075
> print("P-valor con datos del grupo de control usando la prueba de Geary")
[1] "P-valor con datos del grupo de control usando la prueba de Geary"
> y.Geary.p.value
[1] 0.03675674
 \end{verbatim}
 \vspace{-0.5cm}
 Esto indica un buen ajuste a una distribuci\'on normal
 para la poblaci\'on que sigue el tratamiento;
 sin embargo, la poblaci\'on de aquellos que no siguen el tratamiento
 muestran un valor $P$ de $0.03675674$ para el ajuste de bondad.
 Se tomar\'a el riesgo ante la baja probabilidad
 y se tomar\'a la decisi\'on de suponer que ambas muestras vienen
 de poblaciones con distribuciones normales.
 \par
 Para la prueba, se requiere ahora saber si las varianzas poblacionales
 de donde vienen ambas muestras se deben de considerar iguales
 o diferentes. Este implica una prueba de proporci\'on de varianzas,
 cuya hip\'otesis nula se puede enunciar como
 $H_0: \, \frac{\sigma_1}{\sigma_2} = 1$, contra la hip\'otesis
 alternativa $H_1: \, \frac{\sigma_1}{\sigma_2} \neq 1$,
 donde $\sigma_1$ y $\sigma_2$ representan las varianzas poblacionales
 de $X_1$ y $X_2$, respectivamente.
 Para ello se hace una prueba $F$ con ayuda del archivo adjunto
 \texttt{P14\_Prueba\_de\_dos\_varianzas\_02.r},
 que a su vez usa el archivo \texttt{DB42\_Problema\_111.csv},
 con los siguientes cambios:
 \begin{verbatim}
> datos<-read.csv("DB42_Problema_111.csv",sep=";",encoding="UTF-8")
> varInteres<-c("Colesterol.nivel")
> varSel<-c("Grupo")
> alfa<-NULL
> cola<-'D'
 \end{verbatim}
 \vspace{-0.5cm}
 lo cual arroja el siguiente resultado:
 \begin{verbatim}
          variable Freq n1 n2   media1 media2 varianza1 varianza2 v1 v2 alpha
1 Colesterol.nivel   33 15 18 156.3333    170  1094.952  947.8824 14 17  0.05
     PValor Estadistico             RegionRechazo
1 0.7678441    1.155156 < 0.3447969 y > 2.7526407
 \end{verbatim}
 \vspace{-0.5cm}
 en donde se observa que el estad\'{\i}stico $F=\frac{\sigma_1}{\sigma_2}$
 toma el valor de $1.155156$ con $v_1 = 14$ y $v_2 = 17$ grados
 de libertad para la distribuci\'on $F$,
 obteniendo as\'{\i} un valor $P$ de $0.7678441$,
 lo cual, al ser muy alto, no se rechaza la hip\'otesis nula y,
 por lo tanto, se puede suponer las varianzas iguales para la prueba $t$.
 \par 
 Para la prueba se enuncia la hip\'otesis nula como sigue:
 \begin{eqnarray*}
  H_0: & & \mu_1 - \mu_2 \leq 0 \\
  H_1: & & \mu_1 - \mu_2   >  0
 \end{eqnarray*}
 donde $\mu_1$ y $\mu_2$ representan las medias poblacionales
 de $X_1$ y $X_2$, respectivamente,
 con un nivel de significancia de $\alpha = 0.05$.
 La regi\'on de rechazo, el estad\'{\i}stico y la decisi\'on
 se calculan con ayuda del archivo adjunto
 \texttt{P06\_Prueba\_de\_dos\_medias\_02.r},
 que nuevamente usan la base de datos almacenado en el archivo adjunto
 \texttt{DB42\_Problema\_111.csv}, con los siguientes cambios:
 \begin{verbatim}
> datos<-read.csv("DB42_Problema_111.csv",sep=";",encoding="UTF-8")
> varInteres<-c("Colesterol.nivel")
> varSel<-c("Grupo")
> mu<-0
> desv.iguales<-TRUE
> alfa<-0.05
> cola<-'I'
> par<-FALSE
 \end{verbatim}
 \vspace{-0.5cm}
 lo cual arroja el siguiente resultado:
 \begin{verbatim}
              Var1 Freq    Poblaciones H0 valorPVar     suposicionVar n1 n2
1 Colesterol.nivel   33 Independientes  0 0.7678441 Var no diferentes 15 18
    media1 media2 diferencia desv.est1 desv.est2   est.sp error.est grados alpha
1 156.3333    170  -13.66667  33.09006   30.7877 31.84809  11.13419     31  0.05
     PValor Estadistico RegionRechazoInfT RegionRechazoInfX        Resultado
1 0.1144465   -1.227451         -1.695519         -18.87822 No se rechaza H0
 \end{verbatim}
 \vspace{-0.5cm}
 de donde se observa que la regi\'on de rechazo est\'a dado
 para $T < -1.695519$,
 el estad\'{\i}stico obtenido es $t = -1.227451$ y la decisi\'on tomada,
 que aunque es directo tambi\'en se muestra en el programa,
 es la de no rechazar $H_0$ y, por lo tanto, concluir
 que el colesterol promedio no se reduce como consecuencia del programa,
 que es a lo que se quer\'{\i}a llegar.${}_{\blacksquare}$
\end{solucion}
