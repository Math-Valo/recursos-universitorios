\begin{enunciado}
 Un genetista se interesa en la proporci\'on de hombres y mujeres
 de una poblaci\'on que tiene cierto trastorno sangu\'{\i}neo menor.
 En una muestra aleatoria de $100$ hombres, se encuentra
 que $31$ lo padecen, mientras que s\'olo $24$ de $100$ mujeres parecen
 tener el trastorno.
 ¿Con un nivel de significancia de $0.01$ podemos concluir
 que la proporci\'on de hombres en la poblaci\'on
 con este trastorno sangu\'{\i}neo es significantemente mayor
 que la proporci\'on de mujeres afectadas?
\end{enunciado}

\begin{solucion}
 Usando el archivo anexo \texttt{P09\_Prueba\_de\_dos\_proporciones\_01.r}, con los siguientes cambios:
 \begin{verbatim}
> n1<-100
> n2<-100
> x1<-31
> x2<-24
> p1<-NULL
> p2<-NULL
> alfa<-0.01
> alternativa<-'>'
 \end{verbatim}
 \vspace{-0.5cm}
 el programa de R lanza el siguiente resultado:
 \begin{verbatim}
  alternativa  n1  n2 x1 x2   p1   p2 pEstimada DifProp  error.est alpha
1     p1 > p2 100 100 31 24 0.31 0.24     0.275    0.07 0.06314665  0.01
     PValor Estadistico RegionRechazoZ        Resultado
1 0.1338164    1.108531   >= 2.3263479 No se rechaza H0
 \end{verbatim}
 \vspace{-0.5cm}
 El cual indica que los datos: $n_1 = n_2 = 100$,
 con $x_1 = 31$ y $x_2 = 24$; la hip\'otesis nula: $H_0: p_1 \leq p_2$
 contra la hip\'otesiis alternativa: $H_1: p_1 > p_2$; el nivel
 de significancia: $\alpha = 0.01$; el valor del estad\'{\i}stico $z \approx
 \frac{\hat{p}_1 - \hat{p}_2}{\sqrt{\hat{p}\hat{q}(1/n_1+1/n_2)}} \approx
 1.108531$; la regi\'on de rechazo $z > 2.3263479$, con todo y la decisi\'on
 de no rechazar $H_0$; e, incluso, el valor $P$ de $P(z > 1.108531) \approx
 0.1338164$.
 \par 
 Luego entonces se concluye que los datos no arrojan evidencia suficiente
 para rechazar $H_0$; es decir, no se puede concluir que la proporci\'on
 de hombres en la poblaci\'on con el trastorno sangu\'{\i}neo
 se significantemente mayor que la proporci\'on de mujeres afectadas,
 que es a lo que se quer\'{\i}a llegar.${}_{\blacksquare}$
\end{solucion}
