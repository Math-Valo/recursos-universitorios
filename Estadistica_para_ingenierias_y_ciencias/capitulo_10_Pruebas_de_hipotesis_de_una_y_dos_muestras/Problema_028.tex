\begin{enunciado}
 De acuerdo con el \textit{Chemical Engineering} una propiedad importante de la fibra es su absorci\'on de agua. Se encuentra que el porcentaje promedio de absorci\'on de $25$ piezas de fibra de algod\'on que se seleccionan al azar es $20$ con una desviaci\'on est\'andar de $1.5$. Una muestra aleatoria de $25$ piezas de acetato dan un porcentaje promedio de $12$ con una desviaci\'on est\'andar de $1.25$. Hay evidencia s\'olida de que el porcentaje promedio de absorci\'on de la poblaci\'on para la fibra de algod\'on es significativamente mayor que la media para el acetato. Suponga que el porcentaje de absorci\'on se distribuye de forma aproximadamente normal, y que las varianzas de la poblaci\'on en el porcentaje de absorci\'on para las dos fibras son las mismas. Utilice un nivel de significancia de $0.05$.
\end{enunciado}

\begin{solucion}
 \begin{datos}
  $\phantom{0}$
  \begin{itemize}
   \item $X_i \sim n\left( \mu_i, \sigma_i \right)$, para cada $i \in \{ 1, 2 \}$.
   \item $n_1 = n_2 = 25$.
   \item $\bar{x}_1 = 20$ y $\bar{x}_2 = 12$.
   \item $s_1 = 1.5$ y $s_2 = 1.25$.
   \item $\sigma_1 = \sigma_2$.
  \end{itemize}
 \end{datos}

 \begin{hipotesis}
  \begin{eqnarray*}
   H_0: \mu_1 - \mu_2 & = & 0 \\
   H_1: \mu_1 - \mu_2 & > & 0
  \end{eqnarray*}
 \end{hipotesis}

 \begin{significancia}
  $\alpha = 0.05$.
 \end{significancia}

 \begin{region}
  De la tabla A.4, se tiene el valor cr\'{\i}tico $t_{\alpha,n_1+n_2-2} = t_{0.05,48}$ est\'a entre $t_{0.05,60} = 1.671$ y $t_{0.05,40} = 1.684$, por lo que se interpola a $1.6762$. Mientras tanto, usando R, se tiene un valor m\'as pr\'oximo como:
  \begin{verbatim}
> options(digits=22)
> qt(0.05,48,lower.tail=FALSE)
[1] 1.677224196124339261615
  \end{verbatim}
  \vspace{-0.5cm}
  por lo que m\'as precisamente es $t_{0.05,48} = 1.6772$.
 \end{region}

 \begin{estadistico}
  Dado que
  \begin{eqnarray*}
   s_p^2 & = & \frac{s_1^2\left(n_1 - 1\right) + s_2^2\left(n_2 - 1\right)}{n_1 + n_2 - 2} = \frac{1.5^2(25-1) + 1.25^2(25-1)}{25+25-2} = \frac{2.25(24) + 1.5625(24)}{48} \\
   & = & \frac{\cancel{24}(2.25+1.5625)}{\cancelto{2}{48}} = \frac{3.8125}{2} = \frac{61}{32} = 1.90625
  \end{eqnarray*}
  entonces
  \begin{equation*}
   s_p = \sqrt{s_p^2} = \sqrt{\frac{61}{32}} = \frac{\sqrt{122}}{8} \approx 1.3806701271484\ldots
  \end{equation*}
  y
  \begin{eqnarray*}
   t & = & \frac{\left( \bar{x}_1 - \bar{x}_2 \right) - d_0}{s_p\sqrt{\frac{1}{n_1} + \frac{1}{n_2}}} = \frac{(20-12)-0}{\frac{\sqrt{122}}{8}\sqrt{\frac{1}{25}+\frac{1}{25}}} = \frac{8(8)}{\sqrt{\frac{122(1+1)}{25}}} = \frac{64}{\frac{\sqrt{244}}{5}} = \frac{320}{2\sqrt{61}} \\
   & = & \frac{160\sqrt{61}}{61} \approx 20.485900789263\ldots 
  \end{eqnarray*}
 \end{estadistico}

 \begin{decision}
  Se rechaza $H_0$ a favor de $H_1$.
 \end{decision}

 \begin{conclusion}
  El porcentaje de absorci\'on del algod\'on es, en efecto, mayor al porcentaje de absorci\'on del acetato como se menciona en el enunciado.
 \end{conclusion}

 Finalmente, usando el archivo anexo
 \texttt{P05\_Prueba\_de\_dos\_medias\_01.r},
 con los siguientes cambios:
 \begin{verbatim}
> n1<-25
> n2<-25
> mu<-0
> m1<-20
> m2<-12
> m<-NULL
> sigma1<-NULL
> sigma2<-NULL
> s1<-1.5
> s2<-1.25
> sD<-NULL
> desv.iguales<-TRUE
> alfa<-0.05
> cola<-'S'
> par<-FALSE
 \end{verbatim}
 \vspace{-0.5cm}
 el programa de R lanza el siguiente resultado:
 \begin{verbatim}
  Prueba var.pobl H0 n1 n2 DifMedias desv.est1 desv.est2  est.sp error.est
1      t  Iguales  0 25 25         8       1.5      1.25 1.38067 0.3905125
  grados.libertad alpha PValor Estadistico RegionRechazoT RegionRechazoX
1              48  0.05      0     20.4859    > 1.6772242     > 0.654977
      Resultado
1 Se rechaza H0
 \end{verbatim}
 \vspace{-0.5cm}
 El cual coincide con los resultados obtenidos,
 que es a lo que se quer\'{\i}a llegar.${}_{\blacksquare}$
\end{solucion}
