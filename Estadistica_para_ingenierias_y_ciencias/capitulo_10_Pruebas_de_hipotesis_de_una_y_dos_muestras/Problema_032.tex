\begin{enunciado}
 El \textit{Amstat News} (diciembre de 2004) lista los sueldos medios de profesores asociados de estad\'{\i}stica en instituciones de investigaci\'on, en escuelas de humanidades y en otras instituciones en Estados Unidos. Suponga que una muestra de $200$ profesores asociados de instituciones de investigaci\'on que tienen un sueldo promedio de $\$70,750$ anuales con una desviaci\'on est\'andar de $\$6000$. Suponga tambi\'en una muestra de $200$ profesores asociados de otros tipos de instituci\'on que tienen un sueldo promedio de $\$65,200$ con una desviaci\'on est\'andar de $\$5000$. Pruebe la hip\'otesis de que el sueldo medio de profesores asociados en instituciones de investigaci\'on es $\$2000$ mayor que los de los de otras instituciones. Utilice un nivel de significancia de $0.01$.
\end{enunciado}

\begin{solucion}
 \begin{datos}
  $\phantom{0}$
  \begin{itemize}
   \item $n_1 = n_2 = 200$.
   \item $\bar{x}_1 = 70\,750$ y $\bar{x}_2 = 65\,200$.
   \item $s_1 = 6\,000$ y $s_2 = 5\,000$
  \end{itemize}
 \end{datos}

 \begin{hipotesis}
  \begin{eqnarray*}
   H_1: \mu_1 - \mu_2 & \leq & 2\,000 \\
   H_1: \mu_1 - \mu_2 & > & 2\,000
  \end{eqnarray*}
 \end{hipotesis}

 \begin{significancia}
  $\alpha = 0.01$.
 \end{significancia}

 \begin{region}
  De la tabla A.3, se tiene el valor cr\'{\i}tico $z_{\alpha} = z_{0.01} \approx 2.33$, por lo que la regi\'on de rechazo est\'a dado para $z > 2.33$, donde $z = \frac{\left( \bar{x}_1 - \bar{x}_2 \right) - d_0}{\sqrt{\sigma_1^2/n_1 + \sigma_2^2/n_2}}$.
 \end{region}

 \begin{estadistico}
  Ya que el tama\~no de las muestras es grande, se puede aproximar las desviaciones est\'andar muestrales a las poblacionales con lo que se usa el siguiente estad\'{\i}stico:
  \begin{eqnarray*}
   z & = & \frac{\left( \bar{x}_1 - \bar{x}_2 \right) - d_0}{\sqrt{\sigma_1^2/n_1 + \sigma_2^2/n_2}} \approx \frac{\left( \bar{x}_1 - \bar{x}_2 \right) - d_0}{\sqrt{s_1^2/n_1 + s_2^2/n_2}} = \frac{(70\,750 - 65\,200) - 2\,000}{\sqrt{\frac{6\,000^2}{200} + \frac{5\,000^2}{200}}} = \frac{5\,550 - 2\,000}{\sqrt{\frac{36\,000\,000 + 25\,000\,000}{200}}} \\
   & = & \frac{3\,550}{\sqrt{\frac{61000000}{200}}} = \frac{3\,550}{\sqrt{305\,000}} = \frac{3\,550}{50\sqrt{122}} = \frac{71\sqrt{122}}{122} \approx 6.428037969
  \end{eqnarray*}
 \end{estadistico}

 \begin{decision}
  Se rechaza $H_0$ a favor de $H_1$.
 \end{decision}

 \begin{conclusion}
  El sueldo medio de profesores asociados en instituciones de investigaci\'on es, en efecto, $\$2\,000$ mayor que los de las otras instituciones.
 \end{conclusion}

 Finalmente, usando el archivo anexo
 \texttt{P05\_Prueba\_de\_dos\_medias\_01.r},
 con los siguientes cambios:
 \begin{verbatim}
> n1<-200
> n2<-200
> mu<-2000
> m1<-70750
> m2<-65200
> m<-NULL
> sigma1<-NULL
> sigma2<-NULL
> s1<-6000
> s2<-5000
> sD<-NULL
> desv.iguales<-NULL
> alfa<-0.01
> cola<-'S'
> par<-FALSE
 \end{verbatim}
 \vspace{-0.5cm}
 el programa de R lanza el siguiente resultado:
 \begin{verbatim}
  Prueba   H0  n1  n2 DifMedias desv.est1 desv.est2 error.est alpha PValor
1      Z 2000 200 200      5550      6000      5000  552.2681  0.01      0
  Estadistico RegionRechazoZ RegionRechazoX     Resultado
1    6.428038    > 2.3263479 > 3284.7676204 Se rechaza H0
 \end{verbatim}
 \vspace{-0.5cm}
 El cual coincide con los datos obtenidos,
 que es a lo que se quer\'{\i}a llegar.${}_{\blacksquare}$
\end{solucion}
