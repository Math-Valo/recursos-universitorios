\begin{enunciado}
 Por experiencia se sabe
 que el tiempo que se requiere para que los estudiantes de preparatoria
 de \'ultimo a\~no completen una prueba estandarizada es una variable aleatoria
 normal, con una desviaci\'on est\'andar de $6$ minutos.
 Pruebe la hip\'otesis de que $\sigma = 6$ contra la alternativa
 de que $\sigma < 6$,
 si una muestra aleatoria de $20$ estudiantes de preparatoria de \'ultimo a\~no
 tiene una desviaci\'on est\'andar $s = 4.51$.
 Utilice un nivel de significancia de $0.05$.
\end{enunciado}

\begin{solucion}
 \begin{datos}
  $\phantom{0}$
  \begin{itemize}
   \item $X \sim n\left( \mu, \sigma \right)$.
   \item $n = 20$.
   \item $s = 4.51$.
  \end{itemize}
 \end{datos}

 \begin{hipotesis}
  \begin{eqnarray*}
   H_0: \sigma^2 & = & 36 \\
   H_1: \sigma^2 & < & 36
  \end{eqnarray*}
 \end{hipotesis}

 \begin{significancia}
  $\alpha = 0.05$.
 \end{significancia}

 \begin{region}
  De la tabla A.5, se tiene el valor cr\'{\i}tico
  $\chi^2_{1-\alpha,n-1}=\chi^2_{0.95,19} \approx 10.117$,
  por lo que la regi\'on de rechazo est\'a dado para $\chi^2 < 10.117$,
  donde $\chi^2 = \frac{(n-1)s^2}{\sigma_0^2}$.
 \end{region}

 \begin{estadistico}
  \begin{equation*}
   \chi^2 = \frac{(n-1)s^2}{\sigma_0^2} = \frac{19(4.51)^2}{36}
   = \frac{19(20.3401)}{36} = \frac{386.4619}{36}
   = \frac{3\,864\,619}{360\,000} = 10.735052\bar{7}
  \end{equation*}
 \end{estadistico}

 \begin{decision}
  No se rechaza $H_0$.
 \end{decision}

 \begin{conclusion}
  Seg\'un el criterio del nivel de significancia, se concluye
  que no hay prueba suficiente para rechazar la hip\'otesis nula;
  es decir, se mantiene la afirmaci\'on de que los estudiantes
  var\'{\i}an en al menos $6$ minutos para completar una prueba estandarizada.
 \end{conclusion}

 En el c\'odigo registrado en el archivo anexo
 \texttt{P11\_Prueba\_de\_una\_varianza\_02.r}, en R,
 se realiza este procedimiento.
 El c\'odigo permite modificar los valores iniciales correspondientes a:
 \texttt{n} para el tama\~no de la muestra;
 \texttt{sigma2} para la varianza poblacional supuesta en la hip\'otesis nula;
 \texttt{s2} para la varianza muestral;
 \texttt{alfa} para el nivel de significancia;
 \texttt{cola} para indicar si la prueba es de dos colas,
 con \texttt{'D'}, de cola inferior, con \texttt{'I'},
 o de cola superior, con \texttt{'S'}.
 \par 
 El programa espera al menos los datos correspondientes a la muestra:
 el tama\~no de la muestra y la varianza muestral;
 se indica tambi\'en el tipo de prueba (cola izquierda o derecha,
 o dos colas);
 y, finalmente, la varianza poblacional,
 que corresponde a la hip\'otesis nula $H_0$.
 En caso de no fijar una probabilidad de cometer un error de tipo I
 a la variable \texttt{alfa}, se le asigna el valor \texttt{NULL}.
 \par 
 La prueba de hip\'otesis siempre usar\'a un estad\'{\i}stico
 con distribuci\'on $\chi^2$.
 Para ello, se est\'a suponiendo de antemano la normalidad
 de la poblaci\'on de las que viene la muestra.
 \par
 Independientemente del tipo de prueba, el resultado muestra lo siguiente:
 \texttt{n} indica el tama\~no de la muestra;
 \texttt{H0} para indicar el valor propuesto en la hip\'otesis nula
 de la varianza poblacional;
 \texttt{var.muestral}, para indicar la varianza muestral de los datos;
 \texttt{grados}, indica los grados de libertad de la distribuci\'on $\chi^2$
 correspondiente al estadístico de la prueba;
 \texttt{error.est}, para indicar el error est\'andar,
 que corresponde a la estimaci\'on de la varianza poblacional
 usando la varianza muestral, calculado como $e.e.(s^2)=\sigma^2/n$;
 \texttt{alfa}, para el nivel de significancia dado,
 el cual muestra por defecto $0.05$
 en caso de asignarle \texttt{NULL} a \texttt{alfa};
 \texttt{PValor}, para el valor $P$, la probabilidad
 de haber obtenido la varianza muestral
 que se obtuvo suponiendo que la hip\'otesis nula sea cierta;
 \texttt{Estadistico}, para el valor resultante
 del estad\'{\i}stico de prueba;
 \texttt{RegionRechazoJi} para la regi\'on de rechazo del estad\'{\i}stico
 de prueba;
 y, \texttt{RegionRechazoX} para la regi\'on de rechazo de la varianza muestral.
 \par 
 Adem\'as, seg\'un el tipo de prueba, pueden darse el valor
 de \texttt{Resultado} para indicar si se rechaza o no
 la hip\'otesis nula, que aparece al final de los resultados
 cuando se asigna un valor a \texttt{alfa}.
 \par 
 El c\'odigo junto con el resultado se muestra a continuaci\'on:
 \begin{verbatim}
> n<-20
> sigma2<-36
> s2<-4.51^2
> alfa<-0.05
> cola<-'I'
> TestVar<-function(n,varP,varM,alfa=0.05,colas='D'){
+   v<-n-1
+   r<-data.frame(n=n,
+                 H0=varP,
+                 var.muestral=varM,
+                 grados=v)
+   if(n<2){
+     r$error.est<-NA
+     r$alpha<-alfa
+     r$Pvalor<-NA
+     r$estadistico<-NA
+     r$RegionRechazoJi<-NA
+     r$RegionRechazoX<-NA
+   }else{
+     error<-varP/v
+     estadistico<-varM*v/varP
+     r$error.est<-round(error,7)
+     r$alpha<-alfa
+     if(colas=='I'){
+       criticojiL<-qchisq(alfa,v)
+       criticoXL<-criticojiL*error
+       r$PValor<-round(pchisq(estadistico,v),7)
+       r$Estadistico<-estadistico
+       r$RegionRechazoJi<-paste("<",round(criticojiL,7))
+       r$RegionRechazoX<-paste("<",round(criticoXL,7))
+     }else{
+       if (colas=='S') {
+         criticojiU<-qchisq(alfa,v,lower.tail=F)
+         criticoXU<-criticojiU*error
+         r$PValor<-round(pchisq(estadistico,v,lower.tail=F),7)
+         r$Estadistico<-estadistico
+         r$RegionRechazoJi<-paste(">",round(criticojiU,7))
+         r$RegionRechazoX<-paste(">",round(criticoXU,7))
+       }else{
+         criticojiU<-round(qchisq(alfa/2,v,lower.tail=F),7)
+         criticoXU<-round(criticojiU*error,7)
+         criticojiL<-round(qchisq(alfa/2,v),7)
+         criticoXL<-round(criticojiL*error,7)
+         pvalor<-round(pchisq(estadistico,v),7)
+         r$PValor<-ifelse(varM<varP,2*pvalor,2*(1-pvalor))
+         r$Estadistico<-estadistico
+         r$RegionRechazoJi<-paste("<",criticojiL,"y >",criticojiU)
+         r$RegionRechazoX<-paste("<",criticoXL,"y >",criticoXU)
+       }
+     }
+     return(r)
+   }
+ }
> if(is.null(alfa)){
+   Test<-TestVar(n,sigma2,s2,colas=cola)
+ }else{
+   Test<-TestVar(n,sigma2,s2,alfa=alfa,colas=cola)
+   resultado<-ifelse(Test[,"PValor"]>=alfa,
+                     "No se rechaza H0","Se rechaza H0")
+   Test$Resultado<-resultado
+   Test$Resultado[is.na(Test$Resultado)]<-"Pocos datos"
+ }
> Test
   n H0 var.muestral grados error.est alpha    PValor Estadistico RegionRechazoJi
1 20 36      20.3401     19  1.894737  0.05 0.0675709    10.73505    < 10.1170131
  RegionRechazoX        Resultado
1   < 19.1690774 No se rechaza H0
 \end{verbatim}
 El cual coincide con los c\'alculos obtenidos,
 adem\'as de ofrecer m\'as informaci\'on, como el valor $P$
 para el estad\'{\i}stico calculado, el cual se observar es bastante bajo,
 aunque no llega a ser menor que el nivel de significancia $\alpha$.
 Esto implica que, aunque no se rechace la hip\'otesis nula,
 se tiene algo de evidencia que podr\'{\i}a indicar una diferencia
 en la varianza supuesta en la hip\'otesis nula y quiz\'as valga la pena
 generar una nueva muestra.
 Que es a lo que se quer\'{\i}a llegar.${}_{\blacksquare}$
\end{solucion}
