\begin{enunciado}
 Un fabricante desarrolla un nuevo sedal para pesca que, seg\'un afirma, tiene una resistencia media a la rotura de $15$ kilogramos con una desviaci\'on est\'andar de $0.5$ kilogramos. Para probar la hip\'otesis de que $\mu = 15$ kilogramos contra la alternativa de que $\mu < 15$ kilogramos, se prueba una muestra aleatoria de $50$ sedales. La regi\'on cr\'{\i}tica se define como $\bar{x} < 14.9$.
 \begin{enumerate}
  \item Encuentre la probabilidad de cometer un error tipo I cuando $H_0$ es verdadera.
  
  \item Eval\'ue $\beta$ para las alternativas $\mu = 14.8$ y $\mu = 14.9$ kilogramos.
 \end{enumerate}
\end{enunciado}

\begin{solucion}
 Sea $X$ la resistencia a la rotura, en kilogramos, de este nuevo sedal y sea $\overline{X}$ la variable aleatoria de la muestra, del enunciado se tiene lo siguiente, en donde $x_{\text{inf}}$ representa el valor cr\'{\i}tico inferior, en el que se incluye la regi\'on de aceptaci\'on.
 \begin{itemize}
  \item $X \sim n(\mu,\sigma)$.
  \item $\overline{X} \sim n\left( \mu, \sigma/\sqrt{n} \right)$
  \item $n = 50$.
  \item $\sigma = 0.5\,$kg.
  \item $\sigma_{\overline{X}} = 0.5/\sqrt{50} = \frac{0.5\sqrt{2}}{10} = 0.05\sqrt{2}$
  \item $x_{\text{inf}} = 14.9$.
 \end{itemize}
 Con lo que se realiza lo pedido en los incisos como sigue.
 \begin{enumerate}
  \item Suponiendo que
  \begin{itemize}
   \item $\mu = 15$.
  \end{itemize}
  el error tipo I se aproxima usando la tabla A.3 como sigue:
  \begin{eqnarray*}
   \alpha & = & P(\overline{X} < 14.9) = P\left( Z < \frac{14.9-15}{0.05\sqrt{2}} \right) = P\left( Z < -\frac{0.1\sqrt{2}}{0.1} \right) = P\left( Z < -\sqrt{2} \right) \\
   & \approx & P(Z < -1.41) \approx 0.0793
  \end{eqnarray*}
  Finalmente, usando R, se puede calcular la probabilidad usando el script en el archivo anexo \texttt{P02\_Probabilidad\_de\_error\_normal\_1.r}, cambiando las siguientes l\'{\i}neas de c\'odigo:
  \begin{verbatim}
> n<-50
> CriticoInf<-14.9
> CriticoSup<-NULL
> desv<-0.05*sqrt(2)
> media0<-15
> media1<-NULL
> p0<-0.6
> p1<-0.5
  \end{verbatim}
  \vspace{-0.5cm}
  con lo que se obtiene
  \begin{verbatim}
$`Probabilidad de error tipo I`
  HipotesisNula  n media       desv CriticoInf     alpha
1      mu =  15 50    15 0.07071068       14.9 0.0786496
  \end{verbatim}
  \vspace{-0.5cm}
  Por lo tanto, se tiene lo siguiente:
  \begin{itemize}
   \item La aproximaci\'on las tablas da $\alpha = 0.0793$.
   \item La aproximaci\'on con R da $\beta = 0.0786496$.${}_{\square}$
  \end{itemize}

  \item Suponiendo que
  \begin{itemize}
   \item $\mu = 14.8$
  \end{itemize}
  el error tipo II se aproxima usando la tabla A.3 como sigue:
  \begin{eqnarray*}
   \beta & = & P\left(\overline{X} \geq 14.9\right) = 1 - P\left(\overline{X} < 14.9\right) = 1 - P\left( Z < \frac{14.9-14.8}{0.05\sqrt{2}} \right) \\
   & = & 1 - P\left( Z < \frac{0.1\sqrt{2}}{0.1} \right) = 1 - P\left( Z < \sqrt{2} \right) = 1 - P(Z < 1.41) \approx 1 - 0.9207 = 0.0793.
  \end{eqnarray*}
  Bajo supuesto siguiente
  \begin{itemize}
   \item $\mu = 14.9$.
  \end{itemize}
  se tiene ahora que el error tipo II se calcula como sigue:
  \begin{eqnarray*}
   \beta & = & P\left(\overline{X} \geq 14.9\right) = 1 - P\left(\overline{X} < 14.9\right) = 1 - P\left( Z < \frac{14.9-14.9}{0.05\sqrt{2}} \right) \\
   & = & 1 - P(Z < 0) = 1 - 0.5 = 0.5
  \end{eqnarray*}
  Finalmente, usando R, se puede calcular estas probabilidades usando el script en el archivo anexo \texttt{P02\_Probabilidad\_de\_errror\_normal\_1.r}, cambiando las siguientes l\'{\i}neas de c\'odigo para el primer caso:
  \begin{verbatim}
> n<-50
> CriticoInf<-14.9
> CriticoSup<-NULL
> desv<-0.05*sqrt(5)
> media0<-NULL
> media1<-14.8
> p0<-NULL
> p1<-NULL
  \end{verbatim}
  \vspace{-0.5cm}
  con lo que se obtiene
  \begin{verbatim}
$`Probabilidad de error tipo II`
  HipotesisAlternativa  n media       desv CriticoInf      beta
1           mu =  14.8 50  14.8 0.07071068       14.9 0.0786496
  \end{verbatim}
  \vspace{-0.5cm}
  mientras que para el segundo caso se cambian las siguientes l\'{\i}neas de c\'odigo:
  \begin{verbatim}
> n<-50
> CriticoInf<-14.9
> CriticoSup<-NULL
> desv<-0.5
> media0<-NULL
> media1<-14.9
> p0<-NULL
> p1<-NULL
  \end{verbatim}
  \vspace{-0.5cm}
  con lo que se obtiene
  \begin{verbatim}
$`Probabilidad de error tipo II`
  HipotesisAlternativa  n media desv CriticoInf beta
1           mu =  14.9 50  14.9  0.5       14.9  0.5
  \end{verbatim}
  \vspace{-0.5cm}
  Por lo tanto, se tiene el siguiente resumen:
  \begin{itemize}
   \item Bajo el supuesto $\mu = 14.8$:
   \begin{itemize}
    \item La aproximaci\'on con las tablas da $\beta = 0.4207$.
    \item La aproximaci\'on con R da $\beta = 0.4207403$.
   \end{itemize}
   
   \item Y, bajo el supuesto $p = 14.9$:
   \begin{itemize}
    \item El valor exacto, ya sea con tablas o con R, da $\beta = 0.5$.
   \end{itemize}
  \end{itemize}
  que es a lo que se quer\'{\i}a llegar.${}_{\blacksquare}$
 \end{enumerate}
\end{solucion}
