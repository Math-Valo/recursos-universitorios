\begin{enunciado}
 En el ejercicio 9.88 de la p\'agina 315, utilice la distribuci\'on $t$ para probar la hip\'otesis de que la dieta reduce el peso de un individuo en $4.5$ kilogramos, en promedio, contra la hip\'otesis alternativa de que la diferencia media en peso es menor que $4.5$ kilogramos. Utilice un valor $P$.
\end{enunciado}

\begin{solucion}
 \begin{datos}
  Usando los datos del ejercicio 9.88, se tiene lo siguiente:
  \begin{itemize}
   \item $X_i \sim n\left( \mu_i, \sigma_i \right)$,
   para cada $i \in \{ 1, 2 \}$.
   \item $n_1 = n_2 = 7$.
   \item $\bar{d} = \frac{249}{70} = 3.5\overline{571428}$.
   \item $s_D = \frac{\sqrt{339\,843}}{210}
   \approx2.7760026073817863$.
  \end{itemize}
  Adem\'as, por la suposici\'on de normalidad en la distribuci\'on
  de las diferencias poblacionales pareadas,
  se sabe que el siguiente estad\'{\i}stico, que se va a requerir,
  se aproxima a la distribuci\'on mostrada
  con el respectivo par\'ametro que se indica:
  \begin{itemize}
   \item $T = \frac{\overline{D} - d_0}{S_D/\sqrt{n}} \sim t(v)$.
   \item $v = n_1 - 1 = n_2 - 1 = 6$.
  \end{itemize}
 \end{datos}

 \begin{hipotesis}
  \begin{eqnarray*}
   H_0: \mu_D = \mu_1 - \mu_2 & \geq & 4.5 \\
   H_1: \mu_D = \mu_1 - \mu_2 &   <  & 4.5
  \end{eqnarray*}
 \end{hipotesis}

 \begin{estadistico}
  En el c\'alculo siguiente, se aprovech\'o que se ten\'{\i}a
  el valor de $s_D/\sqrt{n}$ en el ejercicio 9.88.
  \begin{eqnarray*}
   t & = & \frac{\bar{d} - d_0}{s_D/\sqrt{n}}
   = \frac{\frac{249}{70} - 4.5}{\frac{\sqrt{48\,549}}{210}}
   = \frac{
   \frac{249 - 9(35)}{\cancel{70}} \sqrt{48\,549} 
   }{
   \frac{48\,549}{\cancelto{3}{210}}
   }
   = -\frac{-66\sqrt{48\,549}}{16\,183}
   \approx -0.898617859805857047351
  \end{eqnarray*}
 \end{estadistico}

 \begin{valorp}
  De la tabla A.4, se tiene que:
  \begin{eqnarray*}
   P\left( T < t \right) & \approx & P(T < -0.898617859805857047351)
   \\
   & = & P(T > 0.898617859805857047351)
   \approx P(T>0.906) \approx 0.2
  \end{eqnarray*}
  N\'otese que $0.898617859805857047351$ a $0.906$
  es debido a que es, por mucho, el valor cr\'{\i}tico m\'as cercano,
  ya que en la tabla el valor cr\'{\i}tico pr\'oximo despu\'es
  de ese es $0.553$.
 \end{valorp}

 \begin{conclusion}
  Como el valor $P$ es alto, se concluye
  que la reducci\'on de peso usando la dieta
  no es significativamente menor a los $4.5$ kilogramos,
  y no hay evidencia para contradecir al fabricante si
  \'el afirma que su dieta reducir\'a en $4.5$ kilogramos el peso
  de un individuo, en un lapso de 2 semanas.
 \end{conclusion}

 Finalmente, usando el archivo anexo
 \texttt{P05\_Prueba\_de\_dos\_medias\_01.r},
 con los siguientes cambios:
 \begin{verbatim}
> n1<-7
> n2<-7
> mu<-4.5
> m1<-NULL
> m2<-NULL
> m<-249/70
> sigma1<-NULL
> sigma2<-NULL
> s1<-NULL
> s2<-NULL
> sD<-sqrt(339843)/210
> desv.iguales<-NULL
> alfa<-NULL
> cola<-'I'
> par<-TRUE
 \end{verbatim}
 \vspace{-0.5cm}
 el programa de R lanza el siguiente resultado:
 \begin{verbatim}
  Prueba              Tipo  H0 n MediaPareada desv.par error.est
1      t Muestras pareadas 4.5 7     3.557143 2.776003   1.04923
  grados.libertad alpha    PValor Estadistico RegionRechazoT RegionRechazoX
1               6  0.05 0.2017374  -0.8986179   < -1.9431803    < 2.4611562
 \end{verbatim}
 \vspace{-0.5cm}
 El cual coincide con los datos obtenidos,
 que es a lo que se quer\'{\i}a llegar.${}_{\blacksquare}$
\end{solucion}
