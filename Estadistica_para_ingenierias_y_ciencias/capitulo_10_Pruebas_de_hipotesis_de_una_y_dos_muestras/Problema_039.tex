\begin{enunciado}
 Los siguientes datos representan los tiempos de duraci\'on de pel\'{\i}culas producidas por 2 compa\~n\'{\i}as cinematogr\'aficas:
 \begin{center}
  \begin{tabular}{crrrrrrr}
   \textbf{Compa\~n\'{\i}a} & \multicolumn{7}{c}{\textbf{Tiempo (minutos)}} \\
   \hline 
   1 & $102$ & $86$ & $98$ & $109$ & $92$ & & \\
   2 & $81$ & $165$ & $97$ & $134$ & $92$ & $87$ & $114$
  \end{tabular}
 \end{center}
 Pruebe la hip\'otesis de que el tiempo de duraci\'on promedio de las pel\'{\i}culas producidas por la compa\~n\'{\i}a 2 excede el tiempo promedio de duraci\'on de las que produce la compa\~n\'{\i}a 1 en $10$ minutos, contra la alternativa unilateral de que la diferencia es de menos de $10$ minutos. Utilice un nivel de significancia de $0.1$ y suponga que las distribuciones de los tiempos son aproximadamente normales con varianzas iguales.
\end{enunciado}

\begin{solucion}
 \begin{datos}
  Resumido, se tiene que
  \begin{itemize}
   \item $X_i \sim n\left( \mu_i, \sigma_i \right)$,
   para cada $i \in \{ 1,2\}$.
   \item $n_1 = 5$ y $n_2 = 7$.
   \item $\sigma_1^2 = \sigma_2^2$.
  \end{itemize}
  Para obtener las medias y desviaciones est\'andar muestrales,
  se calcula lo siguiente:
  \begin{eqnarray*}
   \sum_{i=1}^5 x_{1,i} & = & 102 + 86 + 98 + 109 + 92 = 487 \\
   \sum_{i=1}^5 x_{1,i}^2 & = & 102^2 + 86^2 + 98^2 + 109^2 + 92^2
   = 47\,749 \\
   \sum_{i=1}^7 x_{2,i} & = & 81 + 165 + 97 + 134 + 92 + 87 + 114
   = 770 \\
   \sum_{i=1}^7 x_{2,i}^2 & = & 
   81^2 + 165^2 + 97^2 + 134^2 + 92^2 + 87^2 + 114^2 = 90\,180
  \end{eqnarray*}
  Por lo que el valor de cada media muestral es:
  \begin{eqnarray*}
   \bar{x}_1 & = & \frac{1}{5}\sum_{i=1}^5 x_{1,i} = \frac{487}{5}
   = 97.4 \\
   \bar{x}_2 & = & \frac{1}{7}\sum_{i=1}^7 x_{2,i} = \frac{770}{7}
   = 110
  \end{eqnarray*}
  y las varianzas muestrales se calculan, usando el teorema 8.1,
  como sigue:
  \begin{eqnarray*}
   s_1^2 & = &
   \frac{1}{5(4)}
   \left[
   5\sum_{i=1}^5 x_{1,i}^2 - \left( \sum_{i=1}^5 x_{1,i} \right)^2 \right]
   = \frac{5(47\,749) - 487^2}{20} = \frac{238\,745-237\,169}{20}
   = \frac{1\,576}{20} \\
   & = & \frac{394}{5} = 78.8 \\
   s_2^2 & = &
   \frac{1}{7(6)}
   \left[
   7\sum_{i=1}^7 x_{2,i}^2 - \left( \sum_{i=1}^7 x_{2,i} \right)^2
   \right]
   = \frac{7(90\,180) - 770^2}{42} = \frac{631\,260 - 592\,900}{42}
   = \frac{38\,360}{42} \\
   & = & \frac{2\,740}{3} = 913.\overline{3}
  \end{eqnarray*}
  por lo que el valor de cada desviaci\'on est\'andar muestral es:
  \begin{eqnarray*}
   s_1 & = & \sqrt{s_1^2} = \sqrt{\frac{394}{5}} =
   \frac{\sqrt{1\,970}}{5} \approx 8.8769364 \\
   s_2 & = & \sqrt{s_2^2} = \sqrt{\frac{2\,740}{3}}
   = \frac{2\sqrt{685}\sqrt{3}}{3} = \frac{2\sqrt{2\,055}}{3}
   \approx 30.2214052
  \end{eqnarray*}
  Por lo tanto, se resume el resto de los datos como sigue:
  \begin{itemize}
   \item $\bar{x}_1 = \frac{487}{5} = 97.4$ y $\bar{x}_2 = 110$.
   \item $s_1 = \frac{\sqrt{1\,970}}{5} \approx 8.8769364$
   y $s_2 = \frac{2\sqrt{2\,055}}{3} \approx 30.2214052$.
  \end{itemize}
 \end{datos}

 \begin{hipotesis}
  \begin{eqnarray*}
   H_0: \mu_1 - \mu_2 & \leq & -10 \\
   H_1: \mu_1 - \mu_2 &   >  & -10
  \end{eqnarray*}
 \end{hipotesis}

 \begin{significancia}
  $\alpha = 0.1$.
 \end{significancia}

 \begin{region}
  De la tabla A.4, se tiene el valor cr\'{\i}tico
  $t_{\alpha, n_1+n_2-2} = t_{0.1,10} \approx 1.372$,
  por lo que la regi\'on de rechazo est\'a dado para $t > 1.372$,
  donde $t =
  \frac{
  \left( \bar{x}_1 - \bar{x}_2 \right) - d_0
  }{
  s_p\sqrt{1/n_1 + 1/n_2}
  }$.
 \end{region}

 \begin{estadistico}
  Dado que
  \begin{eqnarray*}
   s_p^2 & = & \frac{
   s_1^2 \left( n_1 - 1 \right) + s_2^2 \left( n_2 - 1\right)
   }{
   n_1 + n_2 -2
   }
   = \frac{\frac{394}{5}(5-1) + \frac{2\,740}{3}(7-1)}{10}
   = \frac{\frac{394(4) + 2\,740(2)(5)}{5}}{10} \\
   & = & \frac{1\,576 + 27\,400}{50} = \frac{28\,976}{50}
   = \frac{14\,488}{25} = 579.52
  \end{eqnarray*}
  entonces
  \begin{equation*}
   s_p = \sqrt{s_p^2} = \sqrt{\frac{14\,488}{25}}
   = \frac{2\sqrt{3\,622}}{5} \approx 24.0732216373
  \end{equation*}
  y
  \begin{eqnarray*}
   t & = & 
   \frac{
   \left( \bar{x}_1 - \bar{x}_2 \right) - d_0
   }{
   s_p\sqrt{\frac{1}{n_1} + \frac{1}{n_2}}
   }
   = \frac{
   (97.4 - 110) - (-10)
   }{
   \frac{2\sqrt{3\,622}}{5}\sqrt{\frac{1}{5} + \frac{1}{7}}
   }
   = \frac{
   (-12.6 + 10)\left(5\sqrt{3\,622}\right)
   }{
   2(3\,622)\sqrt{\frac{7+5}{35}}
   }
   = \frac{-13\sqrt{3\,622}\sqrt{35}}{7\,244\sqrt{12}} \\
   & = & -\frac{13\sqrt{126\,770}\sqrt{3}}{7\,244(2)(3)}
   = -\frac{13\sqrt{380\,310}}{43\,464}
   \approx -0.18445164501914
  \end{eqnarray*}
 \end{estadistico}

 \begin{decision}
  No se rechaza $H_0$.
 \end{decision}

 \begin{conclusion}
  No hay evidencia suficiente para creer
  que el tiempo promedio de duraci\'on de las pel\'{\i}culas producidas
  por la compa\~n\'{\i}a 2 es menor al tiempo promedio de duraci\'on
  de las que produce la compa\~n\'{\i}a 1 en 10 minutos.
  \par 
  N\'otese que desde que la media muestral del tiempo de duraci\'on
  de las pel\'{\i}culas producidas por la compa\~n\'{\i}a 2 fue mayor
  a la media muestral del tiempo de duraci\'on de las que produce
  la compa\~n\'{\i}a 1 en 10 minutos, entonces no se hubiese
  podido rechazar la hip\'otesis nula,
  ya que el signo del estad\'{\i}stico quedaba opuesto al signo del
  valor cr\'{\i}tico de la regi\'on de rechazo;
  sin embargo, a\'un cambiando el sentido de desigualdad
  de las hip\'otesis, tampoco se podr\'{\i}a rechazar la hip\'otesis
  nula de que el tiempo promedio de duraci\'on de las pel\'{\i}culas
  producidas por la compa\~{\i}a 2 sea menor al tiempo promedio
  de duraci\'on de las producidas por la compa\~{\i}a 1
  en 10 minutos,
  puesto que, como se observan en las cuentas, el valor absoluto
  del estad\'{\i}stico sigue siendo menor al del valor cr\'{\i}tico.
  \par 
  Por lo tanto, se tiene que la prueba no es concluyente.
 \end{conclusion}
 Finalmente, usando el archivo anexo
 \texttt{P06\_Prueba\_de\_dos\_medias\_02.r},
 que a su vez requiere los datos del archivo
 \texttt{DB03\_Problema\_039.csv},
 con los siguientes cambios:
 \begin{verbatim}
> datos<-read.csv("DB03_Problema_039.csv",sep=";",encoding="UTF-8")
> varInteres<-c("Tiempo.min")
> varSel<-c("Compañía")
> mu<--10
> desv.iguales<-TRUE
> alfa<-0.1
> cola<-'S'
> par<-FALSE
 \end{verbatim}
 \vspace{-0.5cm}
 el programa de R lanza el siguiente resultado:
 \begin{verbatim}
        Var1 Freq    Poblaciones  H0  valorPVar     suposicionVar n1 n2 media1
1 Tiempo.min   12 Independientes -10 0.03297718 Var no diferentes  5  7   97.4
  media2 diferencia desv.est1 desv.est2   est.sp error.est grados alpha
1    110      -12.6  8.876936  30.22141 24.07322  14.09584     10   0.1
    PValor Estadistico RegionRechazoSupT RegionRechazoSupX        Resultado
1 0.571327  -0.1844516          1.372184          9.342074 No se rechaza H0
 \end{verbatim}
 \vspace{-0.5cm}
 El cual coincide con los datos obtenidos.
 \par
 N\'otese que a\'un cambiando la cola de la prueba,
 esto es, cambiando la siguiente l\'{\i}nea:
 \begin{verbatim}
> cola<-'I'
 \end{verbatim}
 \vspace{-0.5cm}
 el programa de R lanza el nuevo resultado:
 \begin{verbatim}
        Var1 Freq    Poblaciones  H0  valorPVar     suposicionVar n1 n2 media1
1 Tiempo.min   12 Independientes -10 0.03297718 Var no diferentes  5  7   97.4
  media2 diferencia desv.est1 desv.est2   est.sp error.est grados alpha
1    110      -12.6  8.876936  30.22141 24.07322  14.09584     10   0.1
    PValor Estadistico RegionRechazoInfT RegionRechazoInfX        Resultado
1 0.428673  -0.1844516         -1.372184         -29.34207 No se rechaza H0
 \end{verbatim}
 \vspace{-0.5cm}
 Lo cual ofrece m\'as detalle de la menci\'on final
 en la conclusi\'on, reforzando lo ya dicho,
 que es a lo que se quer\'{\i}a llegar.${}_{\blacksquare}$
\end{solucion}
