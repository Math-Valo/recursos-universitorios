\begin{enunciado}
 ¿Qu\'e tan grandes deber\'{\i}an ser las muestras del ejercicio 10.31, si la potencia de nuestra prueba debe ser $0.95$ cuando la diferencia real entre los tipos de hilo $A$ y $B$ es $8$ kilogramos?
\end{enunciado}

\begin{solucion}
 Usando los datos del ejercicios 10.31
 y la informaci\'on del enunciado,
 se tienen los siguientes datos y supuestos:
 \begin{itemize}
  \item $\alpha = 0.05$.
  \item $1-\beta = 0.95$, entonces $\beta = 0.05$.
  \item $\mu_1 - \mu_2 \geq 12$, seg\'un la hip\'otesis nula.
  \item $\mu_1 - \mu_2 = 8$, seg\'un una alternativa espec\'{\'i}fica,
  entonces $\delta = -4$.
%   \item $\sigma_1$ y $\sigma_2$
 \end{itemize}
 Los c\'alculos requeridos para obtener los tama\~nos de muestra requieren,
 en este caso, de conocer las varianzas poblacionales o, alternativamente,
 de suponer que las desviaciones est\'andar poblacionales son iguales
 y conocer o suponer el valor proporcional de $\delta$
 con respecto al supuesto valor al que son iguales las desviaciones,
 esto es $\Delta = \frac{|\delta|}{\sigma}$, para apoyarse de la tabla A.9.
 Sin emabrgo, se desconocen las varianzas y no hay manera de suponer que son iguales,
 adem\'as de que no se enuncia alg\'un supuesto al respecto de la proporci\'on
 mencionada, por lo que no se puede proceder a implementar el c\'alculo con el apoyo de la tabla A.9.
 Entonces, el camino por el que se proceder\'a para calcular lo solicitado
 es usar estimados de los valores de $\sigma_1$ y $\sigma_2$,
 us\'andolos, no como estimados, sino como si fueran los valores reales.
 \par
 En el ejercicio 10.31 se tienen los valores de $s_1$ y $s_2$, que pueden ser usados
 como aproximaciones de $\sigma_1$ y $\sigma_2$, respectivamente,
 cuando el tama\~no de las muestras es al menos de $30$ observaciones.
 Por el momento, estas ser\'an las aproximaciones de $\sigma_1$ y $\sigma_2$ usadas,
 bajo la precauci\'on de que el resultado ser\'a v\'alido
 s\'olo en caso de que se obtenga un tama\~no de muestras de al menos de 30
 observaciones. Entonces
 \begin{itemize}
  \item $\sigma_1 \approx s_1 = 6.28$ y $\sigma_2 \approx s_2 = 5.61$
 \end{itemize}
 Adem\'as, de la tabla A.3, se tiene que
 \begin{itemize}
  \item $z_{\alpha} = 1.645$.
  \item $z_{\beta} = 1.645$.
 \end{itemize}
 Entonces se puede calcular el tama\~no de muestra
 requerido para la calidad buscada en la prueba
 con los siguientes c\'alculos
 \begin{eqnarray*}
  n (=n_1=n_2) & = & \left\lceil \frac{\left( z_{\alpha} + z_{\beta} \right)^2\left( \sigma_1 + \sigma_2 \right)^2}{\delta^2} \right\rceil = \left\lceil \frac{(1.645 + 1.645)^2(6.28 + 5.61)^2}{(-4)^2} \right\rceil \\
  & = & \left\lceil \frac{3.29^2\times 11.89^2}{16} \right\rceil 
  = \left\lceil \frac{ 10.8241 \times 141.3721}{16} \right\rceil
  = \left\lceil \frac{1\,530.22574761}{16} \right\rceil \\
  & = & \lceil 95.6391 \rceil = 96
 \end{eqnarray*}
 Por lo tanto, y como no se contradice la condici\'on para usar $s_1$ y $s_2$,
 se concluye que, a partir de un tama\~no muestral de $n_1 = n_2 = 96$,
 se puede realizar una prueba de hip\'otesis de que
 la resistencia a la tensi\'on promedio del hilo A excede la resistencia
 a la tensi\'on promedio del hilo B en al menos 12 kilogramos,
 esto es $\mu_1 - \mu_2 \geq 12$, con un nivel de significancia de $0.05$
 considerando una potencia de prueba de $0.95$ para la hip\'otesis alternativa
 de que dicha diferencia es $\mu_1 - \mu_2 = 8$,
 es decir con una diferencia entre hip\'otesis de $|\delta| = 4$.${}_{\blacksquare}$
\end{solucion}
