\begin{enunciado}
 ¿Qu\'e tan grande se requiere que sea la muestra del ejercicio 10.24 si la potencia de nuestra prueba ser\'a $0.8$ cuando el tiempo medio real de meditaci\'on exceda el valor hipot\'etico en $1.2\,\sigma$? Utilice $\alpha = 0.05$.
\end{enunciado}

\begin{solucion}
 Usando los datos del ejercicio 10.24
 y la informaci\'on del enunciado,
 se tienen los siguientes datos y supuestos:
 \begin{itemize}
  \item $\alpha = 0.05$.
  \item $1 - \beta = 0.8$, entonces $\beta = 0.2$.
  \item $\mu = 8$, seg\'un la hip\'otesis nula.
  \item $\delta = 1.2\sigma$,
  donde $\mu + \delta$ es una alternativa espec\'{\i}fica.
 \end{itemize}
%  Entonces, de la tabla A.3, se tiene que
%  \begin{itemize}
%   \item $z_{\alpha} = 1.645$.
%   \item $z_{\beta} = 1.645$.
%  \end{itemize}
 Entonces se puede calcular el tama\~no de muestra
 requerido para la calidad buscada en la prueba
 usando la tabla A.8.
 As\'{\i}, pues, se busca en la tabla la columna
 que corresponde el valor de $\alpha=0.05$ para pruebas unilaterales,
 con $\beta = 0.2$, cruzando en la fila
 con el valor de $\Delta = \frac{|\delta|}{\sigma} = \frac{1.2\sigma}{\sigma} = 1.2$,
 con lo que se llega al valor $n = 8$.
 \par 
 Por lo tanto, a partir de un tama\~no muestral de $n = 8$ se puede realizar una prueba
 de hip\'otesis de que, de las personas que usan la meditaci\'on trascendental,
 la cantidad de horas promedio que meditan es de no m\'as de 8 horas a la semana,
 esto es $\mu \leq 8$, con un nivel de significancia de $0.05$
 considerando una potencia de prueba de $0.8$ para una hip\'otesis alternativa
 de que el promedio real de horas de meditaci\'on para estas personas
 en realidad difiere de la hip\'otesis nula en un valor de $\delta = 1.2\sigma$,
 es decir contra la hip\'otesis alternativa
 de que $\mu = 8 + 1.2\sigma$.${}_{\blacksquare}$
\end{solucion}
