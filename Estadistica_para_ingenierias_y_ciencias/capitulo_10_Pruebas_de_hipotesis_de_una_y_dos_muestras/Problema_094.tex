\begin{enunciado}
 El hospital de una universidad realiz\'o un experimento para determinar el grado de alivio que brindan tres remedios para la tos.
 Cada medicamento para la tos se trata en $50$ estudiantes y se registran los sigientes datos:
 \begin{center}
  \begin{tabular}{lccc}
   & \multicolumn{3}{c}{\textbf{Remedio para la tos}} \\
   & \textbf{NyQuil} & \textbf{Robitussin} & \textbf{Triaminic} \\
   \hline 
   Sin alivio & $11$ & $13$ & $9$ \\
   Cierto alivio & $32$ & $28$ & $27$ \\
   Alivio completo & $7$ & $9$ & $14$
  \end{tabular}
 \end{center}
 Con un nivel de significancia de $0.05$, pruebe la hip\'otesis de que los tres remedios para la tos son igualmente efectivos.
\end{enunciado}

\begin{solucion}
 \begin{datos}
  $\phantom{0}$
  \begin{itemize}
   \item Tamaño de muestra total: $150$.
   \item Estudiantes medicados por NyQuil: $50$.
   \item Estudiantes medicados por Robitussin: $50$.
   \item Estudiantes medicados por Triaminic: $50$.
   \item Estudiantes que no sintieron alivio: $11 + 13 + 9 = 33$.
   \item Estudiantes que sintieron cierto alivio: $32 + 28 + 27 = 87$.
   \item Estudiantes completamente aliviados: $7 + 9 + 14 = 30$.
   \item Frecuencias observadas y esperadas: $o_{i,j}$
   y $e_{i,j}=\frac{R_i C_j}{n}$, respectivamente,
   donde $R_i$ y $C_j$ son los marginales del rengl\'on $i$ y la columna $j$,
   respectivamente, y $n$ es el total de toda la muestra.
   As\'{\i}, pues, redondeando a un decimal, se muestra el resumen 
   en la siguiente tabla,
   en donde aparece entre par\'entesis la frecuencia esperada
   y a la izquierda el valor observado:
   \begin{center}
    \begin{tabular}{lccc|c}
     & \multicolumn{3}{c}{\textbf{Remedio para la tos}} &
     \textbf{Marginal por} \\
     \cline{2-4}
     & \textbf{NyQuil} & \textbf{Robitussin} & \textbf{Triaminic} &
     \textbf{efectividad} \\
     \hline 
     Sin alivio & $11 (11)$ & $13 (11)$ & $9 (11)$ & $33$ \\
     Cierto alivio & $32 (29)$ & $28 (29)$ & $27 (29)$ & $87$ \\
     Alivio completo & $7 (10)$ & $9 (10)$ & $14 (10)$ & $30$ \\
     \hline 
     \textbf{Marginal por} & \multirow{2}{*}{$50$} & \multirow{2}{*}{$50$} &
     \multirow{2}{*}{$50$} & \textbf{TOTAL} \\
     \textbf{medicamento} & & & & $n = 150$
    \end{tabular}
   \end{center}
   \item Tama\~no de la tabla de contingencia: $r\times c = 3\times 3$.
   \item Grados de libertad de la prueba $\chi^2$: $v = (r-1)(c-1) = 4$.
  \end{itemize}
 \end{datos}
 
 \begin{hipotesis}
  \begin{eqnarray*}
   H_0: & & \text{La efectividad de los medicamentos son homog\'eneos.} \\
   H_1: & & \text{La efectividad de los medicamentos no son homog\'eneos.}
  \end{eqnarray*}
 \end{hipotesis}

 \begin{significancia}
  $\alpha = 0.05$.
 \end{significancia}

 \begin{region}
  De la tabla A.5, se tiene el valor cr\'{\i}tico
  $\chi^2_{\alpha,v} = \chi^2_{0.05,4} \approx 9.488$,
  por lo que la regi\'on de rechazo est\'a dado
  para $\chi^2 > 9.488$, donde
  $\chi^2 = \sum_{i} \frac{\left( o_i - e_i \right)^2}{e_i}$.
 \end{region}

 \begin{estadistico}
  \begin{eqnarray*}
   \chi^2 & = & \sum_{i} \frac{\left( o_i - e_i \right)^2}{e_i} \\
   & \approx & \frac{(11 - 11)^2}{11} + \frac{(13 - 11)^2}{11} +
   \frac{(9 - 11)^2}{11} + \frac{(32 - 29)^2}{29} + \frac{(28 - 29)^2}{29} \\
   & & \frac{(27 - 29)^2}{29} + \frac{(7 - 10)^2}{10} + 
   \frac{(9 - 10)^2}{10} + \frac{(14 - 10)^2}{10} \\
   & = & \frac{0 + 4 + 4}{11} + \frac{9 + 1 + 4}{29} + 
   \frac{9 + 1+ 16}{10} = \frac{8}{11} + \frac{14}{29} + \frac{26}{10}
   \approx 0.72727 + 0.48276 + 2.6 \\
   & = & 3.81003
  \end{eqnarray*}
 \end{estadistico}

 \begin{decision}
  No se rechaza $H_0$.
 \end{decision}

 \begin{conclusion}
  No hay evidencia para que indique que los medicamentos no son homog\'eneos
  y, por lo tanto, se puede considerar que tienen la misma efectividad.
 \end{conclusion}

 Finalmente, usando el archivo anexo
 \texttt{P18\_Prueba\_de\_independencia\_y\_homogeniedad\_01.r},
 que a su vez requiere los datos del archivo
 \texttt{BD30\_Problema\_094.csv}, con los siguientes cambios:
 \begin{verbatim}
> datos<-read.csv("DB30_Problema_094.csv",sep=";",encoding="UTF-8")
> varInteres<-c("Medicamento","Efectividad")
> varFrecuencia<-"Frecuencia"
> pruebas<-c(1,2,3)
 \end{verbatim}
 \vspace{-0.5cm}
 el programa de R lanza el siguiente resultado:
 \begin{verbatim}
$tabla
            Efectividad
Medicamento  Alivio completo Cierto alivio Sin alivio
  NyQuil                   7            32         11
  Robitussin               9            28         13
  Triaminic               14            27          9

$listaPruebas
$listaPruebas[[1]]

	Pearson's Chi-squared test

data:  tbl1
X-squared = 3.81, df = 4, p-value = 0.4323


$listaPruebas[[2]]

	Log likelihood ratio (G-test) test of independence without correction

data:  tbl1
Log likelihood ratio statistic (G) = 3.7389, X-squared df = 4, p-value =
0.4425


$listaPruebas[[3]]

	Log likelihood ratio (G-test) test of independence with Williams'
	correction

data:  tbl1
Log likelihood ratio statistic (G) = 3.6555, X-squared df = 4, p-value =
0.4546
 \end{verbatim}
 \vspace{-0.5cm}
 Lo cual coincide con los resultados obtenidos, adem\'as de brindar m\'as
 informaci\'on y los valores $P$ junto con otros estad\'{\i}sticos,
 que es lo que se quer\'{\i}a llegar.${}_{\blacksquare}$
\end{solucion}
