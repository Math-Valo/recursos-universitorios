\begin{enunciado}
 Repita el ejercicio 10.10 cuando se administra el extracto de mejill\'on a $70$ pacientes y la regi\'on cr\'{\i}tica se define como $x < 24$, donde $x$ es el n\'umero de pacientes con artritis \'osea que obtienen alivio.
\end{enunciado}

\begin{solucion}
 Usando los t\'erminos del ejercicio 10.10, se tiene ahora que:
 \begin{itemize}
  \item $X \sim b(n,p) \sim n\left( \mu = np, \sigma = \sqrt{npq} \right)$.
  \item $n = 70$.
  \item $x_{\text{inf}} = 24$.
 \end{itemize}
 El resultado se calcular\'a a trav\'es de la aproximaci\'on de la binomial a la normal. El valor preciso binomial es poco pr\'actico obtenerlo, por lo que no se precisar\'a. Finalmente, se realizar\'a las comparaciones con los c\'alculos binomiales y normales en R.
 \par 
 Para el primer supuesto se tiene adem\'as que
 \begin{itemize}
  \item $p = 0.4$.
  \item $\mu = np = (70)(0.4) = 28$.
  \item $\sigma^2 = npq = 28(0.6) = 16.8 = 84/5$.
 \end{itemize}
 as\'{\i}, el error tipo I se aproxima usando la tabla A.3 como sigue:
 \begin{eqnarray*}
  \alpha & = & P(X < 24) \sim P\left( Z < \frac{23.5 - 28}{\sqrt{84/5}} \right) = P\left( Z < -\frac{4.5\sqrt{105}}{42} \right) = P\left( Z < -\frac{9\sqrt{105}}{84} \right) \\
  & = & P\left( Z < -\frac{3\sqrt{105}}{28} \right) \approx P(Z < -1.1) = 0.1357
 \end{eqnarray*}
 Para el siguiente supuesto, se tiene que
 \begin{itemize}
  \item $p = 0.3$.
  \item $\mu = np = (70)(0.3) = 21$.
  \item $\sigma^2 = npq = 21(0.7) = 14.7 = 147/10$.
 \end{itemize}
 as\'{\i}, el error tipo II se aproxima usando la tabla A.3 como sigue:
 \begin{eqnarray*}
  \beta & = & P(X \geq 24) = 1 - P(X < 24) \approx 1 - P\left( Z < \frac{23.5 - 21}{\sqrt{147/10}} \right) \\
  & = & 1 - P\left( Z < \frac{2.5\sqrt{30}}{21} \right) = 1 - P\left( Z < \frac{5\sqrt{30}}{42} \right) \approx 1 - P(Z < 0.65) \approx 1 - 0.7422 = 0.2578
 \end{eqnarray*}
 Finalmente, usando R, se calcula primero estas probabilidades con el script del archivo anexo \texttt{P01\_Probabilidad\_de\_error\_binomial\_1.r}, cambiando las siguientes l\'{\i}neas de c\'odigo:
 \begin{verbatim}
> n<-70
> CriticoInf<-24
> CriticoSup<-NULL
> p0<-0.4
> p1<-0.3
 \end{verbatim}
 \vspace{-0.5cm}
 con lo que se obtiene
 \begin{verbatim}
$`Probabilidad de error tipo I`
  HipotesisNula  n CriticoInf     alpha
1           0.4 70         24 0.1356779

$`Probabilidad de error tipo II`
  HipotesisAlternativa  n CriticoInf      beta
1                  0.3 70         24 0.2540753
 \end{verbatim}
 \vspace{-0.5cm}
 mientras que con el script del archivo anexo \texttt{P02\_Probabilidad\_de\_error\_normal\_1.r}, cambiando las siguientes l\'{\i}neas de c\'odigo:
 \begin{verbatim}
> n<-70
> CriticoInf<-24
> CriticoSup<-NULL
> desv<-NULL
> media0<-NULL
> media1<-NULL
> p0<-0.4
> p1<-0.3
 \end{verbatim}
 \vspace{-0.5cm}
 se obtiene lo siguiente:
 \begin{verbatim}
$`Probabilidad de error tipo I`
  HipotesisNula  n media    desv CriticoInf     alpha
1      p =  0.4 70    28 4.09878       23.5 0.1361268

$`Probabilidad de error tipo II`
  HipotesisAlternativa  n media     desv CriticoInf      beta
1             p =  0.3 70    21 3.834058       23.5 0.2571842
 \end{verbatim}
 \vspace{-0.5cm}
 Por lo tanto, se tiene el siguiente resumen:
 \begin{itemize}
  \item Bajo el supuesto $p = 0.4$:
  \begin{itemize}
   \item La aproximaci\'on con las tablas normales da $\alpha = 0.1357$.
   \item La  aproximaci\'on binomial con R da $\alpha = 0.1356779$.
   \item La aproximaci\'on normal con R da $\alpha = 0.1361268$.
  \end{itemize}

  \item Y, bajo el supuesto $p = 0.3$:
  \begin{itemize}
   \item La aproximaci\'on con las tablas normales da $\beta = 0.2578$.
   \item La aproximaci\'on binomial con R da $\beta = 0.2540753$.
   \item La aproximaci\'on normal con R da $\beta = 0.2571842$.
  \end{itemize}
 \end{itemize}
 que es a lo que se quer\'{\i}a llegar.${}_{\blacksquare}$
\end{solucion}
