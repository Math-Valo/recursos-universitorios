\begin{enunciado}
 Se dice que una m\'aquina despachadora de bebida gaseosa est\'a fuera de control
 si la varianza de los contenidos excede $1.15$ decilitros.
 Si una muestra aleatoria de $25$ bebidas de esta m\'aquina tiene una varianza
 de $2.03$ decilitros, ¿esto significa con un nivel de significancia de $0.05$
 que la m\'aquina est\'a fuera de control?
 Suponga que los contenidos se distribuyen de forma aproximadamente normal.
\end{enunciado}

\begin{solucion}
 \begin{datos}
  $\phantom{0}$
  \begin{itemize}
   \item $X \sim n\left( \mu, \sigma \right)$.
   \item $n = 25$.
   \item $s^2 = 2.03$.
  \end{itemize}
 \end{datos}

 \begin{hipotesis}
  \begin{eqnarray*}
   H_0: \sigma^2 & = & 1.15 \\
   H_1: \sigma^2 & > & 1.15
  \end{eqnarray*}
 \end{hipotesis}

 \begin{significancia}
  $\alpha = 0.05$.
 \end{significancia}

 \begin{region}
  De la tabla A.5, se tiene el valor cr\'{\i}tico
  $\chi^2_{\alpha,n-1}=\chi^2_{0.05,24} \approx 36.415$,
  por lo que la regi\'on de rechazo est\'a dado para $\chi^2 > 36.415$,
  donde $\chi^2 = \frac{(n-1)s^2}{\sigma_0^2}$.
 \end{region}

 \begin{estadistico}
  \begin{equation*}
   \chi^2 = \frac{(n-1)s^2}{\sigma_0^2}
   = \frac{24(2.03)}{1.15}
   = \frac{24\left(\frac{203}{100} \right)}{\frac{23}{20}}
   = \frac{24(203)}{23(5)}
   = \frac{4\,872}{115} \approx 42.3\overline{6521739130434782608695}
  \end{equation*}
 \end{estadistico}

 \begin{decision}
  Se rechaza $H_0$ a favor de $H_1$.
 \end{decision}

 \begin{conclusion}
  Se concluye que, en efecto, m\'aquina tienen una varianza mayor
  a $1.15$ y, por lo tanto, est\'a fuera de control.
 \end{conclusion}
 
 Finalmente, usando el archivo anexo \texttt{P11\_Prueba\_de\_una\_varianza\_02.r}, con los siguientes cambios:
 \begin{verbatim}
> n<-25
> sigma2<-1.15
> s2<-2.03
> alfa<-0.05
> cola<-'S'
 \end{verbatim}
 \vspace{-0.5cm}
 el programa de R lanza el siguiente resultado:
 \begin{verbatim}
   n   H0 var.muestral grados error.est alpha    PValor Estadistico
1 25 1.15         2.03     24 0.0479167  0.05 0.0117412    42.36522
  RegionRechazoJi RegionRechazoX     Resultado
1    > 36.4150285    > 1.7448868 Se rechaza H0
 \end{verbatim}
 \vspace{-0.5cm}
 El cual coincide con los resultados obtenidos,
 que es a lo que se quer\'{\i}a llegar.${}_{\blacksquare}$
\end{solucion}
