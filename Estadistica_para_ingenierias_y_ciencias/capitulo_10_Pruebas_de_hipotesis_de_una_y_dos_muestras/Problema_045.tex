\begin{enunciado}
 De acuerdo con informes publicados, el ejercicio bajo condiciones de fatiga altera los mecanismos que determinan el desempe\~no. Se realiz\'o un experimento donde se usaron $15$ estudiantes universitarios hombres, entrenados para realizar un movimiento horizontal continuo del brazo, de derecha a izquierda, desde un microinterruptor hasta una barrera, golpeando sobre la barrera en coincidencia con la llegada de una manecilla del reloj a la posici\'on de las 6 en punto. Se registra el valor absoluto de la diferencia entre el tiempo, en milisegundos, que toma golpear sobre la barrera y el tiempo para que la manecilla alcance la posici\'on de las 6 en punto (500 mseg). Cada participante ejecuta la tarea cinco veces en condiciones sin fatiga y con fatiga, y se registraron las sumas de las diferencias absolutas para las cinco ejecuciones como sigue:
 \begin{center}
  \begin{tabular}{ccc}
   & \multicolumn{2}{c}{\textbf{Diferencias absolutas de tiempo}} \\
   \hline 
   \textbf{Sujeto} & $\phantom{000}$\textbf{Sin fatiga}$\phantom{000}$ & $\phantom{000}$\textbf{Con fatiga}$\phantom{000}$ \\
   \hline 
   $\phantom{1}1$ & $158$ & $\phantom{1}91$ \\
   $\phantom{1}2$ & $\phantom{1}92$ & $\phantom{1}59$ \\
   $\phantom{1}3$ &  $\phantom{1}65$ & $215$ \\
   $\phantom{1}4$ & $\phantom{1}98$ & $226$ \\
   $\phantom{1}5$ &  $\phantom{1}33$ & $223$ \\
   $\phantom{1}6$ & $\phantom{1}89$ & $\phantom{1}91$ \\
   $\phantom{1}7$ & $148$ & $\phantom{1}92$ \\
   $\phantom{1}8$ & $\phantom{1}58$ & $177$ \\
   $\phantom{1}9$ & $142$ & $134$ \\
   $10$ & $117$ & $116$ \\
   $11$ & $\phantom{1}74$ & $153$ \\
   $12$ & $\phantom{1}66$ & $219$ \\
   $13$ & $109$ & $143$ \\
   $14$ & $\phantom{1}57$ & $164$ \\
   $15$ & $\phantom{1}85$ & $100$
  \end{tabular}
 \end{center}
 Un aumento en las diferencias medias absolutas de tiempo cuando la tarea se ejecuta bajo condiciones de fatiga apoyar\'{\i}a la afirmaci\'on de que el ejercicio, en condiciones de fatiga, altera el mecanismo que determina el desempe\~no. Suponiendo de que las poblaciones se distribuyen normalmente, pruebe tal afirmaci\'on.
\end{enunciado}

\begin{solucion}
 \begin{datos}
  Resumido, se tiene que
  \begin{itemize}
   \item $X_i \sim n\left( \mu_i, \sigma_i \right)$,
   para cada $i \in \{ 1, 2 \}$.
   \item $n_1 = n_2 = 15$.
  \end{itemize}
  Como las observaciones fueron sobre las mismas unidades
  experimentales, entonces se trata de observaciones pareadas,
  por lo que se usar\'an las diferencias de los datos,
  que muestra la tabla siguiente:
  \begin{center}
   \begin{tabular}{cccc}
    & \multicolumn{2}{c}{\textbf{Diferencias absolutas de tiempo}}
    & \\
    \hline 
    \textbf{Sujeto} & $\phantom{000}$\textbf{Sin fatiga}$
    \phantom{000}$ & $\phantom{000}$\textbf{Con fatiga}$
    \phantom{000}$ & $\phantom{00000}d_i\phantom{00000}$ \\
    \hline 
    $\phantom{1}1$ & $158$ & $\phantom{1}91$ &
    $\phantom{-1}67$ \\
    $\phantom{1}2$ & $\phantom{1}92$ & $\phantom{1}59$ &
    $\phantom{-1}33$ \\
    $\phantom{1}3$ &  $\phantom{1}65$ & $215$ & $-150$  \\
    $\phantom{1}4$ & $\phantom{1}98$ & $226$ & $-128$ \\
    $\phantom{1}5$ &  $\phantom{1}33$ & $223$ & $-190$ \\
    $\phantom{1}6$ & $\phantom{1}89$ & $\phantom{1}91$ &
    $-\phantom{15}2$ \\
    $\phantom{1}7$ & $148$ & $\phantom{1}92$ & $\phantom{-1}56$ \\
    $\phantom{1}8$ & $\phantom{1}58$ & $177$ & $-119$ \\
    $\phantom{1}9$ & $142$ & $134$ & $\phantom{-15}8$ \\
    $10$ & $117$ & $116$ & $\phantom{-15}1$ \\
    $11$ & $\phantom{1}74$ & $153$ & $-\phantom{1}79$ \\
    $12$ & $\phantom{1}66$ & $219$ & $-153$ \\
    $13$ & $109$ & $143$ & $-\phantom{1}34$ \\
    $14$ & $\phantom{1}57$ & $164$ & $-107$ \\
    $15$ & $\phantom{1}85$ & $100$ & $-\phantom{5}15$
   \end{tabular}
  \end{center}
  Para obtener la media y desviaci\'on est\'andar 
  de las diferencias muestrales, se calcula lo siguiente:
  \begin{eqnarray*}
   \sum_{i=1}^{15} d_{i} & = &
   67 + 33 -150 -128 -190 -2 +56 -119 + 8 + 1 -79 -153 -34 -107 -15
   \\
   & = & -812 \\
   \sum_{i=1}^{15} d_{i}^2 & = &
   67^2 + 33^2 + (-150)^2 + (-128)^2 + (-190)^2 + (-2)^2 + 56^2
   + (-119)^2 + 8^2 + \\
   & & + 1^2 + (-79)^2 + (-153)^2 + (-34)^2 + (-107)^2 + (-15)^2
   \\
   & = & 140\,408
  \end{eqnarray*}
  por lo que la media de las diferencias muestrales es:
  \begin{equation*}
   \bar{d} = \frac{1}{15} \sum_{i=1}^{15} d_i
   = \frac{-812}{15} = -54.1\overline{3}
  \end{equation*}
  y la varianza de las diferencias muestrales se calcula,
  usando el teorema 8.1, como sigue:
  \begin{eqnarray*}
   s_D^2 & = & \frac{1}{15(14)} \left[
   15 \sum_{i=1}^{15} d_i^2 - \left( \sum_{i=1}^{15} d_i \right)^2
   \right]
   = \frac{15(140\,408) - (-812)^2}{210}
   = \frac{2\,106\,120 - 659\,344}{210} \\
   & = & \frac{1\,446\,776}{210} = \frac{723\,388}{105}
   \approx 6\,889.4\overline{095238}
  \end{eqnarray*}
  por lo que la desviaci\'on est\'andar de las diferencias
  muestrales es:
  \begin{equation*}
   s_D = \sqrt{s_D^2} = \sqrt{\frac{723\,388}{105}}
   = \frac{2\sqrt{180\,847}\sqrt{105}}{105}
   = \frac{2\sqrt{18\,988\,935}}{105}
   \approx 83.002466974
  \end{equation*}
  Por lo tanto, se resume el resto de los datos como sigue:
  \begin{itemize}
   \item $\bar{d} = -\frac{812}{15} = -54.1\overline{3}$.
   \item $S_D = \frac{2\sqrt{18\,988\,935}}{105}
   \approx 83.002466974$
  \end{itemize}
  Adem\'as, por la suposici\'on de normalidad
  en las distribuciones poblacionales,
  se tiene que la distribuci\'on de las diferencias pareadas
  es normal,
  entonces el siguiente estad\'{\i}stico, que se va a requerir,
  se aproxima a la distribuci\'on mostrada
  con el respectivo par\'ametro que se indica:
  \begin{itemize}
   \item $T = \frac{\overline{D} - d_0}{S_d/\sqrt{n}} \sim t(v)$.
   \item $v = n_1 - 1 = n_2 - 1 = 14$.
  \end{itemize}
 \end{datos}

 \begin{hipotesis}
  \begin{eqnarray*}
   H_0: \mu_D = \mu_1 - \mu_2 & \geq & 0 \\
   H_0: \mu_D = \mu_1 - \mu_2 &   <  & 0
  \end{eqnarray*}
 \end{hipotesis}

 \begin{estadistico}
  \begin{eqnarray*}
   t & = & \frac{\bar{d} - d_0}{s_D/\sqrt{n}}
   = \frac{
   -\frac{812}{15}
   }{
   \frac{2\sqrt{18\,988\,935}}{105}/\sqrt{15}
   }
   = - \frac{812(105)}{30\sqrt{1\,265\,929}}
   = - \frac{812(7)\sqrt{1\,265\,929}}{2(1\,265\,929)} \\
   & = & - \frac{406\sqrt{1\,265\,929}}{180\,847}
   \approx -2.5259188806974
  \end{eqnarray*}
 \end{estadistico}

 \begin{valorp}
  Dado que:
  \begin{equation*}
   P(T<t) \approx P(T < -2.5259188806974) = P(T > 2.5259188806974)
  \end{equation*}
  Y ya que $2.415 < 2.5259188806974 < 2.624$,
  en donde, de la tabla A.4, se tiene que
  $P(T > 2.415) = 0.015$ y $P(T > 2.624) = 0.01$,
  entonces, interpolando, se considerar\'a la aproximaci\'on
  de $P(T > 2.5381052) \approx 0.012$, luego entonces:
  \begin{equation*}
   P(T<t) \approx P(T > 2.5381052) \approx 0.012
  \end{equation*}
 \end{valorp}

 \begin{conclusion}
  Por lo tanto, se concluye que la afirmaci\'on es correcta,
  es decir, hay evidencia para afirmar
  que, bajo condiciones de fatiga, hay un aumento
  en las diferencias medias absolutas de tiempo
  cuando la tarea se ejecuta.
 \end{conclusion}

 Finalmente, usando el archivo anexo
 \texttt{P06\_Prueba\_de\_dos\_medias\_02.r},
 que a su vez requiere los datos del archivo
 \texttt{DB08\_Problema\_045.csv},
 con los siguientes cambios:
 \begin{verbatim}
> datos<-read.csv("DB08_Problema_045.csv",sep=";",encoding="UTF-8")
> varInteres<-c("Tiempo.mseg")
> varSel<-c("Fatiga")
> mu<-0
> desv.iguales<-NULL
> alfa<-NULL
> cola<-'I'
> par<-TRUE
 \end{verbatim}
 \vspace{-0.5cm}
 el programa de R lanza el siguiente resultado:
 \begin{verbatim}
         Var1 Freq Poblaciones H0  n diferencia desv.par error.est grados alpha
1 Tiempo.mseg   15    Pareadas  0 15  -54.13333 83.00247  21.43114     14  0.05
      PValor Estadistico RegionRechazoInfT RegionRechazoInfX
1 0.01211081   -2.525919          -1.76131         -37.74689
 \end{verbatim}
 \vspace{-0.5cm}
 El cual coincide con los datos obtenidos,
 que es a lo que se quer\'{\i}a llegar.${}_{\blacksquare}$
\end{solucion}
