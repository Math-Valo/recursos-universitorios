\begin{enunciado}
 Establezca las hip\'otesis nula y alternativa a utilizar en la prueba de las siguientes afirmaciones, y determine de manera general d\'onde se ubica la regi\'on cr\'{\i}tica:
 \begin{enumerate}
  \item A lo m\'as $20\%$ de la cosecha de trigo del pr\'oximo a\~no se exportar\'a a la Uni\'on Sovi\'etica.
  \item En promedio, las amas de casa estadounidenses beben $3$ tazas de caf\'e al d\'{\i}a.
  \item La proporci\'on de graduados en Virginia este a\~no que se especializan en ciencias sociales es al menos $0.15$.
  \item La donaci\'on promedio a la Asociaci\'on Estadounidense del Pulm\'on es no m\'as de $\$10$.
  \item Los residentes del suburbio Richmond recorren, en promedio, $15$ kil\'ometros hasta su lugar de trabajo.
 \end{enumerate}
\end{enunciado}

\begin{solucion}
 $\phantom{0}$
 \begin{enumerate}
  \item Nombrando $p_1$ a la proporci\'on de trigo
  que el pr\'omixo a\~no se exportar\'a a la Uni\'on Sovi\'etica,
  las hip\'otesis se pueden escribir como:
  \begin{eqnarray*}
   H_0: & & p_1 \leq 0.2 \\
   H_1: & & p_1 > 0.2
  \end{eqnarray*}
  La regi\'on cr\'{\i}tica se puede escribrir como todos los valores reales
  $x \in [0,1]$ tales que $x > x_1$, donde $x_1$ es el valor cr\'{\i}tico
  elegido para la prueba y donde $1 > x_1 > 0.2$.

  \item Nombrando $\mu_2$ al promedio de tazas de caf\'e
  que las amas de casa estadounidenses beben al d\'{\i}a,
  las hip\'otesis se pueden escribir como:
  \begin{eqnarray*}
   H_0: & & \mu_2 = 3 \\
   H_1: & & \mu_2 \neq 3
  \end{eqnarray*}
  La regi\'on cr\'{\i}tica se puede escribir como todos los valores
  $x\in \mathbb{N}$ tales que $x > x_{2_{inf}}$ y $x < x_{2_{sup}}$,
  donde $x_{2_{inf}}$ y $x_{2_{sup}}$ son los valores cr\'{\i}ticos
  inferior y superior elegidos para la prueba, respectivamente,
  de modo que $x_{2_{inf}} < 3 < x_{2_{sup}}$
  y $x_{2_{inf}}, x_{2_{sup}} \in \mathbb{N}$.
  
  \item Nombrando $p_3$ a la proporci\'on de graduados en Virginia este
  a\~no que se especializan en ciencias sociales,
  las hip\'otesis se pueden escribir como:
  \begin{eqnarray*}
   H_0: & & p_3 \geq 0.15 \\
   h_0: & & p_3   <  0.15
  \end{eqnarray*}
  La regi\'on cr\'{\i}tica se puede escribir como todos los valores reales
  $x \in [0,1]$ tales $x < x_3$, donde $x_3$ es el valor cr\'{\i}tico
  elegido para la prueba y donde $0 < x_3 < 0.15$.
  
  \item Nombrando $\mu_4$ al promedio de donaci\'on a la Asociaci\'on
  Estadounidense del Pulm\'on, en d\'olares,
  las hip\'otesis se pueden escribir como:
  \begin{eqnarray*}
   H_0: & & \mu_4 \leq 10 \\
   H_1: & & \mu_4   >  10
  \end{eqnarray*}
  La regi\'on cr\'{\i}tica se puede escribir como todos los valores reales
  tales que $x > x_4$, donde $x_4$ es el valor cr\'{\i}tico elegido
  para al prueba.
  
  \item Nombrando $\mu_5$ al promedio de kil\'ometros que los residentes
  del suburbio Richmond recorren hasta su lugar de trabajo,
  las hip\'otesis se pueden escribir como:
  \begin{eqnarray*}
   H_0: & & \mu_5   =  15 \\
   H_1: & & \mu_5 \neq 15
  \end{eqnarray*}
  La regi\'on cr\'{\i}tica se puede escribir como todos los valores reales
  $x$ tales que $x < x_{5_{inf}}$ y $x > x_{5_{sup}}$,
  donde $x_{5_{inf}}$ y $x_{5_{sup}}$ son los valores cr\'{\i}ticos
  inferior y superior elegidos para la prueba, respectivamente.
 \end{enumerate}
\end{solucion}
