\begin{enunciado}
 Una muestra aleatoria de $90$ adultos se clasifica de acuerdo con su g\'enero
 y el n\'umero de horas que pasan viendo la televisi\'on durante una semana:
 \begin{center}
  \begin{tabular}{lcc}
   & \multicolumn{2}{c}{\textbf{Sexo}} \\
   \cline{2-3}
   & \textbf{Masculino} & \textbf{Femenino} \\
   \hline 
   M\'as de 25 horas & $15$ & $29$ \\
   Menos de 25 horas & $27$ & $19$
  \end{tabular}
 \end{center}
 Utilice un nivel de significancia de $0.01$ y pruebe la hip\'otesis
 de que el tiempo que pasan viendo televisi\'on es independiente
 de si el espectador es hombre o mujer.
\end{enunciado}

\begin{solucion}
 \begin{datos}
  $\phantom{0}$
  \begin{itemize}
   \item Tamaño de muestra total: $90$.
   \item Personas que ven m\'as de $25$ horas la T.V. en una semana:
   $15 + 29 = 44$.
   \item Personas que ven menos de $25$ horas la T.V. en una semana:
   $27 + 19 = 46$.
   \item Cantida de hombres en la muestra: $15 + 27 = 42$.
   \item Cantidad de mujeres en la muestra: $29 + 19 = 48$.
   \item Frecuencias observadas y esperadas: $o_{i,j}$
   y $e_{i,j}=\frac{R_i C_j}{n}$, respectivamente,
   donde $R_i$ y $C_j$ son los marginales del rengl\'on $i$ y la columna $j$,
   respectivamente, y $n$ es el total de toda la muestra.
   As\'{\i}, pues, redondeando a un decimal, se muestra el resumen 
   en la siguiente tabla,
   en donde aparece entre par\'entesis la frecuencia esperada
   y a la izquierda el valor observado:
   \begin{center}
    \begin{tabular}{lcc|c}
     & \multicolumn{2}{c}{\textbf{Sexo}} \\
     \cline{2-3}
     & \textbf{Masculino} & \textbf{Femenino} &
     \textbf{Marginal seg\'un el tiempo} \\
     \hline 
     M\'as de 25 horas & $15 (20.5)$ & $29 (23.5)$ & $44$ \\
     Menos de 25 horas & $27 (21.5)$ & $19 (24.5)$ & $46$ \\
     \hline 
     \textbf{Marginal por sexo} & $42$ & $48$ & $n=90$
    \end{tabular}
   \end{center}
   \item Tama\~no de la tabla de contingencia: $r\times c = 2\times 2$.
   \item Grados de libertad de la prueba $\chi^2$: $v = (r-1)(c-1) = 1$.
  \end{itemize}
 \end{datos}

 \begin{hipotesis}
  \begin{eqnarray*}
   H_0: & & \text{Ver m\'as o menos de 25 horas de T.V. a la semana es independiente del sexo.} \\
   H_1: & & \text{Ver m\'as o menos de 25 horas de T.V. a la semana es dependiente del sexo.}
  \end{eqnarray*}
 \end{hipotesis}

 \begin{significancia}
  $\alpha = 0.01$.
 \end{significancia}

 \begin{region}
  De la tabla A.5, se tiene el valor cr\'{\i}tico
  $\chi^2_{\alpha,v} = \chi^2_{0.01,1} \approx 6.635$,
  por lo que la regi\'on de rechazo est\'a dado
  para $\chi^2 > 6.635$, donde
  $\chi^2 = \sum_{i} \frac{\left( o_i - e_i \right)^2}{e_i}$.
  N\'otese que aqu\'{\i} no se har\'a la correcci\'on de Yates,
  ya que las frecuencias esperadas son grandes en cada caso (m\'as de 10).
 \end{region}

 \begin{estadistico}
  \begin{eqnarray*}
   \chi^2 & = & \sum_{i} \frac{\left( o_i - e_i \right)^2}{e_i}
   = \frac{(15 - 20.5)^2}{20.5} + \frac{(29 - 23.5)^2}{23.5} +
   \frac{(27 - 21.5)^2}{21.5} + \frac{(19 - 24.5)^2}{24.5} \\
   & = & \frac{30.25}{20.5} + \frac{30.25}{23.5} + \frac{30.25}{21.5} + 
   \frac{30.25}{24.5}
   \approx 1.47560976 + 1.287234 + 1.40697674 + 1.23469388 \\
   & = & 5.40451438
  \end{eqnarray*}
 \end{estadistico}

 \begin{decision}
  No se rechaza $H_0$.
 \end{decision}

 \begin{conclusion}
  No hay evidencia suficiente para rechazar la hip\'otesis nula 
  y se considera que el tiempo de televisi\'on es independiente
  del sexo del individuo.
 \end{conclusion}
 
 Finalmente, usando el archivo anexo
 \texttt{P18\_Prueba\_de\_independencia\_y\_homogeniedad\_01.r},
 que a su vez requiere los datos del archivo
 \texttt{BD27\_Problema\_091.csv}, con los siguientes cambios:
 \begin{verbatim}
> datos<-read.csv("DB27_Problema_091.csv",sep=";",encoding="UTF-8")
> varInteres<-c("Tiempo.tv.horas","Sexo")
> varFrecuencia<-"Frecuencia"
> pruebas<-c(1,2,3)
 \end{verbatim}
 \vspace{-0.5cm}
 el programa de R lanza el siguiente resultado:
 \begin{verbatim}
$tabla
                   Sexo
Tiempo.tv.horas     Femenino Masculino
  Más de 25 horas         29        15
  Menos de 25 horas       19        27

$listaPruebas
$listaPruebas[[1]]

	Pearson's Chi-squared test

data:  tbl1
X-squared = 5.4702, df = 1, p-value = 0.01934


$listaPruebas[[2]]

	Log likelihood ratio (G-test) test of independence without correction

data:  tbl1
Log likelihood ratio statistic (G) = 5.531, X-squared df = 1, p-value =
0.01868


$listaPruebas[[3]]

	Log likelihood ratio (G-test) test of independence with Williams'
	correction

data:  tbl1
Log likelihood ratio statistic (G) = 5.4398, X-squared df = 1, p-value =
0.01968


$listaPruebas[[4]]

	Pearson's Chi-squared test with Yates' continuity correction

data:  tbl1
X-squared = 4.5262, df = 1, p-value = 0.03338
 \end{verbatim}
 \vspace{-0.5cm}
 Lo cual coincide con los resultados obtenidos, adem\'as de brindar m\'as
 informaci\'on y los valores $P$ usando otros estad\'{\i}sticos, 
 el cual no es menor al nivel de significancia de $0.01$ en ning\'un caso.
 De cualquier forma, se prueba tambi\'en la correcci\'on de Yates
 para verificar lo que se hubiese obtenido en caso de haberse necesitado,
 lo cual se sabe que no es as\'{\i} gracias a que las frecuencias no son
 tan bajas, que es lo que se quer\'{\i}a llegar.${}_{\blacksquare}$
\end{solucion}
