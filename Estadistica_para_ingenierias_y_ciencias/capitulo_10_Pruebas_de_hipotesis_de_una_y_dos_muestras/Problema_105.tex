\begin{enunciado}
 En un conjunto de datos analizados por el Centro de Consulta
 Estad\'{\i}stica del Instituto Polit\'ecnico y Universidad Estatal
 de Virginia, se solicit\'o a un grupo de sujetos completar cierta tarea
 en la computadora.
 La respuesta medida fue el tiempo de terminaci\'on.
 El prop\'osito del experimento fue probar un grupo de herramientas
 de ayuda desarrolladas por el Departamento de Ciencias Computacionales
 del mismo instituto.
 Participaron $10$ sujetos.
 Con una asignaci\'on al azar, a $5$ se les dio un procedimiento est\'andar
 con lenguaje Fortran para completar la tarea.
 A los otros $5$ se les pidi\'o realizar la tarea usando las herramientas
 de ayuda.
 A continuaci\'on se presentan los datos de los tiempos de terminaci\'on de la tarea.
 \begin{center}
  \begin{tabular}{cc}
   \textbf{Grupo 1} & \textbf{Grupo 2} \\
   \textbf{(Procedimiento est\'andar)} & \textbf{(Herramienta de ayuda)} \\
   \hline 
   $161$ & $132$ \\
   $169$ & $162$ \\
   $174$ & $134$ \\
   $158$ & $138$ \\
   $163$ & $133$
  \end{tabular}
 \end{center}
 Suponiendo que las distribuciones poblacionales son noramles y las varianzas son las mismas para los dos grupos, apoye o rechace la conjetura de que las herramientas de ayuda aumentan la velocidad con la que se realiza la tarea.
\end{enunciado}

\begin{solucion}
 Ya que se est\'a suponiendo distribuciones poblacionales normales
 y varianzas iguales en ambos grupos, se omitir\'an las pruebas
 de los mismos y se proceder\'a a usar el archivo anexo
 \texttt{P06\_Prueba\_de\_dos\_medias\_02.r}, que a su vez usa
 el archivo \texttt{DB38\_Problema\_105.csv}, con los siguientes cambios:
 \begin{verbatim}
> datos<-read.csv("DB38_Problema_105.csv",sep=";",encoding="UTF-8")
> varInteres<-c("Tiempos.terminación")
> varSel<-c("Grupo")
> mu<-0
> desv.iguales<-TRUE
> alfa<-NULL
> cola<-'D'
> par<-FALSE
 \end{verbatim}
 \vspace{-0.5cm}
 que corresponden a los datos iniciales, la hip\'otesis de prueba:
 \begin{eqnarray*}
  H_0: & & \mu_1 - \mu_2  =   0 \\
  H_1: & & \mu_1 - \mu_2 \neq 0
 \end{eqnarray*}
 donde $\mu_1$ corresponde a la media poblacional de tiempos
 de finalizaci\'on de procesos usando el proceso est\'andar
 y $\mu_2$ la media poblacional de tiempos de finalizaci\'on,
 usando la herramienta de ayuda,
 entonces se obtiene el siguiente resultado:
 \begin{verbatim}
                 Var1 Freq    Poblaciones H0 valorPVar     suposicionVar n1 n2
1 Tiempos.terminación   10 Independientes  0 0.2211745 Var no diferentes  5  5
  media1 media2 diferencia desv.est1 desv.est2   est.sp error.est grados alpha
1    165  139.8       25.2  6.442049  12.61745 10.01748  6.335614      8  0.05
       PValor Estadistico RegionRechazoInfT RegionRechazoSupT RegionRechazoInfX
1 0.004075838    3.977515         -2.306004          2.306004         -14.60995
  RegionRechazoSupX
1          14.60995
 \end{verbatim}
 \vspace{-0.5cm}
 el cual indica las regiones cr\'{\i}ticas bajo la suposici\'on
 de un nivel de significancia $\alpha=0.05$,
 el valor del estad\'{\i}stico $t = \frac{
 \left( \bar{x}_1 - \bar{x}_2 \right) - d_0}{s_p\sqrt{1/n_1 + 1/n_2}}$,
 donde $\bar{x}_1 = 165$ y $x_2 = 139.8$ son las medias muestrales,
 $d_0 = 0$ es el valor correspondiente al valor de la diferencia
 de medias poblacionales en la hip\'otesis nula, $n_1 = n_2 = 5$
 son los tama\~nos muestrales, $s_p = 10.01748$ es el estimador
 de la desviaci\'on est\'andar a la que se supone son iguales
 las desviaciones poblacionales, y cuya f\'ormula viene de la ra\'{\i}z
 de $s_p^2 = \frac{s_1^2 \left( n_1-1\right) +s_2^2\left( n_2-1\right)}{
 n_1+n_2-2}$, donde $s_1^2 = \left( s_1 \right)^2 = 6.442049^2$
 y $s_2^2 = \left( s_2 \right)^2 = 12.61745^2$
 son las varianzas muestrales y, finalmente $t$ sigue una distribuci\'on
 $t$ de student con $v = n_1 + n_2 - 2 = 8$ grados de libertad.
 \par 
 Entonces el valor del c\'alculo del estad\'{\i}stica es $t = 3.977515$,
 cuyo valor $P$ es $0.004075838$, el cual es muy bajo como para aceptar
 $H_0$. Por lo tanto, la muestra arroja evidencia suficiente
 para rechazar la hip\'otesis nula y concluir
 que el tiempo promedio usando la herramienta de ayuda es diferente,
 y m\'as espec\'{\i}fico es menor, al tiempo promedio
 con el procedimiento est\'andar,
 que es a lo que se quer\'{\i}a llegar.${}_{\blacksquare}$
\end{solucion}
