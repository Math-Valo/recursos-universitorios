\begin{enunciado}
 Datos hist\'oricos indican que la cantidad de dinero que aportaron los residentes
 trabajadores de una ciudad grande para un escuadr\'on de rescate voluntario es
 una variable aleatoria normal con una desviaci\'on est\'andar de $\$1.40$.
 Se sugiere que las contribuciones al escuadr\'on de rescate s\'olo
 de los empleados del departamento de sanidad son mucho m\'as variables.
 Si las contribuciones de una muestra aleatoria de $12$ empleados
 del departamento de sanidad tienen una desviaci\'on est\'andar de $\$1.75$,
 ¿podemos concluir con un nivel de significancia de $0.01$
 que la desviaci\'on est\'andar de las contribuciones
 de todos los trabajadores de sanidad es mayor que la de todos los trabajadores
 que viven en dicha ciudad?
\end{enunciado}

\begin{solucion}
 \begin{datos}
  $\phantom{0}$
  \begin{itemize}
   \item $X \sim n\left( \mu, \sigma \right)$.
   \item $n = 12$.
   \item $s = 1.75$,
   entonces $s^2 = \left(\frac{7}{4}\right)^2 = \frac{49}{16} = 3.0625$.
  \end{itemize}
 \end{datos}

 \begin{hipotesis}
  \begin{eqnarray*}
   H_0: \sigma^2 & = & 1.96 \\
   H_1: \sigma^2 & > & 1.96
  \end{eqnarray*}
 \end{hipotesis}

 \begin{significancia}
  $\alpha = 0.01$.
 \end{significancia}

 \begin{region}
  De la tabla A.5, se tiene el valor cr\'{\i}tico
  $\chi^2_{\alpha,n-1}=\chi^2_{0.01,11} \approx 24.725$,
  por lo que la regi\'on de rechazo est\'a dado para $\chi^2 > 24.725$,
  donde $\chi^2 = \frac{(n-1)s^2}{\sigma_0^2}$.
 \end{region}

 \begin{estadistico}
  \begin{equation*}
   \chi^2 = \frac{(n-1)s^2}{\sigma_0^2}
   = \frac{11\left(\frac{\cancel{49}}{16}\right)}{\frac{\cancel{49}}{25}}
   = \frac{11(25)}{16} = \frac{275}{16} = 17.1875
  \end{equation*}
 \end{estadistico}

 \begin{decision}
  No se rechaza $H_0$.
 \end{decision}

 \begin{conclusion}
  Por el criterio del nivel de significancia, se concluye
  que no hay prueba suficiente para rechazar la hip\'otesis nula;
  es decir, la desviaci\'on est\'andar de los contribuyentes de todos
  los trabajadores de sanidad no es mayor que la de todos los trabajadores
  que viven en dicha ciudad.
 \end{conclusion}
 
 Finalmente, usando el archivo anexo \texttt{P11\_Prueba\_de\_una\_varianza\_02.r}, con los siguientes cambios:
 \begin{verbatim}
> n<-12
> sigma2<-1.4^2
> s2<-1.75^2
> alfa<-0.01
> cola<-'S'
 \end{verbatim}
 \vspace{-0.5cm}
 el programa de R lanza el siguiente resultado:
 \begin{verbatim}
   n   H0 var.muestral grados error.est alpha    PValor Estadistico
1 12 1.96       3.0625     11 0.1781818  0.01 0.1024504     17.1875
  RegionRechazoJi RegionRechazoX        Resultado
1    > 24.7249703    > 4.4055402 No se rechaza H0
 \end{verbatim}
 \vspace{-0.5cm}
 El cual coincide con los resultados obtenidos,
 que es a lo que se quer\'{\i}a llegar.${}_{\blacksquare}$
\end{solucion}
