\begin{enunciado}
 En un informe de investigaci\'on de Richard H. Weindruch de la Escuela de Medicina de la \texttt{UCLA}, se afirma que los ratones con una vida promedio de 32 meses vivir\'{\i}an hasta alrededor de $40$ meses de edad, cuando 40\% de las calor\'{\i}as en su dieta se reemplacen con vitaminas y prote\'{\i}nas. ¿Hay alguna raz\'on para creer que $\mu < 40$, si $64$ ratones que se sujetan a esa dieta tienen una vida promedio de $38$ meses con una desviaci\'on est\'andar de $5.8$ meses? Utilice un valor $P$ en su conclusi\'on.
\end{enunciado}

\begin{solucion}
 \begin{datos}
  $\phantom{0}$
  \begin{itemize}
   \item $n = 64$.
   \item $\bar{x} = 38$.
   \item $s = 5.8$.
  \end{itemize}
  Adem\'as, por el teorema del l\'{\i}mite central, como $n \geq 30$, se puede aproximar la variable aleatoria de la media muestral a una normal, y usar el estad\'{\i}stico siguiente:
  \begin{itemize}
   \item $\overline{X} \sim n\left( \mu , \sigma/\sqrt{n} \right)$.
   \item $Z = \frac{\overline{X}-\mu}{S/\sqrt{n}} \approx
   \frac{\overline{X} - \mu}{\sigma/\sqrt{n}} \sim n(0,1)$.
  \end{itemize}
 \end{datos}

 \begin{hipotesis}
  \begin{eqnarray*}
   H_0: \mu & = & 40 \\
   H_1: \mu & < & 40
  \end{eqnarray*}
 \end{hipotesis}

 \begin{estadistico}
  \begin{equation*}
   z = \frac{\bar{x}-\mu_0}{\sigma/\sqrt{n}} \approx \frac{\bar{x} - \mu_0}{s/\sqrt{n}} = \frac{38-40}{5.8/\sqrt{64}} = -\frac{2(8)}{5.8} = -\frac{80}{29} \approx -2.75862
  \end{equation*}
 \end{estadistico}

 \begin{valorp}
  De la tabla A.3, se tiene que:
  \begin{equation*}
   P(Z<z) \approx P(Z < -2.76) \approx 0.0029
  \end{equation*}
 \end{valorp}

 \begin{conclusion}
  Dado que el valor $P$ es muy peque\~no,
  tanto que \'unicamente hay un aproximado $0.3\%$ de las muestras
  que pueden dar resultados similares,
  es evidencia suficiente para rechazar la afirmaci\'on de la escuela
  de Medicina de la UCLA y concluir que la vida promedio de los ratones
  con una vida promedio de 32 meses que sigue una dieta
  en el que se reemplaza el 40\% de calor\'{\i}as con vitaminas
  y prote\'{\i}nas es significativamente menor a los 40 meses.
 \end{conclusion}
 Finalmente, usando el archivo anexo \texttt{P03\_Prueba\_de\_una\_media\_01.r}, con los siguientes cambios:
 \begin{verbatim}
> n<-64
> mu<-40
> m<-38
> desv<-5.8
> pobl<-FALSE
> alfa<-NULL
> cola<-'I'
> val<-FALSE
 \end{verbatim}
 \vspace{-0.5cm}
 el programa de R lanza el siguiente resultado:
 \begin{verbatim}
  Prueba H0  n MediaMuestral desv.est error.est alpha    PValor Estadistico
1      Z 40 64            38      5.8     0.725  0.05 0.0029023   -2.758621
  RegionRechazoZ RegionRechazoX
1   < -1.6448536   < 38.8074811
 \end{verbatim}
 \vspace{-0.5cm}
 El cual coincide con los resultados obtenidos, que es a lo que se quer\'{\i}a llegar.${}_{\blacksquare}$
\end{solucion}
