\begin{enunciado}
 Se cree que al menos $60\%$ de los residentes de cierta \'area favorecen
 una demanda de anexi\'on de una ciudad vecina.
 ¿Qu\'e conclusi\'on extraer\'{\i}a, si s\'olo $110$ en una muestra de $200$ votantes
 est\'an a favor de la demanda?
 Utilice un nivel de significancia de $0.05$
\end{enunciado}

\begin{solucion}
 \begin{datos}
  $\phantom{0}$
  \begin{itemize}
   \item $n = 200$.
   \item $x = 110$.
  \end{itemize}
 \end{datos}

 \begin{hipotesis}
  \begin{eqnarray*}
   H_0: p & \geq & 0.6 \\
   H_1: p &  <   & 0.6
  \end{eqnarray*}
 \end{hipotesis}

 \begin{significancia}
  $\alpha = 0.05$
 \end{significancia}

 \begin{region}
  De la tabla A.3, se tiene el valor cr\'{\i}tico $z_{\alpha} = z_{0.05} = 1.645$,
  por lo que la regi\'on de rechazo est\'a dado para $z < -1.645$,
  donde $z = \frac{x - np}{\sqrt{npq}} 
  = \frac{\widehat{p} - p}{\sqrt{pq/n}}$.
 \end{region}

 \begin{estadistico}
  \begin{equation*}
   z = \frac{x - np}{\sqrt{npq}}
   = \frac{110 - (200)(0.6)}{\sqrt{(200)(0.6)(0.4)}}
   = \frac{110 - 120}{\sqrt{48}}
   = - \frac{10}{4\sqrt{3}}
   = - \frac{5\sqrt{3}}{6}
   \approx - 1.443375672974
  \end{equation*}
 \end{estadistico}

 \begin{decision}
  No se rechaza $H_0$.
 \end{decision}

 \begin{conclusion}
  No hay pruebas suficientes para refutar la creencia y se puede tomar por cierto
  que al menos el $60\%$ de los residentes est\'an a favor de la demanda.
 \end{conclusion}
 Finalmente, usando el archivo anexo \texttt{P08\_Prueba\_de\_una\_proporcion\_01.r},
 con los siguientes cambios:
 \begin{verbatim}
> n<-200
> x<-110
> p0<-NULL
> p<-0.6
> alfa<-0.05
> cola<-'I'
> distr<-NULL
 \end{verbatim}
 \vspace{-0.5cm}
 el programa de R lanza el siguiente resultado:
 \begin{verbatim}
   distr   p   n   x pMuestral media desv.est alpha    PValor Estadistico
1 Normal 0.6 200 110      0.55   120 6.928203  0.05 0.0744573   -1.443376
  RegionRechazoZ RegionRechazoX        Resultado
1  <= -1.6448536         <= 108 No se rechaza H0
 \end{verbatim}
 \vspace{-0.5cm}
 El cual coincide con los resultados obtenidos,
 que es a lo que se quer\'{\i}a llegar.${}_{\blacksquare}$
\end{solucion}
