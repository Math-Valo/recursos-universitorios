\begin{enunciado}
 El administrador de una compa\~n\'{\i}a de taxis trata de decidir si el uso de llantas radiales en lugar de llantas regulares cinturadas mejora la econom\'{\i}a de combustible. Se equipan $12$ autom\'oviles con llantas radiales y se manejan durante un recorrido de prueba preestablecido. Sin cambiar a los conductores, los mismos autom\'oviles se equipan con llantas regulares cinturadas y se manejan otra vez en el recorrido de prueba. El consumo de gasolina, en kil\'ometros por litro, se registr\'o de la siguiente manera:
 \begin{center}
  \textbf{Kil\'ometros por litro} \\
  \begin{tabular}{ccc}
   \hline 
   \textbf{Autom\'ovil} & \textbf{Llantas radiales} & \textbf{Llantas cinturadas} \\
   \hline
   $1$  & $4.2$ & $4.1$ \\
   $2$  & $4.7$ & $4.9$ \\
   $3$  & $6.6$ & $6.2$ \\
   $4$  & $7.0$ & $6.9$ \\
   $5$  & $6.7$ & $6.8$ \\
   $6$  & $4.5$ & $4.4$ \\
   $7$  & $5.7$ & $5.7$ \\
   $8$  & $6.0$ & $5.8$ \\
   $9$  & $7.4$ & $6.9$ \\
   $10$ & $4.9$ & $4.7$ \\
   $11$ & $6.1$ & $6.0$ \\
   $12$ & $5.2$ & $4.9$
  \end{tabular}
 \end{center}
 ¿Podemos concluir que los autom\'oviles equipados con llantas radiales dan una econom\'{\i}a de combustible mejor que aquellos equipados con llantas cinturadas? Suponga que las poblaciones se distribuyen normalmente. Utilice un valor $P$ en su conclusi\'on.
\end{enunciado}

\begin{solucion}
 \begin{datos}
  Resumido, se tiene que
  \begin{itemize}
   \item $X_i \sim n\left( \mu_i, \sigma_i \right)$,
   para cada $i \in \{ 1, 2 \}$.
   \item $n_1 = n_2 = 12$.
  \end{itemize}
  Como las observaciones fueron sobre las mismas unidades
  experimentales, entonces se trata de observaciones pareadas,
  por lo que se usar\'an las diferencias de los datos,
  que muestra la tabla siguiente:
  \begin{center}
   \textbf{Kil\'ometros por litro} \\
   \begin{tabular}{cccc}
    \hline 
    \textbf{Autom\'ovil} & \textbf{Llantas radiales} & \textbf{Llantas cinturadas} & \hspace{1cm} $d_i$ \hspace{1cm} \\
    \hline
    $1$  & $4.2$ & $4.1$ & $\phantom{-}0.1$ \\
    $2$  & $4.7$ & $4.9$ & $-0.2$ \\
    $3$  & $6.6$ & $6.2$ & $\phantom{-}0.4$ \\
    $4$  & $7.0$ & $6.9$ & $\phantom{-}0.1$ \\
    $5$  & $6.7$ & $6.8$ & $-0.1$ \\
    $6$  & $4.5$ & $4.4$ & $\phantom{-}0.1$ \\
    $7$  & $5.7$ & $5.7$ & $\phantom{-}0\phantom{.1}$ \\
    $8$  & $6.0$ & $5.8$ & $\phantom{-}0.2$ \\
    $9$  & $7.4$ & $6.9$ & $\phantom{-}0.5$ \\
    $10$ & $4.9$ & $4.7$ & $\phantom{-}0.2$ \\
    $11$ & $6.1$ & $6.0$ & $\phantom{-}0.1$ \\
    $12$ & $5.2$ & $4.9$ & $\phantom{-}0.3$
   \end{tabular}
  \end{center}
  Para obtener la media y desviaci\'on est\'andar
  de las diferencias muestrales, se calcula lo siguiente:
  \begin{eqnarray*}
   \sum_{i=1}^{12} d_i & = &
   0.1 - 0.2 + 0.4 + 0.1 - 0.1 + 0.1 + 0 + 0.2 + 0.5 + 0.2 + 0.1 + 0.3
   = 1.7 \\
   \sum_{i=1}^{12} d_i^2 & = &
   0.1^2 + (-0.2)^2 + 0.4^2 + 0.1^2 + (-0.1)^2 + 0.1^2 + 0^2 + 0.2^2
   + 0.5^2 + 0.2^2 + 0.1^2 + 0.3^2 \\
   & = & 0.67
  \end{eqnarray*}
  por lo que la media de las diferencias muestrales es:
  \begin{equation*}
   \bar{d} = \frac{1}{12} \sum_{i=1}^{12} d_i = \frac{1.7}{12}
   = \frac{17}{120} = 0.141\overline{6}
  \end{equation*}
  y la varianza de las diferencias muestrales se calcula,
  usando el teorema 8.1, como sigue:
  \begin{eqnarray*}
   s_D^2 & = &
   \frac{1}{12(11)} \left[
   12 \sum_{i=1}^{12} d_i^2 - \left( \sum_{i=1}^{12} d_i \right)^2
   \right]
   = \frac{12(0.67) - 1.7^2}{132} = \frac{8.04 - 2.89}{132} 
   = \frac{5.15}{132} \\
   & = & \frac{103}{2\,640} \approx 0.0390\overline{15}
  \end{eqnarray*}
  por lo que la desviaci\'on est\'andar de las diferencias
  muestrales es:
  \begin{equation*}
   s_D = \sqrt{s_D^2} = \sqrt{\frac{103}{2\,640}}
   = \frac{\sqrt{103}\sqrt{165}}{660}
   = \frac{\sqrt{16\,995}}{660}
   \approx 0.197522534
  \end{equation*}
  Por lo tanto, se resume el resto de los datos como sigue:
  \begin{itemize}
   \item $\bar{d} = \frac{17}{120} = 0.141\overline{6}$.
   \item $s_D = \frac{\sqrt{16\,995}}{660} \approx 0.197522534$.
  \end{itemize}
  Adem\'as, por la suposici\'on de normalidad
  en las distribuciones poblacionales,
  se tiene que la distribuci\'on de las diferencias pareadas
  es normal,
  entonces el siguiente estad\'{\i}stico, que se va a requerir,
  se aproxima a la distribuci\'on mostrada
  con el respectivo par\'ametro que se indica:
  \begin{itemize}
   \item $T = \frac{\overline{D} - d_0}{S_D/\sqrt{n}} \sim t(v)$.
   \item $v = n_1 - 1 = n_2 - 1 = 11$.
  \end{itemize}
 \end{datos}

 \begin{hipotesis}
  \begin{eqnarray*}
   H_0: \mu_D = \mu_1 - \mu_2 & \leq & 0 \\
   H_1: \mu_D = \mu_1 - \mu_2 &   >  & 0
  \end{eqnarray*}
 \end{hipotesis}

 \begin{estadistico}
  \begin{eqnarray*}
   t & = & \frac{\bar{d} - d_0}{s_D/\sqrt{n}} 
   = \frac{\frac{17}{120} - 0}{\frac{\sqrt{16\,995}}{660}/\sqrt{12}}
   = \frac{
   17\cancelto{11}{(660)}\left( \cancel{2}\cancel{\sqrt{3}} \right)
   }{
   \cancel{120}\sqrt{\cancelto{5\,665}{16\,995}}
   } \hspace{0.5cm}
   = \frac{187\sqrt{5\,665}}{5\,665}
   \approx 2.484515150684
  \end{eqnarray*}
 \end{estadistico}

 \begin{valorp}
  De la tabla A.4, se tiene que:
  \begin{equation*}
   P\left( T > t \right) \approx P(T > 2.484515150684)
   \approx P(T > 2.491) \approx 0.015
  \end{equation*}
  N\'otese que la aproximaci\'on de $2.484515150684$ a $2.491$
  es debido a que es, por mucho, el valor cr\'{\i}tico m\'as cercano,
  ya que en la tabla el valor cr\'{\i}tico pr\'oximo despu\'es
  de ese es $2.328$.
 \end{valorp}

 \begin{conclusion}
  Por lo tanto, como el valor $P$ es peque\~no,
  se concluye que el consumo de gasolina promedio
  de los autom\'oviles equipados con llantas radiales
  es significativamente mayor que el consumo promedio
  de aquellos equipados con llantas cinturadas
  y, por lo tanto, una mejor econom\'{\i}a de combustible
  est\'a en aquellos autom\'oviles que est\'an equipados
  con llantas cinturadas.
 \end{conclusion}

 Finalmente, usando el archivo anexo
 \texttt{P06\_Prueba\_de\_dos\_medias\_02.r},
 que a su vez requiere los datos del archivo
 \texttt{DB07\_Problema\_043.csv},
 con los siguientes cambios:
 \begin{verbatim}
> datos<-read.csv("DB07_Problema_043.csv",sep=";",encoding="UTF-8")
> varInteres<-c("Distancia.kpl")
> varSel<-c("Llantas")
> mu<-0
> desv.iguales<-NULL
> alfa<-NULL
> cola<-'S'
> par<-TRUE
 \end{verbatim}
 \vspace{-0.5cm}
 el programa de R lanza el siguiente resultado:
 \begin{verbatim}
           Var1 Freq Poblaciones H0  n diferencia  desv.par  error.est grados
1 Distancia.kpl   12    Pareadas  0 12  0.1416667 0.1975225 0.05701984     11
  alpha     PValor Estadistico RegionRechazoSupT RegionRechazoSupX
1  0.05 0.01516475    2.484515          1.795885         0.1024011
 \end{verbatim}
 \vspace{-0.5cm}
 El cual coincide con los datos obtenidos,
 que es a lo que se quer\'{\i}a llegar.${}_{\blacksquare}$
\end{solucion}
