\begin{enunciado}
 En un estudio que realiza el Departamento de Ingenier\'{\i}a Mec\'anica y que analiza el Centro de Consulta Estad\'{\i}stica del Instituto Polit\'ecnico y Universidad Estata de Virginia, se comparan las varillas de acero que proveen dos compa\~n\'{\i}as diferentes.
 Se fabrican diez resortes de muestra con las varillas proporcionadas por cada compa\~n\'{\i}a y se estudia la ``capacidad de rebote''.
 Los datos son los siguientes:
 \begin{verbatim}
Compañía A
 9.3 8.8 6.8  8.7  8.5 6.7  8.0  6.5  9.2 7.0
Compañía B
11.0 9.8 9.9 10.2 10.1 9.7 11.0 11.1 10.2 9.6
 \end{verbatim}
 \vspace{-0.5cm}
 ¿Puede concluir que casi no hay diferencia entre las varillas de acero proporcionadas por las dos compa\~n\'{\i}as?
 Utilice un valor $P$ para llegar a su conclusi\'on.
 ¿Las varianzas deber\'{\i}an combinarse aqu\'{\i}?
\end{enunciado}

\begin{solucion}
 Para realizar una prueba de diferencia de medias con pocos datos
 (menos de 30), el estad\'{\i}stico usado requiere una condici\'on inicial
 que es la suposici\'on de que las muestras provengan de poblaciones
 con distribuci\'on normal.
 Entonces se realizar\'an dos pruebas de bondad de ajuste
 para verificar que, en efecto, las muestras provienen de poblaciones
 normales. Entonces, la hip\'otesis nula para estas pruebas se pueden
 escribir de la forma
 $X_i$ sigue una distribuci\'on normal, para cada $i \in \{1,2\}$,
 donde $X_1$ y $X_2$ representan las medidas poblacionales de capacidad
 de rebote de los resortes de las compa\~n\'{\i}as A y B,
 respectivamente, contra la hip\'otesis alternativa
 de que $X_i$, para cada $i \in \{1,2\}$ no sigue una distribuci\'on
 normal.
 \par 
 Debido a que la muestra es peque\~na y no se consideran par\'ametros
 en la suposici\'on de las normalidades,
 la prueba con $\chi^2$ crea insuficientes clases, en cada muestra,
 que al calcular los grados de libertad junto con los grados que se restan
 por no considerar par\'ametros previos, entonces se obtiene
 un valor negativo, lo cual no puede corresponde a grados de libertad.
 Por lo tanto, se proceder\'a a realizar las pruebas de Geary,
 simult\'aneamente, con las siguientes l\'{\i}neas de c\'odigo
 escritas en R, que usa la base de datos
 \texttt{DB42\_Problema\_111.csv},
 que se presenta a continuaci\'on junto con los resultados:
 \begin{verbatim}
> datos<-read.csv("DB43_Problema_112.csv",sep=";",encoding="UTF-8")
> varInteres<-"Rebote.capacidad"; distincion<-"Compañía"
> lista<-split(datos[,varInteres],datos[,distincion])
> x<-lista$A; y<-lista$B
> n1<-length(x); n2<-length(y)
> U1<-sqrt(pi/2)*mean(abs(x-mean(x)))/sqrt(var(x)*(n1-1)/n1)
> U2<-sqrt(pi/2)*mean(abs(y-mean(y)))/sqrt(var(y)*(n2-1)/n2)
> x.Geary.statistic<-(U1-1)/(0.2661/sqrt(n1))
> y.Geary.statistic<-(U2-1)/(0.2661/sqrt(n2))
> x.Geary.p.value<-2*pnorm(abs(x.Geary.statistic),lower.tail=FALSE)
> y.Geary.p.value<-2*pnorm(abs(y.Geary.statistic),lower.tail=FALSE)
> print("P-valor para los productos de la compañia A usando la prueba de Geary")
[1] "P-valor para los productos de la compañia A usando la prueba de Geary"
> x.Geary.p.value
[1] 0.06671429
> print("P-valor para los productos de la compañia B usando la prueba de Geary")
[1] "P-valor para los productos de la compañia B usando la prueba de Geary"
> y.Geary.p.value
[1] 0.3699307
 \end{verbatim}
 \vspace{-0.5cm}
 Esto indica un buen ajuste a una distribuci\'on normal
 para la poblaci\'on de la capacidad de rebote de los productos
 de la compa\~n\'{\i}a B;
 sin embargo, la poblaci\'on de esta capacidad para los productos
 de la compa\~n\'{\i}a A
 muestran un valor $P$ de $0.06671429$ para el ajuste de bondad.
 Se tomar\'a el riesgo ante la baja probabilidad
 y se tomar\'a la decisi\'on de suponer que ambas muestras vienen
 de poblaciones con distribuciones normales.
 \par
 Para la prueba, se requiere ahora saber si las varianzas poblacionales
 de donde vienen ambas muestras se deben de considerar iguales
 o diferentes. Este implica una prueba de proporci\'on de varianzas,
 cuya hip\'otesis nula se puede enunciar como
 $H_0: \, \frac{\sigma_1}{\sigma_2} = 1$, contra la hip\'otesis
 alternativa $H_1: \, \frac{\sigma_1}{\sigma_2} \neq 1$,
 donde $\sigma_1$ y $\sigma_2$ representan las varianzas poblacionales
 de $X_1$ y $X_2$, respectivamente.
 Para ello se hace una prueba $F$ con ayuda del archivo adjunto
 \texttt{P14\_Prueba\_de\_dos\_varianzas\_02.r},
 que a su vez usa el archivo \texttt{DB43\_Problema\_112.csv},
 con los siguientes cambios:
 \begin{verbatim}
> datos<-read.csv("DB43_Problema_112.csv",sep=";",encoding="UTF-8")
> varInteres<-c("Rebote.capacidad")
> varSel<-c("Compañía")
> alfa<-NULL
> cola<-'D'
 \end{verbatim}
 \vspace{-0.5cm}
 lo cual arroja el siguiente resultado:
 \begin{verbatim}
          variable Freq n1 n2 media1 media2 varianza1 varianza2 v1 v2 alpha
1 Rebote.capacidad   20 10 10   7.95  10.26  1.207222 0.3248889  9  9  0.05
     PValor Estadistico             RegionRechazo
1 0.0637265      3.7158 < 0.2483859 y > 4.0259942
 \end{verbatim}
 \vspace{-0.5cm}
 en donde se observa que el estad\'{\i}stico $F=\frac{\sigma_1}{\sigma_2}$
 toma el valor de $3.7158$ con $v_1 = 9$ y $v_2 = 9$ grados
 de libertad para la distribuci\'on $F$,
 obteniendo as\'{\i} un valor $P$ de $0.0637265$,
 lo cual, aunque no es muy alto, es suficiente para decidir rechazar
 la hip\'otesis nula y, por lo tanto, se puede suponer que las varianzas iguales para la prueba $t$.
 \par 
 Para la prueba se enuncia la hip\'otesis nula como sigue:
 \begin{eqnarray*}
  H_0: & & \mu_1 - \mu_2   =   0 \\
  H_1: & & \mu_1 - \mu_2 \neq  0
 \end{eqnarray*}
 donde $\mu_1$ y $\mu_2$ representan las medias poblacionales
 de $X_1$ y $X_2$, respectivamente.
 El estad\'{\i}stico y el valor $P$ se calculan con ayuda
 del archivo adjunto \texttt{P06\_Prueba\_de\_dos\_medias\_02.r},
 que nuevamente usan la base de datos almacenado en el archivo adjunto
 \texttt{DB43\_Problema\_112.csv}, con los siguientes cambios:
 \begin{verbatim}
> datos<-read.csv("DB43_Problema_112.csv",sep=";",encoding="UTF-8")
> varInteres<-c("Rebote.capacidad")
> varSel<-c("Compañía")
 mu<-0
> desv.iguales<-TRUE
> alfa<-NULL
> cola<-'D'
> par<-FALSE
 \end{verbatim}
 \vspace{-0.5cm}
 lo cual arroja el siguiente resultado:
 \begin{verbatim}
              Var1 Freq    Poblaciones H0  valorPVar     suposicionVar n1 n2
1 Rebote.capacidad   20 Independientes  0 0.06372648 Var no diferentes 10 10
  media1 media2 diferencia desv.est1 desv.est2   est.sp error.est grados alpha
1   7.95  10.26      -2.31  1.098737 0.5699903 0.875246 0.3914219     18  0.05
        PValor Estadistico RegionRechazoInfT RegionRechazoSupT RegionRechazoInfX
1 1.379592e-05    -5.90156         -2.100922          2.100922        -0.8223469
  RegionRechazoSupX
1         0.8223469
 \end{verbatim}
 \vspace{-0.5cm}
 de donde se observa que el valor del estad\'{\i}stico
 es $t = -5.90156$, con un valor $P$ aproximado
 de $1.379592\times 10^{-5}$, 
 lo cual, al ser tan bajo, y teniendo en cuenta que $\bar{x}_1 = 7.95$
 y $\bar{x}_2 = 10.26$, se concluye que la capacidad de rebote promedio
 de los resortes son diferentes entre ambas compa\~n\'{\i}as,
 siendo que los resortes de la compa\'{\i}a A tiene una capacidad
 de reborte menor que los resortes de la compa\~n\'{\i}a B,
 que es a lo que se quer\'{\i}a llegar.${}_{\blacksquare}$
\end{solucion}
