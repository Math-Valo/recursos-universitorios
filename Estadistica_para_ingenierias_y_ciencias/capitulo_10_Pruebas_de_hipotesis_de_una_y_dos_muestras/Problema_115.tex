\begin{enunciado}
 Estudios indican que la concentraci\'on de \texttt{PCB} es mucho
 m\'as alta en tejido maligno de pecho que en tejido normal de pecho.
 Si un estudio de $50$ mujeres con c\'ancer de seno revela
 una concentraci\'on promedio de \texttt{PCB} de $22.8 \times 10^{-4}$
 gramos, con una desviaci\'on est\'andar de $4.8 \times 10^{-4}$ gramos,
 ¿la concentraci\'on media de \texttt{PCB} es menor
 que $24 \times 10^{-4}$ gramos?
\end{enunciado}

\begin{solucion}
 El enunciado hace referencia a una concentraci\'on media,
 es decir que se trata de una prueba de una media.
 Entonces, representando con $X$ a la variable aleatoria
 de la concentraci\'on de PCB, en $10^{-4}$ gramos,
 donde $\mu$ y $\sigma$ representan su media y desviaci\'on
 est\'andar poblacional,
 la hip\'otesis de la prueba se puede escribir como:
 \begin{eqnarray*}
  H_0: & & \mu \geq 24 \\
  H_1: & & \mu   <  24
 \end{eqnarray*}
 Como el tama\~no de la muestra es de $50$, que es mayor o igual a $30$,
 entonces, por teorema del l\'{\i}mite central, se tiene
 que $\overline{X}$, la variable aleatoria de la media muestral de $X$,
 tiene una distribuci\'on normal
 y, adem\'as, que la desviac\'on est\'andar muestral aproxima bien
 a $\sigma$, por lo que es v\'alido realizar la prueba
 con el estad\'{\i}stico $z$, el cual se procede a hacer
 con ayuda del archivo \texttt{P03\_Prueba\_de\_una\_media\_01.r},
 con los siguientes cambios:
 \begin{verbatim}
> n<-50
> mu<-24
> m<-22.8
> desv<-4.8
> pobl<-FALSE
> alfa<-NULL
> cola<-'I'
> val<-FALSE
 \end{verbatim}
 \vspace{-0.5cm}
 que, al ejecutarlo, R muestra el siguiente resultado:
 \begin{verbatim}
  Prueba H0  n MediaMuestral desv.est error.est alpha    PValor Estadistico
1      Z 24 50          22.8      4.8 0.6788225  0.05 0.0385499   -1.767767
  RegionRechazoZ RegionRechazoX
1   < -1.6448536   < 22.8834364
 \end{verbatim}
 \vspace{-0.5cm}
 Esto muestra que el estad\'{\i}stico
 $z = \frac{\overline{x} - \mu}{\sigma/\sqrt{n}} \approx \frac{\overline{x} - \mu}{s/\sqrt{n}}$ adquiere el valor $-1.767767$,
 cuyo valor $P$ es $0.0385499$.
 Este valor $P$ es muy bajo y, por lo tanto, arroja evidencia suficiente
 para rechazar $H_0$ a favor de $H_1$;
 es decir, que la concentraci\'on media de PCB en mujeres
 con c\'ancer de seno es menor a $24\times 10^{-4}$ gramos,
 que es a lo que se quer\'{\i}a llegar.${}_{\blacksquare}$
\end{solucion}
