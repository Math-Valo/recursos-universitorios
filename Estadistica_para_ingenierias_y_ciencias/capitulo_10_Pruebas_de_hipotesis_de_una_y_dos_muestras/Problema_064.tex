\begin{enunciado}
 En un estudio sobre la fertilidad de mujeres casadas conducido por Martin O'Connell
 y Carolyn C. Rogers para la Oficina de Censos en 1979,
 se seleccionaron al azar dos grupos de esposas con edades de 25 y 29 a\~nos
 y sin hijos, y a cada una se le pregunt\'o si a final de cuentas planeaba tener
 un hijo.
 Se seleccion\'o un grupo entre las mujeres con menos de dos a\~nos de casadas
 y otro entre las que ten\'{\i}an cinco a\~nos de casadas.
 Suponga que $240$ de las $300$ con menos de dos a\~nos de casadas planean
 tener un hijo alg\'un d\'{\i}a,
 comparadas con $288$ de las $400$ con cinco a\~nos de casadas.
 ¿Podemos concluir que la proporci\'on de mujeres con menos de dos a\~nos de casadas
 que planean tener hijos es significativamente mayor que la proporci\'on
 con cinco a\~nos de casadas? Utilice un valor $P$.
\end{enunciado}

\begin{solucion}
 \begin{datos}
  $\phantom{0}$
  \begin{itemize}
   \item $n_1 = 300$ y $n_2 = 400$.
   \item $x_1 = 240$ y $x_2 = 288$.
  \end{itemize}
  Adem\'as, se pueden calcular las proporciones estimadas de cada muestra
  y la estimaci\'on combinada de la proporci\'on a la que se supone que son iguales
  en ambas poblaciones, como se muestra a continuaci\'on:
  \begin{itemize}
   \item $\widehat{p}_1 = \frac{240}{300} = \frac{8}{10} = 0.8$.
   \item $\widehat{p}_2 = \frac{288}{400} = \frac{18}{25} = 0.72$.
   \item $\widehat{p} = \frac{x_1+x_2}{n_1+n_2} = \frac{240+288}{300+400}
   = \frac{528}{700} = \frac{132}{175} = 0.75\overline{428571}$.
  \end{itemize}
 \end{datos}

 \begin{hipotesis}
  \begin{eqnarray*}
   H_0: p_1 & \leq & p_2 \\
   H_1: p_1 &  >   & p_2
  \end{eqnarray*}
 \end{hipotesis}

 \begin{estadistico}
  \begin{eqnarray*}
   z & \approx &
   \frac{\widehat{p}_1 - 
   \widehat{p}_2}{\sqrt{\widehat{p}\widehat{q}\left(1/n_1+1/n_2\right)}}
   = \frac{
   \displaystyle{ \frac{8}{10} - \frac{18}{25}}
   }{
   \displaystyle{ \sqrt{
   \left(\frac{132}{175}\right)\left(\frac{43}{175}\right)
   \left(\frac{1}{300} + \frac{1}{400}\right)
   }}} \\
   & = & \frac{
   \displaystyle{ \frac{8}{100}}
   }{\displaystyle{
   \frac{1}{175}\sqrt{ \left( \cancel{2^2}\cdot\cancel{3}\cdot 11\cdot 43 \right)
   \left( \frac{7}{2^{\cancelto{2}{4}} \cdot \cancel{3} \cdot 5^2} \right) } 
   }}
   = \frac{\frac{1\,400}{100}}{\displaystyle{\frac{1}{10}\sqrt{3\,311}}}
   = \frac{14(10)\sqrt{3\,311}}{3\,311} \\
   & = & \frac{140\sqrt{3\,311}}{3\,311} \approx 2.43303549781
  \end{eqnarray*}
 \end{estadistico}

 \begin{valorp}
  De la tabla A.3 se tiene que:
  \begin{equation*}
   P(Z > z) = 1 - P(Z < z) \approx 1 - P(Z < 2.43) \approx 1 - 0.9925 = 0.0075
  \end{equation*}
 \end{valorp}

 \begin{conclusion}
  Por lo tanto, como el valor $P$ es muy peque\~no,
  se tiene evidencia suficiente para rechazar la hip\'otesis nula
  y concluir que, en efecto, la proporci\'on de mujeres con menos de dos a\~nos
  de casadas que pleanean tener hijos es significativamente mayor
  que la proporci\'on con cinco a\~nos de casadas.
 \end{conclusion}

 Finalmente, usando el archivo anexo \texttt{P09\_Prueba\_de\_dos\_proporciones\_01.r}, con los siguientes cambios:
 \begin{verbatim}
> n1<-300
> n2<-400
> x1<-240
> x2<-288
> p1<-NULL
> p2<-NULL
> alfa<-NULL
> alternativa<-'>'
 \end{verbatim}
 \vspace{-0.5cm}
 el programa de R lanza el siguiente resultado:
 \begin{verbatim}
  alternativa  n1  n2  x1  x2  p1   p2 pEstimada DifProp  error.est alpha
1     p1 > p2 300 400 240 288 0.8 0.72 0.7542857    0.08 0.03288074  0.05
     PValor Estadistico RegionRechazoZ
1 0.0074864    2.433035   >= 1.6448536
 \end{verbatim}
 \vspace{-0.5cm}
 El cual coincide con los resultados obtenidos,
 que es a lo que se quer\'{\i}a llegar.${}_{\blacksquare}$
\end{solucion}
