\begin{enunciado}
 Se comparan dos tipos de instrumentos para medir la cantidad de mon\'oxidos
 de azufre en la atm\'osfera, en un experimento sobre la contaminaci\'on del aire.
 Se desea determinar si los dos tipos de instrumentos dan mediciones que tengan la misma variabilidad.
 Se registran las siguientes lecturas para los dos instrumentos:
 \begin{center}
  \begin{tabular}{cc}
   \multicolumn{2}{c}{\textbf{Mon\'oxido de azufre}} \\
   \hline 
   \textbf{Instrumento A} & \textbf{Instrumento B} \\ 
   \hline 
   $0.86$ & $0.87$ \\
   $0.82$ & $0.74$ \\
   $0.75$ & $0.63$ \\
   $0.61$ & $0.55$ \\
   $0.89$ & $0.76$ \\
   $0.64$ & $0.70$ \\
   $0.81$ & $0.69$ \\
   $0.68$ & $0.57$ \\
   $0.65$ & $0.53$
  \end{tabular}
 \end{center}
 Suponiendo que las poblaciones de mediciones se distribuyen
 de forma aproximadamente normal,
 pruebe la hip\'otesis de que $\sigma_A = \sigma_B$, contra la alternativa
 $\sigma_A \neq \sigma_B$. Use un valor $P$.
\end{enunciado}

\begin{solucion}
 \begin{datos}
  Resumido se tiene que:
  \begin{itemize}
   \item $X_i \sim n\left( \mu_i, \sigma_i \right)$, para cada $i \in \{1,2\}$.
   \item $n_1 = n_2 = 9$.
  \end{itemize}
  Para obtener las varianzas se calcula primero lo siguiente:
  \begin{eqnarray*}
   \sum_{i=1}^{9} x_{A,i} & = &
   0.86 + 0.82 + 0.75 + 0.61 + 0.89 + 0.64 + 0.81 + 0.68 + 0.65 = 6.71 \\
   \sum_{i=1}^{9} x_{A,i}^2 & = & 
   0.86^2 + 0.82^2 + 0.75^2 + 0.61^2 + 0.89^2 + 0.64^2 + 0.81^2 + 0.68^2 +
   0.65^2 = 5.0893 \\
   \sum_{i=1}^{9} x_{B,i} & = &
   0.87 + 0.74 + 0.63 + 0.55 + 0.76 + 0.70 + 0.69 + 0.57 + 0.53 = 6.04 \\
   \sum_{i=1}^{9} x_{B,i}^2 & = &
   0.87^2 + 0.74^2 + 0.63^2 + 0.55^2 + 0.76^2 + 0.70^2 + 0.69^2 + 0.57^2 +
   0.53^2 = 4.1534
  \end{eqnarray*}
  por lo que las varianzas muestrales se calculan, usando el teorema 8.1,
  como sigue:
  \begin{eqnarray*}
   s_A^2 & = &
   \frac{1}{9(8)}
   \left[ 9\sum_{i=1}^9 x_{A,i}^2 -
   \left( \sum_{i=1}^9 x_{A,i} \right)^2 \right]
   = \frac{9(5.0893) - 6.71^2}{72} \\
   & = & \frac{45.8037 - 45.0241}{72} = \frac{0.7796}{72}
   = \frac{1\,949}{180\,000} = 0.01082\bar{7} \\
   s_B^2 & = &
   \frac{1}{9(8)}
   \left[ 9\sum_{i=1}^9 x_{B,i}^2 -
   \left( \sum_{i=1}^9 x_{B,i} \right)^2 \right]
   = \frac{9(4.1534) - 6.04^2}{72} \\
   & = & \frac{37.3806 - 36.4816}{72}
   = \frac{0.899}{72} = \frac{899}{72\,000} = 0.012486\bar{1}
  \end{eqnarray*}
  Por lo que el resto de los datos se resume como sigue:
  \begin{itemize}
   \item $s_A^2 = \frac{1\,949}{180\,000} = 0.01082\bar{7}$
   y $s_B^2 = \frac{899}{72\,000} = 0.012486\bar{1}$.
  \end{itemize}
  Adem\'as, por la suposici\'on de normalidad en las distribuciones
  poblacionales, se tiene la distribuci\'on siguiente:
  \begin{itemize}
   \item $\frac{s_A^2}{s_B^2} \sim f(v_1,v_2)$.
   \item $v_1 = n_1 - 1 = 8$ y $v_2 = n_2 - 1 = 8$.
  \end{itemize}
 \end{datos}

 \begin{hipotesis}
  \begin{eqnarray*}
   H_0: \sigma_A &  =   & \sigma_B \\
   H_1: \sigma_A & \neq & \sigma_B
  \end{eqnarray*}
 \end{hipotesis}

 \begin{estadistico}
  \begin{equation*}
   f = \frac{s_A^2}{s_B^2}
   = \frac{\displaystyle{\frac{1\,949}{\cancelto{5}{180\,000}}}}
   {\displaystyle{\frac{899}{\cancelto{2}{72\,000}}}}
   = \frac{1\,949(2)}{899(5)}
   = \frac{3\,898}{4\,495} \approx 0.86718576
  \end{equation*}
 \end{estadistico}

 \begin{valorp}
  De la tabla A.6 se observa que el valor $f$
  est\'a muy alejado de $f_{0.05,8,8}$;
  sin embargo, como $f_{0.95,8,8} = \frac{1}{f_{0.05,8,8}}$,
  se tiene que $\frac{1}{f} = \frac{4\,495}{3\,898} \approx 1.153155464$,
  pero a\'un as\'{\i} el nuevo valor se encuentra lejos de $f_{0.05,8,8}$,
  por lo que se entiende que $P(F<f) > 0.05$ y, por lo tanto, 
  se tiene que:
  \begin{equation*}
   2\times P\left( F_{v_1,v_2} < f \right)
   = 2\times P\left( F_{v_2,v_1} > \frac{1}{f} \right) > 2\times(0.05) = 0.1
  \end{equation*}
 \end{valorp}

 \begin{conclusion}
  Por lo tanto, como el valor $P$ es muy grande,
  entonces no hay evidencia suficiente para rechazar la hip\'otesis nula,
  y, por lo tanto, se afirma que la varianza en las mediciones
  de los instrumentos no es significativamente diferente.
 \end{conclusion}

 Finalmente, usando el archivo anexo \texttt{P14\_Prueba\_de\_dos\_varianzas\_02.r}, con los siguientes cambios:
 \begin{verbatim}
> datos<-read.csv("DB12_Problema_076.csv",sep=";",encoding="UTF-8")
> varInteres<-c("MonoxidoAzufre.ppm")
> varSel<-c("Instrumento")
> alfa<-NULL
> cola<-'D'
 \end{verbatim}
 \vspace{-0.5cm}
 el programa de R lanza el siguiente resultado:
 \begin{verbatim}
            variable Freq n1 n2    media1    media2 varianza1 varianza2 v1 v2
1 MonoxidoAzufre.ppm   18  9  9 0.7455556 0.6711111 0.0108278 0.0124861  8  8
  alpha    PValor Estadistico             RegionRechazo
1  0.05 0.8451865   0.8671858 < 0.2255676 y > 4.4332599
 \end{verbatim}
 \vspace{-0.5cm}
 El cual coincide con los resultados obtenidos,
 adem\'as de ofrecer un valor m\'as preciso para el valor $P$,
 que es a lo que se quer\'{\i}a llegar.${}_{\blacksquare}$
\end{solucion}
