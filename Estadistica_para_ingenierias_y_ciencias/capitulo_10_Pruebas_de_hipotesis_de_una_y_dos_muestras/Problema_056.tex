\begin{enunciado}
 Suponga que, en el pasado, $40\%$ de todos los adultos favorec\'{\i}an
 la pena capital.
 ¿Tenemos raz\'on para creer que la proporci\'on de adultos que actualmente favorecen
 la pena capital ha aumentado
 si, en una muestra aleatoria de $15$ adultos, $8$ est\'an a favor de la pena capital?
 Utilice un nivel de significancia de $0.05$.
\end{enunciado}

\begin{solucion}
 \begin{datos}
  $\phantom{0}$
  \begin{itemize}
   \item $n = 15$.
   \item $x = 8$.
  \end{itemize}
 \end{datos}

 \begin{hipotesis}
  \begin{eqnarray*}
   H_0: p & = & 0.4 \\
   H_1: p & > & 0.4
  \end{eqnarray*}
 \end{hipotesis}

 \begin{significancia}
  $\alpha = 0.05$
 \end{significancia}

 \begin{estadistico}
  Variable binomial $X$ con $p = 0.4$ y $n = 15$.
 \end{estadistico}

 \begin{valorp}
  De la tabla A.1, se tiene la siguiente aproximaci\'on:
  \begin{equation*}
   P\left( X \geq 8 | p = 0.4 \right) = 1 - P\left( X < 8 | p = 0.4 \right)
   = 1 - \sum_{i=0}^{7} b\left( x; 15, 0.4 \right)
   \approx 1 - 0.7869 = 0.2131 > 0.05
  \end{equation*}
  Mientras que el valor preciso se obtiene como sigue:
  \begin{eqnarray*}
   P\left( X \geq 8 | p = 0.4 \right)
   & = & 1 - \sum_{i=0}^{7} b\left( x; 15, 0.4 \right) \\
   & = & 1 - \frac{1}{10^{15}} \left( 6^{15} + 15\cdot 4\cdot 6^{14} +
   105\cdot 4^2 \cdot 6^{13} + 455 \cdot 4^3 \cdot 6^{12} +
   1\,365 \cdot 4^4 \cdot 6^{11} + \right. \\
   & & \phantom{1 - \frac{1}{10^{15}}} \left.
   3\,003 \cdot 4^5 \cdot 6^{10} +
   5\,005 \cdot 4^6 \cdot 6^9 + 6\,435 \cdot 4^7 \cdot 6^8 \right) \\
   & = & 1 - \frac{4\cdot 6^8}{10^{15}} \left( 9 \cdot 6^5 + 15\cdot 6^6 +
   105\cdot 4 \cdot 6^5 + 455 \cdot 4^2 \cdot 6^4 +
   1\,365 \cdot 4^3 \cdot 6^3 + \right. \\
   & & \phantom{1 - \frac{6^8}{10^15}} \left. 
   3\,003 \cdot 4^4 \cdot 6^2 +
   5\,005 \cdot 4^5 \cdot 6 + 6\,435 \cdot 4^6 \right) \\
   & = & 1 - \frac{6\,718\,464}{10^{15}}
   (69\,984 + 699\,840 + 3\,265\,920 + 9\,434\,880 + \\
   & & \phantom{1 - \frac{6\,718\,464}{10^{15}}(}
   18\,869\,760 + 27\,675\,648 +
   30\,750\,720 + 26\,357\,760) \\
   & = & 1 - \frac{6\,718\,464 \cdot 117\,124\,512}{1\,000\,000\,000\,000\,000}
   = 1 - \frac{786\,896\,817\,389\,568}{1\,000\,000\,000\,000\,000} \\
   & = & \frac{213\,103\,182\,610\,432}{1\,000\,000\,000\,000\,000}
   = 0.213103182610432
  \end{eqnarray*}

 \end{valorp}

 \begin{decision}
  No se rechaza $H_0$.
 \end{decision}

 \begin{conclusion}
  No hay raz\'on para creer que la proporci\'on de adultos que actualmente favorecen la pena
  capital ha aumentado.
 \end{conclusion}

 Finalmente, usando el archivo anexo \texttt{P08\_Prueba\_de\_una\_proporcion\_01.r},
 con los siguientes cambios:
 \begin{verbatim}
> n<-15
> x<-8
> p0<-NULL
> p<-0.4
> alfa<-0.05
> cola<-'S'
> distr<-NULL
 \end{verbatim}
 \vspace{-0.5cm}
 el programa de R lanza el siguiente resultado:
 \begin{verbatim}
     distr   p  n x pMuestral media desv.est alpha
1 Binomial 0.4 15 8 0.5333333     6 1.897367  0.05
                          Estadistico    PValor RegionRechazoX        Resultado
1 Var. binomial X con p= 0.4  y n= 15 0.2131032          >= 10 No se rechaza H0  
 \end{verbatim}
 \vspace{-0.5cm}
 El cual coincide con los resultados obtenidos,
 que es a lo que se quer\'{\i}a llegar.${}_{\blacksquare}$
\end{solucion}
