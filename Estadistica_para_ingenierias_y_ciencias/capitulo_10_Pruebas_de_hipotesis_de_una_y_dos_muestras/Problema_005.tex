\begin{enunciado}
 Repita el ejercicio 10.4 cuando se seleccionan $200$ adultos y la regi\'on de aceptaci\'on se define como $110 \leq x \leq 130$, donde $x$ es el n\'umero de graduados universitarios en nuestra muestra. Utilice la aproximaci\'on normal.
\end{enunciado}

\begin{solucion}
 Usando los t\'erminos del ejercicio 10.4, se tiene ahora que:
 \begin{itemize}
  \item $X \sim b(n,p) \sim n\left(\mu = np, \sigma = \sqrt{npq}\right)$.
  \item $n = 200$.
  \item $x_{\text{inf}} = 110$ y $x_{\text{sup}} = 130$.
 \end{itemize}
 Para el primer supuesto, se tiene adem\'as que
 \begin{itemize}
  \item $p = 0.6$.
  \item $\mu = np = (200)(0.6) = 120$.
  \item $\sigma^2 = npq = 120(0.4) = 48$.
 \end{itemize}
 as\'{\i}, el error Tipo I se calcula como sigue:
 \begin{eqnarray*}
  \alpha & = & P(X < 110) + P(X > 130) = 1 + P(X < 110) - P(X \leq 130) \\
  & \approx & 1 + P\left( Z < \frac{109.5-120}{\sqrt{48}} \right) - P\left(Z < \frac{130.5 - 120}{\sqrt{48}} \right) \\
  & = & 1 + P\left(Z < -\frac{10.5\sqrt{3}}{12} \right) - P\left(Z < \frac{10.5\sqrt{3}}{12} \right) \\
  & = & 1 + P\left(Z < -\frac{21\sqrt{3}}{24} \right) - \left[ 1 - P\left(Z < -\frac{21\sqrt{3}}{24} \right) \right] \\
  & = & 2P\left( Z < -\frac{7\sqrt{3}}{8} \right) \approx 2P(Z < -1.52)
 \end{eqnarray*}
 Esto se puede aproximar usando la Tabla A.1, con lo que se obtiene lo siguiente:
 \begin{equation*}
  \alpha = 2P(Z < -1.52) = 2(0.0643) = 0.1286
 \end{equation*}
 Para el siguiente supuesto, se tiene que
 \begin{itemize}
  \item $p = 0.5$.
  \item $\mu = np = (200)(0.5) = 100$.
  \item $\sigma^2 = npq = 100(0.5) = 50$.
 \end{itemize}
 as\'{\i}, el error Tipo II se calcula como sigue:
 \begin{eqnarray*}
  \beta & = & P(110 \leq X \leq 130) = P(X \leq 130) - P(X < 110) \\
  & \approx & P\left( Z < \frac{130.5-100}{\sqrt{50}} \right) - P\left( Z < \frac{109.5-100}{\sqrt{50}} \right) = P\left( Z < \frac{30.5\sqrt{2}}{10} \right) - P\left( Z < \frac{9.5\sqrt{2}}{10} \right) \\
  & = & P\left(Z < 3.05\sqrt{2}\right) - P\left(Z < 0.95\sqrt{2}\right) \approx P(Z < 4.31) - P(Z < 1.34)
 \end{eqnarray*}
 Esto se puede aproximar usando la Tabla A.1, con lo que se obtiene lo siguiente:
 \begin{equation*}
  \beta \approx P(Z < 4.31) - P(Z < 1.34) \approx 1 - 0.9099 = 0.0901
 \end{equation*}
 y, para el \'ultimo supuesto, se tiene que
 \begin{itemize}
  \item $p = 0.7$.
  \item $\mu = np = (200)(0.7) = 140$.
  \item $\sigma^2 = npq = 140(0.3) = 42$.
 \end{itemize}
 as\'{\i}, el error Tipo II se calcula como sigue:
 \begin{eqnarray*}
  \beta & = & P(X \leq 130) - P(X < 100) \approx P\left( Z < \frac{130.5-140}{\sqrt{42}} \right) - P\left( Z < \frac{109.5-140}{\sqrt{42}} \right) \\
  & = & P\left( Z < -\frac{30.5\sqrt{42}}{42} \right) - P\left( Z < -\frac{9.5\sqrt{42}}{42} \right) \approx P(Z < -1.47) - P(Z < -4.71)
 \end{eqnarray*}
 Esto se puede aproximar usando la Tabla A.1, con lo que se obtiene:
 \begin{equation*}
  \beta \approx P(Z < -1.47) - P(Z < -4.71) \approx 0.0708-0 = 0.0708
 \end{equation*}
 Usando R tambi\'en se puede calcular esta probabilidad usando el script que se encuentra en el archivo \texttt{P02\_Probabilidad\_de\_error\_normal\_1.r}, el cual permite calcular $\alpha$ o $\beta$, o ambos, suponiendo que viene de una distribuci\'on normal. Esto permite hacerse como la aproximaci\'on de pruebas binomiales o directamente de una poblaci\'on normal con desviaci\'on est\'andar conocida, por lo que tambi\'en se permite suponiendo una hipotesis nula o alternativa que precise una proporci\'on o directamente la media de la distribuci\'on.
 \par 
 Los valores modificables por el usuario son: el tama\~no de la muestra, \texttt{n}; el valor cr\'{\i}tico inferior, \texttt{CriticoInf}, o superior, \texttt{CriticoSup}, o ambos, tales que si la cantidad de casos favorables es una cantidad entre \'estas, inclusive, entonces se encuentra en la regi\'on aceptaci\'on; bajo el supuesto de que se d\'e directamente una distribuci\'on normal, se agrega la desviaci\'on est\'andar poblacionale, \texttt{desv}, la media real suponiendo la hip\'otesis nula, \texttt{media0}, o la media real suponiendo una hip\'otesis alternativa, \texttt{media1}, o ambas; o, bien, bajo el supuesto de que sea una aproximaci\'on a la normal de pruebas binomiales, la proporci\'on real suponiendo la hip\'otesis nula, \texttt{p0}, o la proporci\'on real suponiendo una hip\'otesis alternativa, \texttt{p1}, o ambas.
 \par 
 El c\'odigo completo se muestra a continuaci\'on.
 \begin{verbatim}
> n<-200
> CriticoInf<-110
> CriticoSup<-130
> desv<-NULL
> media0<-NULL
> media1<-NULL
> p0<-0.6
> p1<-0.5
> alpha1<-0
> alpha2<-0
> beta1<-0
> beta2<-0
> hipprop0<-FALSE
> hipprop1<-FALSE
> binf<-!is.null(CriticoInf)
> bsup<-!is.null(CriticoSup)
> if(is.null(desv)){
+     if(!is.null(p0)){
+         media0<-n*p0
+         desv0<-sqrt(n*p0*(1-p0))
+         hipprop0<-TRUE
+     }
+     if(!is.null(p1)){
+         media1<-n*p1
+         desv1<-sqrt(n*p1*(1-p1))
+         hipprop1<-TRUE
+     }
+     if(binf) CriticoInf<-CriticoInf-0.5
+     if(bsup) CriticoSup<-CriticoSup+0.5
+ } else{
+     desv0<-desv
+     desv1<-desv
+ }
> balpha<-!is.null(media0)
> bbeta<-!is.null(media1)
> if(binf){
+     if(balpha) alpha1<-pnorm(CriticoInf,media0,desv0)
+     if(bbeta) beta1<-pnorm(CriticoInf,media1,desv1)
+ }
> if(bsup){
+     if(balpha) alpha2<-1-pnorm(CriticoSup,media0,desv0)
+     if(bbeta) beta2<-1-pnorm(CriticoSup,media1,desv1)
+ }
> alpha<-alpha1+alpha2
> beta<-1-(beta1+beta2)
> if(balpha){
+     if(binf&bsup){
+         Resultado1<-data.frame(HipotesisNula
+                                =ifelse(hipprop0,
+                                        paste("p = ",p0),
+                                        paste("mu = ",media0)),
+                                n=n,
+                                media=media0,
+                                desv=desv0,
+                                CríticoInf=CriticoInf,
+                                CriticoSup=CriticoSup,
+                                alpha=alpha)
+     } else{
+         if(binf){
+             Resultado1<-data.frame(HipotesisNula
+                                    =ifelse(hipprop0,
+                                            paste("p = ",p0),
+                                            paste("mu = ",media0)),
+                                    n=n,
+                                    media=media0,
+                                    desv=desv0,
+                                    CriticoInf=CriticoInf,
+                                    alpha=alpha)
+         }
+         else{
+             Resultado1<-data.frame(HipotesisNula
+                                    =ifelse(hipprop0,
+                                            paste("p = ",p0),
+                                            paste("mu = ",media0)),
+                                    n=n,
+                                    media=media0,
+                                    desv=desv0,
+                                    CriticoSup=CriticoSup,
+                                    alpha=alpha)
+         }
+     }
+ } else Resultado1<-NULL
> if(bbeta){
+     if(binf&bsup){
+         Resultado2<-data.frame(HipotesisAlternativa
+                                =ifelse(hipprop1,
+                                        paste("p = ",p1),
+                                        paste("mu = ",media1)),
+                                n=n,
+                                media=media1,
+                                desv=desv1,
+                                CríticoInf=CriticoInf,
+                                CriticoSup=CriticoSup,
+                                beta=beta)
+     } else{
+         if(binf){
+             Resultado2<-data.frame(HipotesisAlternativa
+                                    =ifelse(hipprop1,
+                                            paste("p = ",p1),
+                                            paste("mu = ",media1)),
+                                    n=n,
+                                    media=media1,
+                                    desv=desv1,
+                                    CriticoInf=CriticoInf,
+                                    beta=beta)
+         }
+         else{
+             Resultado2<-data.frame(HipotesisAlternativa
+                                    =ifelse(hipprop1,
+                                            paste("p = ",p1),
+                                            paste("mu = ",media1)),
+                                    n=n,
+                                    media=media1,
+                                    desv=desv1,
+                                    CriticoSup=CriticoSup,
+                                    beta=beta)
+         }
+     }
+ } else Resultado2<-NULL
> if(balpha&bbeta){
+     Resultado<-list(Resultado1,Resultado2)
+     names(Resultado)<-c("Probabilidad de error tipo I",
+                         "Probabilidad de error tipo II")
+ } else{
+     if(balpha){
+         Resultado<-list(Resultado1)
+         names(Resultado)<-c("Probabilidad de error tipo I")
+     } else{
+         Resultado<-list(Resultado2)
+         names(Resultado)<-c("Probabilidad de error tipo II")
+     }
+ }
> Resultado
$`Probabilidad de error tipo I`
  HipotesisNula   n media     desv CríticoInf CriticoSup     alpha
1      p =  0.6 200   120 6.928203      109.5      130.5 0.1296346

$`Probabilidad de error tipo II`
  HipotesisAlternativa   n media     desv CríticoInf CriticoSup       beta
1             p =  0.5 200   100 7.071068      109.5      130.5 0.08954656
 \end{verbatim}
 \vspace{-0.5cm}
 En donde al final se muestral $\alpha$ para el supuesto de $p = 0.6$ y $\beta$ bajo el supuesto de $p = 0.5$, para obtener el valor de $\beta$ para el supuesto $p = 0.7$, se vuelve a ejecutar el script con el siguiente cambio:
 \begin{verbatim}
> n<-200
> CriticoInf<-110
> CriticoSup<-130
> desv<-NULL
> media0<-NULL
> media1<-NULL
> p0<-NULL
> p1<-0.7
 \end{verbatim}
 \vspace{-0.5cm}
 con lo que se obtiene el siguiente resultado:
 \begin{verbatim}
$`Probabilidad de error tipo II`
  HipotesisAlternativa   n media     desv CríticoInf CriticoSup       beta
1             p =  0.7 200   140 6.480741      109.5      130.5 0.07133898
 \end{verbatim}
 \vspace{-0.5cm}
 En resumen, se tiene lo siguiente:
 \begin{itemize}
  \item Bajo el supuesto real $p = 0.6$:
  \begin{itemize}
   \item La aproximaci\'on con las tablas da $\alpha = 0.1286$.
   \item La aproximaci\'on con R da $\alpha = 0.1296346$.
  \end{itemize}

  \item Bajo el supuesto $p = 0.5$:
  \begin{itemize}
   \item La aproximaci\'on con las tablas da $\beta = 0.0901$.
   \item La aproximaci\'on en R da $\beta = 0.08954656$.
  \end{itemize}

  \item Y, bajo el supuesto $p = 0.7$:
  \begin{itemize}
   \item La aproximaci\'on con las tablas da $\beta = 0.0708$.
   \item La aproximaci\'on en R da $\beta = 0.07133898$.
  \end{itemize}
 \end{itemize}
 En conclusi\'on, este nuevo procedimiento a disminuido y las probabilidades de error tipo II, pero la regi\'on de aceptaci\'on hace que la probabilidad de cometer un error tipo I aumente, ya que \'este es considerablemente grande, que es a lo que se quer\'{\i}a llegar.${}_{\blacksquare}$
\end{solucion}

