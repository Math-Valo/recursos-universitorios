\begin{enunciado}
 En un restaurante de carnes asadas una m\'aquina de bebidas gaseosas se ajusta de manera que la cantidad de bebida que sirva est\'e distribuida de forma aproximadamente normal, con una media de $200$ mililitros y una desviaci\'on est\'andar de $15$ mililitros. La m\'aquina se verifica peri\'odicamente tomando una muestra de $9$ bebidas y calculando el contenido promedio. Si $\bar{x}$ cae en el intervalo $191 < \bar{x} < 209$, se considera que la m\'aquina opera de forma satisfactoria; de otro modo, concluimos que $\mu \neq 200$ mililitros.
 \begin{enumerate}
  \item Encuentre la probabilidad de cometer un error tipo I cuando $\mu = 200$ mililitros.
  
  \item Encuentre la probabilidad de cometer un error tipo II cuando $\mu = 215$ mililitros.
 \end{enumerate}
\end{enunciado}

\begin{solucion}
 Sea $X$ la cantidad de mililitros que sirve la m\'aquina de bebidas gaseosas y sea $\overline{X}$ la variable aleatoria de la muestra, del enunciado se tiene lo siguiente, en donde $x_{\text{inf}}$ y $x_{\text{sup}}$ representan el valor cr\'{\i}tico inferior y superior, respectivamente. 
 \begin{itemize}
  \item $X \sim n(\mu, \sigma)$.
  \item $\overline{X} \sim n\left( \mu, \sigma/\sqrt{n} \right)$
  \item $n = 9$.
  \item $\sigma = 15\,$ml.
  \item $\sigma_{\overline{X}} = 15/\sqrt{9} = 15/3 = 5$.
  \item $x_{\text{inf}} = 191$ y $x_{\text{sup}} = 209$.
 \end{itemize}
 Con lo que sea realiza lo pedido en los incisos como sigue.
 \begin{enumerate}
  \item Se tiene lo siguiente bajo el supuesto dado.
  \begin{itemize}
   \item $\mu = 200\,$ml.
  \end{itemize}
  el error tipo I se aproxima usando la tabla A.3 como sigue:
  \begin{eqnarray*}
   \alpha & = & P\left(\overline{X} \leq 191\right) + P\left(\overline{X} \geq 209\right) = P\left( Z < \frac{191 - 200}{5} \right) + P\left( Z > \frac{209 - 200}{5} \right) \\
   & = & P\left( Z < -\frac{9}{5} \right) + P\left(Z > \frac{9}{5}\right) = 2P\left( Z < -\frac{9}{5} \right) = 2P(Z < -1.8) \approx 2(0.0359) \\ 
   & = & 0.0718
  \end{eqnarray*}
  Finalmente, en R se puede calcular esta probabilidad usando el script en el archivo anexo \texttt{P02\_Probabilidad\_de\_error\_normal\_1.r}, cambiando las siguientes l\'{\i}neas de c\'odigo:
  \begin{verbatim}
> n<-9
> CriticoInf<-191
> CriticoSup<-209
> desv<-5
> media0<-200
> media1<-NULL
> p0<-NULL
> p1<-NULL
  \end{verbatim}
  \vspace{-0.5cm}
  con lo que se obtiene
  \begin{verbatim}
$`Probabilidad de error tipo I`
  HipotesisNula n media desv CríticoInf CriticoSup      alpha
1     mu =  200 9   200    5        191        209 0.07186064
  \end{verbatim}
  \vspace{-0.5cm}
  Por lo tanto, se tiene lo siguiente:
  \begin{itemize}
   \item La aproximaci\'on con las tablas da $\alpha = 0.0718$.
   \item La aproximaci\'on con R da $\alpha = 0.07186064$.${}_{\square}$
  \end{itemize}

  \item Suponiendo que
  \begin{itemize}
   \item $\mu = 215$
  \end{itemize}
  el error tipo II se aproxima usando la tabla A.3 como sigue:
  \begin{eqnarray*}
   \beta & = & P\left(191 < \overline{X} < 209\right) = P\left(\frac{191 - 215}{5} < Z < \frac{209 - 215}{5}\right) = P\left( -\frac{24}{5} < Z < -\frac{6}{5} \right) \\
   & = & P( -4.8 < Z < -1.2 ) = P(Z < -1.2) - P(Z \leq -4.8) \approx 0.1151 - 0 = 0.1151
  \end{eqnarray*}
  Finalmente, en R se puede calcular esta probabilidad usando el script en el archivo anexo \texttt{P02\_Probabilidad\_de\_error\_normal\_1.r}, cambiando las siguientes l\'{\i}neas de c\'odigo:
  \begin{verbatim}
> n<-9
> CriticoInf<-191
> CriticoSup<-209
> desv<-5
> media0<-NULL
> media1<-215
> p0<-NULL
> p1<-NULL
  \end{verbatim}
  \vspace{-0.5cm}
  con lo que se obtiene
  \begin{verbatim}
$`Probabilidad de error tipo II`
  HipotesisAlternativa n media desv CríticoInf CriticoSup      beta
1            mu =  215 9   215    5        191        209 0.1150689
  \end{verbatim}
  \vspace{-0.5cm}
  Por lo tanto, se tiene lo siguiente:
  \begin{itemize}
   \item La aproximaci\'on con las tablas da $\beta = 0.1151$.
   \item La aproximaci\'on con R da $\beta = 0.1150689$.
  \end{itemize}
  que es a lo que se quer\'{\i}a llegar.${}_{\blacksquare}$
 \end{enumerate}
\end{solucion}
