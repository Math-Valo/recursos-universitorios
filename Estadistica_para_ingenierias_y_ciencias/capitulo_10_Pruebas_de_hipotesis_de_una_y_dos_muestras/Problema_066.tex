\begin{enunciado}
 En un invierno con epidemia de gripe, una compa\~n\'{\i}a farmac\'eutica
 bien conocida estudi\'o a $2\,000$ beb\'es,
 para determinar si el nuevo medicamento de la comp\~n\'{\i}a era eficaz despu\'es
 de dos d\'{\i}as.
 Entre $120$ beb\'es que ten\'{\i}an gripe y se les suministr\'o el medicamento,
 $29$ se curaron dentro de dos d\'{\i}as.
 Entre $280$ beb\'es que ten\'{\i}an gripe pero que no recibieron el f\'armaco,
 $56$ se curaron dentro de dos d\'{\i}as.
 ¿Hay alguna indicaci\'on significativa que apoye la afirmaci\'on
 de la compa\~n\'{\i}a de la efectividad del medicamento?
\end{enunciado}

\vspace{2cm}
\begin{solucion}
 El m\'etodo para la soluci\'on de este problema ser\'a a trav\'es del valor $P$,
 la probabilidad de cometer un error de tipo I ante las muestras obtenidas.
 \begin{datos}
  $\phantom{0}$
  \begin{itemize}
   \item $n_1 = 120$ y $n_2 = 280$.
   \item $x_1 = 29$ y $x_2 = 56$.
  \end{itemize}
  Adem\'as, se pueden calcular las proporciones estimadas de cada muestra
  y la estimaci\'on combinada de la proporci\'on a la que se supone que son iguales
  en ambas poblaciones, como se muestra a continuaci\'on:
  \begin{itemize}
   \item $\widehat{p}_1 = \frac{29}{120} = 0.241\bar{6}$.
   \item $\widehat{p}_2 = \frac{56}{280} = \frac{2}{10} = 0.2$.
   \item $\widehat{p} = \frac{x_1+x_2}{n_1+n_2} = \frac{29+56}{120+280}
   = \frac{85}{400} = \frac{17}{80} = 0.2125$.
  \end{itemize}
 \end{datos}

 \begin{hipotesis}
  \begin{eqnarray*}
   H_0: p_1 & \leq & p_2 \\
   H_1: p_1 &  >   & p_2
  \end{eqnarray*}
 \end{hipotesis}

 \begin{estadistico}
  \begin{eqnarray*}
   z & \approx &
   \frac{
   \widehat{p}_1 - \widehat{p}_2
   }{
   \sqrt{\widehat{p}\widehat{q}\left(1/n_1+1/n_2\right)}
   }
   = \frac{
   \displaystyle{ \frac{29}{120} - \frac{2}{10}}
   }{
   \displaystyle{ \sqrt{
   \left(\frac{17}{80}\right)\left(\frac{63}{80}\right)
   \left(\frac{1}{120} + \frac{1}{280}\right)
   }}}
   = \frac{
   \displaystyle{\frac{29-24}{120}}
   }{\displaystyle{
   \sqrt{\left(\frac{17\cdot 7\cdot 3^2}{80^2}\right)\left( \frac{7+3}{840} \right)}
   }} \\
   & = & \frac{\displaystyle{\frac{\cancel{5}}{\cancelto{24}{120}}}}{\displaystyle{
   \frac{3}{80} \sqrt{
   \frac{
   17\cdot \cancel{7} \cdot \cancel{10}
   }{
   \cancel{7}\cdot 2^2 \cdot 3 \cdot \cancel{2\cdot 5}
   } }
   }}
   = \frac{80}{\displaystyle{
   \frac{ \cancelto{12}{24}(3)}{\cancel{2}}\sqrt{\frac{17}{3}}
   }}
   = \frac{\cancelto{20}{80}\sqrt{51}}{\cancelto{9}{36}(17)}
   = \frac{20\sqrt{51}}{153} \approx 0.93352
  \end{eqnarray*}
 \end{estadistico}

 \begin{valorp}
  De la tabla A.3 se tiene que:
  \begin{equation*}
   P(Z > z) = 1 - P(Z < z) \approx 1-P(Z<0.93) \approx 1 - 0.8238
   = 0.1762
  \end{equation*}
 \end{valorp}

 \begin{conclusion}
  Las muestras se\~nalan, con ayuda del c\'alculo del $P$-valor,
  que no alguna indicaci\'on significativa que apoye la afirmaci\'on
  de la compa\~n\'{\i}a acerca de la efectividad del medicamente;
  es decir, no se puede asegurar que el medicamente sea eficaz
  para curar la gripe en beb\'es.
 \end{conclusion}

 Finalmente, usando el archivo anexo \texttt{P09\_Prueba\_de\_dos\_proporciones\_01.r}, con los siguientes cambios:
 \begin{verbatim}
> n1<-120
> n2<-280
> x1<-29
> x2<-56
> p1<-NULL
> p2<-NULL
> alfa<-NULL
> alternativa<-'>'
 \end{verbatim}
 \vspace{-0.5cm}
 el programa de R lanza el siguiente resultado:
 \begin{verbatim}
  alternativa  n1  n2 x1 x2        p1  p2 pEstimada    DifProp  error.est alpha
1     p1 > p2 120 280 29 56 0.2416667 0.2    0.2125 0.04166667 0.04463393  0.05
     PValor Estadistico RegionRechazoZ
1 0.1752758   0.9335201   >= 1.6448536
 \end{verbatim}
 \vspace{-0.5cm}
 El cual coincide con los resultados obtenidos,
 que es a lo que se quer\'{\i}a llegar.${}_{\blacksquare}$
\end{solucion}
