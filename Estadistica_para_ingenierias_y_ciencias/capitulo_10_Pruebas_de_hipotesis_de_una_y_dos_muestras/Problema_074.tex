\begin{enunciado}
 En el ejercicio 10.41 de la p\'agina 359, pruebe la hip\'otesis
 al nivel de significancia de $0.05$ de que $\sigma_1^2 = \sigma_2^2$
 contra la alternativa de que $\sigma_1^2 \neq \sigma_2^2$,
 donde $\sigma_1^2$ y $\sigma_2^2$ son las varianzas
 para el n\'umero de organismos por metro cuadrado
 en los dos diferentes lugares de Cedar Run.
\end{enunciado}

\begin{solucion}
 Resumiendo los datos del ejercicio 10.41, se tiene lo siguiente:
 \begin{itemize}
  \item $X_i \sim n\left( \mu_i, \sigma_i \right)$,
  para cada $i \in \{ 1,2 \}$.
  \item $n_1 = 16$ y $n_2 = 12$.
  \item $s_1^2 = 62\,005\,060$
  y $s_2^2 = \frac{202\,881\,875}{33} = 6\,147\,935.\overline{60}$.
 \end{itemize}

 \begin{hipotesis}
  \begin{eqnarray*}
   H_0: \sigma_1^2 &  =   & \sigma_2^2 \\
   H_1: \sigma_1^2 & \neq & \sigma_2^2
  \end{eqnarray*}
 \end{hipotesis}

 \begin{significancia}
  $\alpha = 0.05$.
 \end{significancia}

 \begin{region}
  En la tabla A.6 \'unicamente se tiene valores cr\'{\i}ticos
  para significancias unilaterales de $\alpha=0.05$ o $\alpha=0.01$
  y para significancias bilaterales de $\alpha=0.1$ y $\alpha=0.02$,
  por lo que no es posible obtener de ah\'{\i} los valores cr\'{\i}ticos
  $f_{\alpha/2,n-1,n-2}=f_{0.05,15,11}$
  ni tampoco de $f_{1-\alpha/2,n-1,n-2}=f_{0.95,15,11}$.
  Por lo tanto, se tendr\'a que obtener dichos valores
  con las l\'{\i}neas de c\'odigo escritas en R
  que se muestran a continuaci\'on con sus respectivos resultados:
  \begin{verbatim}
> qf(0.025,16-1,12-1)
[1] 0.3324659
> qf(0.025,16-1,12-1,lower.tail = F)  # = 1/qf(0.025,12-1,16-1)
[1] 3.329935
  \end{verbatim}
  \vspace{-0.5cm}
  por lo que la regi\'on de rechazo est\'a dado
  para $0.3324659 < f < 3.329935$, donde $f = \frac{s_1^2}{s_2^2}$.
 \end{region}

 \begin{estadistico}
  \begin{equation*}
   f = \frac{s_1^2}{s_2^2}
   = \frac{62\,005\,060}{\frac{202\,881\,875}{33}}
   = \frac{33(\cancelto{12\,401\,012}{62\,005\,060})}
   {\cancelto{40\,576\,375}{202\,881\,875}} \hspace{1cm}
   = \frac{409\,233\,396}{40\,576\,375} \approx 10.0855
  \end{equation*}
 \end{estadistico}

 \begin{decision}
  Se rechaza $H_0$ a favor de $H_1$.
 \end{decision}

 \begin{conclusion}
  Se concluye entonces que son diferentes las varianzas
  del n\'umero de organismos por metro cuadrado
  en los dos diferentes lugares de Cedar Run.
 \end{conclusion}

 En el c\'odigo registrado en el archivo anexo
 \texttt{P14\_Prueba\_de\_dos\_varianzas\_02.r}, en R,
 se realiza este procedimiento.
 El c\'odigo permite modificar los valores iniciales
 que corresponden a:
 \texttt{datos}, que guarda los datos de la lectura
 de un archivo,
 en este caso se lee el arhivo \texttt{DB05\_Problema\_041.csv},
 y este \'ultimo nombre es el que se modifica 
 para leer otros archivos;
 \texttt{varInteres} para indicar el nombre de la columna
 que corresponden a los datos en la base anterior;
 \texttt{varSel} para indicar el nombre de la columna
 en donde se indica a cu\'al muestra corresponde cada dato;
 \texttt{alfa} para el nivel de significancia;
 \texttt{cola} para indicar si la prueba es de dos colas,
 con \texttt{'D'}, de cola inferior, con \texttt{'I'},
 o de cola superior, con \texttt{'S'}.
 \par 
 El programa espera todos los datos, excepto \texttt{alfa} que es opcional.
 En caso de no fijar una probabilidad de cometer un error de tipo I
 a esta variable, se le asigna el valor \texttt{NULL}.
 \par 
 La prueba de hip\'otesis siempre usar\'a un estad\'{\i}stico
 con distribuci\'on $f$ de Fisher.
 Para ello, se est\'a suponiendo de antemano la normalidad
 de las poblaciones de donde provienen las muestras.
 \par 
 Independientemente del tipo de prueba, el resultado muestra lo siguiente:
 \texttt{variable} para conocer lo que est\'a midiendo los datos,
 incluyendo el tipo de unidad que se usa;
 \texttt{Freq} indica el tama\~no de las unidades experimentales,
 es decir, la suma de los tama\~nos muestrales;
 \texttt{n1} y \texttt{n2} indica el tama\~no de cada muestra;
 \texttt{media1} y \texttt{media2} indica la media muestral de cada muestra;
 \texttt{varianza1} y \texttt{varianza2} indica la varianza muestral
 de cada muestra;
 \texttt{v1} y \texttt{v2} indica los grados de libertad usados
 en el estad\'{\i}stico $f$ de Fisher;
 \texttt{alfa} para el nivel de significancia dado,
 el cual muestra por defecto $0.05$
 en caso de asignarle \texttt{NULL} a \texttt{alfa};
 \texttt{PValor} indica el valor $P$, la probabilidad
 de haber obtenido la proporci\'on de varianzas muestrales
 que se obtuvo suponiendo que la hip\'otesis nula sea cierta;
 \texttt{Estadistico} indica el valor resultante
 del estad\'{\i}stico de prueba;
 \texttt{RegionRechazo} indica la regi\'on de rechazo del estad\'{\i}stico
 de prueba, el cual es precisamente la regi\'on de rechazo de la proporci\'on
 de varianzas muestrales.
 \par 
 Adem\'as, seg\'un el tipo de prueba, pueden darse el valor
 de \texttt{Resultado} para indicar si se rechaza o no
 la hip\'otesis nula, que aparece al final de los resultados
 cuando se asigna un valor a \texttt{alfa}.
 \par 
 El c\'odigo junto con el resultado se muestra a continuaci\'on:
 \begin{verbatim}
> datos<-read.csv("DB05_Problema_041.csv",sep=";",encoding="UTF-8")
> varInteres<-c("AcidoAscórbico.mgpcml")
> varSel<-c("Estación")
> alfa<-0.05
> cola<-'D'
> w<-data.frame(factor(datos[,varSel]))
> names(w)<-varSel
> varBin<-c(varSel)
> datos<-data.frame(datos,w)
> if (length(varBin)<1){
+   stop("Debe al menos indicar una variable binaria")
+ }else{
+   sonbinarios<-ifelse(length(table(datos[,varBin]))!=2,1,0)
+ }
> if (sonbinarios!=0)  stop("La variable no es binaria")
> valores<-unlist(datos[,c(varInteres)])
> variables<-factor(rep(varInteres,each=dim(datos)[1]))
> agrupaciones<-data.frame(datos[1:dim(datos)[1],varBin])
> names(agrupaciones)<-varBin
> datos2<-data.frame(agrupaciones,variable=variables,valor=valores)
> propVar<-function(l,cola,alfa=0.05){
+   x<-l[[1]][!is.na(l[[1]])]
+   y<-l[[2]][!is.na(l[[2]])]
+   n1<-length(x)
+   n2<-length(y)
+   m1<-mean(x)
+   m2<-mean(y)
+   s1<-var(x)
+   s2<-var(y)
+   v1<-n1-1
+   v2<-n2-1
+   if ( n1< 2 | n2 < 2){
+     r<-data.frame(n1=n1,n2=n2,
+                   media1=round(m1,7),media2=round(m2,7),
+                   varianza1=round(s1,7),varianza2=round(s2,7),
+                   v1=v1,v2=v2,
+                   alpha=alfa,
+                   PValor=NA,
+                   Estadistico=NA,
+                   RegionRechazo=NA
+                   )
+   }else{
+     estadistico = round(s1/s2,7)
+     if(cola=='D'){
+       regionL<-round(qf(alfa/2,v1,v2),7)
+       regionU<-round(qf(alfa/2,v1,v2,lower.tail=F),7)
+       pvalor<-2*min(pf(estadistico,v1,v2,lower.tail=F),
+                     pf(estadistico,v1,v2))
+       r<-data.frame(n1=n1,n2=n2,
+                     media1=round(m1,7),media2=round(m2,7),
+                     varianza1=round(s1,7),varianza2=round(s2,7),
+                     v1=v1,v2=v2,
+                     alpha=alfa,
+                     PValor=round(pvalor,7),
+                     Estadistico=estadistico,
+                     RegionRechazo=paste("<",regionL,"y >",regionU)
+                     )
+     }else{
+       if(cola=='I'){
+         region<-round(qf(alfa,v1,v2),7)
+         pvalor<-round(pf(estadistico,v1,v2),7)
+         r<-data.frame(n1=n1,n2=n2,
+                       media1=round(m1,7),media2=round(m2,7),
+                       varianza1=round(s1,7),varianza2=round(s2,7),
+                       v1=v1,v2=v2,
+                       alpha=alfa,
+                       PValor=pvalor,
+                       Estadistico=estadistico,
+                       RegionRechazo=paste("<",region)
+                       )
+       }else{
+         region<-round(qf(alfa,v1,v2,lower.tail=F),7)
+         pvalor<-round(pf(estadistico,v1,v2,lower.tail=F),7)
+         r<-data.frame(n1=n1,n2=n2,
+                       media1=round(m1,7),media2=round(m2,7),
+                       varianza1=round(s1,7),varianza2=round(s2,7),
+                       v1=v1,v2=v2,
+                       alpha=alfa,
+                       PValor=pvalor,
+                       Estadistico=estadistico,
+                       RegionRechazo=paste(">",region)
+                       )
+       }
+     }
+   }
+   return(r)
+ }
> listaG<-datos2[,c(varBin,"valor")]
> listaG1<-split(listaG$"valor", listaG[,varBin])
> if(as.matrix(listaG)[1]==names(listaG1[2])){
+   listaG1=listaG1[c(2,1)]
+ }
> listaG1<-list(listaG1)
> names(listaG1)<-varInteres
> if(is.null(alfa)){
+   Test<-t(sapply(listaG1,propVar,cola))
+ }else{
+   Test<-t(sapply(listaG1,propVar,cola,alfa=alfa))
+ }
> t1<-as.data.frame.table(table(datos2[,c("variable")]))
> identif<-t1[t1$Freq>0,]
> names(identif)[names(identif)=="Var1"]<-"variable"
> Test<-data.frame(identif,Test)
> if(!is.null(alfa)){
+   Test$Resultado<-ifelse(Test$PValor>=alfa,
+                          "No se rechaza H0","Se rechaza H0")
+   Test$Resultado[is.na(Test$Resultado)]<-"Pocos datos"
+ }
> Test
               variable Freq n1 n2 media1   media2 varianza1 varianza2 v1 v2
1 AcidoAscórbico.mgpcml   28 16 12 9897.5 4120.833  62005060   6147936 15 11
  alpha   PValor Estadistico             RegionRechazo     Resultado
1  0.05 0.000452    10.08551 < 0.3324659 y > 3.3299348 Se rechaza H0
 \end{verbatim}
 \vspace{-0.5cm}
 El cual coincide con los c\'alculos obtenidos,
 adem\'as de ofrecer m\'as informaci\'on como las medias muestrales,
 los cuales pueden servir en futuros c\'alculos,
 as\'{\i} como tambi\'en ofrecer valores m\'as precisos
 de lo que se hubiese obtenido si realizar a mano los c\'alculos,
 como, la regi\'on de rechazo que no se encontraba en la tabla A.6,
 con lo que se prueba la eficiencia del programa,
 que es a lo que se quer\'{\i}a llegar.${}_{\blacksquare}$
\end{solucion}
