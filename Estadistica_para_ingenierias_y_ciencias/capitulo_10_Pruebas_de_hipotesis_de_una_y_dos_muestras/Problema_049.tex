\begin{enunciado}
 ¿Qu\'e tan grande se requiere que sea la muestra del ejercicio 10.22, si la potencia de nuestra prueba debe ser $0.95$ cuando la estatura promedio real difiere de $162.5$ en $3.1$ cent\'{\i}metros?
\end{enunciado}

\begin{solucion}
 Usando los datos del ejercicio 10.22 y la informaci\'on del enunciado, se tiene los siguientes datos y supuestos:
 \begin{itemize}
%   \item $\alpha = $.
  \item $1 - \beta = 0.95$, entonces $\beta = 0.05$.
  \item $\mu = 162.5$, seg\'un la hip\'otesis nula.
  \item $\delta = 3.1$, para una alternativa espec\'{\i}fica.
  \item $\sigma = 6.9$.
 \end{itemize}
 El valor de $\alpha$ es requerido para estos c\'alculos; sin embargo, se desconoce. Por lo tanto, se supondr\'a que $\alpha$ es alg\'un valor entre los dos m\'as usados; es decir, se supondr\'a que:
 \begin{itemize}
  \item $\alpha \in [ 0.01, 0.05 ]$
 \end{itemize}
 Entonces, de la tabla A.3, se deduce adem\'as que:
 \begin{itemize}
  \item $z_{\alpha/2} \in [1.96, 2.575]$.
  \item $z_{\beta} = 1.645$.
 \end{itemize}
 Entonces se puede calcular el tama\~no de muestra requerido para la calidad buscada en la prueba con la siguiente f\'ormula:
 \begin{equation*}
  n =
  \left\lceil
  \frac{\left( z_{\alpha/2} + z_{\beta} \right)^2 \sigma^2}{\delta^2}
  \right\rceil
  = \left\lceil
  \frac{\left( z_{\alpha/2} + 1.645 \right)^2 (6.9)^2}{3.1^2}
  \right\rceil
  = \left\lceil
  \frac{\left( z_{\alpha/2} + 1.645 \right)^2 \times 47.61}{9.61}
  \right\rceil
 \end{equation*}
 Por lo que, en el caso en que $\alpha = 0.01$, se obtiene lo siguiente:
 \begin{eqnarray*}
  n =
  \left\lceil \frac{\left( 2.575 + 1.645 \right)^2 \times 47.61}{9.61} \right\rceil
  = \left\lceil
  \frac{4.22^2\times 4\,761}{961}
  \right\rceil
  = \left\lceil \frac{84\,785.7924}{961} \right\rceil
  = \lceil 88.226631 \rceil = 89
 \end{eqnarray*}
 Por otro lado, en el caso en que $\alpha = 0.05$, se tienen los siguiente c\'alculos:
 \begin{eqnarray*}
  n =
  \left\lceil \frac{(1.96 + 1.28)^2 \times 47.61}{9.61} \right\rceil
  = \left\lceil \frac{3.605^\times 4\,761}{961} \right\rceil
  = \left\lceil \frac{61\,874.07502}{961} \right\rceil
  = \lceil 64.385\ldots \rceil = 65
 \end{eqnarray*}
 Sin embargo, en el ap\'endice del libro se indica que el resultado
 del c\'alculo del tama\~no de muestra, siendo \'este:
 $n = \lceil 78.28 \rceil = 79$. 
 Entonces, haciendo un proceso inverso, se puede deducir el valor que se esperaba de $\alpha$, esto es:
 \begin{eqnarray*}
  & & 78.28 \approx \frac{\left( z_{\alpha/2} + 1.645 \right)^2 \times 47.61}{9.61} \\
  \Leftrightarrow & z_{\alpha/2} &
  \approx \sqrt{ \frac{78.28 \times 9.61}{47.61} } - 1.645 
  = \sqrt{\frac{752.2708}{47.61}} - 1.645
  = \sqrt{15.80068893\ldots} - 1.645 \\
  & & \approx 3.975 - 1,.645 = 2.33 \approx z_{0.01}
 \end{eqnarray*}
 Luego, entonces, $\alpha = 2(0.01) = 0.02$ es el valor que se deb\'{\i}a de suponer.
 \par 
 Por lo tanto, a partir de un tama\~no muestral de $n = 79$
 se puede realizar una prueba de hip\'otesis
 de que la estatura media de mujeres en el grupo actual de primer a\~no
 de la universidad a la que hace referencia el problema 10.22
 no ha cambiado de 162.5 cm, esto es $\mu = 162.5$,
 con un nivel de significancia de $0.02$
 considerando una potencia de prueba de $0.95$
 para la hip\'otesis alternativa de que ahora hay una diferencia de $3.1$ cm,
 esto es que $\left| \mu - 162.5 \right| = |\delta| = 3.1$.${}_{\blacksquare}$
\end{solucion}
