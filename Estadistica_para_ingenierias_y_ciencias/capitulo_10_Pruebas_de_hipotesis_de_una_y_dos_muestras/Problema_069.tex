\begin{enunciado}
 Se deben supervisar las aflotoxinas ocasionadas por moho en cosechas de cacahuate
 en Virgnia.
 Una muestra de $64$ lotes de cacahuate revela niveles de $24.17$ ppm,
 en promedio, con una varianza de $4.25$ ppm.
 Pruebe la hip\'otesis de que $\sigma^2 = 4.2$ ppm
 con la alternativa de que $\sigma^2 \neq 4.2$ ppm.
 Utilice un valor $P$ en sus conclusiones.
\end{enunciado}

\begin{solucion}
 \begin{datos}
  $\phantom{0}$
  \begin{itemize}
%    \item $X \sim n\left( \mu, \sigma \right)$.
   \item $n = 64$.
   \item $\bar{x} = 24.17$
   \item $s^2 = 4.25$.
  \end{itemize}
 \end{datos}

 \begin{hipotesis}
  \begin{eqnarray*}
   H_0: \sigma^2 &  =   & 4.2 \\
   H_1: \sigma^2 & \neq & 4.2
  \end{eqnarray*}
 \end{hipotesis}

 \begin{estadistico}
  \begin{equation*}
   \chi^2 = \frac{(n-1)s^2}{\sigma_0^2}
   = \frac{63(4.25)}{4.2} = \frac{63(85)}{84} = \frac{3(85)}{4}
   = \frac{255}{4} = 63.75
  \end{equation*}
 \end{estadistico}

 \begin{valorp}
  De la tabla A.5 no se puede obtener el valor deseado,
  ya que la cantidad m\'axima de grados de libertad que contiene es de $60$,
  entonces se apoyar\'a para el c\'alculo del valor $P$ el software
  estad\'{\i}stico R con la probabilidad de obtener un valor mayor o igual
  al estad\'{\i}stico, como se muestra a continuaci\'on:
  \begin{verbatim}
> pchisq(63.75,63,lower.tail = F)
[1] 0.4498938
  \end{verbatim}
  Luego entonces, el valor $P$ buscado se calcula como sigue:
  \begin{equation*}
   2P\left( X^2 > \chi_{\alpha/2}^2 \right) \approx 2(0.44989) = 0.89978
  \end{equation*}
 \end{valorp}

 \begin{conclusion}
  Como el valor $P$ es muy alto, se tiene que no hay pruebas para rechazar
  la hip\'otesis nula y se concluye que los niveles de aflotoxinas ocasionadas
  por moho en cocsechas de cacahuate en Virginia mantienen una varianza
  de $4.2$ partes por mill\'on.
 \end{conclusion}

 Finalmente, usando el archivo anexo \texttt{P11\_Prueba\_de\_una\_varianza\_02.r}, con los siguientes cambios:
 \begin{verbatim}
> n<-64
> sigma2<-4.2
> s2<-4.25
> alfa<-NULL
> cola<-'D'
 \end{verbatim}
 \vspace{-0.5cm}
 el programa de R lanza el siguiente resultado:
 \begin{verbatim}
   n  H0 var.muestral grados error.est alpha    PValor Estadistico
1 64 4.2         4.25     63 0.0666667  0.05 0.8997876       63.75
              RegionRechazoJi            RegionRechazoX
1 < 42.9502749 y > 86.8295906 < 2.8633517 y > 5.7886394
 \end{verbatim}
 \vspace{-0.5cm}
 El cual coincide con los resultados obtenidos,
 que es a lo que se quer\'{\i}a llegar.
 Como nota final, se debe recordar que el m\'etodo usado para la pruebas
 de hip\'otesis requer\'{\i}a de la suposici\'on de que poblaci\'on se distribuya
 normalmente, por lo que, a falta de la menci\'on de dicha distribuci\'on,
 se ha hecho dicha supoci\'on impl\'{\i}citamente.${}_{\blacksquare}$
\end{solucion}
