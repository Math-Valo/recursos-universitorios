\begin{enunciado}
 Se lleva a cabo un estudio en Indiana, Kentucky y Ohio,
 para determinar la postura de los votantes con respecto al transporte
 escolar.
 Una encuesta de $200$ votantes de cada uno de estos estados
 da los siguientes resultados:
 \begin{center}
  \begin{tabular}{lccc}
   & \multicolumn{3}{c}{\textbf{Postura del votante}} \\
   \cline{2-4}
   & & \textbf{No} & \\
   \textbf{Estado} & \textbf{Apoya} & \textbf{apoya} & \textbf{Indeciso} \\
   \hline 
   Indiana & $82$ & $97$ & $21$ \\
   Kentucky & $107$ & $66$ & $27$ \\
   Ohio & $93$ & $74$ & $33$
  \end{tabular}
 \end{center}
 Con un nivel de significancia de $0.025$, pruebe la hip\'otesis nula
 de que las proporciones de votantes dentro de cada categor\'{\i}a
 de postura son las mismas para cada uno de los tres estados.
\end{enunciado}

\begin{solucion}
 \begin{datos}
  $\phantom{0}$
  \begin{itemize}
   \item Tamaño de muestra total: $600$.
   \item Votantes encuestados de Indiana: $200$.
   \item Votantes encuestados de Kentucky: $200$.
   \item Votantes encuestados de Ohio: $200$.
   \item Votantes a favor del transporte escolar: $82+107+93 = 282$.
   \item Votantes en contra del transporte escolar: $97+66+74 = 237$.
   \item Votantes indecisos: $21+27+33 = 81$.
   \item Frecuencias observadas y esperadas: $o_{i,j}$
   y $e_{i,j}=\frac{R_i C_j}{n}$, respectivamente,
   donde $R_i$ y $C_j$ son los marginales del rengl\'on $i$ y la columna $j$,
   respectivamente, y $n$ es el total de toda la muestra.
   As\'{\i}, pues, redondeando a un decimal, se muestra el resumen 
   en la siguiente tabla,
   en donde aparece entre par\'entesis la frecuencia esperada
   y a la izquierda el valor observado:
   \begin{center}
    \begin{tabular}{lccc|c}
     & \multicolumn{3}{c}{\textbf{Postura del votante}} \\
     \cline{2-4}
     & & \textbf{No} & & \textbf{Marginal} \\
     \textbf{Estado} & \textbf{Apoya} & \textbf{apoya} & \textbf{Indeciso} &
     \textbf{por estado} \\
     \hline 
     Indiana & $82 (94)$ & $97 (79)$ & $21 (27)$ & $200$ \\
     Kentucky & $107 (94)$ & $66 (79)$ & $27 (27)$ & $200$ \\
     Ohio & $93 (94)$ & $74 (79)$ & $33 (27)$ & $200$ \\
     \hline 
     \textbf{Marginal} & \multirow{2}{*}{$282$} & \multirow{2}{*}{$237$} &
     \multirow{2}{*}{$81$} & \textbf{TOTAL} \\
     \textbf{por postura} & & & & $n=600$
    \end{tabular}
   \end{center}
   \item Tama\~no de la tabla de contingencia: $r\times c = 3\times 3$.
   \item Grados de libertad de la prueba $\chi^2$: $v = (r-1)(c-1) = 4$.
  \end{itemize}
 \end{datos}
 
 \begin{hipotesis}
  \begin{eqnarray*}
   H_0: & & \text{Los votantes por estado son homog\'eneos respecto a las posturas.} \\
   H_1: & & \text{Los votantes por estado no son homog\'eneos respecto a las posturas.}
  \end{eqnarray*}
 \end{hipotesis}

 \begin{significancia}
  $\alpha = 0.025$.
 \end{significancia}

 \begin{region}
  De la tabla A.5, se tiene el valor cr\'{\i}tico
  $\chi^2_{\alpha,v} = \chi^2_{0.025,4} \approx 11.143$,
  por lo que la regi\'on de rechazo est\'a dado
  para $\chi^2 > 11.143$, donde
  $\chi^2 = \sum_{i} \frac{\left( o_i - e_i \right)^2}{e_i}$.
 \end{region}

 \begin{estadistico}
  \begin{eqnarray*}
   \chi^2 & = & \sum_{i} \frac{\left( o_i - e_i \right)^2}{e_i} \\
   & = & \frac{(82 - 94)^2}{94} + \frac{(107 - 94)^2}{94} +
   \frac{(93 - 94)^2}{94} + \frac{(97 - 79)^2}{79} + 
   \frac{(66 - 79)^2}{79} + \\
   & & \frac{(74 - 79)^2}{79} + \frac{(21 - 27)^2}{27} +
   \frac{(27 - 27)^2}{27} + \frac{(33 - 27)^2}{27} \\
   & = & \frac{144 + 169 + 1}{94} + \frac{324 + 169 + 25}{79} + 
   \frac{36 + 0 + 36}{27}
   = \frac{314}{94} + \frac{518}{79} + \frac{72}{27} \\
   & \approx & 3.3404 + 6.557 + 2.6667 = 12.5641
  \end{eqnarray*}
 \end{estadistico}

 \begin{decision}
  Se rechaza $H_0$ a favor de $H_1$.
 \end{decision}

 \begin{conclusion}
  A un nivel de significancia de $0.025$, los resultados arrojan
  evidencia suficiente para decir que no todas
  las proporciones de votantes dentro de cada categor\'{\i}a
  de postura son las mismas para cada uno de los tres estados.
 \end{conclusion}

 Finalmente, usando el archivo anexo
 \texttt{P18\_Prueba\_de\_independencia\_y\_homogeniedad\_01.r},
 que a su vez requiere los datos del archivo
 \texttt{BD34\_Problema\_098.csv}, con los siguientes cambios:
 \begin{verbatim}
> datos<-read.csv("DB34_Problema_098.csv",sep=";",encoding="UTF-8")
> varInteres<-c("Estado","Postura")
> varFrecuencia<-"Frecuencia"
> pruebas<-c(1,2,3)
 \end{verbatim}
 \vspace{-0.5cm}
 el programa de R lanza el siguiente resultado:
 \begin{verbatim}
$tabla
          Postura
Estado     Apoya Indeciso No apoya
  Indiana     82       21       97
  Kentucky   107       27       66
  Ohio        93       33       74

$listaPruebas
$listaPruebas[[1]]

	Pearson's Chi-squared test

data:  tbl1
X-squared = 12.564, df = 4, p-value = 0.01361


$listaPruebas[[2]]

	Log likelihood ratio (G-test) test of independence without correction

data:  tbl1
Log likelihood ratio statistic (G) = 12.433, X-squared df = 4, p-value =
0.0144


$listaPruebas[[3]]

	Log likelihood ratio (G-test) test of independence with Williams' correction

data:  tbl1
Log likelihood ratio statistic (G) = 12.357, X-squared df = 4, p-value =
0.01488
 \end{verbatim}
 \vspace{-0.5cm}
 Lo cual coincide con los resultados obtenidos,
 adem\'as de brindar m\'as informaci\'on como el $P-$valor
 y las pruebas con otros estad\'{\i}sticos,
 que es lo que se quer\'{\i}a llegar.${}_{\blacksquare}$
\end{solucion}
