\begin{enunciado}
 Un experto en marketing de una compa\~n\'{\i}a fabricante de pasta
 considera que $40\%$ de los amantes de la pasta prefieren la lasagna.
 Si $9$ de $20$ amantes de la pasta eligen lasagna sobre otras pastas,
 ¿qu\'e se puede concluir acerca de la afirmaci\'on del experto?
 Utilice un nivel de significancia de $0.05$.
\end{enunciado}

\begin{solucion}
 \begin{datos}
  $\phantom{0}$
  \begin{itemize}
   \item $n = 20$.
   \item $x = 9$.
  \end{itemize}
 \end{datos}

 \begin{hipotesis}
  \begin{eqnarray*}
   H_0: p & = & 0.4 \\
   H_1: p & > & 0.4
  \end{eqnarray*}
 \end{hipotesis}

 \begin{significancia}
  $\alpha = 0.05$
 \end{significancia}

 \begin{estadistico}
  Variable binomial $X$ con $p = 0.4$ y $n = 20$.
 \end{estadistico}

 \begin{valorp}
  De la tabla A.1, se tiene la siguiente aproximaci\'on:
  \begin{equation*}
   P\left( X \geq 9 | p = 0.4 \right) = 1 - P\left( X < 9 | p = 0.4 \right)
   = 1 - \sum_{i=0}^{8} b\left( x; 20, 0.4 \right)
   \approx 1 - 0.5956 = 0.4044 > 0.05
  \end{equation*}
  Mientras que el valor preciso se obtiene como sigue:
  \begin{eqnarray*}
   P\left( X \geq 9 | p = 0.4 \right) & = & 1 - \sum_{i=0}^{8} b\left( x; 20, 0.4 \right)
   = 1 - \frac{1}{10^{20}} \sum_{i=0}^{8} \binom{20}{x} 4^x 6^{20-x} \\
   & = & 1 - \frac{1}{10^{20}}\left( 6^{20} + 20\cdot 4\cdot 6^{19} + 
   190 \cdot 4^{2}6^{18} + 1\,140 \cdot 4^{3}\cdot 6^{17} +
   4\,845 \cdot 4^{4}\cdot 6^{16} + \right. \\
   & & \phantom{1 - \frac{1}{10^{20}} (}
   15\,504 \cdot 4^{5}\cdot 6^{15} +
   38\,760 \cdot 4^{6} + \cdot 6^{14} + 
   77\,520 \cdot 4^{7} \cdot 6^{13} + \\
   & & \phantom{1 - \frac{1}{10^{20}}}
   \left. 125\,970 \cdot 4^{8} \cdot 6^{12} \right) \\
   & = & 1 - \frac{4\cdot 6^{12}}{10^{20}} \left(9\cdot 6^6 + 
   20\cdot 6^7 + 190 \cdot 4\cdot 6^6 + 1\,140 \cdot 4^2\cdot 6^5 +
   4\,845 \cdot 4^3\cdot 6^4 + \right. \\
   & & \phantom{1 - \frac{4\cdot 6^{13}}{10^{20}}} \left.
   15\,504 \cdot 4^4\cdot 6^3 +
   38\,760 \cdot 4^5 + \cdot 6^2 + 
   77\,520 \cdot 4^6 \cdot 6 + 125\,970 \cdot 4^7 \right) \\
   & = & 1 - \frac{8\,707\,129\,344}{10^{20}} \left(
   419\,904 + 5\,598\,720 + 35\,458\,560 + 141\,834\,240 + \right. \\
   & & \phantom{1 - \frac{8\,707\,129\,344}{10^{20}} (}
   401\,863\,680 + 857\,309\,184 + 1\,428\,848\,640 + \\
   & & \phantom{1 - \frac{8\,707\,129\,344}{10^{20}}}
   \left. 1\,905\,131\,520 +  2\,063\,892\,480 \right) \\ 
   & = & 1 - 
   \frac{8\,707\,129\,344 \cdot 6\,840\,356\,928}{100\,000\,000\,000\,000\,000\,000}
   = 1-\frac{59\,559\,872\,531\,222\,495\,232}{100\,000\,000\,000\,000\,000\,000} \\
   & = & \frac{40\,440\,127\,468\,777\,504\,768}{100\,000\,000\,000\,000\,000\,000}
   = 0.40440127468777504768
  \end{eqnarray*}
 \end{valorp}

 \begin{decision}
  No se rechaza $H_0$.
 \end{decision}

 \begin{conclusion}
  No hay pruebas suficientes para rechazar la afirmaci\'on del experto en marketing;
  esto es, que se puede considerar como verdadero que, entre los amantes de la pasta,
  el porcentaje de personas que prefieren la lasagna es aproximadamente del $40\%$.
 \end{conclusion}

 En el c\'odigo registrado en el archivo anexo 
 \texttt{P08\_Prueba\_de\_una\_proporcion\_01.r}, en R, se realiza este procedimiento.
 El c\'odigo permite modificar los valores iniciales que corresponden a:
 \texttt{n} para el tama\~no de la muestra;
 \texttt{x} para la cantidad de casos favorables
 o, en su defecto, \texttt{p0} para la proporci\'on de \'exitos muestrales;
 \texttt{p} para el valor de la proporci\'on poblacional
 supuesta en la hip\'otesis nula;
 \texttt{alfa} para el nivel de significancia;
 \texttt{cola} para indicar si la prueba es de dos colas, con \texttt{'D'},
 de cola inferior, con \texttt{'I'}, o de cola superior, con \texttt{'S'};
 y, \texttt{distr}, un par\'ametro opcional para hacer comparaciones did\'acticas
 forzando al programa a realizar los c\'alculos usando la distribuci\'on binomial,
 con \texttt{'B'}, la aproximaci\'on por la distribuci\'on de Poisson,
 con \texttt{'P'}, o la aproximaci\'on por la distribuci\'on normal, con \texttt{'N'}.
 \par 
 El programa espera al menos los datos correspondientes a la muestra:
 el tama\~no de muestra; la cantidad de casos favorables en la muestra
 o, en su defecto, la proporci\'on muestral; la hip\'otesis nula;
 y, el tipo de prueba (cola izquierda o derecha, o dos colas).
 Por lo tanto, el valor $P$ siempre se obtiene,
 independientemente de si se desea una prueba de hip\'otesis
 fijando la probabilidad del error tipo I o si se realiza una prueba de significancia
 (aproximaci\'on al valor $P$).
 Los valores no dados, ya sea por redundancia o por ser opcionales, se indican
 con el valor \texttt{NULL}.
 \par
 Si no ha sido asignado un valor al par\'ametro para forzar un tipo de distribuci\'on
 en los c\'alculos, el c\'odigo determinar\'a la distribuci\'on
 seg\'un las siguientes reglas: si la muestra es peque\~na, esto es si $n\leq 20$,
 se usar\'an los c\'alculos precisos por la distribuci\'on binomial
 con par\'ametros $n$ igual al tama\~no de la muestra
 y $p$ igual a la proporci\'on supuesta en la hip\'otesis nula;
 si la muestra no es peque\~na pero el valor hipot\'etico, $p$, est\'a cerca
 de $0$ o $1$, en menos de $0.05$ unidades, se usar\'a una aproximaci\'on
 con la distribuci\'on de Poisson, con par\'ametro $\mu = np$;
 y, en cualquier otra muestra no peque\~na, se usar\'a una aproximaci\'on
 con la distribuci\'on normal con par\'ametros $\mu=np$ y $\sigma^2=np(1-p)$,
 que a su vez usar\'a como estad\'{\i}stico la distribuci\'on normal est\'andar
 a trav\'es del valor de la transformaci\'on $z = \frac{x-np}{\sqrt{np(1-p)}}$.
 \par 
 Independientemente del tipo de prueba, el resultado muestra lo siguiente:
 \texttt{distr} para saber el tipo de distribuci\'on que se us\'o en los c\'alculos;
 \texttt{p} para el valor propuesto en la hip\'otesis nula;
 \texttt{n} para el tama\~no de la muestra;
 \texttt{x} para la cantidad de casos favorables en la muestra;
 \texttt{pMuestral} para la proporci\'on muestral;
 \texttt{media} para la media poblacional, considerando el valor hipot\'etico $p$;
 \texttt{desv.est} para la desviaci\'on est\'andar poblacional,
 considerando el valor hipot\'etico $p$;
 \texttt{alpha} para el nivel de significancia dado, el cual muestra por defecto $0.05$
 en caso de asignar \texttt{NULL} a \texttt{alfa};
 \texttt{Estadistico} para indicar la distribuci\'on binomial o de Poisson
 con sus respectivos par\'ametros, seg\'un sea el caso,
 o, si se us\'o una aproximaci\'on con la distribuci\'on normal, el valor resultante
 del estad\'{\i}stico de prueba; \texttt{PValor} para el valor $P$,
 la probabilidad de haber obtenido una muestra como se obtuvo,
 suponiendo que la hip\'otesis nula sea cierta;
 y, \texttt{RegionRechazoX} para indicar la regi\'on de n\'umeros enteros
 para los que la cantidad de casos favorables llevar\'{\i}an al rechazo de la prueba
 seg\'un el valor de $\alpha$ (posiblemente el dado por defecto).
 \par 
 Adem\'as, seg\'un el tipo de prueba, puede darse los siguientes valores:
 \texttt{RegionRechazoZ} para indicar en d\'onde se encuentra la regi\'on de rechazo
 para el valor obtenido por el estad\'{\i}sitco de prueba $z$
 al realizar los c\'alculos por medio de la aproximaci\'on a la distribuci\'on normal,
 seg\'un el valor de $\alpha$ (posiblemente el dado por defecto);
 y, \texttt{Resultado} para indicar si se rechaza o no la hip\'otesis nula,
 que aparece cuando se asigna un valor a \texttt{alfa},
 y se encuentra siempre al final de los resultados.
 \par
 El c\'odigo junto con los resultados se muestra a continuaci\'on:
 \begin{verbatim}
> n<-20
> x<-9
> p0<-NULL
> p<-0.4
> alfa<-0.05
> cola<-'S'
> distr<-NULL
> if(is.null(x)){
+   x<-round(p0*n)
+ }
> TestProp<-function(n,x,p,alfa=0.05,colas='D',preferencia=NULL){
+   p0<-x/n
+   if(is.null(preferencia)){
+     if(n <= 20){
+       distr<-'B'
+     }else{
+       if(abs(0.5 - p) > 0.4){
+         distr<-'P'
+       }
+       else{
+         distr<-'N'
+       }
+     }
+   }else{
+     distr<-preferencia
+   }
+   if(distr=='B'){
+     r<-data.frame(distr="Binomial",
+                   p=p,
+                   n=n,
+                   x=x,
+                   pMuestral=p0,
+                   media=n*p,
+                   desv.est=sqrt(n*p*(1-p)),
+                   alpha=alfa,
+                   Estadistico=paste("Var. binomial X con p=",p," y n=",n)
+                   )
+     if(colas=='D'){
+       if(p0<p){
+         pvalor<-round(2*pbinom(x,n,p),7)
+       }else{
+         pvalor<-round(2*pbinom(x-1,n,p,lower.tail=F),7)
+       }
+       xInf<-qbinom(alfa/2,n,p)-1
+       xSup<-qbinom(alfa/2,n,p,lower.tail=F)+1
+       r$PValor<-pvalor
+       r$RegionRechazoX<-paste("<=",xInf," y >=",xSup)
+     }else{
+       if(colas=='I'){
+         r$PValor<-round(pbinom(x,n,p),7)
+         r$RegionRechazoX<-paste("<=",qbinom(alfa,n,p)-1)
+       }else{
+         r$PValor<-round(pbinom(x-1,n,p,lower.tail=F),7)
+         r$RegionRechazoX<-paste(">=",qbinom(alfa,n,p,lower.tail=F)+1)
+       }
+     }
+   }else{
+     if(distr=='N'){
+       r<-data.frame(distr="Normal",
+                     p=p,
+                     n=n,
+                     x=x,
+                     pMuestral=p0,
+                     media=n*p,
+                     desv.est=sqrt(n*p*(1-p)),
+                     alpha=alfa
+                     )
+       estadistico<-(p0-p)/sqrt(p*(1-p)/n)
+       if(colas=='D'){
+         pvalor<-round(2*pnorm(abs(estadistico),lower.tail=F),7)
+         criticoz<-round(qnorm(1-alfa/2),7)
+         criticoxInf<-floor(n*p-criticoz*sqrt(n*p*(1-p)))
+         criticoxSup<-ceiling(n*p+criticoz*sqrt(n*p*(1-p)))
+         r$PValor<-pvalor
+         r$Estadistico<-estadistico
+         r$RegionRechazoZ<-paste("<=",-criticoz," y >=",criticoz)
+         r$RegionRechazoX<-paste("<=",criticoxInf," y >=",criticoxSup)
+       }else{
+         criticoz<-round(qnorm(1-alfa),7)
+         if(colas=='I'){
+           pvalor<-round(pnorm(estadistico),7)
+           criticox<-floor(n*p-criticoz*sqrt(n*p*(1-p)))
+           r$PValor<-pvalor
+           r$Estadistico<-estadistico
+           r$RegionRechazoZ<-paste("<=",-criticoz)
+           r$RegionRechazoX<-paste("<=",criticox)
+         }else{
+           pvalor<-round(pnorm(estadistico,lower.tail=F),7)
+           criticox<-ceiling(n*p+criticoz*sqrt(n*p*(1-p)))
+           r$PValor<-pvalor
+           r$Estadistico<-estadistico
+           r$RegionRechazoZ<-paste(">=",criticoz)
+           r$RegionRechazoX<-paste(">=",criticox)
+         }
+       }
+     }else{
+       r<-data.frame(distr="Poisson",
+                     p=p,
+                     n=n,
+                     x=x,
+                     pMuestral=p0,
+                     media=n*p,
+                     desv.est=sqrt(n*p*(1-p)),
+                     alpha=alfa,
+                     Estadistico=paste("Var. Poisson X con mu=",n*p)
+                     )
+       if(colas=='D'){
+         if(p0<p){
+           pvalor<-round(2*ppois(x,n*p),7)
+         }else{
+           pvalor<-round(2*ppois(x-1,n*p,lower.tail=F),7)
+         }
+         xInf<-qpois(alfa/2,n*p)-1
+         xSup<-qpois(alfa/2,n*p,lower.tail=F)+1
+         r$PValor<-pvalor
+         r$RegionRechazoX<-paste("<=",xInf," y >=",xSup)
+       }else{
+         if(colas=='I'){
+           r$PValor<-round(ppois(x,n*p),7)
+           r$RegionRechazoX<-paste("<=",qpois(alfa,n*p)-1)
+         }else{
+           r$PValor<-round(ppois(x-1,n*p,lower.tail=F),7)
+           r$RegionRechazoX<-paste(">=",qpois(alfa,n*p,lower.tail=F)+1)
+         }
+       }
+     }
+   }
+   return(r)
+ }
> if(is.null(alfa)){
+   Test<-TestProp(n,x,p,colas=cola,preferencia=distr)
+ }else{
+   Test<-TestProp(n,x,p,alfa,cola,distr)
+   resultado<-ifelse(Test[,"PValor"]>=alfa,"No se rechaza H0","Se rechaza H0")
+   Test$Resultado<-resultado
+ }
> Test
     distr   p  n x pMuestral media desv.est alpha
1 Binomial 0.4 20 9      0.45     8  2.19089  0.05
                          Estadistico    PValor RegionRechazoX        Resultado
1 Var. binomial X con p= 0.4  y n= 20 0.4044013          >= 13 No se rechaza H0
 \end{verbatim}
 Lo cual coincide con los resultados obtenidos, adem\'as de dar m\'as informaci\'on
 como: la media y la desviaci\'on est\'andar de la distribuci\'on binomial
 y la cantidad de casos favorables a partir del cual se rechazar\'{\i}a la prueba,
 que es a lo que se quer\'{\i}a llegar${}_{\blacksquare}$
 
\end{solucion}
