\begin{enunciado}
 En un experimento para estudiar la dependencia de la hipertensi\'on
 con respecto a los h\'abitos de fumar, se tomaron los siguientes datos
 de $180$ individuos:
 \begin{center}
  \begin{tabular}{lccc}
   & \textbf{No} & \textbf{Fumadores} & \textbf{Fumadores} \\
   & \textbf{fumadores} & \textbf{moderados} & \textbf{empedernidos} \\
   \hline 
   \textbf{Con} \\
   $\,$ \textbf{hipertensi\'on} & $21$ & $36$ & $30$ \\
   \textbf{Sin} \\
   $\,$ \textbf{hipertensi\'on} & $48$ & $26$ & $19$
  \end{tabular}
 \end{center}
 Pruebe la hip\'otesis de que la presencia o ausencia de la hipertensi\'on es
 independiente de los h\'abitos de fumar.
 Utilice un nivel de significancia de $0.05$.
\end{enunciado}

\begin{solucion}
 \begin{datos}
  $\phantom{0}$
  \begin{itemize}
   \item Tamaño de muestra total: $180$.
   \item Tama\~no de muestra de individuos con hipertensi\'on: $21+36+30= 87$.
   \item Tama\~no de muestra de individuos sin hipertensi\'on: $48+26+19= 93$.
   \item Tama\~no de muestra de no fumadores: $21 + 48 = 69$.
   \item Tama\~no de muestra de fumadores moderados: $36 + 26 = 62$.
   \item Tama\~no de muestra de fumadores empedernidos: $30 + 19 = 49$.
   \item Frecuencias observadas y esperadas: $o_{i,j}$
   y $e_{i,j}=\frac{R_i C_j}{n}$, respectivamente,
   donde $R_i$ y $C_j$ son los marginales del rengl\'on $i$ y la columna $j$,
   respectivamente, y $n$ es el total de toda la muestra.
   As\'{\i}, pues, redondeando a un decimal, se muestra el resumen 
   en la siguiente tabla,
   en donde aparece entre par\'entesis la frecuencia esperada
   y a la izquierda el valor observado:
   \begin{center}
    \begin{tabular}{lccc|c}
     & \textbf{No} & \textbf{Fumadores} & \textbf{Fumadores} &
     \textbf{marginal seg\'un} \\
     & \textbf{fumadores} & \textbf{moderados} & \textbf{empedernidos} & 
     \textbf{la hipertensi\'on} \\
     \hline 
     \textbf{Con} & & & & \\
     $\,$ \textbf{hipertensi\'on} & $21 (33.3)$ & $36 (30)$ & $30 (23.7)$ &
     $87$ \\
     \textbf{Sin} & & & & \\
     $\,$ \textbf{hipertensi\'on} & $48 (35.7)$ & $26 (32)$ & $19 (25.3)$ &
     $93$ \\
     \hline 
     \textbf{Marginal seg\'un} & & & & \textbf{TOTAL} \\
     $\,$ \textbf{tipo de fumador} & $69$ & $62$ & $49$ & $n=180$
    \end{tabular}
   \end{center}
   \item Tama\~no de la tabla de contingencia: $r\times c = 2\times 3$.
   \item Grados de libertad de la prueba $\chi^2$: $v = (r-1)(c-1) = 2$.
  \end{itemize}
 \end{datos}

 \begin{hipotesis}
  \begin{eqnarray*}
   H_0: & & \text{La hipertensi\'on es independiente del tipo de fumador.} \\
   H_1: & & \text{La hipertensi\'on depende del tipo de fumador.}
  \end{eqnarray*}
 \end{hipotesis}

 \begin{significancia}
  $\alpha = 0.05$.
 \end{significancia}

 \begin{region}
  De la tabla A.5, se tiene el valor cr\'{\i}tico
  $\chi^2_{\alpha,v} = \chi^2_{0.05,2} \approx 5.991$,
  por lo que la regi\'on de rechazo est\'a dado
  para $\chi^2 > 5.991$, donde
  $\chi^2 = \sum_{i} \frac{\left( o_i - e_i \right)^2}{e_i}$.
 \end{region}

 \begin{estadistico}
  \begin{eqnarray*}
   \chi^2 & = & \sum_{i} \frac{\left( o_i - e_i \right)^2}{e_i} \\
   & = & \frac{(21 - 33.3)^2}{33.3} + \frac{(36 - 30)^2}{30} +
   \frac{(30 - 23.7)^2}{23.7} + \frac{(48 - 35.7)^2}{35.7} + 
   \frac{(26 - 32)^2}{32} + \frac{(19 - 25.3)^2}{25.3} \\
   & = & \frac{151.29}{33.3} + \frac{36}{30} + \frac{39.69}{23.7} + 
   \frac{151.29}{35.7} + \frac{36}{32} + \frac{39.69}{25.3} \\
   & \approx & 4.5432 + 1.2 + 1.67468 + 4.237815 + 1.125 + 1.56877
   = 14.349465
  \end{eqnarray*}
 \end{estadistico}

 \begin{decision}
  Se rechaza $H_0$ a favor de $H_1$.
 \end{decision}

 \begin{conclusion}
  Con un nivel de significancia de $0.05$, se concluye
  que la hipertensi\'on depende de los h\'abitos de fumar.
 \end{conclusion}

 En el c\'odigo registrado en el archivo
 \texttt{P18\_Prueba\_de\_independencia\_y\_homogeniedad\_01.r}
 se realiza este procedimiento.
 El c\'odigo permite modificar los valores iniciales que corresponden a:
 \texttt{datos} que guarda los datos de la lectura de un archivo,
 siendo en este caso \texttt{BD26\_Problema\_090.csv} es el archivo
 le\'{\i}do;
 \texttt{varInteres} en el que se indican los dos nombres de las columnas
 en la base de datos que corresponden a las dos categor\'{\i}as
 de la tabla de contingencia;
 \texttt{varFrecuencia} para indicar, en caso de ser necesario, el nombre
 de la columna en donde se encuentra el resume de frecuencias
 de correspondientes a cada intersecci\'on de ambas categor\'{\i}as,
 y, en caso de que no se haya contabilizado las frecuencias,
 se asigna \texttt{NULL} a esta variable y el programa realiza dicho resumen;
 y, \texttt{pruebas}, para indicar las pruebas que se desea realizar
 con el programa de entre una lista de 7 pruebas.
 \par 
 El programa espera al menos los datos correspondientes a la base de datos,
 escrito en un archivo \texttt{.csv} con dos columnas de categor\'{\i}as,
 identificados en \texttt{varInteres},
 as\'{\i} como tambi\'en al menos una de las pruebas disponibles,
 indic\'andose en una lista de n\'umeros, seg\'un la prueba correspondiente:
 \texttt{1} para la prueba $\chi^2$ de Pearson, que describe el libro
 y que aqu\'{\i} se realiz\'o a mano;
 \texttt{2} para la prueba $G$ sin correcci\'on de Williams,
 tambi\'en conocido como raz\'on de verosimilitud,
 que se parece a la prueba $\chi^2$ porque cuantifica diferencias
 entre valores esperados y observados,
 se contrasta con los valores de $\chi^2$ y se calculan los grados de libertad
 del mismo modo,
 pero con el estad\'{\i}stico
 $G = 2\sum_{i} o_i \ln \left(\frac{o_i}{e_i}\right)$, donde $i$ recorre
 todos los elementos de la tabla;
 \texttt{3} para la prueba $G$ con correcci\'on de Williams, que se aplica
 cuando las diferentes categor\'{\i}as tienen frecuencias similares
 y la diferencia se centra en el estad\'{\i}stico que se calcula como
 $G' = \frac{G}{q}$, en donde, si se nombra $R_1, R_2, \ldots, R_r$ y
 $C_1, C_2, \ldots, C_c$ a los totales del rengl\'on $1$, $2$ y as\'{\i}
 seguidamente hasta el \'ultimo rengl\'on, $r$, y a los totales
 de la columna $1$, $2$ y as\'{\i} seguidamente hasta la \'ultima columna,
 $c$, respectivamente, y a $n$ como el tama\~no total de la muestra,
 se define el denominador de $G'$ como $q = 1
 + \frac{\left[ n\left(\frac{1}{R_1} + \cdots \frac{1}{R_r}\right)-1 \right]\left[ n\left( \frac{1}{C_1} + \cdots + \frac{1}{C_c} \right) - 1 \right]}{
 6n(r-1)(c-1)}$;
 \texttt{4} para la prueba $\chi^2$ con correcci\'on de Yates, 
 la cual s\'olo se aplica en tablas de $2\times 2$;
 \texttt{5} para la prueba exacta de Fisher, que s\'olo se aplica en tablas
 de $2\times2$, y, en este caso, se considera para pruebas bilaterales;
 \texttt{6} para la prueba exacta de Fisher,
 pero ahora considerando su uso para pruebas unilaterales
 del tipo ``menor que'';
 y, \texttt{7} para la prueba exacta de Fisher,
 considerado para pruebas unilaterales del tipo ``mayor que''.
 \par 
 El programa mostrar\'a en su salida los resultados de las pruebas en R,
 usando de forma general funciones previamente creadas para la comunidad de R,
 y estos resultados se mostrar\'an en forma de lista describiendo
 el nombre del m\'etodo as\'{\i} como del estad\'{\i}stico, el valor $P$
 o lo grados de libertad. seg\'un sea el caso.
 Cabe hacer notar que, como se usan funciones de la comunidad de R,
 los nombres o descripciones de los diferentes m\'etodos realizados
 en los resultados se encuentran en ingl\'es.
 \par 
 El c\'odigo junto con el resultado se muestra a continuaci\'on,
 en donde se consider\'o, para este problema,
 el uso de los pruebas $\chi^2$, G sin correcci\'on y G
 con correcci\'on de Williams.
 \begin{verbatim}
> datos<-read.csv("DB26_Problema_090.csv",sep=";",encoding="UTF-8")
> varInteres<-c("Estado.hipertensión","Hábito.fumar")
> varFrecuencia<-"Frecuencia"
> pruebas<-c(1,2,3)
> g.test <- function(x, correct="none",
+                    p = rep(1/length(x), length(x)))
+ {
+   DNAME <- deparse(substitute(x))
+   if (is.matrix(x)) {
+     if (min(dim(x)) == 1) 
+       x <- as.vector(x)
+   }
+   if (any(x < 0) || any(is.na(x))) 
+     stop("Todas las entradas deben de ser no negativas")
+   if ((n <- sum(x)) == 0) 
+     stop("Al menos una entrada debe de ser positiva")
+   nrows<-nrow(x)
+   ncols<-ncol(x)
+   sr <- apply(x,1,sum)
+   sc <- apply(x,2,sum)
+   E <- outer(sr,sc, "*")/n
+   # Calcular G
+   g <- 0
+   for (i in 1:nrows){
+     for (j in 1:ncols){
+       if (x[i,j] != 0) g <- g + x[i,j] * log(x[i,j]/E[i,j])
+     }
+   }
+   q <- 1
+   if (correct=="williams"){
+     row.tot <- col.tot <- 0    
+     for (i in 1:nrows){ row.tot <- row.tot + 1/(sum(x[i,])) }
+     for (j in 1:ncols){ col.tot <- col.tot + 1/(sum(x[,j])) }
+     q <- 1+ ((n*row.tot-1)*(n*col.tot-1))/(6*n*(ncols-1)*(nrows-1))
+   }
+   STATISTIC <- G <- 2 * g / q
+   PARAMETER <- (nrow(x)-1)*(ncol(x)-1)
+   PVAL <- 1-pchisq(STATISTIC,df=PARAMETER)
+   base<-"Log likelihood ratio (G-test) test of independence "
+   if(correct=="none")
+     METHOD<-paste(base,"without correction")
+   if(correct=="williams")
+     METHOD<-paste(base,"with Williams' correction")
+   names(STATISTIC) <- "Log likelihood ratio statistic (G)"
+   names(PARAMETER) <- "X-squared df"
+   names(PVAL) <- "p.value"
+   structure(list(statistic=STATISTIC,parameter=PARAMETER,p.value=PVAL,
+                  method=METHOD,data.name=DNAME, observed=x, expected=E),
+             class="htest")
+ }
> tablaDobleEntrada<-function(datos,varInteres,pruebas){
+   tbl1<-table(datos[,varInteres])
+   listaPruebas<-NULL
+   for (i in pruebas){
+     if (i==1) listaPruebas<-c(listaPruebas,
+                               list(chisq.test(tbl1,correct=FALSE)))
+     if (i==2) listaPruebas<-c(listaPruebas,list(g.test(tbl1)))
+     if (i==3) listaPruebas<-c(listaPruebas,
+                               list(g.test(tbl1,correct="williams")))
+     if (i==4) listaPruebas<-c(listaPruebas,list(chisq.test(tbl1,correct=TRUE)))
+     if (i==5) listaPruebas<-c(listaPruebas,list(fisher.test(tbl1)))
+     if (i==6) listaPruebas<-c(listaPruebas,
+                               list(fisher.test(tbl1,alternative="less")))
+     if (i==7) listaPruebas<-c(listaPruebas,
+                               list(fisher.test(tbl1,alternative="greater")))
+   }
+   return(list(tabla=tbl1,listaPruebas=listaPruebas))
+ }
> if(length(varInteres)!=2) stop("Deben ser dos variables de interés")
> if(!is.null(varFrecuencia)){
+   datos<-datos[rep(row.names(datos), datos$Frecuencia),
+                 which(colnames(datos)!=varFrecuencia)]
+ }
> listaR<-tablaDobleEntrada(datos,varInteres,pruebas)
> listaR
$tabla
                   Hábito.fumar
Estado.hipertensión Fumador empedernido Fumador moderado No fumador
   Con hipertensión                  30               36         21
   Sin hipertensión                  19               26         48

$listaPruebas
$listaPruebas[[1]]

	Pearson's Chi-squared test

data:  tbl1
X-squared = 14.464, df = 2, p-value = 0.0007232


$listaPruebas[[2]]

	Log likelihood ratio (G-test) test of independence without correction

data:  tbl1
Log likelihood ratio statistic (G) = 14.763, X-squared df = 2, p-value =
0.0006226


$listaPruebas[[3]]

	Log likelihood ratio (G-test) test of independence with Williams'
	correction

data:  tbl1
Log likelihood ratio statistic (G) = 14.597, X-squared df = 2, p-value =
0.0006765
 \end{verbatim}
 \vspace{-0.5cm}
 Lo cual coincide con lo obtenido en la prueba,
 adem\'as de brindar m\'as informaci\'on destacando el valor $P$
 y las otras pruebas, siendo determinante el rechazo de la hip\'otesis nula,
 que es a lo que se quer\'{\i}a llegar.${}_{\blacksquare}$
\end{solucion}
