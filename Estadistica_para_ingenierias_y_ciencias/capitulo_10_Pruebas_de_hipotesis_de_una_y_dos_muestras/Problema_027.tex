\begin{enunciado}
 Un estudio de la Universidad de Colorado en Boulder muestra que correr aumenta el porcentaje de la tasa metab\'olica de descanso (\texttt{TMD}) en mujeres ancianas. La \texttt{TMD} promedio de $30$ ancianas corredoras fue $34.0\%$ m\'as alta que la \texttt{TMD} promedio de $30$ ancianas sedentarias, en tanto que las desviaciones est\'andar reportadas fueron de $10.5$ y $10.2\%$, respectivamente. ¿Hay un aumento significativo en la \texttt{TMD} de las corredoras con respecto a las sedentarias? Suponga que las poblaciones se distribuyen de forma aproximadamente normal con varianzas iguales. Utilice un valor $P$ en sus conclusiones.
\end{enunciado}

\begin{solucion}
 \begin{datos}
  $\phantom{0}$
  \begin{itemize}
   \item $X_i \sim n\left( \mu_i, \sigma_i \right)$, para cada $i \in \{ 1, 2 \}$.
   \item $n_1 = 30$ y $n_2 = 30$.
   \item $\bar{x}_1 - \bar{x}_2 = 34$.
   \item $s_1 = 10.5$ y $s_2 = 10.2$.
   \item $\sigma_1^2 = \sigma_2^2$.
  \end{itemize}
  Ya que $n_1, n_2 \geq 30$,
  como resultado del teorema del teorema del l\'{\i}mite central
  se obtiene adem\'as que los siguientes estadísticos se aproximan
  a las distribuciones indicadas:
  \begin{itemize}
   \item $\overline{X}_i \sim n\left( \mu_i, \sigma_i/n \right)$,
   para cada $i \in \{ 1, 2 \}$.
   \item $Z = \frac{\left( \overline{X}_1 - \overline{X}_2 \right) - d_0}{\sqrt{S_1^2/n_1 + S_2^2/n_2}} \approx
   \frac{\left( \overline{X}_1 - \overline{X}_2 \right) - d_0}{\sqrt{\sigma_1^2/n_1 + \sigma_2^2/n_2}}
   \sim n\left( 0, 1 \right)$.
  \end{itemize}
  A\'un as\'{\i}, si no se contara con dichos resultados,
  por las suposiciones de normalidad en la distribuciones
  poblacionales y que las varianzas son iguales, se sabe
  que el siguiente estad\'{\i}stico,
  que se aproxima a la distribuci\'on mostrada
  con el respectivo par\'ametro que se indica,
  se puede usar en vez del anterior.
  \begin{itemize}
   \item $T = 
   \frac{\left( \overline{X}_1 - \overline{X}_2 \right) - d_0}{S_p\sqrt{1/n_1 + 1/n_2}} \sim t(v)$.
   \item $v = n_1 + n_2 - 2 = 58$.
  \end{itemize}
 \end{datos}

 \begin{hipotesis}
  \begin{eqnarray*}
   H_0: \mu_1 - \mu_2 & = & 0 \\
   H_1: \mu_1 - \mu_2 & > & 0
  \end{eqnarray*}
 \end{hipotesis}

 \begin{estadistico}
  Usando el estad\'{\i}stico obtenido con el teorema
  del l\'{\i}mite central, se calcula que:
  \begin{eqnarray*}
   z & = & \frac{\left( \bar{x}_1 - \bar{x}_2 \right) - \left( \mu_1 - \mu_2 \right)}{\sqrt{s_1^2/n_1 + s_2^2/n_2}}
   \approx \frac{34-0}{\sqrt{10.5^2/30 + 10.2^2/30}}
   = \frac{34\sqrt{30}}{\sqrt{110.25+104.04}}
   = \frac{34(10)\sqrt{30}}{\sqrt{21\,429}} \\
   & = & \frac{340\sqrt{10}}{\sqrt{7\,143}}
   = \frac{340\sqrt{71\,430}}{7143} \approx 12.7215079
  \end{eqnarray*}
  A\'un as\'{\i}, el valor con el otro estad\'{\i}stico
  se obtiene calculando primero lo siguiente:
  \begin{eqnarray*}
   s_p^2 & = &
   \frac{
   s_1^2 \left( n_1 - 1 \right) + s_2^2\left( n_2 - 1 \right)
   }{
   n_1 + n_2 - 2
   }
   = \frac{10.5^2(30-1) + 10.2(30-1)}{30+30-2}
   = \frac{\cancel{29}(110.25 + 104.04)}{\cancelto{2}{58}} \\
   & = & \frac{21\,429}{200}
   = 107.145
  \end{eqnarray*}
  entonces
  \begin{equation*}
   s_p = \sqrt{s_p^2} = \sqrt{\frac{21\,429}{200}}
   = \frac{3\sqrt{2\,381}\sqrt{2}}{20}
   = \frac{3\sqrt{4\,762}}{20}
   \approx 10.351086899451670625994809052
  \end{equation*}
  y, por lo tanto,
  \begin{eqnarray*}
   t & = & 
   \frac{
   \left( \bar{x}_1 - \bar{x}_2 \right) - d_0
   }{
   s_p\sqrt{1/n_1 + 1/n_2}
   }
   = \frac{34-0}{\frac{3\sqrt{4\,762}}{20}\sqrt{\frac{1}{30}+\frac{1}{30}}}
   = \frac{34(20)}{3\sqrt{\frac{4\,762}{15}}}
   = \frac{680\sqrt{15}\sqrt{4\,762}}{3(4\,762)} \\
   & = & \frac{340\sqrt{71\,430}}{7\,143}
   \approx 12.7215079
  \end{eqnarray*}
 \end{estadistico}

 \begin{valorp}
  De la tabla A.3 se tiene que:
  \begin{equation*}
   P(Z > z) \approx P(Z > 12.72) \approx 0 
  \end{equation*}
  Por otro lado, usando el estad\'{\i}stico $T$, se tiene que:
  \begin{equation*}
   P(T > t) \approx P(T > 12.721507900588463860701)
  \end{equation*}
  Pero usando la tabla A.4 con 58 grados de libertad,
  se buscar\'{\i}a interpolar con los valores m\'as cercanos;
  sin embargo, estos est\'an muy por afuera del rango de la tabla,
  a saber, con $60$ grados de libertad,
  el valor m\'as grande es $3.460$
  y $P(T > 3.460) = 0.0005$,
  mientras que con $40$ grados de libertad,
  el valor m\'as grande es $3.551$
  y $P(T > 3.551) = 0.0005$, por lo que se puede asegurar que
  \begin{equation*}
   P(T > t) \approx P(T > 12.721507900588463860701) < 0.0005
  \end{equation*}
 \end{valorp}

 \begin{conclusion}
  Por lo tanto, como el valor $P$ es muy peque\~no, entonces se tiene evidencia suficiente para concluir que las mujeres ancianas corredoras tienen un porcentaje en la tasa metab\'olica de descanso mayor que las mujeres ancianas sedentarias.
 \end{conclusion}

 En el c\'odigo registrado en el archivo anexo
 \texttt{P05\_Prueba\_de\_dos\_medias\_01.r}, en R,
 se realiza este procedimiento.
 El c\'odigo permite modificar los valores iniciales
 que corresponden a:
 \texttt{n1} y \texttt{n2} para el tama\~no de las muestras;
 \texttt{mu} para el valor de la diferencia de las medias
 poblacionales supuesta en la hip\'otesis nula;
 \texttt{m1} y \texttt{m2} para las medias muestrales o,
 en su defecto, \texttt{m} para la diferencia de estas
 si se da el dato directamente;
 \texttt{sigma1} y \texttt{sigma2} para las desviaciones est\'andar
 poblacionales, si se indican,
 o, en su defecto, \texttt{s1} y \texttt{s2}
 para las desviaciones est\'andar muestrales;
 \texttt{desv.iguales} indica,
 en caso de que se desconozcan las desviaciones est\'andar
 o varianzas poblacionales, si se est\'an suponiendo iguales,
 con el valor \texttt{TRUE}, o diferentes, con \texttt{FALSE};
 \texttt{alfa} para el nivel de significancia;
 \texttt{cola} para indicar si la prueba es de dos colas,
 con \texttt{'D'}, de cola inferior, con \texttt{'I'},
 o de cola superior, con \texttt{'S'};
 adicionalmente, la rutina puede realizar pruebas pareadas
 con \texttt{par} al darle el valor \texttt{TRUE},
 que por defecto aparece como \texttt{FALSE}.
 Esta \'ultima prueba puede necesitar del valor de la desviaci\'on
 est\'andar de la diferencia de los datos pareados,
 para lo que se usa el valor de \texttt{sD}.
 \par 
 El programa espera al menos los datos correspondientes
 a la muestra:
 el tama\~no de ambas muestras;
 la diferencia de las medias muestrales,
 que en caso de dar los valores individuales,
 la rutina calcula la diferencia previo a la funci\'on principal;
 las desviaciones est\'andar, ya sean poblacionales
 o muestrales, o en caso de ser muestras pareadas,
 la desviaci\'on estandar muestral de la diferencia;
 y el tipo de prueba (cola izquierda o derecha, o dos colas);
 y, en caso de ser muestras peque\~nas,
 la suposici\'on sobre si son o no iguales
 las desviaciones est\'andar poblacionales.
 Por lo tanto, el valor $P$ siempre se obtiene,
 independientemente de si se desea una prueba de hip\'otesis
 fijando la probabilidad del error tipo I
 o si se realiza una prueba de significancia
 (aproximaci\'on al valor $P$).
 Los valores no dados, ya sea por redundancia o por ser opcionales,
 se indican con el valor \texttt{NULL},
 excepto \texttt{par} que por defecto tiene el valor \texttt{FALSE}.
 \par 
 La prueba de hip\'otesis usar\'a alg\'un estad\'{\i}stico,
 seg\'un corresponda, que tiene distribuci\'on $Z$ o $t$.
 Se considerar\'a la de distribuci\'on $Z$ si se dan
 las desviaciones est\'andar poblacionales o,
 como estimadores de \'estas,
 las desviaciones est\'andar muestrales.
 Esto \'ultimo ocurrir\'a \'unicamente
 si los tama\~nos de cada muestra son mayores o iguales que $30$.
 Si no se dan las desviaciones est\'andar poblacionales
 y las muestras son peque\~nas,
 se usar\'a el estad\'{\i}stico con distribuci\'on $t$
 y, en dado caso, una condici\'on necesaria y que se va a suponer
 es que las distribuciones poblacionales son aproximadamente
 normales.
 Adem\'as, como ya se mencion\'o, se considerar\'a la prueba
 de muestras pareadas.
 \par
 Independientemente del tipo de prueba,
 el resultado muestra lo siguiente:
 \texttt{Prueba} para saber si se us\'o un estad\'{\i}stico
 con distribuci\'on $Z$ o con distribuci\'on $t$;
 \texttt{H0} para el valor propuesto de la diferencia
 de las medias poblacionales;
 \texttt{n1} y \texttt{n2} para el tama\~no de cada muestra;
 \texttt{DifMedias} o \texttt{MediaPareada}
 para la diferencia de las medias muestrales,
 seg\'un son muestras independientes o pareadas;
 \texttt{desv.est1} y \texttt{desv.est2} para las desviaciones
 est\'andar, ya sea poblacional o muestral seg\'un sea el caso,
 que, en el caso de las muestras pareadas, se sustituye
 por \texttt{desv.par};
 \texttt{error.est} para el error est\'andar,
 cuya f\'ormula var\'{\i}a entre un estad\'{\i}stico y otro,
 pero, independientemente,
 cuando este error divide a
 $\left(\bar{x}_1 - \bar{x}_2\right) - d_0$
 se obtiene el estad\'{\i}stico,
 y representa la estimaci\'on de la desviaci\'on est\'andar
 de $\overline{X}_1 - \overline{X}_2$;
 \texttt{alpha} para el nivel de significancia dado,
 el cual muestra por defecto $0.05$
 en caso de asignar \texttt{NULL} a \texttt{alfa};
 \texttt{PValor} para el valor $P$,
 la probabilidad de haber obtenido una muestra como se obtuvo,
 suponiendo que la hip\'otesis nula sea cierta;
 \texttt{Estadistico} para el valor resultante
 del estad\'{\i}stico de prueba;
 \texttt{RegionRechazoZ} o \texttt{RegionRechazoT},
 seg\'un el tipo de prueba realizada,
 que indica en d\'onde se encuentra la regi\'on de rechazo para
 los valores obtenidos por el estad\'{\i}stico de prueba,
 seg\'un el valor de $\alpha$ (posiblemente el dado por defecto);
 \texttt{RegionRechazoX} para indicar la regi\'on de rechazo,
 en t\'erminudos de las unidades originales del problema,
 para el valor de \texttt{m},
 la diferencia de las medias muestrales.
 \par 
 Adem\'as, seg\'un el tipo de prueba,
 pueden darse los siguientes valores:
 \texttt{Resultado} para indicar si se rechaza o no la hip\'otesis
 nula, que aparece cuando se asigna un valor a \texttt{alfa}, y se encuentra siempre al final de los resultados;
 \texttt{var.pobl} aparece cuando se realiza una prueba $t$, 
 no se conocen las desviaciones est\'andar poblacionales,
 y este valor indica si se est\'an suponiendo iguales o distintos,
 aparece despu\'es de \texttt{Prueba};
 \texttt{est.sp} aparece cuando se realiza una prueba $t$
 y se est\'a suponiendo varianzas iguales,
 este valor indica el resultado
 de calcular $s_p$, el estimador de la desviaci\'on est\'andar
 poblacional com\'un $\sigma$ de las distribuciones poblacionales
 de las que se est\'an obteniendo las muestras,
 aparece despu\'es de las desviaciones est\'andar muestrales;
 \texttt{grados.libertad} aparece cuando se realiza una prueba $t$,
 indica los grados de libertad
 que es el par\'ametro de la distribuci\'on $t$
 del estad\'{\i}stico,
 entre el valor del error est\'andar y $\alpha$;
 \texttt{Tipo} aparece \'unicamente cuando se trata de una prueba
 con muestras dependientes, es decir muestras pareadas,
 indicando el texto: \textit{Muestras pareadas}.
 \par 
 El c\'odigo junto con el resultado se muestra a continuaci\'on:
 \begin{verbatim}
> n1<-30
> n2<-30
> mu<-0
> m1<-NULL
> m2<-NULL
> m<-34
> sigma1<-NULL
> sigma2<-NULL
> s1<-10.5
> s2<-10.2
> sD<-NULL
> desv.iguales<-TRUE
> alfa<-NULL
> cola<-'S'
> par<-FALSE
> if(is.null(m)) m<-m1-m2
> if(n1>=30 & n2>=30){
+    if(is.null(sigma1)) sigma1<-s1
+    if(is.null(sigma2)) sigma2<-s2
+ }
> if(par & n1 != n2){
+    stop("Las muestras pareadas deben tener el mismo tamaño.")
+ }
> TestMedia<-function(n1,n2,mu,m,desv1,desv2,sD=NULL,alfa=0.05,colas='D',
+                     conocidas=FALSE,iguales=FALSE,pareadas=FALSE){
+   if(n1 < 2 | n2 < 2 | par & is.null(sD)){
+     if(is.null(sD)){
+        error<-ifelse(is.na(desv1)|is.na(desv2),
+                      NA,sqrt(desv1^2/n1+desv2^2/n2))
+     }else{
+        error<-sD
+     }
+     r<-data.frame(Prueba=NA, H0=mu,
+                   n1=n1, n2=n2,
+                   DifMedias=m,
+                   desv.est1=ifelse(is.na(desv1),NA,desv1),
+                   desv.est2=ifelse(is.na(desv2),NA,desv2),
+                   error.est=error,
+                   alpha=alfa,
+                   PValor=NA,
+                   Estadistico=NA,
+                   RegionRechazoZ=NA,
+                   RegionRechazoX=NA)
+   }
+   estadistico<-m-mu
+   if(!pareadas){
+     if(conocidas){
+       desv<-sqrt(desv1^2/n1+desv2^2/n2)
+       estadistico<-estadistico/desv
+       r<-data.frame(Prueba="Z",H0=mu,
+                     n1=n1, n2=n2,
+                     DifMedias=m,
+                     desv.est1=desv1, desv.est2=desv2,
+                     error.est=desv,
+                     alpha=alfa)
+       if(colas=='D'){
+         pvalor<-round(2*pnorm(abs(estadistico),lower.tail=F),7)
+         criticoz<-round(qnorm(1-alfa/2),7)
+         criticox<-round(criticoz*desv,7)
+         r$PValor<-pvalor
+         r$Estadistico<-estadistico
+         r$RegionRechazoZ<-paste("<",-criticoz," y >",criticoz)
+         r$RegionRechazoX<-paste("<",mu-criticox," y >",mu+criticox)
+       }else{
+         criticoz<-round(qnorm(1-alfa),7)
+         criticox<-round(criticoz*desv,7)
+         if(colas=='I'){
+           pvalor<-round(pnorm(estadistico),7)
+           r$PValor<-pvalor
+           r$Estadistico<-estadistico
+           r$RegionRechazoZ<-paste("<",-criticoz)
+           r$RegionRechazoX<-paste("<",mu-criticox)
+         }else{
+           pvalor<-round(pnorm(estadistico,lower.tail=F),7)
+           r$PValor<-pvalor
+           r$Estadistico<-estadistico
+           r$RegionRechazoZ<-paste(">",criticoz)
+           r$RegionRechazoX<-paste(">",mu+criticox)
+         }
+       }
+     }else{
+       if(iguales){
+         grados<-n1+n2-2
+         sp<-sqrt((desv1^2*(n1-1)+desv2^2*(n2-1))/grados)
+         desv<-sp*sqrt(1/n1+1/n2)
+         estadistico<-estadistico/desv
+         r<-data.frame(Prueba="t",
+                       var.pobl="Iguales",
+                       H0=mu,
+                       n1=n1,n2=n2,
+                       DifMedias=m,
+                       desv.est1=desv1,
+                       desv.est2=desv2,
+                       est.sp=sp,
+                       error.est=desv,
+                       grados.libertad=grados,
+                       alpha=alfa)
+       }else{
+         numerador<-(desv1^2/n1 + desv2^2/n2)^2
+         denominador<-(desv1^2/n1)^2/(n1-1) + (desv2^2/n2)^2/(n2-1)
+         grados<-round(numerador/denominador,0)
+         desv<-sqrt(desv1^2/n1 + desv2^2/n2)
+         estadistico<-estadistico/desv
+         r<-data.frame(Prueba="t",
+                       var.pobl="Diferentes",
+                       H0=mu,
+                       n1=n1,n2=n2,
+                       DifMedias=m,
+                       desv.est1=desv1,
+                       desv.est2=desv2,
+                       error.est=desv,
+                       grados.libertad=grados,
+                       alpha=alfa)
+       }
+       if(colas=='D'){
+         pvalor<-round(2*pt(abs(estadistico),grados,lower.tail=F),7)
+         criticot<-round(qt(1-alfa/2,grados),7)
+         criticox<-round(criticot*desv,7)
+         r$PValor<-pvalor
+         r$Estadistico<-estadistico
+         r$RegionRechazoT<-paste("<",-criticot," y >",criticot)
+         r$RegionRechazoX<-paste("<",mu-criticox," y >",mu+criticox)
+       }else{
+         criticot<-round(qt(1-alfa,grados),7)
+         criticox<-round(criticot*desv,7)
+         if(colas=='I'){
+           pvalor<-round(pt(estadistico,grados),7)
+           r$PValor<-pvalor
+           r$Estadistico<-estadistico
+           r$RegionRechazoT<-paste("<",-criticot)
+           r$RegionRechazoX<-paste("<",mu-criticox)
+         }else{
+           pvalor<-round(pt(estadistico,grados,lower.tail=F),7)
+           r$PValor<-pvalor
+           r$Estadistico<-estadistico
+           r$RegionRechazoT<-paste(">",criticot)
+           r$RegionRechazoX<-paste(">",mu+criticox)
+         }
+       }
+     }
+   }
+   else{
+     desv<-sD/sqrt(n1)
+     estadistico<-estadistico/desv
+     grados<-n1-1
+     r<-data.frame(Prueba="t",
+                   Tipo="Muestras pareadas",
+                   H0=mu,
+                   n=n1,
+                   MediaPareada=m,
+                   desv.par=sD,
+                   error.est=desv,
+                   grados.libertad=grados,
+                   alpha=alfa)
+     if(colas=='D'){
+       pvalor<-round(2*pt(abs(estadistico),grados,lower.tail=F),7)
+       criticot<-round(qt(1-alfa/2,grados),7)
+       criticox<-round(criticot*desv,7)
+       r$PValor<-pvalor
+       r$Estadistico<-estadistico
+       r$RegionRechazoT<-paste("<",-criticot," y >",criticot)
+       r$RegionRechazoX<-paste("<",mu-criticox," y >",mu+criticox)
+     }else{
+       criticot<-round(qt(1-alfa,grados),7)
+       criticox<-round(criticot*desv,7)
+       if(colas=="I"){
+         pvalor<-round(pt(estadistico,grados),7)
+         r$PValor<-pvalor
+         r$Estadistico<-estadistico
+         r$RegionRechazoT<-paste("<",-criticot)
+         r$RegionRechazoX<-paste("<",mu-criticox)
+       }else{
+         pvalor<-round(pt(estadistico,grados,lower.tail=F),7)
+         r$PValor<-pvalor
+         r$Estadistico<-estadistico
+         r$RegionRechazoT<-paste(">",criticot)
+         r$RegionRechazoX<-paste(">",mu+criticox)
+       }
+     }
+   }
+   return(r)
+ }
> if(is.null(alfa)){
+   if(!is.null(sigma1)){
+     Test<-TestMedia(n1,n2,mu,m,sigma1,sigma2,colas=cola,conocidas=TRUE,
+                     pareadas=par)
+   }else{
+     Test<-TestMedia(n1,n2,mu,m,s1,s2,sD=sD,colas=cola,conocidas=FALSE,
+                     iguales=desv.iguales,pareadas=par)
+   }
+ }else{
+   if(!is.null(sigma1)){
+     Test<-TestMedia(n1,n2,mu,m,sigma1,sigma2,alfa=alfa,colas=cola,
+                     conocidas=TRUE,pareadas=par)
+   }else{
+     Test<-TestMedia(n1,n2,mu,m,s1,s2,sD=sD,alfa=alfa,colas=cola,
+                     conocidas=FALSE,iguales=desv.iguales,pareadas=par)
+   }
+   resultado<-ifelse(Test[,"PValor"]>=alfa,"No se rechaza H0","Se rechaza H0")
+   Test$Resultado<-resultado
+   Test$Resultado[is.na(Test$Resultado)]<-"Pocos datos"
+ }
> Test
  Prueba H0 n1 n2 DifMedias desv.est1 desv.est2 error.est alpha PValor
1      Z  0 30 30        34      10.5      10.2  2.672639  0.05      0
  Estadistico RegionRechazoZ RegionRechazoX
1    12.72151    > 1.6448536    > 4.3961001
 \end{verbatim}
 \vspace{-0.5cm}
 N\'otese que el c\'odigo muestra en el resultado
 que se us\'o una prueba $Z$,
 esto es porque los tama\~nos muestrales son mayores
 o iguales que $30$.
 Por lo tanto, nunca va a mostrar una prueba $t$.
 En la pr\'actica no importa
 porque la prueba $Z$ es muy pr\'oxima a la prueba $t$
 en estos casos y, por ello, con esta prueba basta para concluir;
 sin embargo, y \'unicamente para demostrar la coincidencia
 con los resultados de las cuentas realizadas en este ejercicio,
 se har\'a un cambio s\'olo para este ejercicio
 y as\'{\i} muestre el resultado de la prueba $t$.
 Esto se hace eliminando el siguiente fragmento del c\'odigo:
 \begin{verbatim}
> if(n1>=30 & n2>=30){
+    if(is.null(sigma1)) sigma1<-s1
+    if(is.null(sigma2)) sigma2<-s2
+ }
 \end{verbatim}
 \vspace{-0.5cm}
 que se encuentra casi inmediatamente despu\'es
 de la secci\'on modificable por el usuario.
 Entonces, sin hacer ning\'un otro cambio, el programa de R lanza
 el siguiente resultado:
 \begin{verbatim}
> Test
  Prueba var.pobl H0 n1 n2 DifMedias desv.est1 desv.est2   est.sp error.est
1      t  Iguales  0 30 30        34      10.5      10.2 10.35109  2.672639
  grados.libertad alpha PValor Estadistico RegionRechazoT RegionRechazoX
1              58  0.05      0    12.72151    > 1.6715528    > 4.4674574
 \end{verbatim}
 \vspace{-0.5cm}
 Lo cual coincide con los resultados obtenidos,
 adem\'as de dar m\'as informaci\'on como el error est\'andar
 y una regi\'on de rechazo, tanto para el estad\'{\i}stico 
 como para $\bar{x}_1-\bar{x}_2$, suponiendo que $\alpha=0.05$,
 que es a lo que se quer\'{\i}a llegar.${}_{\blacksquare}$
\end{solucion}
