\begin{enunciado}
 Las siguientes respuestas con respecto al est\'andar de vida al momento
 de una encuesta de opini\'on independiente de $1\,000$ familias
 contra un a\~no antes parece estar de acuerdo con los resultados
 de un estudio publicado en \textit{Across the Board} (junio de 1981):
 \begin{center}
  \begin{tabular}{lcccr}
   & \multicolumn{3}{c}{\textbf{Est\'andar de vida}} & \\
   \cline{2-4}
   & \textbf{Algo} & & \textbf{No tan} & \\
   \textbf{Periodo} & \textbf{mejor} & \textbf{Igual} & \textbf{bueno} &
   \textbf{Total} \\
   \hline
   1980: Enero & $72$ & $144$ & $84$ & $300$ \\
   $\phantom{1980:}$ Mayo & $63$ & $135$ & $102$ & $300$ \\
   $\phantom{1980:}$ Septiembre & $47$ & $100$ & $53$ & $200$ \\
   1981: Enero & $40$ & $105$ & $55$ & $200$
  \end{tabular}
 \end{center}
 Pruebe la hip\'otesis
 de que las proporciones de familias dentro de cada est\'andar de vida
 son las mismas para cada uno de los cuatro periodos.
 Utilice un valor $P$.
\end{enunciado}

\begin{solucion}
 \begin{datos}
  $\phantom{0}$
  \begin{itemize}
   \item Tamaño de muestra total: $1\,000$.
   \item Familias encuestadas en Enero de 1980: $300$.
   \item Familias encuestadas en Mayo de 1980: $300$.
   \item Familias encuestadas en Septiembre de 1980: $200$.
   \item Familias encuestadas en Enero de 1981: $200$.
   \item Opiniones de est\'andar ``Algo mejor'': $72+63+47+40 = 222$.
   \item Opiniones de est\'andar ``Igual'': $144+135+100+105 = 484$.
   \item Opiniones de est\'andar ``No tan bueno'': $84+102+53+55 = 294$.
   \item Frecuencias observadas y esperadas: $o_{i,j}$
   y $e_{i,j}=\frac{R_i C_j}{n}$, respectivamente,
   donde $R_i$ y $C_j$ son los marginales del rengl\'on $i$ y la columna $j$,
   respectivamente, y $n$ es el total de toda la muestra.
   As\'{\i}, pues, redondeando a un decimal, se muestra el resumen 
   en la siguiente tabla,
   en donde aparece entre par\'entesis la frecuencia esperada
   y a la izquierda el valor observado:
   \begin{center}
    \begin{tabular}{lcccr}
     & \multicolumn{3}{c}{\textbf{Est\'andar de vida}} & \\
     \cline{2-4}
     & \textbf{Algo} & & \textbf{No tan} & \\
     \textbf{Periodo} & \textbf{mejor} & \textbf{Igual} & \textbf{bueno} &
     \textbf{Total} \\
     \hline
     1980: Enero & $72 (66.6)$ & $144 (145.2)$ & $84 (88.2)$ & 
     $300$ \\
     $\phantom{1980:}$ Mayo & $63 (66.6)$ & $135 (145.2)$ & $102 (88.2)$ &
     $300$ \\
     $\phantom{1980:}$ Septiembre & $47(44.4)$ & $100(96.8)$ & $53(58.8)$ &
     $200$ \\
     1981: Enero & $40 (44.4)$ & $105 (96.8)$ & $55 (58.8)$ & 
     $200$ \\
     \hline
     \textbf{Marginal por est\'andar} & $222$ & $484$ & $294$ & $n=1\,000$
    \end{tabular}
   \end{center}
   \item Tama\~no de la tabla de contingencia: $r\times c = 4\times 3$.
   \item Grados de libertad de la prueba $\chi^2$: $v = (r-1)(c-1) = 6$.
  \end{itemize}
 \end{datos}
 
 \begin{hipotesis}
  \begin{eqnarray*}
   H_0: & & \text{Las familias en cada est\'andar de vida
   son homog\'eneas.} \\
   H_1: & & \text{Las familias en cada est\'andar de vida
   no son homog\'eneas.}
  \end{eqnarray*}
 \end{hipotesis}

 \begin{estadistico}
  \begin{eqnarray*}
   \chi^2 & = & \sum_{i} \frac{\left( o_i - e_i \right)^2}{e_i} \\
   & = & \frac{(72 - 66.6)^2}{66.6} + \frac{(63 - 66.6)^2}{66.6} +
   \frac{(47 - 44.4)^2}{44.4} + \frac{(40 - 44.4)^2}{44.4} + \\
   & & \frac{(144 - 145.2)^2}{145.2} + \frac{(135 - 145.2)^2}{145.2} + \frac{(100 - 96.8)^2}{96.8} + \frac{(105 - 96.8)^2}{96.8} + \\
   & & \frac{(84 - 88.2)^2}{88.2} + \frac{(102 - 88.2)^2}{88.2} +
   \frac{(53 - 58.8)^2}{58.8} + \frac{(55 - 58.8)^2}{58.8} \\
   & = & \frac{29.16 + 12.96}{66.6} + \frac{6.76 + 19.36}{44.4} +
   \frac{1.44 + 104.04}{145.2} + \frac{10.24 + 67.24}{96.8} + \\
   & & \frac{17.64 + 190.44}{88.2} + \frac{33.64 + 14.44}{58.8} \\
   & \approx & 0.6324 + 0.5883 + 0.7264 + 0.8004 + 2.3592 + 0.8177 \\
   & = & 5.9244
  \end{eqnarray*}
 \end{estadistico}

 \begin{valorp}
  De la tabla A.5 se observa que, con $6$ grados de libertad, se tienen
  $P\left(\chi^2 > 5.348\right) \approx 0.5$ y 
  $P\left(\chi^2 > 7.231\right) \approx 0.3$, por lo que, interpolando,
  se aproxima que $P\left(\chi^2 > 5.9244 \right) \approx 0.44$
  lo cual es una probabilidad considerablemente alta.
 \end{valorp}

 \begin{decision}
  No se rechaza $H_0$.
 \end{decision}

 \begin{conclusion}
  La muestra no lanza evidencia para rechazar la hip\'otesis nula
  y, por lo tanto, se considera que est\'andares de vida
  en los diferentes periodos son homog\'eneas, es decir, 
  las proporciones de familias dentro de cada est\'andar de vida
  son las mismas.
 \end{conclusion}

 Finalmente, usando el archivo anexo
 \texttt{P18\_Prueba\_de\_independencia\_y\_homogeniedad\_01.r},
 que a su vez requiere los datos del archivo
 \texttt{BD33\_Problema\_097.csv}, con los siguientes cambios:
 \begin{verbatim}
> datos<-read.csv("DB33_Problema_097.csv",sep=";",encoding="UTF-8")
> varInteres<-c("Período","Estándar.vida")
> varFrecuencia<-"Frecuencia"
> pruebas<-c(1,2,3)
 \end{verbatim}
 \vspace{-0.5cm}
 el programa de R lanza el siguiente resultado:
 \begin{verbatim}
$tabla
                 Estándar.vida
Período          Algo mejor Igual No tan bueno
  1980.Enero              72   144           84
  1980.Mayo               63   135          102
  1980.Septiembre         47   100           53
  1981.Enero              40   105           55

$listaPruebas
$listaPruebas[[1]]

	Pearson's Chi-squared test

data:  tbl1
X-squared = 5.9245, df = 6, p-value = 0.4317


$listaPruebas[[2]]

	Log likelihood ratio (G-test) test of independence without correction

data:  tbl1
Log likelihood ratio statistic (G) = 5.8504, X-squared df = 6, p-value =
0.4402


$listaPruebas[[3]]

	Log likelihood ratio (G-test) test of independence with Williams' correction

data:  tbl1
Log likelihood ratio statistic (G) = 5.8276, X-squared df = 6, p-value =
0.4428
 \end{verbatim}
 \vspace{-0.5cm}
 Lo cual coincide con los resultados obtenidos, adem\'as de brindar m\'as
 precisi\'on e informaci\'on de otros estad\'{\i}sticos,
 que es lo que se quer\'{\i}a llegar.${}_{\blacksquare}$
\end{solucion}
