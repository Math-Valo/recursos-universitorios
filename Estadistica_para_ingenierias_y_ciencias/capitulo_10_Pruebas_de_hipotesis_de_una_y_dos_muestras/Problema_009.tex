\begin{enunciado}
 Repita el ejercicio 10.8 cuando se tratan $100$ manchas y la regi\'on cr\'{\i}tica se define como $x > 82$, donde $x$ es el n\'umero de manchas que se eliminan.
\end{enunciado}

\begin{solucion}
 Usando los t\'erminos del ejercicio 10.8, se tiene ahora que:
 \begin{itemize}
  \item $X \sim b(n,p) \sim n\left( \mu = np, \sigma = \sqrt{npq} \right)$.
  \item $n = 100$.
  \item $x_{\text{sup}} = 81$.
 \end{itemize}
 El resultado se calcular\'a a trav\'es de la aproximaci\'on de la binomial a la normal. El valor preciso binomial es poco pr\'actico obtenerlo, por lo que no se precisar\'a. Finalmente, se realizar\'a las comparaciones con los c\'alculos binomiales y normales en R.
 \par 
 Para el primer supuesto, se tiene adem\'as que
 \begin{itemize}
  \item $p = 0.7$.
  \item $\mu = np = (100)(0.7) = 70$.
  \item $\sigma^2 = npq = 70(0.3) = 21$.
 \end{itemize}
 as\'{\i}, el error tipo I se aproxima usando la tabla A.1 como sigue:
 \begin{eqnarray*}
  \alpha & = & P(X \geq 82) = 1 - P(X < 82) \approx 1 - P\left( Z < \frac{81.5 - 70}{\sqrt{21}} \right) \\
  & = & 1 - P\left( Z < \frac{11.5\sqrt{21}}{21} \right) = 1- P\left( Z < \frac{23\sqrt{21}}{42} \right) \approx 1 - P(Z <  2.51) \approx 1 - 0.994 \\
  & = & 0.006
 \end{eqnarray*}
 Para el siguiente supuesto, se tiene que
 \begin{itemize}
  \item $p = 0.9$.
  \item $\mu = np = (100)(0.9) = 90$.
  \item $\sigma^2 = npq = 90(0.1) = 9$
 \end{itemize}
 as\'{\i}, el error tipo II se aproximar usando la tabla A.1 como sigue:
 \begin{eqnarray*}
  \beta & = & P(X < 82) \approx P\left( Z < \frac{81.5 - 90}{\sqrt{9}} \right) = P\left( Z < -\frac{8.5}{3} \right) = P\left( Z < -\frac{17}{6} \right) \\
  & \approx & P(Z < -2.83) \approx 0.0023
 \end{eqnarray*}
 Finalmente, usando R, se calcula primero estas probabilidades con el script del archivo anexo \texttt{P01\_Probabilidad\_de\_error\_binomial\_1.r}, cambiando las siguientes l\'{\i}neas de c\'odigo:
 \begin{verbatim}
> n<-100
> CriticoInf<-NULL
> CriticoSup<-81
> p0<-0.7
> p1<-0.9
 \end{verbatim}
 \vspace{-0.5cm}
 con lo que se obtiene
 \begin{verbatim}
$`Probabilidad de error tipo I`
  HipotesisNula   n CriticoSup       alpha
1           0.7 100         81 0.004522639

$`Probabilidad de error tipo II`
  HipotesisAlternativa   n CriticoSup        beta
1                  0.9 100         81 0.004580754
 \end{verbatim}
 \vspace{-0.5cm}
 Mientras que con el script del archivo anexo \texttt{P02\_Probabilidad\_de\_error\_normal\_1.r}, cambiando las siguientes l\'{\i}neas de c\'odigo:
 \begin{verbatim}
> n<-100
> CriticoInf<-NULL
> CriticoSup<-81
> desv<-NULL
> media0<-NULL
> media1<-NULL
> p0<-0.7
> p1<-0.9
 \end{verbatim}
 \vspace{-0.5cm}
 se obtiene lo siguiente:
 \begin{verbatim}
$`Probabilidad de error tipo I`
  HipotesisNula   n media     desv CriticoSup       alpha
1      p =  0.7 100    70 4.582576       81.5 0.006045013

$`Probabilidad de error tipo II`
  HipotesisAlternativa   n media desv CriticoSup        beta
1             p =  0.9 100    90    3       81.5 0.002303266
 \end{verbatim}
 \vspace{-0.5cm}
 Por lo tanto, se tiene el siguiente resumen:
 \begin{itemize}
  \item Bajo el supuesto real $p = 0.7$:
  \begin{itemize}
   \item La aproximaci\'on con las tablas normales da $\alpha = 0.006$.
   \item La aproximaci\'on binomial con R da $\alpha = 0.004522639$.
   \item La aproximaci\'on normal con R da $\alpha = 0.006045013$.
  \end{itemize}

  \item Y, bajo el supuesto $p  = 0.9$:
  \begin{itemize}
   \item La aproximaci\'on con las tablas normales da $\beta = 0.0023$.
   \item La aproximaci\'on binomial con R da $\beta = 0.004580754$.
   \item La aproximaci\'on normal con R da $\beta = 0.002303266$.
  \end{itemize}
 \end{itemize}
 que es a lo que se quer\'{\i}a llegar.${}_{\blacksquare}$
\end{solucion}
