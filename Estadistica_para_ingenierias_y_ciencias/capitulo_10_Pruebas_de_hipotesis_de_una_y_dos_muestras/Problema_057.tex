\begin{enunciado}
 Se lanza $20$ veces una moneda, con un resultado de $5$ caras.
 ¿\'Esta es suficiente evidencia para rechazar la hip\'otesis
 de que la moneda est\'a balanceada, a favor de la alternativa
 de que las caras ocurren menos de $50\%$ de las veces?
 Cite un valor $P$.
\end{enunciado}

\begin{solucion}
 \begin{datos}
  $\phantom{0}$
  \begin{itemize}
   \item $n = 20$.
   \item $x = 5$.
  \end{itemize}
 \end{datos}

 \begin{hipotesis}
  \begin{eqnarray*}
   H_0: p & = & 0.5 \\
   H_1: p & < & 0.5
  \end{eqnarray*}
 \end{hipotesis}

 \begin{estadistico}
  Variable binomial $X$ con $p = 0.5$ y $n = 20$.
 \end{estadistico}

 \begin{valorp}
  De la tabla A.1, se tiene la siguiente aproximaci\'on:
  \begin{equation*}
   P\left( X \leq 5 | p = 0.5 \right) = \sum_{i=0}^{5} b\left( x; 20, 0.5 \right)
   \approx 0.0207
  \end{equation*}
  Mientras que el valor preciso se obtiene como sigue:
  \begin{eqnarray*}
   P\left( X \leq 5 | p = 0.5 \right) & = & \sum_{i=0}^{5} b\left( x; 20, 0.5 \right) \\
   & = & \frac{1}{10^{20}} \left( 5^{20} + 20\cdot 5\cdot 5^{19} + 
   190 \cdot 5^2 \cdot 5^{18} + 1\,140\cdot 5^3 \cdot 5^{17} +
   4\,845 \cdot 5^4 \cdot 5^{16} + \right. \\
   & & \left. 15\,504\cdot 5^5 \cdot 5^{15} \right) \\
   & = & \frac{5^{20}}{10^{20}} (1 + 20 + 190 + 1\,140 + 4\,845 + 15\,504)
   = \frac{95\,367\,431\,640\,625 \cdot 21\,700}{100\,000\,000\,000\,000\,000\,000} \\
   & = & \frac{2\,069\,473\,266\,601\,562\,500}{100\,000\,000\,000\,000\,000\,000}
   = 0.020694732666015625
  \end{eqnarray*}
 \end{valorp}

 \begin{decision}
  Se rechaza $H_0$ a favor de $H_1$.
 \end{decision}

 \begin{conclusion}
  El valor $P$ es muy peque\~no, por lo que se evidencia la falta de probabilidad
  de que est\'e balanceada; es decir, hay pruebas suficientes para rechazar dicha
  hip\'otesis y concluir que la moneda no est\'a balanceada.
 \end{conclusion}
 Finalmente, usando el archivo anexo \texttt{P08\_Prueba\_de\_una\_proporcion\_01.r},
 con los siguientes cambios:
 \begin{verbatim}
> n<-20
> x<-5
> p0<-NULL
> p<-0.5
> alfa<-NULL
> cola<-'I'
> distr<-NULL
 \end{verbatim}
 \vspace{-0.5cm}
 el programa de R lanza el siguiente resultado:
 \begin{verbatim}
     distr   p  n x pMuestral media desv.est alpha
1 Binomial 0.5 20 5      0.25    10 2.236068  0.05
                          Estadistico    PValor RegionRechazoX
1 Var. binomial X con p= 0.5  y n= 20 0.0206947           <= 5
 \end{verbatim}
 \vspace{-0.5cm}
 El cual coincide con los resultados obtenidos,
 que es a lo que se quer\'{\i}a llegar.${}_{\blacksquare}$
\end{solucion}
