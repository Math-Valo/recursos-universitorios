\begin{enunciado}
 De acuerdo con un estudio de la Universidad Johns Hopkins publicado
 En \textit{American Journal of Public Health}, las viudas viven m\'as
 que los viudos.
 Considere los siguientes datos de supervivencia de $100$ viudas
 y $100$ viudos despu\'es de la muerte del c\'onyugue:
 \begin{center}
  \begin{tabular}{lcc}
   \textbf{A\~nos de vida} & \textbf{Viuda} & \textbf{Viudo} \\
   \hline
   Menos de 5 & $25$ & $39$ \\
   de 5 a 10 & $42$ & $40$ \\
   M\'as de 10 & $33$ & $21$
  \end{tabular}
 \end{center}
 ¿Con un nivel de significancia de $0.05$ podemos concluir
 que las proporciones de viudas y viudos son iguales
 con respecto a los diferentes periodos que un c\'onyugue sobrevive
 luego de la muerte de su compa\~nero?
\end{enunciado}

\begin{solucion}
 \begin{datos}
  $\phantom{0}$
  \begin{itemize}
   \item Tamaño de muestra total: $200$.
   \item Tama\~no muestral de viudas: $100$.
   \item Tama\~no muestral de viudos: $100$.
   \item Tama\~no muestral de quienes vivieron menos de 5 a\~nos: $64$.
   \item Tama\~no muestral de quienes vivieron de 5 a 10 a\~nos: $82$.
   \item Tama\~no muestral de quienes vivieron m\'as de 5 a\~nos: $54$.
   \item Frecuencias observadas y esperadas: $o_{i,j}$
   y $e_{i,j}=\frac{R_i C_j}{n}$, respectivamente,
   donde $R_i$ y $C_j$ son los marginales del rengl\'on $i$ y la columna $j$,
   respectivamente, y $n$ es el total de toda la muestra.
   As\'{\i}, pues, redondeando a un decimal, se muestra el resumen 
   en la siguiente tabla,
   en donde aparece entre par\'entesis la frecuencia esperada
   y a la izquierda el valor observado:
   \begin{center}
    \begin{tabular}{lcc|c}
     & & & \textbf{Marginal por a\~nos} \\
     \textbf{A\~nos de vida} & \textbf{Viuda} & \textbf{Viudo} &
     \textbf{de supervivencia} \\
     \hline
     Menos de 5 & $25 (32)$ & $39 (32)$ & $64$ \\
     de 5 a 10 & $42 (41)$ & $40 (41)$ & $82$ \\
     M\'as de 10 & $33 (27)$ & $21 (27)$ & $54$ \\
     \textbf{Marginal por sexo} & $100$ & $100$ & \textbf{TOTAL:} $n=200$
    \end{tabular}
   \end{center}
   \item Tama\~no de la tabla de contingencia: $r\times c = 3\times 2$.
   \item Grados de libertad de la prueba $\chi^2$: $v = (r-1)(c-1) = 2$.
  \end{itemize}
 \end{datos}
 
 \begin{hipotesis}
  \begin{eqnarray*}
   H_0: & & p_1 = p_2 = p_3,
   \text{ las proporciones de supervivencias por g\'enero son iguales} \\
   H_1: & & \text{Alguna de las proporciones de supervivencia por g\'enero
   no son iguales.}
  \end{eqnarray*}
 \end{hipotesis}

 \begin{significancia}
  $\alpha = 0.05$.
 \end{significancia}

 \begin{region}
  De la tabla A.5, se tiene el valor cr\'{\i}tico
  $\chi^2_{\alpha,v} = \chi^2_{0.05,2} \approx 5.991$,
  por lo que la regi\'on de rechazo est\'a dado
  para $\chi^2 > 5.991$, donde
  $\chi^2 = \sum_{i} \frac{\left( o_i - e_i \right)^2}{e_i}$.
 \end{region}

 \begin{estadistico}
  \begin{eqnarray*}
   \chi^2 & = & \sum_{i} \frac{\left( o_i - e_i \right)^2}{e_i} \\
   & \approx & \frac{(25 - 32)^2}{32} + \frac{(39 - 32)^2}{32} +
   \frac{(42 - 41)^2}{41} + \frac{(40 - 41)^2}{41} + 
   \frac{(33 - 27)^2}{27} + \frac{(21 - 27)^2}{27} \\
   & = & \frac{49 + 49}{32} + \frac{1 + 1}{41} + 
   \frac{36 + 36}{27} = \frac{98}{32} + \frac{2}{41} + \frac{72}{27}
   \approx 3.0625 + 0.0488 + 2.6667 \\
   & = & 5.778
  \end{eqnarray*}
 \end{estadistico}

 \begin{decision}
  No se rechaza $H_0$.
 \end{decision}

 \begin{conclusion}
  A un nivel de significancia de $0.05$, los resultados arrojan
  que no hay evidencia que rechace la diferencia de las proporciones,
  por lo que se puede considerar las proporciones de supervivencia por
  el viudo y viudo como iguales.
 \end{conclusion}

 Finalmente, usando el archivo anexo
 \texttt{P18\_Prueba\_de\_independencia\_y\_homogeniedad\_01.r},
 que a su vez requiere los datos del archivo
 \texttt{BD32\_Problema\_096.csv}, con los siguientes cambios:
 \begin{verbatim}
> datos<-read.csv("DB32_Problema_096.csv",sep=";",encoding="UTF-8")
> varInteres<-c("Sexo","Vida.años")
> varFrecuencia<-"Frecuencia"
> pruebas<-c(1)
 \end{verbatim}
 \vspace{-0.5cm}
 el programa de R lanza el siguiente resultado:
 \begin{verbatim}
$tabla
       Vida.años
Sexo    de 5 a 10 Más de 10 Menos de 5
  Viuda        42        33         25
  Viudo        40        21         39

$listaPruebas
$listaPruebas[[1]]

	Pearson's Chi-squared test

data:  tbl1
X-squared = 5.7779, df = 2, p-value = 0.05563
 \end{verbatim}
 \vspace{-0.5cm}
 Lo cual coincide con los resultados obtenidos,
 adem\'as de brindar la informaci\'on del $P-$valor.
 N\'otese que, aunque se ha usado el programa de independencia
 y homogeneidad, el resultado es v\'alidos debido a que corresponden
 a los mismo procedimientos;
 sin embargo, no se ha usado la prueba G ya que no se ha encontrado a\'un
 en la literatura algo que valide dicha prueba para proporciones.
 Que es lo que se quer\'{\i}a llegar.${}_{\blacksquare}$
\end{solucion}
