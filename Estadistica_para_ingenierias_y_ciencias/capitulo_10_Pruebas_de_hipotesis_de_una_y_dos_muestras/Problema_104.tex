\begin{enunciado}
 Se realiza un estudio para determinar
 si, en las bodas, m\'as italianos que estadounidenses prefieren
 la champa\~na blanca en vez de la rosada.
 De los $300$ italianos que se seleccionaron al azar, $72$ prefieren
 champa\~na blanca, y de los $400$ estadounidenses seleccionados
 $70$ prefieren champa\~na blanca en vez de la rosada.
 ¿Podemos concluir que una proporci\'on mayor
 de italianos que de estadounidenses prefiere champa\~na blanca
 en las bodas?
 Utilice un nivel de significancia de $0.05$
\end{enunciado}

\begin{solucion}
 Usando el archivo anexo \texttt{P09\_Prueba\_de\_dos\_proporciones\_01.r}, con los siguientes cambios:
 \begin{verbatim}
> n1<-300
> n2<-400
> x1<-72
> x2<-70
> p1<-NULL
> p2<-NULL
> alfa<-0.05
> alternativa<-'>'
 \end{verbatim}
 \vspace{-0.5cm}
 el programa de R lanza el siguiente resultado:
 \begin{verbatim}
  alternativa  n1  n2 x1 x2   p1    p2 pEstimada DifProp  error.est alpha
1     p1 > p2 300 400 72 70 0.24 0.175 0.2028571   0.065 0.03071296  0.05
     PValor Estadistico RegionRechazoZ     Resultado
1 0.0171566    2.116371   >= 1.6448536 Se rechaza H0
 \end{verbatim}
 \vspace{-0.5cm}
 El cual indica que los datos: $n_1 = 300$ y $n_2 = 400$,
 con $x_1 = 72$ y $x_2 = 70$; la hip\'otesis nula: $H_0: p_1 \leq p_2$
 contra la hip\'otesiis alternativa: $H_1: p_1 > p_2$; el nivel
 de significancia: $\alpha = 0.05$; el valor del estad\'{\i}stico $z \approx
 \frac{\hat{p}_1 - \hat{p}_2}{\sqrt{\hat{p}\hat{q}(1/n_1+1/n_2)}} \approx
 2.116371$; la regi\'on de rechazo $z > 1.6448536$,
 con todo y la decisi\'on de rechazar $H_0$;
 e, incluso, el valor $P$ de $P(z > 2.116371) \approx 0.0171566$.
 \par 
 Luego entonces se concluye que los datos arrojan evidencia suficiente
 para rechazar $H_0$; es decir, se puede concluir que la proporci\'on
 de italianos que prefiere la champa\~na blanca en las bodas
 es mayor que la proporci\'on de estadounidenses
 que la as\'{\i} la prefieren,
 que es a lo que se quer\'{\i}a llegar.${}_{\blacksquare}$
\end{solucion}
