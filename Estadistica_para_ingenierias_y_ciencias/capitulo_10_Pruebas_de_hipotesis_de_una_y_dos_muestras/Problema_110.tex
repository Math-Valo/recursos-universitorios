\begin{enunciado}
 Se lleva a cabo un estudio en el Centro de Medicina Veterinaria Equina
 de la Universidad Regional de Virgini-Maryland, para determinar
 si la realizaci\'on de cierto tipo de cirug\'{\i}a en caballos j\'ovenes
 tiene alg\'un efecto en ciertas clases de c\'elulas sangu\'{\i}neas
 en el animal.
 Se toman muestras del fluido de cada uno de seis potros antes y despu\'es
 de la cirug\'{\i}a.
 Se analizan las muestras para el n\'umero de leucogramos
 de gl\'obulos blancos (\texttt{WBC}) posoperatorios.
 Tambi\'en se realiza una medici\'on de leucogramos \texttt{WBC}
 preoperatorios.
 Utilice una prueba $t$ de una muestra pareada
 para determinar si hay un cambio significativo en los leucogramos
 \texttt{WBC} con la cirug\'{\i}a.
 \begin{center}
  \begin{tabular}{ccc}
   \textbf{Potro} & \textbf{Precirug\'{\i}a*} & \textbf{Postcirug\'{\i}a*}
   \\
   \hline 
   1 & $10.80$ & $10.60$ \\
   2 & $12.90$ & $16.60$ \\
   3 & $ 9.59$ & $17.20$ \\
   4 & $ 8.81$ & $14.00$ \\
   5 & $12.00$ & $10.60$ \\
   6 & $ 6.07$ & $ 8.60$ \\
   \hline
   \multicolumn{3}{l}{*Todos los valores $\times 10^{-3}$.}
  \end{tabular}
 \end{center}
\end{enunciado}

\begin{solucion}
 Como se menciona en el ejercicio, el procedimiento corresponde
 a una prueba de muestras pareada.
 Para ello, primero habr\'a que verificar que las muestras provienen
 de poblaciones normales.
 Esto es, antes que todo, probar la hip\'otesis nula que se enuncia como: $X_1$ y $X_2$, las variables aleatorias correspondientes
 a las mediciones de WBC preoperatorios y posoperatorios,
 tienen distribuciones normales, contra la hip\'otesis alternativa,
 de que no es as\'{\i}.
 \par 
 El proceso que generalmente se usa para probar la normalidad 
 se encuentra registrado en el archivo anexo
 \texttt{P17\_Prueba\_de\_normalidad\_01.r}
 que, en este caso, usar\'{\i}a el archivo 
 \texttt{DB41\_Problema\_110.csv} que contiene la base de datos
 escritos en la tabla;
 sin embargo, se trata de muy pocos datos
 y el programa causa un error debido que,
 al colapsar las frecuencias para tener al menos una frecuencia esperada
 de 5 unidades por clase, se obtiene una \'unica clase,
 lo cual no permite realizar la prueba $\chi^2$
 por no tener una valor positivo de grados de libertad.
 Por lo tanto, se proceder\'a a hacer la prueba de Geary
 con las siguientes l\'{\i}neas de c\'odigo junto con sus resultados:
 \begin{verbatim}
> datos<-read.csv("DB41_Problema_110.csv",sep=";",encoding="UTF-8")
> varInteres<-"WBC.cantidad"; distincion<-"Muestra"
> lista<-split(datos[,varInteres],datos[,distincion])
> x<-lista$Precirugía
> y<-lista$Postcirugía
> n1<-length(x); n2<-length(y)
> U1<-sqrt(pi/2)*mean(abs(x-mean(x)))/sqrt(var(x)*(n1-1)/n1)
> U2<-sqrt(pi/2)*mean(abs(y-mean(y)))/sqrt(var(y)*(n2-1)/n2)
> x.Geary.statistic<-(U1-1)/(0.2661/sqrt(n1))
> y.Geary.statistic<-(U2-1)/(0.2661/sqrt(n2))
> x.Geary.p.value<-2*pnorm(abs(x.Geary.statistic),lower.tail=FALSE)
> y.Geary.p.value<-2*pnorm(abs(y.Geary.statistic),lower.tail=FALSE)
> print("P-valor de datos precirugía usando la prueba de Geary"); x.Geary.p.value
[1] "P-valor de datos precirugía usando la prueba de Geary"
[1] 0.6601252
> print("P-valor de datos precirugía usando la prueba de Geary"); y.Geary.p.value
[1] "P-valor de datos precirugía usando la prueba de Geary"
[1] 0.1278513
 \end{verbatim}
 \vspace{-0.5cm}
 Lo cual indica un buen ajuste a una normal de los datos precirug\'{\i}a
 y postcirug\'{\i}a, al tener, en ambos casos, valores $P$ mayores a 0.1.
 \par 
 Ahora que se tienen las condiciones iniciales, la prueba con muestras
 pareadas se puede realizar definiendo primero la hip\'otesis de prueba:
 \begin{eqnarray*}
  H_0: & & \mu_D = \mu_1 - \mu_2  =   0 \\
  H_1: & & \mu_D = \mu_1 - \mu_2 \neq 0
 \end{eqnarray*}
 usando el estad\'{\i}stico $t$ con $n_1 + n_2 - 2 = 10$ grados
 de libertad.
 Entonces, los valores de los c\'alculos se pueden obtener
 usando el archivo anexo
 \texttt{P06\_Prueba\_de\_dos\_medias\_02.r},
 que a su vez usa la base de datos en el archivo ya mencionado
 \texttt{DB41\_Problema\_110.csv}, con los siguientes cambios:
 \begin{verbatim}
> datos<-read.csv("DB41_Problema_110.csv",sep=";",encoding="UTF-8")
> varInteres<-c("WBC.cantidad")
> varSel<-c("Muestra")
> mu<-0
> desv.iguales<-NULL
> alfa<-NULL
> cola<-'D'
> par<-TRUE
 \end{verbatim}
 \vspace{-0.5cm}
 con lo que se obtiene el siguiente resultado:
 \begin{verbatim}
          Var1 Freq Poblaciones H0 n diferencia desv.par error.est grados alpha
1 WBC.cantidad    6    Pareadas  0 6     -2.905  3.35574  1.369975      5  0.05
      PValor Estadistico RegionRechazoInfT RegionRechazoSupT RegionRechazoInfX
1 0.08745276   -2.120477         -2.570582          2.570582         -3.521633
  RegionRechazoSupX
1          3.521633
 \end{verbatim}
 \vspace{-0.5cm}
 Lo cual muestra el valor del estad\'{\i}stico $t = -2.120477$,
 cuyo valor $P$ es $0.08745276$, el cual rechazar\'{\i}a la hip\'otesis
 nula si se considerara una significancia menor $0.0874528$.
 En este caso, como la prueba es de dos colas, el cual se sabe
 no es tan riguroso sobre cu\'ando rechazar una hip\'otesis nula, 
 entonces se considera este valor $P$ lo suficientemente alto
 como para decidir en este caso rechazar la hip\'otesis nula
 y, por lo tanto, concluir que s\'{\i} hay un cambio significativo
 en los leucogramos WBC con la cirug\'{\i}a,
 que es a lo que se quer\'{\i}a llegar.${}_{\blacksquare}$
\end{solucion}
