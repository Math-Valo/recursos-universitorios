\begin{enunciado}
 En un estudio que conduce el Centro de Recursos Acu\'aticos
 y que analiza el Centro de Consulta Estad\'{\i}stica
 del Instituto Polit\'ecnico y Universidad Estatal de Virginia,
 se comparan dos plantas de tratamiento para aguas residuales.
 La planta \textit{A} se ubica donde el ingreso medio de los hogares
 est\'a por abajo de $\$22,000$ al a\~no,
 y la planta \textit{B} se ubica donde el ingreso medio de los hogres
 est\'a por arriba de $\$60,000$ anuales.
 La cantidad de agua residual que trata cada planta
 (miles de galones/d\'{\i}a) se muestra de forma aleatoria
 durante 10 d\'{\i}as.
 Los datos son los siguientes:
 \begin{verbatim}
Planta A:
21 19 20 23 22 28 32 19 13 18
Planta B:
20 39 24 33 30 28 30 22 33 24
 \end{verbatim}
 \vspace{-0.5cm}
 ¿Con un nivel de significancia de $5\%$ podemos concluir que la cantidad promedio de agua residual tratada en el vecindario de altos ingresos es mayor que la del \'area de bajos ingresos?
\end{enunciado}

\begin{solucion}
 Para realizar una prueba de diferencia de medias con menos de 30 datos,
 se requiere la suposici\'on de que las muestras provengan de poblaciones
 con distribuci\'on normal.
 Para validar el uso de la prueba de dos medias, se realizar\'an pruebas
 de bondad de ajuste que indiquen si las muestras provienen de poblaciones
 normales.
 Entonces, la hip\'otesis nula para estas pruebas se pueden
 escribir de la forma: $X_i$ sigue una distribuci\'on normal,
 para cada $i \in \{1,2\}$,
 donde $X_1$ y $X_2$ representan los cantidades de agua residual,
 en miles de galones por d\'{\i}a, de las plantas A y B, respectivamente,
 contra la hip\'otesis alternativa de que $X_i$,
 para cada $i \in \{1,2\}$, no sigue una distribuci\'on normal.
 \par 
 Debido a que la muestra es peque\~na y no se consideran par\'ametros
 en la suposici\'on de las normalidades,
 la prueba con $\chi^2$ crea insuficientes clases, en cada muestra,
 que al calcular los grados de libertad junto con los grados que se restan
 por no considerar par\'ametros previos, entonces se obtiene
 un valor negativo, lo cual no puede corresponder a grados de libertad.
 Por lo tanto, se proceder\'a a realizar las pruebas de Geary,
 simult\'aneamente, con las siguientes l\'{\i}neas de c\'odigo
 escritas en R, que usa la base de datos
 \texttt{DB44\_Problema\_113.csv},
 que se presenta a continuaci\'on junto con los resultados:
 \begin{verbatim}
> datos<-read.csv("DB44_Problema_113.csv",sep=";",encoding="UTF-8")
> varInteres<-"AguaResidual.KGalPDia"; distincion<-"Planta"
> lista<-split(datos[,varInteres],datos[,distincion])
> x<-lista$A; y<-lista$B
> n1<-length(x); n2<-length(y)
> U1<-sqrt(pi/2)*mean(abs(x-mean(x)))/sqrt(var(x)*(n1-1)/n1)
> U2<-sqrt(pi/2)*mean(abs(y-mean(y)))/sqrt(var(y)*(n2-1)/n2)
> x.Geary.statistic<-(U1-1)/(0.2661/sqrt(n1))
> y.Geary.statistic<-(U2-1)/(0.2661/sqrt(n2))
> x.Geary.p.value<-2*pnorm(abs(x.Geary.statistic),lower.tail=FALSE)
> y.Geary.p.value<-2*pnorm(abs(y.Geary.statistic),lower.tail=FALSE)
> print("P-valor para la planta A usando la prueba de Geary");x.Geary.p.value
[1] "P-valor para la planta A usando la prueba de Geary"
[1] 0.5061923
> print("P-valor para la planta B usando la prueba de Geary");y.Geary.p.value
[1] "P-valor para la planta B usando la prueba de Geary"
[1] 0.4920951
 \end{verbatim}
 \vspace{-0.5cm}
 Esto indica un buen ajuste a una distribuci\'on normal
 para ambas variables aleatorias;
 Por lo tanto, es v\'alido suponer que ambas muestras vienen
 de poblaciones con distribuciones normales.
 \par
 Para la prueba, se requiere ahora saber si las varianzas poblacionales
 de donde vienen ambas muestras se deben de considerar iguales
 o diferentes. Este implica una prueba de proporci\'on de varianzas,
 cuya hip\'otesis nula se puede enunciar como
 $H_0: \, \frac{\sigma_1}{\sigma_2} = 1$, contra la hip\'otesis
 alternativa $H_1: \, \frac{\sigma_1}{\sigma_2} \neq 1$,
 donde $\sigma_1$ y $\sigma_2$ representan las varianzas poblacionales
 de $X_1$ y $X_2$, respectivamente.
 Para ello se hace una prueba $F$ con ayuda del archivo adjunto
 \texttt{P14\_Prueba\_de\_dos\_varianzas\_02.r},
 que a su vez usa el archivo \texttt{DB44\_Problema\_113.csv},
 con los siguientes cambios:
 \begin{verbatim}
> datos<-read.csv("DB44_Problema_113.csv",sep=";",encoding="UTF-8")
> varInteres<-c("AguaResidual.KGalPDia")
> varSel<-c("Planta")
> alfa<-NULL
> cola<-'D'
 \end{verbatim}
 \vspace{-0.5cm}
 lo cual arroja el siguiente resultado:
 \begin{verbatim}
               variable Freq n1 n2 media1 media2 varianza1 varianza2 v1 v2 alpha
1 AguaResidual.KGalPDia   20 10 10   21.5   28.3  28.27778  34.45556  9  9  0.05
     PValor Estadistico             RegionRechazo
1 0.7733006    0.820703 < 0.2483859 y > 4.0259942
 \end{verbatim}
 \vspace{-0.5cm}
 en donde se observa que el estad\'{\i}stico $F=\frac{\sigma_1}{\sigma_2}$
 toma el valor de $0.820703$ con $v_1 = v_2 = 9$ grados
 de libertad para la distribuci\'on $F$,
 obteniendo as\'{\i} un valor $P$ de $0.7733006$,
 lo cual, al ser muy alto, no se rechaza la hip\'otesis nula y,
 por lo tanto, se puede suponer las varianzas iguales para la prueba $t$.
 \par 
 Para la prueba se enuncia la hip\'otesis nula como sigue:
 \begin{eqnarray*}
  H_0: & & \mu_1 - \mu_2 \geq 0 \\
  H_1: & & \mu_1 - \mu_2   <  0
 \end{eqnarray*}
 donde $\mu_1$ y $\mu_2$ representan las medias poblacionales
 de $X_1$ y $X_2$, respectivamente,
 con un nivel de significancia de $\alpha = 0.05$.
 La regi\'on de rechazo, el estad\'{\i}stico y la decisi\'on
 se calculan con ayuda del archivo adjunto
 \texttt{P06\_Prueba\_de\_dos\_medias\_02.r},
 que nuevamente usa la base de datos almacenado en el archivo adjunto
 \texttt{DB44\_Problema\_113.csv}, con los siguientes cambios:
 \begin{verbatim}
> datos<-read.csv("DB44_Problema_113.csv",sep=";",encoding="UTF-8")
> varInteres<-c("AguaResidual.KGalPDia")
> varSel<-c("Planta")
> mu<-0
> desv.iguales<-TRUE
> alfa<-0.05
> cola<-'I'
> par<-FALSE
 \end{verbatim}
 \vspace{-0.5cm}
 lo cual arroja el siguiente resultado:
 \begin{verbatim}
                   Var1 Freq    Poblaciones H0 valorPVar     suposicionVar n1 n2
1 AguaResidual.KGalPDia   20 Independientes  0 0.7733006 Var no diferentes 10 10
  media1 media2 diferencia desv.est1 desv.est2   est.sp error.est grados alpha
1   21.5   28.3       -6.8  5.317685  5.869885 5.600595  2.504662     18  0.05
       PValor Estadistico RegionRechazoInfT RegionRechazoInfX     Resultado
1 0.007096931   -2.714937         -1.734064         -4.343244 Se rechaza H0
 \end{verbatim}
 \vspace{-0.5cm}
 de donde se observa que la regi\'on de rechazo est\'a dado
 para $T < -1.734064$,
 el estad\'{\i}stico obtenido es $t = -2.714937$ y la decisi\'on tomada,
 ya en este punto evidente pero que tambi\'en se muestra en el programa,
 es la de rechazar $H_0$ a favor de $H_1$ y, por lo tanto, concluir
 que, en efecto, la cantidad promedio de agua residual tratada
 en el vecindario de altos ingresos es mayor que la del \'area
 de bajos ingresos, que es a lo que se quer\'{\i}a llegar.${}_{\blacksquare}$
\end{solucion}
