\begin{enunciado}
 Se llev\'o a cabo un estudio en el Departamento de Veterinaria del Instituto Polit\'ecnico y Universidad Estatal de Virginia, para determinar si la ``resistencia'' de una herida de incisi\'on quir\'urgica resulta afectada por la temperatura del bistur\'{\i}. Se utilizaron $8$ perros en el experimento. La incisi\'on se realiz\'o en el abdomen de los animales. Se aplicaron una incisi\'on ``caliente'' y una ``fr\'{\i}a'' a cada perro, y se midi\'o la resistencia. Los datos que resultaron aparecen abajo.
 \begin{center}
  \begin{tabular}{crc}
   \textbf{Perro} & \textbf{Bistur\'{\i}} & \textbf{Resistencia} \\
   \hline 
   $1$ & Caliente & $\phantom{1}5120$ \\
   $1$ & Fr\'{\i}o & $\phantom{1}8200$ \\
   $2$ & Caliente & $10000$ \\
   $2$ & Fr\'{\i}o & $\phantom{1}8600$ \\
   $3$ & Caliente & $10000$ \\
   $3$ & Fr\'{\i}o & $\phantom{1}9200$ \\
   $4$ & Caliente & $10000$ \\
   $4$ & Fr\'{\i}o & $\phantom{1}6200$ \\
   $5$ & Caliente & $10000$ \\
   $5$ & Fr\'{\i}o & $10000$ \\
   $6$ & Caliente & $\phantom{1}7900$ \\
   $6$ & Fr\'{\i}o & $\phantom{1}5200$ \\
   $7$ & Caliente & $\phantom{19}510$ \\
   $7$ & Fr\'{\i}o & $\phantom{19}885$ \\
   $8$ & Caliente & $\phantom{1}1020$ \\
   $8$ & Fr\'{\i}o & $\phantom{19}460$
  \end{tabular}
 \end{center}
 \begin{enumerate}
  \item Escriba una hip\'otesis apropiada para determinar si hay una diferencia significativa en la resistencia entre las incisiones caliente y fr\'{\i}a.
  \item Pruebe la hip\'otesis mediante el uso de una prueba $t$ pareada. Utilice un valor $P$ en su conclusi\'on.
 \end{enumerate}
\end{enunciado}

\begin{solucion}
 $\phantom{0}$
 \begin{enumerate}
  \item Debido a que cada pareja de observaciones fueron sobre la misma unidad
  experimental, entonces la prueba debe de ser tratado como una prueba
  de observaciones pareadas; sin embargo, para ello se requiere la suposici\'on
  de que las distribuciones de cada poblaci\'on son normales.
  Esta suposici\'on no est\'a escrito sobre el enunciado,
  pero como no se conoce otro m\'etodo sin la suposici\'on y que sea apropiado
  para este tipo de problema presentado,
  entonces se considerar\'a la suposici\'on adicional de que las distribuciones de cada poblaci\'on son normales.
  \par 
  Entonces con lo antes mencionado, se procede a escribir una hip\'otesis,
  seg\'un lo pedido. Como se quiere probar que hay una diferencia significativa
  entre las incisiones caliente y fr\'{\i}a, entonces se debe de suponer,
  como hip\'otesis nula, que la media poblacional de las diferencias
  en las observaciones pareadas en cero, en contraste con que sea distinto de cero.
  Esto es:
  \begin{eqnarray*}
   H_0: \mu{}_D = \mu{}_1 - \mu{}_2 &   =  & 0 \\
   H_1: \mu{}_D = \mu{}_1 - \mu{}_2 & \neq & 0
  \end{eqnarray*}
  que es lo que se deseaba encontrar.${}_{\square}$
  
  \item
  \begin{datos}
   Resumiendo, y a\~nadiendo el supuesto adicional mencionado en el inciso anterior,
   se tiene que
   \begin{itemize}
    \item $X_i \sim n\left( \mu{}_i, \sigma_i \right)$, para cada $i \in \{1,2\}$.
    \item $n_1 = n_2 = 8$
   \end{itemize}
   Como ya se mencion\'o en el inciso anterior, se trata de una prueba
   de observaciones pareadas, por lo que se usar\'an las diferencias de los datos,
   que se muestra en la tabla siguiente:
   \begin{center}
    \begin{tabular}{cccr}
     & \textbf{Resistencia al} & \textbf{Resistencia al} & \\
     \textbf{Perro} & \textbf{Bistur\'{\i} caliente} &
     \textbf{Bistur\'{\i} fr\'{\i}o} & $d_i$ \\
     \hline
     $1$ & $5\,120$  & $8\,200$ & $-3\,080$ \\
     $2$ & $10\,000$ & $8\,600$ & $1\,400$  \\
     $3$ & $10\,000$ & $9\,200$ & $800$     \\
     $4$ & $10\,000$ & $6\,200$ & $3\,800$  \\
     $5$ & $10\,000$ & $10\,000$ & $0$      \\
     $6$ & $7\,900$  & $5\,200$ & $2\,700$  \\
     $7$ & $510$     & $885$    & $-375$    \\
     $8$ & $1020$    & $460$    & $560$
    \end{tabular}
   \end{center}
   Para obtener la media y desviaci\'on est\'andar de las diferencias muestrales,
   se calcula lo siguiente:
   \begin{eqnarray*}
    \sum_{i=1}^{8} d_i & = &
    -3\,080 + 1\,400 + 800 + 3\,800 + 0 + 2\,700 - 375 + 560 = 5\,805 \\
    \sum_{i=1}^{8} d_i^2 & = &
    (-3\,080)^2 + 1\,400^2 + 800^2 + 3\,800^2 + 0^2 + 2\,700^2 + (-375)^2 + 560^2
    = 34\,270\,625
   \end{eqnarray*}
   por lo que la media de las diferencias muestrales es:
   \begin{equation*}
    \overline{d} = \frac{1}{8}\sum_{i=1}^{8} d_i = \frac{5\,805}{8} = 725.625
   \end{equation*}
   y la varianza de las diferencias muestrales se calcula, usando el teorema 8.1,
   como sigue:
   \begin{eqnarray*}
    s_D^2 & = &
    \frac{1}{8(7)}
    \left[ 8\sum_{i=1}^{8} d_i^2 - \left( \sum_{i=1}^{8} d_i \right)^2 \right]
    = \frac{8(34\,270\,625) - (5\,805)^2}{56}
    = \frac{274\,165\,000 - 33\,698\,025}{56} \\
    & = & \frac{240\,466\,975}{56} = 4\,284\,053.125
   \end{eqnarray*}
   por lo que la desviaci\'on est\'andar de las diferencias muestrales es:
   \begin{eqnarray*}
    s_D & = & \sqrt{s_D^2} = \sqrt{\frac{240\,466\,975}{56}}
    = \frac{5\sqrt{9\,618\,679}\sqrt{14}}{2(14)}
    = \frac{35\sqrt{2\,748\,194}}{28} \\
    & = & \frac{5\sqrt{2\,748\,194}}{4}
    \approx 2\,072.20972
   \end{eqnarray*}
   Por lo tanto, se resume el resto de los datos como sigue:
   \begin{itemize}
    \item $\overline{d} = \frac{5\,805}{8} = 725.625$.
    \item $s_D = \frac{5\sqrt{2\,748\,194}}{4} \approx 2\,072.20972$
   \end{itemize}
   Adem\'as, por la suposici\'on a\~nadida y necesaria de normalidad
   en las distribuciones poblacionles, se tiene que la distribuci\'on
   de las diferencias pareadas es normal, entonces el siguientes estad\'{\i}stico,
   que se va a requerir, se aproxima a la distribuci\'on mostrada
   con el respectivo par\'ametro que se indica:
   \begin{itemize}
    \item $T = \frac{\overline{D} - d_0}{S_d/\sqrt{n}} \sim t(v)$.
    \item $v = n_1 - 1 = n_2 - 1 = 7$
   \end{itemize}
  \end{datos}

  \begin{hipotesis}
   \begin{eqnarray*}
    H_0: \mu{}_D = \mu{}_1 - \mu{}_2 &   =  & 0 \\
    H_1: \mu{}_D = \mu{}_1 - \mu{}_2 & \neq & 0
   \end{eqnarray*}
  \end{hipotesis}

  \begin{estadistico}
   \begin{eqnarray*}
    t & = & \frac{\bar{d} - d_0}{s_D/\sqrt{n}}
    = \frac{\frac{5\,805}{8}}{\frac{5\sqrt{2\,748\,194}}{4}/\sqrt{8}}
    = \frac{5\,805\left( 2\sqrt{2}\right)}{10\sqrt{2\,748\,194}}
    = \frac{1\,161}{\sqrt{1\,374\,097}} \\
    & = & \frac{1\,161\sqrt{1\,374\,097}}{1\,374\,097}
    \approx 0.9904294
   \end{eqnarray*}
  \end{estadistico}

  \begin{valorp}
   Dado que:
   \begin{equation*}
    P(|T| < |t|) = 2P(T > t) \approx 2P(T > 0.9904294)
   \end{equation*}
   y ya que $0.896 < 0.9904294 < 1.119$, en donde, de la tabla A.4, se tiene que
   $P(0.896) = 0.2$ y $P(1.119) = 0.15$, entonces, interpolando,
   se considerar\'a la aproximaci\'on $P(T > 0.9904294) \approx 0.18$,
   luego entonces:
   \begin{equation*}
    P(|T| < |t|) \approx 2P(T > 0.9904294) \approx 2(0.18) = 0.36
   \end{equation*}
  \end{valorp}

  \begin{conclusion}
   Por lo tanto, como el valor $P$ es alto, se concluye
   que no hay pruebas suficientes para rechazar la hip\'otesis nula;
   es decir, no hay evidencia suficiente para creer que exista una diferencia
   significativa en la resistencia entre las incisiones caliente y fr\'{\i}a.
  \end{conclusion}
  Finalmente, usando el archivo anexo \texttt{P06\_Prueba\_de\_dos\_medias\_02.r},
  que a suvez requiere del archivo \texttt{DB09\_Problema\_053.csv},
  con los siguientes cambios:
  \begin{verbatim}
> datos<-read.csv("DB10_Problema_053.csv",sep=";",encoding="UTF-8")
> varInteres<-c("Resistencia.x")
> varSel<-c("Bisturí")
> mu<-0
> desv.iguales<-NULL
> alfa<-NULL
> cola<-'D'
> par<-TRUE
  \end{verbatim}
  \vspace{-0.5cm}
  el programa de R lanza el siguiente resultado:
  \begin{verbatim}
           Var1 Freq Poblaciones H0 n diferencia desv.par error.est
1 Resistencia.x    8    Pareadas  0 8    725.625  2072.21  732.6368
  grados alpha   PValor Estadistico RegionRechazoInfT RegionRechazoSupT
1      7  0.05 0.354957   0.9904294         -2.364624          2.364624
  RegionRechazoInfX RegionRechazoSupX
1         -1732.411          1732.411
  \end{verbatim}
  El cual coincide con los datos obtenidos,
  que es a lo que se quer\'{\i}a llegar.${}_{\blacksquare}$
 \end{enumerate}
\end{solucion}
