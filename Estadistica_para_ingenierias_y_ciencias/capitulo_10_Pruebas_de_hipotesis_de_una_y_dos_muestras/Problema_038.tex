\begin{enunciado}
 Un investigador de la \texttt{UCLA} afirma que la vida promedio de un rat\'on se puede prolongar hasta $8$ meses m\'as cuando las calor\'{\i}as en su dieta se reducen en aproximadamente $40\%$ desde el momento en que se destetan. Las dietas restringidas se enriquecen a niveles normales con vitaminas y prote\'{\i}na. Suponga que se alimenta a una muestra aleatoria de $10$ ratones con una dieta normal y tiene una vida promedio de $32.1$ meses con una desviaci\'on est\'andar de $3.2$ meses; mientras que una muestra aleatoria de $15$ ratones se alimenta con la dieta restringida y viven un promedio de $37.6$ meses con una desviaci\'on est\'andar de $2.8$ meses. Con un nivel de significancia de $0.05$ pruebe la hip\'otesis de que la vida promedio de los ratones con esta dieta restringida aumenta $8$ meses contra la alternativa de que el aumento es menor que $8$ meses. Suponga que las distribuciones de las vidas con las dietas regular y restringida son aproximadamente normales con varianzas iguales.
\end{enunciado}

\begin{solucion}
 $\phantom{0}$
 \begin{datos}
  $\phantom{0}$
  \begin{itemize}
   \item $X_i \sim n\left( \mu_i, \sigma_i \right)$,
   para cada $i \in \{ 1,2 \}$.
   \item $n_1 = 10$ y $n_2 = 15$.
   \item $\bar{x}_1 = 32.1$ y $\bar{x}_2 = 37.6$.
   \item $s_1 = 3.2$ y $s_2 = 2.8$.
   \item $\sigma_1^2 = \sigma_2^2$.
  \end{itemize}
 \end{datos}

 \begin{hipotesis}
  \begin{eqnarray*}
   H_0: \mu_1 - \mu_2 & \leq & -8 \\
   H_1: \mu_1 - \mu_2 &   >  & -8
  \end{eqnarray*}
 \end{hipotesis}

 \begin{significancia}
  $\alpha = 0.05$.
 \end{significancia}

 \begin{region}
  De la tabla A.4, se tiene el valor cr\'{\i}tico
  $t_{\alpha,n_1+n_2-2} = t_{0.05,23} \approx 1.714$,
  por lo que la regi\'on de rechazo est\'a dado para $t > 1.714$,
  donde $t = \frac{
  \left( \bar{x}_1 - \bar{x}_2 \right) - d_0
  }{
  s_p\sqrt{1/n_1 + 1/n_2}
  }$.
 \end{region}

 \begin{estadistico}
  Dado que
  \begin{eqnarray*}
   s_p^2
   & = &
   \frac{s_1^2\left( n_1-1\right) +s_2^2\left( n_2-1\right)}{n_1+n_2-2}
   = \frac{3.2^2(10-1) + 2.8^2(15-1)}{23}
   = \frac{10.24(9)+7.84(14)}{23} \\
   & = & \frac{201.92}{23}
   = \frac{5\,048}{575}
   \approx 8.77913043
  \end{eqnarray*}
  entonces
  \begin{equation*}
   s_p = \sqrt{s_p^2} = \sqrt{\frac{5\,048}{575}}
   = \frac{2\sqrt{1\,262}\sqrt{23}}{5(23)}
   = \frac{2\sqrt{29\,026}}{115}
   \approx 2.962959742
  \end{equation*}
  y
  \begin{eqnarray*}
   t & = & \frac{
  \left( \bar{x}_1 - \bar{x}_2 \right) - d_0
  }{
  s_p\sqrt{\frac{1}{n_1} + \frac{1}{n_2}}
  }
  =
  \frac{
  (32.1 - 37.6) - (-8)
  }{
  \frac{2\sqrt{29\,026}}{115}\sqrt{ \frac{1}{10} + \frac{1}{15} }
  }
  =
  \frac{
  (-5.5+8)\left(115\sqrt{29\,026}\right)
  }{
  2(29\,026)\sqrt{\frac{2+3}{30}}
  }
  =
  \frac{
  2.5\left( 115\sqrt{29\,026} \right)
  }{
  58\,052 \sqrt{\frac{1}{6}}
  } \\
  & = &
  \frac{
  5 \left( 115\sqrt{29\,026} \right) \left( \sqrt{6} \right)
  }{
  2(58\,052)
  }
  = \frac{
  \left( 575\sqrt{43\,539} \right)(\cancel{2})
  }{
  \cancel{2}(58\,052)
  }
  = \frac{25\sqrt{43\,539}}{2\,524}
  \approx 2.0667592
  \end{eqnarray*}
 \end{estadistico}

 \begin{decision}
  Se rechaza $H_0$ a favor de $H_1$.
 \end{decision}

 \begin{conclusion}
  Se concluye que la vida promedio de los ratones que siguen una dieta
  en la que se reducen las calor\'{\i}as en aproximadamente 40\%
  desde el momento en que se destetan,
  enriquecidas a niveles normales con vitaminas y prote\'{\i}na,
  es menor en $8$ meses que la vida promedio de los ratones
  que siguen una dieta normal
  y, por lo tanto, el investigador est\'a equivocado.
 \end{conclusion}

 Finalmente, usando el archivo anexo
 \texttt{P05\_Prueba\_de\_dos\_medias\_01.r},
 con los siguientes cambios:
 \begin{verbatim}
> n1<-10
> n2<-15
> mu<--8
> m1<-32.1
> m2<-37.6
> m<-NULL
> sigma1<-NULL
> sigma2<-NULL
> s1<-3.2
> s2<-2.8
> sD<-NULL
> desv.iguales<-TRUE
> alfa<-0.05
> cola<-'S'
> par<-FALSE
 \end{verbatim}
 \vspace{-0.5cm}
 el programa de R lanza el siguiente resultado:
 \begin{verbatim}
  Prueba var.pobl H0 n1 n2 DifMedias desv.est1 desv.est2  est.sp error.est
1      t  Iguales -8 10 15      -5.5       3.2       2.8 2.96296  1.209623
  grados.libertad alpha    PValor Estadistico RegionRechazoT RegionRechazoX
1              23  0.05 0.0250968    2.066759    > 1.7138715   > -5.9268612
      Resultado
1 Se rechaza H0
 \end{verbatim}
 \vspace{-0.5cm}
 El cual coincide con los datos obtenidos,
 que es a lo que se quer\'{\i}a llegar.${}_{\blacksquare}$
\end{solucion}
