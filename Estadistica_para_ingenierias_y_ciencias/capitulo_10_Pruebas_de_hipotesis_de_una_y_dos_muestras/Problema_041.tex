\begin{enunciado}
 El Departamento de Zoolog\'{\i}a del Instituto Polit\'ecnico y Universidad Estatal de Virginia llev\'o a cabo un estudio, para determinar si hay una diferencia significativa en la densidad de organismos en dos estaciones diferentes ubicadas en Cedar Run, un r\'{\i}o secundario que se localiza en la cuenca del r\'{\i}o Roanoke. El drenaje de una planta de tratamiento de aguas negras y el sobreflujo del estanque de sedimentaci\'on de la Federal Mogul Corporation entran al flujo cerca del nacimiento del r\'{\i}o. Los siguientes datos dan las medidas de densidad, en n\'umero de organismos por metro cuadrado, en las dos diferentes estaciones colectoras:
 \begin{center}
  \textbf{N\'umero de organismos por metro cuadrado} \\
  \begin{tabular}{rr|rr}
   \hline
   \multicolumn{2}{c}{$\phantom{00000}$\textbf{Estaci\'on 1}$\phantom{00000}$} & \multicolumn{2}{c}{$\phantom{00000}$\textbf{Estaci\'on 2}$\phantom{00000}$} \\
   \hline 
   $5030$ & $4980$ & $2800$ & $2810$ \\
   $13,700$ & $11,910$ & $4670$ & $1330$ \\
   $10,730$ & $8130$ & $6890$ & $3320$ \\
   $11,400$ & $26,850$ & $7720$ & $1230$ \\
   $860$ & $17,660$ & $7030$ & $2130$ \\
   $2200$ & $22,800$ & $7330$ & $2190$ \\
   $4250$ & $1130$ & & \\
   $15,040$ & $1690$ & &
  \end{tabular}
 \end{center}
 ¿Con un nivel de significancia de $0.05$ podemos concluir que son iguales las densidades promedio en las dos estaciones? Suponga que las observaciones provienen de poblaciones normales con varianzas diferentes.
\end{enunciado}

\begin{solucion}
 \begin{datos}
  Resumido, se tiene que
  \begin{itemize}
   \item $X_i \sim n\left( \mu_i, \sigma_i \right)$,
   para cada $i \in \{ 1, 2 \}$.
   \item $n_1 = 16$ y $n_2 = 12$.
   \item $\sigma_1^2 \neq \sigma_2^2$.
  \end{itemize}
  Para obtener las medias y desviaciones est\'andar muestrales,
  se calcula lo siguiente:
  \begin{eqnarray*}
   \sum_{i=1}^{16} x_{1,i} & = &
   5\,030 + 4\,980 + 13\,700 + 11\,910 + 10\,730 + 8\,130 +
   11\,400 + 26\,850 + 860 + \\
   & & + 17\,660 + 2\,200 + 22\,800 + 4\,250 + 1\,130 +
   15\,040 + 1\,690 = 158\,360 \\
   \sum_{i=1}^{16} x_{1,i}^2 & = & 
   5\,030^2 + 4\,980^2 + 13\,700^2 + 11\,910^2 + 10\,730^2 +
   8\,130^2 + 11\,400^2 + 26\,850^2 + \\
   & & 860^2 + 17\,660^2 + 2\,200^2 + 22\,800^2 + 4\,250^2 +
   1\,130^2 + 15\,040^2 + 1\,690^2 \\
   & = & 2\,497\,444\,000 \\
   \sum_{i=1}^{12} x_{2,i} & = &
   2\,800 + 2\,810 + 4\,670 + 1\,330 + 6\,890 + 3\,320 + 7\,720 +
   1\,230 + 7\,030 + 2\,130 + \\
   & & + 7\,330 + 2\,190
   = 49\,450 \\
   \sum_{i=1}^{12} x_{2,i}^2 & = &
   2\,800^2 + 2\,810^2 + 4\,670^2 + 1\,330^2 + 6\,890^2 + 3\,320^2 +
   7\,720^2 + 1\,230^2 + 7\,030^2 + \\
   & & 2\,130^2 + 7\,330^2 + 2\,190^2
   = 271\,402\,500
  \end{eqnarray*}
  Por lo que el valor de cada media muestral es:
  \begin{eqnarray*}
   \bar{x}_1 & = & \frac{1}{16} \sum_{i=1}^{16} x_{1,i} =
   \frac{158\,360}{16} = \frac{19\,795}{2} = 9\,897.5 \\
   \bar{x}_2 & = & \frac{1}{12} \sum_{i=1}^{12} x_{2,i} =
   \frac{49\,450}{12} = \frac{24\,725}{6} = 4\,120.8\overline{3}
  \end{eqnarray*}
  y las varianzas muestrales se calculan, usando el teorema 8.1,
  como sigue:
  \begin{eqnarray*}
   s_1^2 & = &
   \frac{1}{16(15)}\left[
   16\sum_{i=1}^{16} x_{1,i}^2 -\left(\sum_{i=1}^{16} x_{1,i}\right)^2
   \right]
   = \frac{16(2\,497\,444\,000) - 158\,360^2}{240} \\
   & = & \frac{39\,959\,104\,000 - 25\,077\,889\,600}{240}
   = \frac{14\,881\,214\,400}{240}
   = 62\,005\,060 \\
   s_2^2 & = &
   \frac{1}{12(11)}\left[
   12\sum_{i=1}^{12} x_{2,i}^2 -\left(\sum_{i=2}^{12} x_{2,i}\right)^2
   \right]
   = \frac{12(271\,402\,500) - 49\,450^2}{132} \\
   & = & \frac{3\,256\,830\,000 - 2\,445\,302\,500}{132}
   = \frac{811\,527\,500}{132} = \frac{202\,881\,875}{33}
   = 6\,147\,935.\overline{60}
  \end{eqnarray*}
  por lo que el valor de cada desviaci\'on est\'andar muestral es:
  \begin{eqnarray*}
   s_1 & = & \sqrt{s_1^2} = \sqrt{62\,005\,060}
   = 2\sqrt{15\,501\,265}
   \approx 7\,874.32917777762 \\
   s_2 & = & \sqrt{s_2^2} = \sqrt{\frac{202\,881\,875}{33}}
   = \frac{25\sqrt{324\,611}\sqrt{33}}{33}
   = \frac{25\sqrt{10\,712\,163}}{33}
   \approx 2\,479.5030966023
  \end{eqnarray*}
  Por lo tanto, se resume el resto de los datos como sigue:
  \begin{itemize}
   \item $\bar{x}_1 = \frac{19\,795}{2} = 9\,897.5$
   y $\bar{x}_2 = \frac{24\,725}{6} = 4\,120.8\overline{3}$.
   \item $s_1 = 2\sqrt{15\,501\,265} \approx 7\,874.32917777762$ y
   $s_2 = \frac{25\sqrt{10\,712\,163}}{33} \approx 2\,479.5030966023$.
  \end{itemize}
 \end{datos}

 \begin{hipotesis}
  \begin{eqnarray*}
   H_0: \mu_1 - \mu_2 &   =  & 0 \\
   H_1: \mu_1 - \mu_2 & \neq & 0
  \end{eqnarray*}
 \end{hipotesis}

 \begin{significancia}
  $\alpha = 0.05$.
 \end{significancia}

 \begin{region}
  Se considera una distribuci\'on $t$ con $v$ grados de libertad,
  donde $v$ es el entero m\'as pr\'oximo
  de la siguiente expresi\'on:
  \begin{eqnarray*}
   & & \frac{\displaystyle{
   \left( \frac{s_1^2}{n_1} + \frac{s_2^2}{n_2} \right)^2
   }}{\displaystyle{
   \frac{\left( s_1^2/n_1 \right)^2}{n_1 - 1} +
   \frac{\left( s_2^2/n_2 \right)^2}{n_2 - 1}
   }}
   = \frac{\displaystyle{
   \left( \frac{62\,005\,060}{16} + 
   \frac{\frac{202\,881\,875}{33}}{12} \right)^2
   }}{\displaystyle{
   \frac{\left( 62\,005\,060/16 \right)^2}{16 - 1} +
   \frac{\left( \frac{202\,881\,875}{33}/12 \right)^2}{12 - 1}
   }} \\
   & = & \frac{\displaystyle{
   \left( \frac{15\,501\,265(99) + 202\,881\,875}{12(33)} \right)^2
   }}{
   \frac{15\,501\,265^2}{15(4)^2} +
   \frac{202\,881\,875^2}{11(12)^2(33)^2}
   }
   = \frac{
   \displaystyle{\frac{1\,737\,507\,110^2}{\cancel{(12)^2(33)^2}}}
   }{\displaystyle{
   \frac{
   15\,501\,265^2(3)^2(33)^2(11) + 202\,881\,875^2(15)
   }{
   (15)(11)\cancel{(12)^2(33)^2}
   }
   }} \\
   & = & \frac{
   3\,018\,930\,957\,300\,552\,100(15)(11)
   }{
   26\,523\,236\,558\,939\,591\,850
   }
   = \frac{
   3\,320\,824\,053\,030\,607\,310
   }{
   176\,821\,577\,059\,597\,279
   }
   \approx 18.7806494
  \end{eqnarray*}
  De la tabla A.4, se tiene el valor cr\'{\i}tico
  $t_{\alpha/2,v} = t_{0.025,19} \approx 2.093$,
  por lo que la regi\'on de rechazo est\'a dado para $|t| > 2.093$,
  donde $t = \frac{
  \left( \bar{x}_1 - \bar{x}_2 \right) - d_0
  }{
  \sqrt{s_1^2/n_1 + s_2^2/n_2}
  }$.
 \end{region}

 \begin{estadistico}
  \begin{eqnarray*}
   t & = & \frac{
   \left( \bar{x}_1 - \bar{x}_2 \right) - d_0
   }{
   \sqrt{s_1^2/n_1 + s_2^2/n_2}
   }
   = \frac{
   \left( \frac{19\,795}{2} - \frac{24\,725}{6} \right)^2 - 0
   }{
   \sqrt{\frac{62\,005\,060}{16} + \frac{202\,881\,875/33}{12}}
   }
   = \frac{\frac{34660}{6}}{\sqrt{\frac{1\,737\,507\,110}{396}}} \\
   & = & \frac{17330\sqrt{1\,737\,507\,110}\left( 6\sqrt{11} \right)}{3(1\,737\,507\,110)}
   = \frac{3466\sqrt{19\,112\,578\,210}}{173\,750\,711}
   \approx 2.757792624
  \end{eqnarray*}
 \end{estadistico}

 \begin{decision}
  Se rechaza $H_0$ a favor de $H_1$.
 \end{decision}

 \begin{conclusion}
  Se concluye que s\'{\i} hay una diferencia significativa
  entre las densidades promedio de organismos
  en las dos estaciones en Cedar Run,
  en la cuenca del r\'{\i}o Roanoke.  
 \end{conclusion}

 Finalmente, usando el archivo anexo
 \texttt{P06\_Prueba\_de\_dos\_medias\_02.r},
 que a su vez requiere los datos del archivo
 \texttt{DB05\_Problema\_041.csv},
 con los siguientes cambios:
 \begin{verbatim}
> datos<-read.csv("DB05_Problema_041.csv",sep=";",encoding="UTF-8")
> varInteres<-c("AcidoAscórbico.mgpcml")
> varSel<-c("Estación")
> mu<-0
> desv.iguales<-FALSE
> alfa<-0.05
> cola<-'D'
> par<-FALSE
 \end{verbatim}
 \vspace{-0.5cm}
 el programa de R lanza el siguiente resultado:
 \begin{verbatim}
                   Var1 Freq    Poblaciones H0    valorPVar  suposicionVar n1
1 AcidoAscórbico.mgpcml   28 Independientes  0 0.0004520126 Var diferentes 16
  n2 media1   media2 diferencia desv.est1 desv.est2 error.est grados.libertad
1 12 9897.5 4120.833   5776.667  7874.329  2479.503   2094.67              19
  alpha     PValor Estadistico RegionRechazoInfT RegionRechazoSupT
1  0.05 0.01261219    2.757793          -2.09468           2.09468
  RegionRechazoInfX RegionRechazoSupX     Resultado
1         -4387.664          4387.664 Se rechaza H0
 \end{verbatim}
 \vspace{-0.5cm}
 El cual coincide con los datos obtenidos.
 N\'otese que la regi\'on de rechazo del estad\'{\i}stico $t$
 es m\'as preciso en el programa
 debido a que en los c\'alculos hechos a mano se aproxima
 el par\'ametro $v$.
 Que es a lo que se quer\'{\i}a llegar.${}_{\blacksquare}$
\end{solucion}
