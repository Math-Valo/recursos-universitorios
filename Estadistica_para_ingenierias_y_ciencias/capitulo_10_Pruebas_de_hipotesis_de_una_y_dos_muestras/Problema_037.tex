\begin{enunciado}
 En el ejercicio 9.42 de la p\'agina 298, pruebe la hip\'otesis de que los camiones compactos Volkswagen, en promedio, exceden a los camiones compactos Toyota, equipados de forma similar, en cuatro kil\'ometros por litro. Utilice un nivel de significancia de $0.10$.
\end{enunciado}

\begin{solucion}
 En el ejercicio 9.42, se resolvi\'o
 que $(4.324, 5.676)$ es un intervalo de confianza bilateral
 del $90\%$ para la diferencia de las medias,
 para la misma informaci\'on muestral.
 Por lo tanto, no se rechazar\'{\i}a una prueba de hip\'otesis
 \textit{bilateral} sobre la diferencia de las medias
 que incluya cualquier valor hipot\'etico entre $4.324$ y $5.676$
 (que no es este el caso).
 Sin embargo, el problema sufiere realizar una prueba de hip\'otesis
 unilateral, por lo que se procede seg\'un se sabe,
 y ayudado por los c\'alculos que reaparezcan del ejercicio mencionado.
 \begin{datos}
  Usando la notaci\'on y datos del ejercicio 9.42,
  se tiene lo siguiente:
  \begin{itemize}
   \item $X_i \sim n\left( \mu_i, \sigma_i \right)$,
   para cada $i \in \{ 1, 2 \}$.
   \item $n_1 = 12$ y $n_2 = 10$.
   \item $\bar{x}_1 = 16$ y $\bar{x}_2 = 11$.
   \item $s_1 = 1$ y $s_2 = 0.8$.
   \item $s_p = \frac{\sqrt{2\,095}}{50} \approx 0.9154233993$
   \item $\sigma_1 = \sigma_2$.
  \end{itemize}
 \end{datos}
 
 \begin{hipotesis}
  $\phantom{0}$
  \begin{eqnarray*}
   H_0: \mu_1 - \mu_2 & \leq & 4 \\
   H_1: \mu_1 - \mu_2 & > & 4
  \end{eqnarray*}
 \end{hipotesis}

 \begin{significancia}
  $\alpha = 0.1$.
 \end{significancia}

 \begin{region}
  De la tabla A.4, se tiene el valor cr\'{\i}tico
  $t_{\alpha,n_1+n_2-2} = t_{0.1,20} = 1.325$,
  por lo que la regi\'on de rechazo est\'a dado para $t > 1.325$,
  donde $t = 
  \frac{
  \left( \bar{x}_1 - \bar{x}_2 \right) - d_0
  }{
  s_p\sqrt{1/n_1 + 1/n_2}
  }$.
 \end{region}

 \begin{estadistico}
  \begin{eqnarray*}
   t & = & 
   \frac{
   \left( \bar{x}_1 - \bar{x}_2 \right) - d_0
   }{
   s_p\sqrt{1/n_1 + 1/n_2}
   }
   = \frac{(16-11)-4}{\frac{\sqrt{2\,095}}{50}\sqrt{\frac{1}{12}+\frac{1}{10}}}
   = \frac{(5-4)(50)\sqrt{2\,095}}{2\,095\sqrt{\frac{11}{60}}}
   = \frac{50\sqrt{2\,095}\sqrt{60}}{2\,095\sqrt{11}} \\
   & = & \frac{10\sqrt{125\,700}\sqrt{11}}{419(11)}
   = \frac{100\sqrt{13\,827}}{4\,609}
   \approx 2.55127499936194
  \end{eqnarray*}
 \end{estadistico}

 \begin{decision}
  Se rechaza $H_0$ a favor de $H_1$.
 \end{decision}

 \begin{conclusion}
  Se concluye que los camiones compactos Volkswagen, en promedio, exceden a los camiones compactos Toyota, equipados de forma similar, en cuatro kil\'ometros por litro.
 \end{conclusion}

 Finalmente, usando el archivo anexo
 \texttt{P05\_Prueba\_de\_dos\_medias\_01.r},
 con los siguientes cambios:
 \begin{verbatim}
> n1<-12
> n2<-10
> mu<-4
> m1<-16
> m2<-11
> m<-NULL
> sigma1<-NULL
> sigma2<-NULL
> s1<-1
> s2<-0.8
> sD<-NULL
> desv.iguales<-TRUE
> alfa<-0.1
> cola<-'S'
> par<-FALSE
 \end{verbatim}
 \vspace{-0.5cm}
 el programa de R lanza el siguiente resultado:
 \begin{verbatim}
  Prueba var.pobl H0 n1 n2 DifMedias desv.est1 desv.est2    est.sp error.est
1      t  Iguales  4 12 10         5         1       0.8 0.9154234 0.3919609
  grados.libertad alpha    PValor Estadistico RegionRechazoT RegionRechazoX
1              20   0.1 0.0095122    2.551275    > 1.3253407    > 4.5194817
      Resultado
1 Se rechaza H0
 \end{verbatim}
 \vspace{-0.5cm}
 El cual coincide con los datos obtenidos,
 que es a lo que se quer\'{\i}a llegar.${}_{\blacksquare}$
\end{solucion}
