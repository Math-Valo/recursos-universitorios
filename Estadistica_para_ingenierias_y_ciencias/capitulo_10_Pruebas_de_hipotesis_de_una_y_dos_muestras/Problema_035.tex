\begin{enunciado}
 Para indagar si un nuevo suero frena el desarrollo de la leucemia,
 se seleccionan $9$ ratones,
 todos con una etapa avanzada de la enfermedad.
 Cinco ratones reciben el tratamiento y cuatro no.
 Los tiempos de supervivencia, en a\~nos, a partir del momento
 en que comienza el experimento son los siguientes:
 \begin{center}
  \begin{tabular}{l|ccccc}
   Con tratamiento & $2.1$ & $5.3$ & $1.4$ & $4.6$ & $0.9$ \\
   \hline 
   Sin tratamiento & $1.9$ & $0.5$ & $2.8$ & $3.1$
  \end{tabular}
 \end{center}
 ¿Se puede decir en el nivel de significancia de $0.05$
 que el suero es efectivo?
 Suponga que las dos distribuciones se distribuyen de forma normal
 con varianzas iguales.
\end{enunciado}

\begin{solucion}
 \begin{datos}
  Resumido, se tiene que
  \begin{itemize}
   \item $X_i \sim n\left( \mu_i, \sigma_i \right)$, para cada $i \in \{ 1, 2 \}$.
   \item $n_1 = 5$ y $n_2 = 4$.
   \item $\sigma_1^2 = \sigma_2^2$.
  \end{itemize}
  Para obtener las medias y desviaciones est\'andar muestrales,
  se calcula lo siguiente:
  \begin{eqnarray*}
   \sum_{i=1}^{5} x_{1,i} & = & 2.1 + 5.3 + 1.4 + 4.6 + 0.9 = 14.3 \\
   \sum_{i=1}^{5} x_{1,i}^2 & = & 2.1^2 + 5.3^2 + 1.4^2 + 4.6^2 + 0.9^2 = 56.43 \\
   \sum_{i=1}^{4} x_{2,i} & = & 1.9 + 0.5 + 2.8 + 3.1 = 8.3 \\
   \sum_{i=1}^{4} x_{2,i}^2 & = & 1.9^2 + 0.5^2 + 2.8^2 + 3.1^2 = 21.31
  \end{eqnarray*}
  Por lo que el valor de cada media muestral es:
  \begin{eqnarray*}
   \bar{x}_1 & = & \frac{1}{5} \sum_{i=1}^5 x_{1,i} = \frac{14.3}{5} = 2.86 \\
   \bar{x}_2 & = & \frac{1}{4} \sum_{i=1}^4 x_{2,i} = \frac{8.3}{4} = 2.075
  \end{eqnarray*}
  y las varianzas muestrales se calculan, usando el teorema 8.1, como sigue:
  \begin{eqnarray*}
   s_1^2 & = & \frac{1}{5(4)}\left[ 5\sum_{i=1}^5 x_{1,i}^2 - \left( \sum_{i=1}^5 x_{1,i} \right)^2 \right] = \frac{5(56.43) - 14.3^2}{20} = \frac{282.15 - 204.49}{20} = \frac{77.66}{20} \\
   & = & \frac{3\,883}{1\,000} = 3.883 \\
   s_2^2 & = &
   \frac{1}{4(3)}\left[
   4\sum_{i=1}^4 x_{2,i}^2 - 
   \left( \sum_{i=1}^4 x_{2,i} \right)^2
   \right]
   = \frac{4(21.31) -  8.3^2}{12} = \frac{85.24 - 68.89}{12}
   = \frac{16.35}{12} \\
   & = & \frac{327}{240} = \frac{109}{80} = 1.3625
  \end{eqnarray*}
  por lo que el valor de cada desviaci\'on est\'andar muestral es:
  \begin{eqnarray*}
   s_1 & = & \sqrt{s_1^2} = \sqrt{\frac{3\,883}{1\,000}} = \frac{\sqrt{3\,883}}{10\sqrt{10}} = \frac{\sqrt{38\,830}}{100} \approx 1.97053292 \\
   s_2 & = & \sqrt{s_2^2} = \sqrt{\frac{109}{80}} = \frac{\sqrt{109}}{4\sqrt{5}} = \frac{\sqrt{545}}{20} \approx 1.1672617529928752
  \end{eqnarray*}
  Por lo tanto, se resume el resto de los datos como sigue:
  \begin{itemize}
   \item $\bar{x}_1 = \frac{14.3}{5} = 2.86$ y $\bar{x}_2 = \frac{8.3}{4} = 2.075$.
   \item $s_1 = \frac{\sqrt{38\,830}}{100} \approx 1.97053292$ y $s_2 = \frac{\sqrt{1090}}{20} \approx 1.1672617529928752$
  \end{itemize}
 \end{datos}

 \begin{hipotesis}
  \begin{eqnarray*}
   H_0: \mu_1 - \mu_2 & \leq & 0 \\
   H_1: \mu_1 - \mu_2 &   >  & 0
  \end{eqnarray*}
 \end{hipotesis}

 \begin{significancia}
  $\alpha = 0.05$.
 \end{significancia}

 \begin{region}
  De la tabla A.4, se tiene el valor cr\'{\i}tico
  $t_{\alpha,n_1+n_2-2} = t_{0.05,7} \approx 1.895$,
  por lo que la regi\'on de rechazo est\'a dado para $t > 1.895$,
  donde
  $t=\frac{\left(\bar{x}_1-\bar{x}_2\right)-d_0}{s_p\sqrt{1/n_1+1/n_2}}$.
 \end{region}

 \begin{estadistico}
  Dado que
  \begin{eqnarray*}
   s_p^2 & = &
   \frac{s_1^2\left(n_1-1\right) +s_2^2\left(n_2-1\right)}{n_1+n_2-2}
   = \frac{\frac{3\,883}{1\,000}(5-1)+\frac{109}{80}(4-1)}{7}
   = \frac{\frac{15\,532(2) + 327(25)}{2\,000}}{7} \\
   & = & \frac{31\,064 + 8\,175}{14\,000}
   = \frac{39\,239}{14\,000}
   = 2.8027\overline{857142}
  \end{eqnarray*}
  entonces
  \begin{equation*}
   s_p = \sqrt{s_p^2} = \sqrt{\frac{39\,239}{14\,000}}
   = \frac{\sqrt{39\,239}\sqrt{35}}{700}
   = \frac{\sqrt{1\,373\,365}}{700}
   \approx 1.674152237487891746154                                                                                                                                                                                                                                                                                                                                                                                                                                                                                                                                  
  \end{equation*}
  y
  \begin{eqnarray*}
   t & = &
   \frac{ \left( \bar{x}_1 - \bar{x}_2 \right) - d_0
   }{
   s_p \sqrt{ \frac{1}{n_1} + \frac{1}{n_2} }
   }
   =
   \frac{
   \left( 2.86 - 2.075 \right) - 0
   }{
   \frac{\sqrt{1\,373\,365}}{700}\sqrt{\frac{1}{5}+\frac{1}{4}}
   }
   = \frac{700(0.785)}{\sqrt{1\,373\,365}\sqrt{\frac{4+5}{20}}}
   = \frac{549.5}{\sqrt{1\,373\,365}\left( \frac{3}{2\sqrt{5}} \right)}
   \\
   & = & \frac{2(549.5)}{3\sqrt{274\,673}}
   = \frac{1\,099\sqrt{274\,673}}{824\,019}
   = \frac{157\sqrt{274\,673}}{117\,717}
   \approx 0.6989859595927896853
  \end{eqnarray*}
 \end{estadistico}

 \begin{decision}
  No se rechaza $H_0$.
 \end{decision}

 \begin{conclusion}
  No hay evidencia suficiente para creer
  que el nuevo suero aumenta el tiempo de vida de ratones con leucemia,
  lo que quiere decir que no se puede concluir
  que el nuevo suero frena el desarrollo de la leucemia,
  por lo tanto, no se puede decir que el suero sea efectivo.
 \end{conclusion}

 En el c\'odigo registrado en el archivo anexo
 \texttt{P06\_Prueba\_de\_dos\_medias\_02.r}, en R,
 se realiza este procedimiento.
 El c\'odigo permite modificar los valores iniciales
 que corresponden a:
 \texttt{datos}, que guarda los datos de la lectura
 de un archivo,
 en este caso se lee el arhivo \texttt{DB02\_Problema\_035.csv},
 y \'este \'ultimo nombre es el que se modifica 
 para leer otros archivos;
 \texttt{varInteres} para indicar el nombre de la columna
 que corresponden a los datos en la base anterior;
 \texttt{varSel} para indicar el nombre de la columna
 en donde se indica a cu\'al muestra corresponde cada dato;
 \texttt{mu} para el valor de la diferencia de las medias
 poblacionales supuestas en la hip\'otesis nula;
 \texttt{desv.iguales} indica si se est\'a suponiendo
 que las varianzas poblacionales son iguales,
 con el valor \texttt{TRUE}, o diferentes, con \texttt{FALSE};
 \texttt{alfa} para el nivel de significancia;
 \texttt{cola} para indicar si la prueba es de dos colas,
 con \texttt{'D'}, de cola inferior, con \texttt{'I'},
 o de cola superior, con \texttt{'S'};
 adicionalmente, la rutina puede realizar pruebas pareadas
 con \texttt{par} al darle el valor \texttt{TRUE},
 que por defecto aparece como \texttt{FALSE}.
 \par 
 El programa espera al menos los datos correspondientes
 a la base de datos, escrito en un archivo \texttt{.csv},
 con una columna con todos los datos y otra columna
 para distinguir a cu\'al de los dos tipos de muestra corresponde cada dato; ambos nombres de columnas deben ser indicados;
 se indica tambi\'en el tipo de prueba (cola izquierda o derecha,
 o dos colas);
 adem\'as, se indica si las muestras son o no pareadas;
 y, finalmente, la diferencia de las medias poblacionales,
 que corresponde a la hip\'otesis nula $H_0$.
 No es necesario indicar una suposici\'on acerca de la relaci\'on
 entre las varianzas poblacionales,
 pues, de no tener una suposici\'on previa, se calcula
 una prueba de hip\'otesis para la igualdad de las varianzas
 poblacionales (que se ve en uno de los apartados que siguen
 sobre este mismo cap\'{\i}tulo
 y cuya explicaci\'on, por lo tanto, no se dar\'a
 sino hasta que se llegue a dicho apartado).
 Si no hay una instrucci\'on para asignar un valor a $\alpha$
 o suponer algo acerca de las variancias iguales,
 entonces sus respectivas variables se les debe asignar el valor
 \texttt{NULL}.
 \par 
 La prueba de hip\'otesis siempre usar\'a un estad\'{\i}stico
 con distribuci\'on $t$,
 independientemente de los tama\~nos muestrales.
 Para ello, se est\'a suponiendo de antemano la normalidad
 de las poblaciones de las que viene la muestra.
 \par 
 Independientemente del tipo de prueba, el resultado muestra lo siguiente:
 \texttt{Var1} para conocer lo que est\'a midiendo los datos,
 incluyendo el tipo de unidad que se usa;
 \texttt{Freq} indica el tama\~no de las unidades experimentales,
 es decir, la suma de los tama\~nos muestrales en caso de no ser
 muestras pareadas o el tama\~no de una muestra en el otro caso;
 \texttt{Poblaciones} que indica, si es \texttt{Pareadas}, 
 que las poblaciones son dependientes
 y se est\'a realizando una prueba pareada,
 o \texttt{Independientes}, en otro caso;
 \texttt{H0} para indicar el valor propuesto en la hip\'otesis nula
 de la diferencia de las medias poblacionales;
 \texttt{n1} y \texttt{n2}, para el tama\~no de cada muestra
 en caso de ser poblaciones independientes, o \texttt{n},
 en el caso de que se haga una prueba pareada
 (que corresponde igual al valor de la frecuencia);
 \texttt{media1} y \texttt{media2} para indicar las medias
 muestrales; \texttt{diferencia}, para la diferencia
 de las medias muestrales;
 \texttt{desv.est1} y \texttt{desv.est2}, 
 para las desviaciones est\'andar en pruebas no pareadas,
 o, en otro caso, \texttt{desv.par}
 para indicar la desviaci\'on est\'andar muestral de los datos
 que resultan de restar por parejas los datos de cada muestra
 seg\'un correspondan a una misma unidad experimental;
 \texttt{error.est}, para indicar el error est\'andar,
 que corresponde a la estimaci\'on de la desviaci\'on est\'andar
 de $\overline{X}_1 - \overline{X}_2$;
 \texttt{grados} indica los grados de libertad
 de la distribuci\'on $t$ correspondiente al estad\'{\i}stico
 de la prueba;
 \texttt{alfa}, para el nivel de significancia dado,
 el cual muestra por defecto $0.05$
 en caso de asignarle \texttt{NULL} a \texttt{alfa};
 \texttt{PValor}, para el valor $P$, la probabilidad
 de haber obtenido la diferencia de medias muestrales
 que se obtuvo suponiendo que la hip\'otesis nula sea cierta;
 y, \texttt{Estadistico}, para el valor resultante
 del estad\'{\i}stico de prueba.
 \par 
 Adem\'as, tambi\'en se da dos regiones de rechazo:
 la regi\'on de rechazo para el estad\'{\i}stico
 y la regi\'on de rechazo para la diferencia de las medias
 muestrales; adem\'as, las regiones se muestran
 con una o dos variables seg\'un si es una prueba
 de una o dos colas, respectivamente.
 Para el estad\'{\i}stico y para la diferencia de medias
 muestrales, respectivamente, se muestran las siguientes variables
 para cada prueba:
 para la de cola inferior, aparecen las salidas
 \texttt{RegionRechazoInfT} y \texttt{RegionRechazoInfX};
 para la de cola superior, aparecen \'unicamente las salidas
 \texttt{RegionRechazoSupT} y \texttt{RegionRechazoSupX};
 y, para la de dos colas, aparece \texttt{RegionRechazoInfT}
 y \texttt{RegionRechazoSupT}, para el estad\'{\i}stico de prueba,
 y \texttt{RegionRechazoInfX} y \texttt{RegionRechazoSupX},
 para la diferencia de las medias muestrales.
 \par 
 Adem\'as, seg\'un el tipo de prueba, pueden darse
 los siguientes valores:
 \texttt{valorPVar} que indica la probabilidad
 de que las varianzas sean iguales en caso de una prueba
 no pareada;
 \texttt{suposicionVar} aparece siempre que aparezca la salida
 anterior e indica si se est\'a suponiendo
 que las varianzas son iguales, con \texttt{Var no diferentes},
 o diferentes, con \texttt{Var diferentes},
 el cual corresponde a la hip\'otesis dada en los datos de entrada
 o, en su defecto, se da el veredicto de que son iguales
 si el valor $P$ de las varianzas es mayor a $\alpha$
 (que vale $0.05$ si se omite en la entrada),
 o bien dar el veredicto de que las varianzas son distintas
 si el valor $P$ de las varianzas es menor a $\alpha$;
 adem\'as, en el caso de suponer varianzas poblacionales iguales,
 se da el valor de $s_p$, con la variable \texttt{est.sp},
 que es el estimador de la desviaci\'on est\'andar poblacional
 com\'un, $\sigma$, de las distribuciones poblacionales
 de las que se est\'an obteniendo las muestras, 
 \'este valor aparece despu\'es de las desviaciones est\'andar
 muestrales;
 y, finalmente, \texttt{Resultado} para indicar si se rechaza o no
 la hip\'otesis nula, que aparece al final de los resultados
 cuando se asigna un valor a \texttt{alfa}.
 \par 
 El c\'odigo junto con el resultado se muestra a continuaci\'on:
 \begin{verbatim}
> datos<-read.csv("DB02_Problema_035.csv",sep=";",encoding="UTF-8")
> varInteres<-c("supervivencia.años")
> varSel<-c("Registro")
> mu<-0
> desv.iguales<-TRUE
> alfa<-0.05
> cola<-'S'
> par<-FALSE
> w<-data.frame(factor(datos[,varSel]))
> names(w)<-varSel
> varBin<-c(varSel)
> datos<-data.frame(datos,w)
> if (length(varBin)<1){
+  stop("Debe al menos indicar una variable binaria")
+ }else{
+  sonbinarios<-ifelse(length(table(datos[,varBin]))!=2,1,0)
+ }
> if (sonbinarios!=0)  stop("La variable no es binaria")
> valores<-unlist(datos[,c(varInteres)])
> variables<-factor(rep(varInteres,each=dim(datos)[1]))
> 
> agrupaciones<-data.frame(datos[1:dim(datos)[1],varBin])
> names(agrupaciones)<-varBin
> datos2<-data.frame(agrupaciones,variable=variables,valor=valores)
> difMedias<-function(l,mu,cola,alfa=0.05,iguales=NULL,pareadas=FALSE){
+   x<-l[[1]][!is.na(l[[1]])]
+   y<-l[[2]][!is.na(l[[2]])]
+   n1<-length(x)
+   n2<-length(y)
+   m1<-mean(x)
+   m2<-mean(y)
+   sd1<-sd(x)
+   sd2<-sd(y)
+   if ( n1< 2 | n2 < 2){
+      r<-data.frame(H0=mu,Poblaciones=NA,
+                    n1=n1,n2=n2,
+                    media1=m1,media2=m2,
+                    diferencia=m1-m2,
+                    alpha=alfa,
+                    PValor=NA,
+                    valorPMedia=NA,
+                    valorPVar=NA,
+                    varIgual=NA,
+                    Estadistico=NA,
+                    RegionRechazoZ=NA,
+                    RegionRechazoX=NA)
+   }else{
+     if(!pareadas){
+       r1<-var.test(x,y,conf.level=1-alfa)
+       if(is.null(iguales)){
+         varIgual<-(r1$p.value>=alfa)
+       }else{
+         varIgual<-iguales
+       }
+       r2<-t.test(x,y,mu=mu,alternative=cola,conf.level=1-alfa,
+                  var.equal=varIgual)
+       grados<-as.numeric(r2$parameter)
+       r<-data.frame(Poblaciones=TRUE,
+                     H0=mu,
+                     valorPVar=r1$p.value,
+                     suposicionVar=varIgual,
+                     n1=n1, n2=n2,
+                     media1=m1, media2=m2,
+                     diferencia=m1-m2,
+                     desv.est1=sd1,
+                     desv.est2=sd2)
+       if(varIgual){
+         r$est.sp=round(r2$stderr/sqrt(1/n1+1/n2),7)
+         r$error.est=r2$stderr
+         r$grados=grados
+         r$alpha=alfa
+         r$PValor=r2$p.value
+         r$Estadistico=r2$statistic
+         if(cola=="two.sided"){
+           criticot<-round(qt(1-alfa/2,grados),7)
+           criticox<-round(criticot*r2$stderr,7)
+           r$RegionRechazoInfT<--criticot
+           r$RegionRechazoSupT<-criticot
+           r$RegionRechazoInfX<-mu-criticox
+           r$RegionRechazoSupX<-mu+criticox
+         }else{
+           criticot<-round(qt(1-alfa,grados),7)
+           criticox<-round(criticot*r2$stderr,7)
+           if(cola=="less"){
+             r$RegionRechazoInfT<--criticot
+             r$RegionRechazoInfX<-mu-criticox
+           }else{
+             r$RegionRechazoSupT<-criticot
+             r$RegionRechazoSupX<-mu+criticox
+           }
+         }
+       }else{
+         r$error.est=r2$stderr
+         r$grados.libertad=round(grados,0)
+         r$alpha=alfa
+         r$PValor=r2$p.value
+         r$Estadistico=r2$statistic
+         if(cola=="two.sided"){
+           criticot<-round(qt(1-alfa/2,grados),7)
+           criticox<-round(criticot*r2$stderr,7)
+           r$RegionRechazoInfT<--criticot
+           r$RegionRechazoSupT<-criticot
+           r$RegionRechazoInfX<-mu-criticox
+           r$RegionRechazoSupX<-mu+criticox
+         }else{
+           criticot<-round(qt(1-alfa,grados),7)
+           criticox<-round(criticot*r2$stderr,7)
+           if(cola=="less"){
+             r$RegionRechazoInfT<--criticot
+             r$RegionRechazoInfX<-mu-criticox
+           }else{
+             r$RegionRechazoSupT<-criticot
+             r$RegionRechazoSupX<-mu+criticox
+           }
+         }
+       }
+     }else{
+       r1<-t.test(x,y,mu=mu,alternative=cola,conf.level=1-alfa,paired=TRUE)
+       grados<-as.numeric(r1$parameter)
+       r<-data.frame(Poblaciones=FALSE,
+                     H0=mu,
+                     n=n1,
+                     diferencia=m1-m2,
+                     desv.par=sd(x-y),
+                     error.est=r1$stderr,
+                     grados=grados,
+                     alpha=alfa,
+                     PValor=r1$p.value,
+                     Estadistico=r1$statistic)
+       if(cola=="two.sided"){
+         criticot<-round(qt(1-alfa/2,grados),7)
+         criticox<-round(criticot*r1$stderr,7)
+         r$RegionRechazoInfT<--criticot
+         r$RegionRechazoSupT<-criticot
+         r$RegionRechazoInfX<-mu-criticox
+         r$RegionRechazoSupX<-mu+criticox
+       }else{
+         criticot<-round(qt(1-alfa,grados),7)
+         criticox<-round(criticot*r1$stderr,7)
+         if(cola=="less"){
+           r$RegionRechazoInfT<--criticot
+           r$RegionRechazoInfX<-mu-criticox
+         }else{
+           r$RegionRechazoSupT<-criticot
+           r$RegionRechazoSupX<-mu+criticox
+         }
+       }
+     }
+   }
+   return(r)
+ }
> listaG<-datos2[,c(varBin,"valor")]
> listaG1<-split(listaG$"valor", listaG[,varBin])
> if(as.matrix(listaG)[1]==names(listaG1[2])){
+    listaG1=listaG1[c(2,1)]
+ }
> listaG1<-list(listaG1)
> names(listaG1)<-varInteres
> if(cola=='I'){
+    cola<-"less"
+ }else{
+    if(cola=='S'){
+       cola<-"greater"
+    }
+    else{
+       cola<-"two.sided"
+    }
+ }
> if(is.null(alfa)){
+    rFin<-t(sapply(listaG1,difMedias,mu,cola,iguales=desv.iguales,pareadas=par))
+ }else{
+    rFin<-t(sapply(listaG1,difMedias,mu,cola,alfa=alfa,iguales=desv.iguales,
+                   pareadas=par))
+ }
> t1<-as.data.frame.table(table(datos2[,c("variable")]))
> identif<-t1[t1$Freq>0,]
> d<-dim(rFin)
> n<-colnames(rFin)
> rFin<-data.frame(matrix(unlist(rFin),d))
> names(rFin)<-n
> rFin<-data.frame(identif,rFin)
> if(rFin$Poblaciones){
+    rFin<-rFin[with(rFin,n1!=0 | n2!=0),]
+    rFin$Poblaciones<-"Independientes"
+    if(is.null(alfa)){
+      rFin$suposicionVar<-ifelse(rFin$suposicionVar>=0.05,"Var no diferentes",
+                                 "Var diferentes")
+    }else{
+      rFin$suposicionVar<-ifelse(rFin$suposicionVar>=alfa,"Var no diferentes",
+                                 "Var diferentes")
+    }
+ }else{
+    rFin<-rFin[with(rFin,n!=0),]
+    rFin$Freq=rFin$Freq/2
+    rFin$Poblaciones<-"Pareadas"
+ }
> if(!is.null(alfa)){
+    rFin$Resultado<-ifelse(rFin$PValor>=alfa,
+                           "No se rechaza H0","Se rechaza H0")
+    rFin$Resultado[is.na(rFin$Resultado)]<-"Pocos datos"
+ }
> rFin
                Var1 Freq    Poblaciones H0 valorPVar     suposicionVar n1 n2
1 supervivencia.años    9 Independientes  0 0.4160363 Var no diferentes  5  4
  media1 media2 diferencia desv.est1 desv.est2   est.sp error.est grados alpha
1   2.86  2.075      0.785  1.970533  1.167262 1.674152  1.123055      7  0.05
     PValor Estadistico RegionRechazoSupT RegionRechazoSupX        Resultado
1 0.2535567    0.698986          1.894579          2.127717 No se rechaza H0
 \end{verbatim}
 \vspace{-0.5cm}
 Lo cual coincide con los resultados obtenidos,
 adem\'as de dar m\'as informaci\'on
 como: el valor $P$ de una prueba de igualdad de varianzas,
 que refuerza la suposici\'on de que las varianzas son iguales;
 el error est\'andar;
 el valor $P$ para del estad\'{\i}stico de prueba;
 y regiones de rechazo, una para el estad\'{\i}stico de prueba
 y una para $\bar{x}_1 - \bar{x}_2$,
 que es a lo que se quer\'{\i}a llegar.${}_{\blacksquare}$
\end{solucion}
