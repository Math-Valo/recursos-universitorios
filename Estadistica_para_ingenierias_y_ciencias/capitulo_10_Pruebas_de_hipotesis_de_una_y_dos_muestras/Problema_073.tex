\begin{enunciado}
 Se realiza un estudio para comparar la longitud de tiempo entre hombres
 y mujeres para ensamblar cierto producto.
 La experiencia indica que la distribuci\'on de tiempos tanto para hombres
 como para mujeres es aproximadamente normal,
 pero que la varianza de los tiempos para las mujeres es menor
 que para los hombres.
 Una muestra aleatoria de tiempos para $11$ hombres y $14$ mujeres da los siguientes datos:
 \begin{center}
  \begin{tabular}{cc}
   \textbf{Hombres} & \textbf{Mujeres} \\
   \hline 
   $n_1 = 11$  & $n_2 = 14$ \\
   $s_1 = 6.1$ & $s_2 = 5.3$
  \end{tabular}
 \end{center}
 Pruebe la hip\'otesis de que $\sigma_1^2 = \sigma_2^2$ contra la alternativa
 $\sigma_1^2 > \sigma_2^2$.
 Utilice un valor $P$ en su conclusi\'on.
\end{enunciado}

\begin{solucion}
 \begin{datos}
  $\phantom{0}$
  \begin{itemize}
   \item $X_i \sim n\left( \mu_i, \sigma_i \right)$,
   para cada $i \in \{1,2\}$.
   \item $n_1 = 11$ y $n_2 = 14$.
   \item $s_1 = 6.1$ y $s_2 = 5.3$.
  \end{itemize}
 \end{datos}

 \begin{hipotesis}
  \begin{eqnarray*}
   H_0: \sigma_1^2 & = & \sigma_2^2 \\
   H_1: \sigma_1^2 & > & \sigma_2^2
  \end{eqnarray*}
 \end{hipotesis}

 \begin{estadistico}
  \begin{equation*}
   f = \frac{s_1^2}{s_2^2}
   = \frac{6.1^2}{5.3^2} = \frac{37.21}{28.09}
   = \frac{3\,721}{2\,809} \approx 1.32467071
  \end{equation*}
 \end{estadistico}

 \begin{valorp}
  De la tabla A.6 se interpola que:
  \begin{equation*}
   P(F_{v_1,v_2} > f) = P(F_{n_1-1,\,n_2-1} > 1.32467071)
   = P(F_{10,13} > 1.325)) > P(F_{10,13} > 2.67) = 0.05
  \end{equation*}
 \end{valorp}

 \begin{conclusion}
  El valor $P$ sobrepasa el l\'{\i}mite de probabilidad de la tabla A.6, $0.05$,
  por lo que la probabilidad de cometer un error tipo I,
  es decir de errar al rechazar la hip\'otesis nula cuando es cierta,
  es muy alta, por lo que no hay evidencia suficiente para rechazar
  la hip\'otesis nula y, por lo tanto, se afirma que la varianza en los tiempos
  para ensamblar el producto no son significativamente diferentes
  y se pueden considerar como iguales.
 \end{conclusion}
 
 En el c\'odigo registrado en el archivo anexo
 \texttt{P13\_Prueba\_de\_dos\_varianzas\_01.r}, en R,
 se realiza este procedimiento.
 El c\'odigo permite modificar los valores iniciales correspondientes a:
 \texttt{n1} y \texttt{n2} para los tama\~nos de las muestras;
 \texttt{s1} y \texttt{s2} para las varianzas muestrales;
 \texttt{alfa} para el nivel de significancia;
 \texttt{cola} para indicar si la prueba es de dos colas,
 con \texttt{'D'}, de cola inferior, con \texttt{'I'},
 o de cola superior, con \texttt{'S'}.
 \par 
 El programa espera todos los datos, excepto \texttt{alfa} que es opcional.
 En caso de no fijar una probabilidad de cometer un error de tipo I
 a esta variable, se le asigna el valor \texttt{NULL}.
 \par 
 La prueba de hip\'otesis siempre usar\'a un estad\'{\i}stico
 con distribuci\'on $f$ de Fisher.
 Para ello, se est\'a suponiendo de antemano la normalidad
 de las poblaciones de donde provienen las muestras.
 \par
 Independientemente del tipo de prueba, el resultado muestra lo siguiente:
 \texttt{n1} y \texttt{n2} que indican los tama\~nos de las muestras;
 \texttt{s1} y \texttt{s2} para indicar las varianzas muestrales de los datos;
 \texttt{v1} y \texttt{v2} que indican los grados de libertad
 para la distribuci\'on $f$ correspondiente al estadístico de la prueba;
 \texttt{alfa} que indica el nivel de significancia dado,
 el cual muestra por defecto $0.05$
 en caso de que se le asigne \texttt{NULL};
 \texttt{PValor} que indica el valor $P$, la probabilidad
 de haber obtenido la proporci\'on de varianzas muestrales que se obtuvo
 suponiendo que la hip\'otesis nula sea cierta;
 \texttt{Estadistico}, para el valor resultante
 del estad\'{\i}stico de prueba; y,
 \texttt{RegionRechazo} para la regi\'on de rechazo del estad\'{\i}stico
 de prueba, el cual es precisamente la regi\'on de rechazo de la proporci\'on
 de varianzas muestrales.
 \par 
 Adem\'as, seg\'un el tipo de prueba, pueden darse el valor
 de \texttt{Resultado} para indicar si se rechaza o no
 la hip\'otesis nula, que aparece al final de los resultados
 cuando se asigna un valor a \texttt{alfa}.
 \par 
 El c\'odigo junto con el resultado se muestra a continuaci\'on:
 \begin{verbatim}
> n1<-11
> n2<-14
> s1<-6.1^2
> s2<-5.3^2
> alfa<-NULL
> cola<-'S'
> TestVarProp<-function(n1,n2,s1,s2,alfa=0.05,colas='D'){
+   estadistico<-round(s1/s2,7)
+   v1<-n1-1
+   v2<-n2-1
+   if(n1<2 | n2 < 2){
+     r<-data.frame(n1=n1,n2=n2,
+                   s1=s1,s2=s2,
+                   v1=v1, v2=v2,
+                   alpha=alfa,
+                   PValor=NA,
+                   Estadistico=estadistico,
+                   RegionRechazo=NA
+                   )
+   }else{
+     if(colas=='D'){
+       regionL<-round(qf(alfa/2,n1-1,n2-1),7)
+       regionU<-round(qf(alfa/2,n1-1,n2-1,lower.tail=F),7)
+       pvalor<-2*min(pf(estadistico,n1-1,n2-1,lower.tail=F),
+                     pf(estadistico,n1-1,n2-1))
+       r<-data.frame(n1=n1,n2=n2,
+                     s1=s1,s2=s2,
+                     v1=v1,v2=v2,
+                     alpha=alfa,
+                     PValor=round(pvalor,7),
+                     Estadistico=estadistico,
+                     RegionRechazo=paste("<",regionL,"y >",regionU)
+                     )
+     }else{
+       if(colas=='I'){
+         region<-round(qf(alfa,n1-1,n2-1),7)
+         pvalor<-round(pf(estadistico,n1-1,n2-1),7)
+         r<-data.frame(n1=n1,n2=n2,
+                       s1=s1,s2=s2,
+                       v1=v1,v2=v2,
+                       alpha=alfa,
+                       PValor=pvalor,
+                       Estadistico=estadistico,
+                       RegionRechazo=paste("<",region)
+                       )
+       }else{
+         region<-round(qf(alfa,n1-1,n2-1,lower.tail=F),7)
+         pvalor<-round(pf(estadistico,n1-1,n2-1,lower.tail=F),7)
+         r<-data.frame(n1=n1,n2=n2,
+                       s1=s1,s2=s2,
+                       v1=v1,v2=v2,
+                       alpha=alfa,
+                       PValor=pvalor,
+                       Estadistico=estadistico,
+                       RegionRechazo=paste(">",region))
+       }
+     }
+   }
+   return(r)
+ }
> if(is.null(alfa)){
+   Test<-TestVarProp(n1,n2,s1,s2,colas=cola)
+ }else{
+   Test<-TestVarProp(n1,n2,s1,s2,alfa=alfa,colas=cola)
+   resultado<-ifelse(Test[,"PValor"]>=alfa,
+                     "No se rechaza H0","Se rechaza H0")
+   Test$Resultado<-resultado
+   Test$Resultado[is.na(Test$Resultado)]<-"Pocos datos"
+ }
> Test
  n1 n2    s1    s2 v1 v2 alpha    PValor Estadistico RegionRechazo
1 11 14 37.21 28.09 10 13  0.05 0.3118285    1.324671   > 2.6710242
 \end{verbatim}
 \vspace{-0.7cm}
 El cual coincide con los c\'alculos obtenidos,
 adem\'as de ofrecer m\'as informaci\'on, como el valor $P$,
 que en los c\'alculos por computadora es m\'as preciso
 que decir que es mayor a $0.05$, como se hizo en conclusi\'on
 de este ejercicio, que es a lo que se quer\'{\i}a llegar.${}_{\blacksquare}$
\end{solucion}
