\begin{enunciado}
 Se realiza un estudio para determinar si hay una diferencia
 entre las proporciones de padres en los estados
 de Maryland (MD), Virginia (VA), Georgia (GA) y Alabama (AL)
 que est\'an a favor de colocar Biblias en las escuelas primarias.
 En la siguiente tabla se registran las respuestas de $100$ padres
 seleccionados al azar en cada uno de esos estados:
 \begin{center}
  \begin{tabular}{lcccc}
   & \multicolumn{4}{c}{\textbf{Estado}} \\
   \cline{2-5}
   \textbf{Preferencia} & \textbf{MD} & \textbf{VA} & \textbf{GA} &
   \textbf{AL} \\
   \hline 
   S\'{\i} & $65$ & $71$ & $78$ & $82$ \\
   No & $35$ & $29$ & $22$ & $18$
  \end{tabular}
 \end{center}
 ¿Podemos concluir que las proporciones de padres que est\'an a favor
 de colocar Biblias en las escuelas son las mismas
 para estos cuatro estados?
 Utilice un nivel de significancia de $0.01$.
\end{enunciado}

\begin{solucion}
 El problema corresponde a una prueba de varias proporciones,
 el cual se resuelve usando una prueba $\chi^2$ como se hace con la prueba
 de homogeneidad, con $(r-1)\times(c-1) = (2-1)\times (4-1) = 3$ grados
 de libertad. As\'{\i} que, nombrando $p_1$, $p_2$, $p_3$ y $p_4$
 a las proporciones de padres que est\'an a favor de colocar Biblias
 en las escuelas primarias en los estados de Maryland, Virginia, Georgia
 y Alabama, respectivamente, la prueba se escribe como sigue:
 \begin{eqnarray*}
  H_0: & & p_1 = p_2 = p_3 = p_4 \\
  H_1: & & \text{no todos las proporciones } p_1, p_2, p_3 \text{ y } p_4
  \text{ son iguales.}
 \end{eqnarray*}
 con un nivel de significancia $\alpha = 0.01$.
 Esto genera la regi\'on de rechazo $\chi^2 > 11.345$, 
 que corresponde al valor cr\'{\i}tico de la distribuci\'on $\chi^2$
 con $3$ grados de libertad.
 \par 
 El proceso se puede realizar usando el programa de R escrito
 en el archivo adjunto con el nombre
 \texttt{P18\_Prueba\_de\_independencia\_y\_homogeneidad\_01.r},
 que, a su vez, usa la base de datos en el archivo nombrado
 \texttt{DB40\_Problema\_109.csv}, con los siguientes cambios:
 \begin{verbatim}
> datos<-read.csv("DB40_Problema_109.csv",sep=";",encoding="UTF-8")
> varInteres<-c("Preferencia","Estado")
> varFrecuencia<-"Frecuencia"
> pruebas<-c(1)
 \end{verbatim}
 \vspace{-0.5cm}
 el programa de R lanza el siguiente resultado:
 \begin{verbatim}
$tabla
           Estado
Preferencia AL GA MD VA
         No 18 22 35 29
         Sí 82 78 65 71

$listaPruebas
$listaPruebas[[1]]

	Pearson's Chi-squared test

data:  tbl1
X-squared = 8.8358, df = 3, p-value = 0.03156
 \end{verbatim}
 \vspace{-0.5cm}
 Con lo que se obtiene un valor del estad\'{\i}stico $\chi^2 = 8.8358$,
 lo cual lleva a decidir no rechazar $H_0$ y concluir,
 con un nivel de significancia de $0.01$, 
 que la proporci\'on de padres que est\'an a favor de colocar Biblias
 en las escuelas no son significantivamente diferentes.
 Sin embargo, n\'otese que el valor $P$ para el valor del estad\'{\i}stico
 obtenido corresponde a una probabilidad de $0.03156$,
 que es bastante bajo.
 Esto indica que existe un riesgo que hay que asumir
 ante la decisi\'on de no rechazar la hip\'otesis nula,
 que es a lo que se quer\'{\i}a llegar.${}_{\blacksquare}$
\end{solucion}
