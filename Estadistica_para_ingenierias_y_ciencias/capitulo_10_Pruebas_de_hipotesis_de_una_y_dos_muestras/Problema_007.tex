\begin{enunciado}
 Repita el ejercicio 10.6 cuando se seleccionan $50$ pedidos, y se define la regi\'on cr\'{\i}tica como $x \leq 24$, donde $x$ es el n\'umero de pedidos en nuestra muestra que llegan tarde. Utilice la aproximaci\'on normal.
\end{enunciado}

\begin{solucion}
 Usando los t\'erminos del ejercicio 10.6, se tiene ahora que:
 \begin{itemize}
  \item $X \sim b(n,p) \sim n\left( \mu = np, \sigma = \sqrt{npq} \right)$.
  \item $n = 50$.
  \item $x_{\text{inf}} = 25$.
 \end{itemize}
 Para el primer supuesto, se tiene adem\'as que
 \begin{itemize}
  \item $p = 0.6$.
  \item $\mu = np = (50)(0.6) = 30$.
  \item $\sigma^2 = npq = 30(0.4) = 12$.
 \end{itemize}
 as\'{\i}, el error tipo I se calcula como sigue:
 \begin{eqnarray*}
  \alpha & = & P(X \leq 24) = P\left( Z < \frac{24.5 - 30}{\sqrt{12}} \right) \approx P\left( Z < -\frac{5.5\sqrt{3}}{6} \right) = P\left( Z < -\frac{11\sqrt{6}}{12} \right) \\
  & \approx & P\left( Z < - 1.59 \right)
 \end{eqnarray*}
 Esto se puede aproximar usando la tabla A.1, con lo que se obtiene:
 \begin{equation*}
  \alpha \approx P\left( Z < - 1.59 \right) \approx 0.0559
 \end{equation*}
 Para el siguiente supuesto, se tiene que
 \begin{itemize}
  \item $p = 0.3$.
  \item $\mu = np = (50)(0.3) = 15$.
  \item $\sigma^2 = npq = 15(0.7) = 10.5 = 21/2$
 \end{itemize}
 as\'{\i}, el error tipo II se calcula como sigue:
 \begin{eqnarray*}
  \beta & = & P(X > 24) = 1 - P(X \leq 24) \approx 1 - P\left( Z < \frac{24.5 - 15}{\sqrt{21/2}} \right) = 1 - P\left( Z < \frac{9.5\sqrt{42}}{21} \right) \\
  & = & 1 - P\left( Z < \frac{19\sqrt{42}}{42} \right) \approx 1 - P(Z < 2.93)
 \end{eqnarray*}
 Esto se puede aproximar usando la tabla A.1, con lo que se obtiene lo siguiente:
 \begin{equation*}
  \beta \approx 1 - P(Z < 2.93) \approx  1 - 0.9983 = 0.0017
 \end{equation*}
 Para el siguiente supuesto, se tiene que
 \begin{itemize}
  \item $p = 0.4$.
  \item $\mu = np = (50)(0.4) = 20$.
  \item $\sigma^2 = npq = 20(0.6) = 12$.
 \end{itemize}
 as\'{\i}, el erro tipo II se calcula como sigue:
 \begin{eqnarray*}
  \beta & = & 1 - P(X \leq 24) \approx 1 - P\left( Z < \frac{24.5 - 20}{\sqrt{12}} \right) = 1 - P\left( Z < \frac{4.5\sqrt{3}}{6} \right) = 1 - P\left( Z < \frac{9\sqrt{3}}{12} \right) \\
  & = & 1 - P\left( Z < \frac{3\sqrt{3}}{4} \right) \approx 1 - P(Z < 1.3)
 \end{eqnarray*}
 Esto se puede aproximar usando la tabla A.1, con lo que se obtiene lo siguiente:
 \begin{equation*}
  \beta \approx 1 - P(Z < 1.3) \approx 1 - 0.9032 = 0.0968
 \end{equation*}
 Y, para el \'ultimo supuesto, se tiene que
 \begin{itemize}
  \item $p = 0.5$.
  \item $\mu = np = (50)(0.5) = 25$.
  \item $\sigma^2 = npq = 25(0.5) = 12.5 = 25/2$.
 \end{itemize}
 as\'{\i}, el error tipo II se calcula como sigue:
 \begin{eqnarray*}
  \beta & = & 1 - P(X \leq 24) \approx 1 - P\left( Z < \frac{24.5 - 25}{\sqrt{25/2}} \right) = 1 - P\left( Z < -\frac{0.5\sqrt{2}}{5} \right) \\
  & = & 1 - P\left( Z < -\frac{\sqrt{2}}{10} \right) \approx 1 - P\left( Z < -0.14 \right)
 \end{eqnarray*}
 Esto se puede aproximar usando la tabla A.1, con lo que se obtiene:
 \begin{equation*}
  \beta \approx 1 - P\left( Z < -0.14 \right) \approx 1 - 0.4443 = 0.5557
 \end{equation*}
 Finalmente, usando R, se peude calcular estas probabilidades usando el script del archivo anexo \texttt{P02\_Probabilidad\_de\_error\_normal\_1.r}, cambiando las siguientes l\'{\i}neas de c\'odigo:
 \begin{verbatim}
> n<-50
> CriticoInf<-25
> CriticoSup<-NULL
> desv<-NULL
> media0<-NULL
> media1<-NULL
> p0<-0.6
> p1<-0.3
 \end{verbatim}
 \vspace{-0.5cm}
 con lo que se $\alpha$ y, para el supuesto $p = 0.3$, $\beta$, como se muestra a continuaci\'on:
 \begin{verbatim}
$`Probabilidad de error tipo I`
  HipotesisNula  n media     desv CriticoInf     alpha
1      p =  0.6 50    30 3.464102       24.5 0.0561756

$`Probabilidad de error tipo II`
  HipotesisAlternativa  n media    desv CriticoInf        beta
1             p =  0.3 50    15 3.24037       24.5 0.001685216
 \end{verbatim}
 \vspace{-0.5cm}
 Para el valor de $\beta$, bajo el supuesto $p = 0.4$, se ejecuta nuevamente el script con los siguientes cambios:
 \begin{verbatim}
> n<-50
> CriticoInf<-25
> CriticoSup<-NULL
> desv<-NULL
> media0<-NULL
> media1<-NULL
> p0<-NULL
> p1<-0.4
 \end{verbatim}
 \vspace{-0.5cm}
 con lo que se obtiene el siguiente resultado.
 \begin{verbatim}
$`Probabilidad de error tipo II`
  HipotesisAlternativa  n media     desv CriticoInf       beta
1             p =  0.4 50    20 3.464102       24.5 0.09696543
 \end{verbatim}
 \vspace{-0.5cm}
 Finalmente, para el valor de $\beta$ bajo el supuesto $p = 0.5$, se ejecuta una vez m\'as el script con el siguiente cambio:
 \begin{verbatim}
> n<-50
> CriticoInf<-25
> CriticoSup<-NULL
> desv<-NULL
> media0<-NULL
> media1<-NULL
> p0<-NULL
> p1<-0.5
 \end{verbatim}
 \vspace{-0.5cm}
 con lo que se obtiene el siguiente resultado:
 \begin{verbatim}
$`Probabilidad de error tipo II`
  HipotesisAlternativa  n media     desv CriticoInf      beta
1             p =  0.5 50    25 3.535534       24.5 0.5562315
 \end{verbatim}
 \vspace{-0.5cm}
 En resumen, se tiene lo siguiente:
 \begin{itemize}
  \item Bajo el supuesto real $p = 0.6$:
  \begin{itemize}
   \item La aproximaci\'on con las tablas da $\alpha = 0.0559$.
   \item La aproximaci\'on con R da $\alpha = 0.0561756$.
  \end{itemize}
  
  \item Bajo el supuesto $p = 0.3$:
  \begin{itemize}
   \item La aproximaci\'on con las tablas da $\beta = 0.0017$.
   \item La aproximaci\'on con R da $\beta = 0.001685216$.
  \end{itemize}

  \item Bajo el supuesto $p = 0.4$:
  \begin{itemize}
   \item La aproximaci\'on con las tablas da $\beta = 0.0968$.
   \item La aproximaci\'on en R da $\beta = 0.09696543$.
  \end{itemize}

  \item Y, bajo el supuesto $p = 0.5$:
  \begin{itemize}
   \item La aproximaci\'on con las tablas da $\beta = 0.5557$.
   \item La aproximaci\'on en R da $\beta = 0.5562315$.
  \end{itemize}
 \end{itemize}
 Que es a lo que se quer\'{\i}a llegar.${}_{\blacksquare}$
\end{solucion}
