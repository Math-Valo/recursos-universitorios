\begin{enunciado}
 Se lleva a cabo un estudio para saber si el aumento de la concentraci\'on de sustrato tiene un efecto apreciable sobre la velocidad de una reacci\'on qu\'{\i}mica. Con una concentraci\'on de sustrato de 1.5 moles por litro, la reacci\'on se realiz\'o $15$ veces, con una velocidad promedio de $7.5$ micromoles por 30 minutos y una desviaci\'on est\'andar de $1.5$. Con una concentraci\'on de sustrato de 2.0 moles por litro, se realizan $12$ reacciones, que dan una velocidad promedio de $8.8$ micromoles por 30 minutos y una desviaci\'on est\'andar muestral de $1.2$. ¿Hay alguna raz\'on para creer que este incremento en la concentraci\'on de sustrato ocasiona un aumento en la velocidad media de m\'as de $0.5$ micromoles por 30 minutos? Utiilice un nivel de significancia de $0.01$ y suponga que las poblaciones se distribuyen de forma aproximadamente normal con varianzas iguales.
\end{enunciado}

\begin{solucion}
 \begin{datos}
  $\phantom{0}$
  \begin{itemize}
   \item $X_i \sim n\left( \mu_i, \sigma_i \right)$, para cada $i \in \{ 1, 2 \}$.
   \item $n_1 = 15$ y $n_2 = 12$.
   \item $\bar{x}_1 = 7.5$ y $\bar{x}_2 = 8.8$.
   \item $s_1 = 1.5$ y $s_2 = 1.2$.
   \item $\sigma_1^2 = \sigma_2^2$.
  \end{itemize}
 \end{datos}

 \begin{hipotesis}
  \begin{eqnarray*}
   H_0: \mu_1 - \mu_2 & \geq & -0.5 \\
   H_1: \mu_1 - \mu_2 & < &  -0.5
  \end{eqnarray*}
 \end{hipotesis}

 \begin{significancia}
  $\alpha = 0.01$.
 \end{significancia}

 \begin{region}
  De la tabla A.4, se tiene el valor cr\'{\i}tico $t_{\alpha,n_1+n_2-2} = t_{0.01,25} \approx 2.485$, por lo que la regi\'on de rechazo est\'a dado para $t < -2.485$, donde $t = \frac{\left( \bar{x}_1 - \bar{x}_2\right) -d_0}{s_p\sqrt{1/n_1 + 1/n_2}}$.
 \end{region}

 \begin{estadistico}
  Dado que
  \begin{eqnarray*}
   s_p^2 & = & \frac{s_1^2\left( n_1 - 1\right) + s_2^2\left( n_2 - 1 \right)}{n_1 + n_2 - 2} = \frac{1.5^2(15 - 1) + 1.2^2(12 - 1)}{25} = \frac{2.25(14) + 1.44(11)}{25} \\
   & = & \frac{47.34}{25} = \frac{2\,367}{1\,250} = 1.8936
  \end{eqnarray*}
  entonces
  \begin{equation*}
   s_p = \sqrt{s_p^2} = \sqrt{\frac{2\,367}{1\,250}} = \frac{3\sqrt{263}\left( \sqrt{2} \right)}{25} = \frac{3\sqrt{526}}{50} \approx 1.376081
  \end{equation*}
  y
  \begin{eqnarray*}
   t & = & \frac{\left( \bar{x}_1 - \bar{x}_2 \right) - d_0}{s_p\sqrt{\frac{1}{n_1} + \frac{1}{n_2}}} = \frac{(7.5-8.8)-(-0.5)}{\frac{3\sqrt{526}}{50} \sqrt{\frac{1}{15} + \frac{1}{12}}} = \frac{(-1.3 + 0.5)(50)}{3\sqrt{526}\sqrt{\frac{12+15}{180}}} = \frac{-0.8(50)}{\frac{3\sqrt{526}\sqrt{27(5)}}{30}} \\
   & = & -\frac{40(30)}{3\sqrt{71\,010}} = - \frac{400}{3\sqrt{7\,890}} = -\frac{400\sqrt{7\,890}}{23\,670} \approx -1.50106754
  \end{eqnarray*}
 \end{estadistico}

 \begin{decision}
  No se rechaza $H_0$.
 \end{decision}

 \begin{conclusion}
  No hay evidencia suficiente para creer que con una concentraci\'on de sustrato de 1.5 moles por litro haya una velocidad menor de 0.5 micromoles por 30 minutos a que si la concentraci\'on de sustrato fuese de 2.0 moles por litro, por lo tanto, no se tomar\'a por cierto que el aumento de la concentraci\'on de sustrato tenga un efecto apreciable sobre la velocidad de una reacci\'on qu\'{\i}mica.
 \end{conclusion}

 Finalmente, usando el archivo anexo
 \texttt{P05\_Prueba\_de\_dos\_medias\_01.r},
 con los siguientes cambios:
 \begin{verbatim}
> n1<-15
> n2<-12
> mu<--0.5
> m1<-7.5
> m2<-8.8
> m<-NULL
> sigma1<-NULL
> sigma2<-NULL
> s1<-1.5
> s2<-1.2
> sD<-NULL
> desv.iguales<-TRUE
> alfa<-0.01
> cola<-'I'
> par<-FALSE
 \end{verbatim}
 \vspace{-0.5cm}
 el programa de R lanza el siguiente resultado:
 \begin{verbatim}
  Prueba var.pobl   H0 n1 n2 DifMedias desv.est1 desv.est2   est.sp error.est
1      t  Iguales -0.5 15 12      -1.3       1.5       1.2 1.376081  0.532954
  grados.libertad alpha    PValor Estadistico RegionRechazoT RegionRechazoX
1              25  0.01 0.0729318   -1.501068   < -2.4851072   < -1.8244479
         Resultado
1 No se rechaza H0
 \end{verbatim}
 \vspace{-0.5cm}
 El cual coincide con los datos obtenidos,
 que es a lo que se quer\'{\i}a llegar.${}_{\blacksquare}$
\end{solucion}
