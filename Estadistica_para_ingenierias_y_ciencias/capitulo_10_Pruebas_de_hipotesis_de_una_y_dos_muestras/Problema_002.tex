\begin{enunciado}
 Una soci\'ologa se interesa en la eficacia de un curso de entrenamiento dise\~nado para lograr que m\'as conductores utilicen los cinturones de seguridad en los autom\'oviles.
 \begin{enumerate}
  \item ¿Qu\'e hip\'otesis prueba ella si comete un error tipo I al concluir de manera err\'onea que el curso de entrenamiento no es eficaz?
  
  \item ¿Qu\'e hip\'otesis prueba ella si comete un error tipo II al concluir de forma err\'onea que el curso de entrenamiento es eficaz?
 \end{enumerate}
\end{enunciado}

\begin{solucion}
 $\phantom{0}$
 \begin{enumerate}
  \item La hip\'otesis nula que prueba ella es que el curso de entrenamiento s\'{\i} es eficaz.
  
  \item La hip\'otesis nula que prueba ella es, nuevamente, que el curso de entrenamiento s\'{\i} es eficaz, que es a lo que se quer\'{\i}a llegar.${}_{\blacksquare}$
 \end{enumerate}
\end{solucion}
