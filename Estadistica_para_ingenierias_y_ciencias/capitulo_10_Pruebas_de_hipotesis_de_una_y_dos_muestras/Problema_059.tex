\begin{enunciado}
 Una compa\~n\'{\i}a petrolera afirma que un quinto de las casas en cierta ciudad
 se calientan con petr\'oleo.
 ¿Tenemos raz\'on para creer que menos de $1/5$ se calientan con petr\'oleo
 si, en una muestra aleatoria de $1000$ casas en esta ciudad, se encuentra
 que $136$ se calientan con petr\'oleo? Utilice un valor $P$ en su conclusi\'on.
\end{enunciado}

\begin{solucion}
 \begin{datos}
  $\phantom{0}$
  \begin{itemize}
   \item $n = 1\,000$.
   \item $x = 136$.
  \end{itemize}
 \end{datos}

 \begin{hipotesis}
  \begin{eqnarray*}
   H_0: p & \geq & 0.2 \\
   H_1: p &  <   & 0.2
  \end{eqnarray*}
 \end{hipotesis}

 \begin{estadistico}
  \begin{equation*}
   z = \frac{x - np}{\sqrt{npq}}
   = \frac{136 - (1\,000)(0.2)}{\sqrt{(1\,000)(0.2)(0.8)}}
   = \frac{136 - 200}{\sqrt{160}} = - \frac{64}{4\sqrt{10}}
   = - \frac{64\sqrt{10}}{40} = - \frac{16\sqrt{10}}{10} \approx - 5.059644
  \end{equation*}
 \end{estadistico}
 
 \begin{valorp}
  De la tabla A.3, se tiene que:
  \begin{equation*}
   P(Z < z) = P(Z > -Z) \approx P(Z > 5.06) \approx 0
  \end{equation*}
 \end{valorp}
 
 \begin{conclusion}
  Por lo tanto, como el valor $P$ es muy cercano a cero, se tiene evidencia
  suficiente para rechazar la hip\'otesis nula a favor de la alternativa;
  es decir, que s\'{\i} hay raz\'on suficiente para afirmar que menos de $1/5$
  de las casas se calientan con petr\'oleo.
 \end{conclusion}
 Finalmente, usando el archivo anexo \texttt{P08\_Prueba\_de\_una\_proporcion\_01.r},
 con los siguientes cambios:
 \begin{verbatim}
> n<-1000
> x<-136
> p0<-NULL
> p<-1/5
> alfa<-NULL
> cola<-'I'
> distr<-NULL
 \end{verbatim}
 \vspace{-0.5cm}
 el programa de R lanza el siguiente resultado:
 \begin{verbatim}
   distr   p    n   x pMuestral media desv.est alpha PValor Estadistico
1 Normal 0.2 1000 136     0.136   200 12.64911  0.05  2e-07   -5.059644
  RegionRechazoZ RegionRechazoX
1  <= -1.6448536         <= 179
 \end{verbatim}
 \vspace{-0.5cm}
 El cual coincide con los resultados obtenidos,
 que es a lo que se quer\'{\i}a llegar.${}_{\blacksquare}$
\end{solucion}
