\begin{enunciado}
 Con referencia al ejercicio 10.39 de la p\'agina 359, pruebe la hip\'otesis
 de que $\sigma_1^2 = \sigma_2^2$ contra la alternativa
 de que $\sigma_2^2 \neq \sigma_2^2$,
 donde $\sigma_1^2$ y $\sigma_2^2$ son las varianzas para los tiempos
 de duraci\'on de pel\'{\i}culas producidas
 por la compa\~n\'{\i}a 1 y la compa\~n\'{\i}a 2, respectivamente.
 Utilice un valor $P$.
\end{enunciado}

\begin{solucion}
 Resumiendo los datos del ejercicio 10.39, se tiene lo siguiente:
 \begin{itemize}
  \item $X_i \sim n\left( \mu_i, \sigma_i \right)$,
  para cada $i \in \{ 1,2 \}$.
  \item $n_1 = 5$ y $n_2 = 7$.
  \item $s_1^2 = \frac{394}{5} = 78.8$
  y $s_2^2 = \frac{2\,740}{3} = 913.\bar{3}$.
 \end{itemize}

 \begin{hipotesis}
  \begin{eqnarray*}
   H_0: \sigma_1^2 &  =   & \sigma_2^2 \\
   H_1: \sigma_1^2 & \neq & \sigma_2^2
  \end{eqnarray*}
 \end{hipotesis}

 \begin{estadistico}
  \begin{equation*}
   f = \frac{s_1^2}{s_2^2}
   = \frac{\displaystyle{\frac{394}{5}}}{\displaystyle{\frac{2\,740}{3}}}
   = \frac{394(3)}{2\,740(5)} = \frac{197(3)}{1\,370(5)}
   = \frac{591}{6\,850} \approx 0.08\overline{62773722}
  \end{equation*}
 \end{estadistico}

 \begin{valorp}
  De la tabla A.6 se observa que el valor $f$
  est\'a muy alejado de $f_{0.05,4,6}$;
  sin embargo, como $f_{0.95,4,6} = \frac{1}{f_{0.05,6,4}}$,
  se tiene que $\frac{1}{f} = \frac{6\,850}{591} \approx 11.59052$,
  el cual se encuentra entre $f_{0.05,6,4}$ y $f_{0.01,6,4}$, por lo que
  $P\left(F>\frac{1}{f}\right) \approx P(F > 11.59052)$ es un valor
  entre $0.01$ y $0.05$ y se va a considerar el valor interpolado $0.016$,
  entonces:
  \begin{equation*}
   2\times P\left( F_{v_1,v_2} < f\right)
   \approx 2\times P\left( F_{v_2,v_1} > \frac{1}{f}  \right)
   \approx 2\times P\left( F_{v_2,v_1} > 11.59052 \right)
   \approx 2(0.016) = 0.032
  \end{equation*}
 \end{valorp}

 \begin{conclusion}
  Ya que el valor $P$ es muy peque\~no,
  entonces hay evidencia suficiente para rechazar la hip\'otesis nula,
  y, por lo tanto, se afirma que la varianza en los tiempos
  para ensamblar el producto no son significativamente diferentes
  y se pueden considerar como iguales los tiempos de duraci\'on
  de las pel\'{\i}culas producidas por la compa\~n\'{\i}a 1
  y la de las compa\~n\'{\i}a 2.
 \end{conclusion}

 Finalmente, usando el archivo anexo \texttt{P14\_Prueba\_de\_dos\_varianzas\_02.r}, con los siguientes cambios:
 \begin{verbatim}
> datos<-read.csv("DB03_Problema_039.csv",sep=";",encoding="UTF-8")
> varInteres<-c("Tiempo.min")
> varSel<-c("Compañía")
> alfa<-NULL
> cola<-'D'
 \end{verbatim}
 \vspace{-0.5cm}
 el programa de R lanza el siguiente resultado:
 \begin{verbatim}
    variable Freq n1 n2 media1 media2 varianza1 varianza2 v1 v2 alpha    PValor
1 Tiempo.min   12  5  7   97.4    110      78.8  913.3333  4  6  0.05 0.0329772
  Estadistico             RegionRechazo
1   0.0862774 < 0.1087274 y > 6.2271612
 \end{verbatim}
 \vspace{-0.5cm}
 El cual coincide con los resultados obtenidos,
 adem\'as de ofrecer un valor m\'as preciso para el valor $P$,
 que es a lo que se quer\'{\i}a llegar.${}_{\blacksquare}$
\end{solucion}
