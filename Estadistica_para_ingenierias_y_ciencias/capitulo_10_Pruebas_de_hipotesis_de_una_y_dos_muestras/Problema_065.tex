\begin{enunciado}
 Una comunidad urbana quiere demostrar que la incidencia de c\'ancer de seno es mayor
 en ella que en un \'area rural vecina.
 (Se encontr\'o que los niveles de \texttt{PCB} son m\'as altos en el suelo
 de la comunidad urbana.)
 Si se encuentra que $20$ de $200$ mujeres adultas en la comunidad urbana
 tienen c\'ancer de seno y $10$ de $150$ mujeres adultas en la comunidad rural
 tienen c\'ancer de seno,
 ¿podr\'{\i}amos concluir con un nivel de significancia de $0.05$
 que este tipo de c\'ancer prevalece m\'as en la comunidad urbana?
\end{enunciado}

\begin{solucion}
 \begin{datos}
  $\phantom{0}$
  \begin{itemize}
   \item $n_1 = 200$ y $n_2 = 150$.
   \item $x_1 = 20$ y $x_2 = 10$.
  \end{itemize}
  Adem\'as, se pueden calcular las proporciones estimadas de cada muestra
  y la estimaci\'on combinada de la proporci\'on a la que se supone que son iguales
  en ambas poblaciones, como se muestra a continuaci\'on:
  \begin{itemize}
   \item $\widehat{p}_1 = \frac{20}{200} = \frac{1}{10} = 0.1$.
   \item $\widehat{p}_2 = \frac{10}{150} = \frac{1}{15} = 0.0\bar{6}$.
   \item $\widehat{p} = \frac{x_1+x_2}{n_1+n_2} = \frac{20+10}{200+150}
   = \frac{30}{350} = \frac{3}{35} = 0.0\overline{857142}$.
  \end{itemize}
 \end{datos}

 \begin{hipotesis}
  \begin{eqnarray*}
   H_0: p_1 & \leq & p_2 \\
   H_1: p_1 &  >   & p_2
  \end{eqnarray*}
 \end{hipotesis}

 \begin{significancia}
  $\alpha = 0.05$.
 \end{significancia}

 \begin{region}
  De la tabla A.3, se tiene el valor cr\'{\i}tico $z_{\alpha}=z_{0.05}\approx 1.645$,
  por lo que la regi\'on de rechazo est\'a dado para $z > 1.645$,
  donde $z \approx \frac{\hat{p}_1 - \hat{p}_2}{\sqrt{\hat{p}\hat{q}(1/n_1+1/n_2)}}$.
 \end{region}

 \begin{estadistico}
  \begin{eqnarray*}
   z & \approx &
   \frac{\widehat{p}_1 - 
   \widehat{p}_2}{\sqrt{\widehat{p}\widehat{q}\left(1/n_1+1/n_2\right)}}
   = \frac{
   \displaystyle{ \frac{1}{10} - \frac{1}{15}}
   }{
   \displaystyle{ \sqrt{
   \left(\frac{3}{35}\right)\left(\frac{32}{35}\right)
   \left(\frac{1}{200} + \frac{1}{150}\right)
   }}}
   = \frac{\displaystyle{\frac{\cancel{5}}{\cancelto{30}{150}}}}
   {\displaystyle{
   \sqrt{\left( \frac{\cancel{3}\cdot 2^5}{35^{\cancel{2}}} \right)
   \left( \frac{\cancel{35}}{ \cancelto{1\,000}{3\,000} } \right)}
   }} \\
   & = & \frac{1}
   {\displaystyle{
   30\sqrt{\frac{ 2^{\cancelto{2}{5}} }{ 7 \cdot \cancel{2^3} \cdot 5^4  }}
   }}
   = \frac{1}{\displaystyle{
   \frac{\cancelto{12}{60}}{\cancelto{5}{25}}\sqrt{\frac{1}{7}}
   }}
   = \frac{5\sqrt{7}}{12}
   \approx 1.10239637961
  \end{eqnarray*}
 \end{estadistico}

 \begin{decision}
  No se rechaza $H_0$.
 \end{decision}

 \begin{conclusion}
  No hay evidencia suficiente
  que demuestre que la incidencia de c\'ancer de seno es mayor
  en la comunidad urbana que en el \'area rural vecina.
 \end{conclusion}

 Finalmente, usando el archivo anexo \texttt{P09\_Prueba\_de\_dos\_proporciones\_01.r}, con los siguientes cambios:
 \begin{verbatim}
> n1<-200
> n2<-150
> x1<-20
> x2<-10
> p1<-NULL
> p2<-NULL
> alfa<-0.05
> alternativa<-'>'
 \end{verbatim}
 \vspace{-0.5cm}
 el programa de R lanza el siguiente resultado:
 \begin{verbatim}
  alternativa  n1  n2 x1 x2  p1         p2  pEstimada    DifProp  error.est
1     p1 > p2 200 150 20 10 0.1 0.06666667 0.08571429 0.03333333 0.03023716
  alpha    PValor Estadistico RegionRechazoZ        Resultado
1  0.05 0.1351447    1.102396   >= 1.6448536 No se rechaza H0
 \end{verbatim}
 \vspace{-0.5cm}
 El cual coincide con los resultados obtenidos,
 que es a lo que se quer\'{\i}a llegar.${}_{\blacksquare}$
\end{solucion}
