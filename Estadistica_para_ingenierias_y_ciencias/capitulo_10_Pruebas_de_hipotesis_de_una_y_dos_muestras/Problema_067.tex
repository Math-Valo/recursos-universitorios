\begin{enunciado}
 Se sabe que el volumen de los envases de un lubricante espec\'{\i}fico
 se distribuye normalmente con una varianza de $0.03$ litros.
 Pruebe la hip\'otesis de que $\sigma^2 = 0.03$
 contra la alternativa de que $\sigma^2 \neq 0.03$
 para la muestra aleatoria de $10$ envases del ejercicio 10.25
 de la p\'agina 357.
 Use un valor $P$ en sus conclusiones.
\end{enunciado}

\begin{solucion}
 \begin{datos}
  Resumiendo, los datos obtenidos del ejercicio 10.25 para este ejercicio son:
  \begin{itemize}
   \item $X \sim n\left( \mu, \sigma \right)$.
   \item $n = 10$.
%    \item $\bar{x} = \frac{503}{10} = 10.06$.
   \item $s = \frac{2\sqrt{85}}{75} \approx 0.245854518861$.
  \end{itemize}
 \end{datos}

 \begin{hipotesis}
  \begin{eqnarray*}
   H_0: \sigma^2 &  =   & 0.03 \\
   H_1: \sigma^2 & \neq & 0.03
  \end{eqnarray*}
 \end{hipotesis}

 \begin{estadistico}
  \begin{equation*}
   \chi^2 = \frac{(n-1)s^2}{\sigma_0^2}
   = \frac{\displaystyle{ (10-1)\left( \frac{2\sqrt{85}}{75} \right)^2 }}{0.03}
   = \frac{
   \displaystyle{9\cdot \frac{4(\cancelto{17}{85})}{\cancelto{1\,125}{5\,625}}}
   }{
   \displaystyle{\frac{3}{100}}
   }
   = \frac{\cancel{9}(68)(100)}{\cancelto{125}{1\,125}(3)}
   = \frac{68(4)}{3(5)} = \frac{272}{15} = 18.1\bar{3}
  \end{equation*}
 \end{estadistico}

 \begin{valorp}
  De la tabla A.5 se interpola que:
  \begin{equation*}
   P\left( \left| X^2 \right| > \left| \chi^2 \right| \right)
   = 2P\left( X^2 > \chi^2 \right)
   = 2P( X^2 > 18.1\bar{3}) \approx 2(0.03) = 0.06
  \end{equation*}
 \end{valorp}

 \begin{conclusion}
  La probabilidad es suficientemente baja para sospechar que hay algo de evidencia
  que respalde que la varianza en el contenido de los envases de lubricante difiere
  de $0.03$ litros y ser\'{\i}a mejor que quede a la decisi\'on de un experto o
  tomar una decisi\'on respecto al conflicto que podr\'{\i}a ocasionar equivocarse
  con suponer como cierta la hip\'otesis nula.
  En esta ocasi\'on, siendo yo la persona a decidir,
  tomo como evidencia suficiente para rechazar la hip\'otesis nula
  a favor de la alternativa;
  es decir, la varianza en el contenido de los envases de lubricantes difiere
  de $0.03$ litros.
 \end{conclusion}

 En el c\'odigo registrado en el archivo anexo
 \texttt{P10\_Prueba\_de\_una\_varianza\_01.r}, en R,
 se realiza este procedimiento.
 El c\'odigo permite modificar los valores iniciales
 que corresponden a:
 \texttt{datos}, que guarda los datos de la lectura
 de un archivo,
 en este caso se lee el arhivo \texttt{DB01\_Problema\_025.csv},
 y este \'ultimo nombre es el que se modifica 
 para leer otros archivos;
 \texttt{varInteres} para indicar el nombre de la columna
 que corresponden a los datos en la base anterior;
 \texttt{var} para el valor de la varianza poblacional supuesta
 en la hip\'otesis nula;
 \texttt{alfa} para el nivel de significancia;
 \texttt{cola} para indicar si la prueba es de dos colas,
 con \texttt{'D'}, de cola inferior, con \texttt{'I'},
 o de cola superior, con \texttt{'S'}.
 \par 
 El programa espera al menos los datos correspondientes
 a la base de datos, escrito en un archivo \texttt{.csv},
 con una columna con todos los datos;
 se indica tambi\'en el tipo de prueba (cola izquierda o derecha,
 o dos colas);
 y, finalmente, la varianza poblacional,
 que corresponde a la hip\'otesis nula $H_0$.
 Si no hay una instrucci\'on para asignar un valor a $\alpha$
 entonces se le debe asignar el valor \texttt{NULL}.
 \par 
 La prueba de hip\'otesis siempre usar\'a un estad\'{\i}stico
 con distribuci\'on $\chi^2$.
 Para ello, se est\'a suponiendo de antemano la normalidad
 de la poblaci\'on de las que viene la muestra.
 \par 
 Independientemente del tipo de prueba, el resultado muestra lo siguiente:
 \texttt{var} para conocer lo que est\'a midiendo los datos,
 incluyendo el tipo de unidad que se usa;
 \texttt{n} indica el tama\~no de la muestra;
 \texttt{H0} para indicar el valor propuesto en la hip\'otesis nula
 de la varianza poblacional;
 \texttt{var.muestral}, para indicar la varianza muestral de los datos;
 \texttt{grados}, indica los grados de libertad de la distribuci\'on $\chi^2$
 correspondiente al estadístico de la prueba;
 \texttt{error.est}, para indicar el error est\'andar,
 que corresponde a la estimaci\'on de la varianza poblacional
 usando la varianza muestral;
 \texttt{alfa}, para el nivel de significancia dado,
 el cual muestra por defecto $0.05$
 en caso de asignarle \texttt{NULL} a \texttt{alfa};
 \texttt{PValor}, para el valor $P$, la probabilidad
 de haber obtenido la varianza muestral
 que se obtuvo suponiendo que la hip\'otesis nula sea cierta;
 \texttt{Estadistico}, para el valor resultante
 del estad\'{\i}stico de prueba;
 \texttt{RegionRechazoJi} para la regi\'on de rechazo del estad\'{\i}stico
 de prueba;
 y, \texttt{RegionRechazoX} para la regi\'on de rechazo de la varianza muestral.
 \par 
 Adem\'as, seg\'un el tipo de prueba, pueden darse el valor
 de \texttt{Resultado} para indicar si se rechaza o no
 la hip\'otesis nula, que aparece al final de los resultados
 cuando se asigna un valor a \texttt{alfa}.
 \par 
 El c\'odigo junto con el resultado se muestra a continuaci\'on:
 \begin{verbatim}
> datos<-read.csv("DB01_Problema_025.csv",sep=";",encoding="UTF-8")
> varInteres<-c("Contenido.l")
> var<-0.03
> alfa<-NULL
> cola<-'D'
> valores<-unlist(datos[,varInteres])
> variable<-factor(rep(varInteres,each=dim(datos)[1]))
> TestVar<-function(x,varP,alfa=0.05,colas='D'){
+   x<-x[!is.na(x)]
+   n<-length(x)
+   r<-data.frame(n=n,
+                 H0=varP)
+   if(n<2){
+     r$var.muestral<-NA
+     r$grados<-n-1
+     r$error.est<-NA
+     r$alpha<-alfa
+     r$Pvalor<-NA
+     r$estadistico<-NA
+     r$RegionRechazoJi<-NA
+     r$RegionRechazoX<-NA
+   }else{
+     var<-var(x)
+     v<-n-1
+     error<-varP/v
+     estadistico<-var*v/varP
+     r$var.muestral<-round(var,7)
+     r$grados<-v
+     r$error.est<-round(error,7)
+     r$alpha<-alfa
+     if(colas=='I'){
+       criticojiL<-qchisq(alfa,v)
+       criticoXL<-criticojiL*error
+       r$PValor<-round(pchisq(estadistico,v),7)
+       r$Estadistico<-estadistico
+       r$RegionRechazoJi<-paste("<",round(criticojiL,7))
+       r$RegionRechazoX<-paste("<",round(criticoXL,7))
+     }else{
+       criticojiU<-qchisq(alfa,v,lower.tail=F)
+       criticoXU<-criticojiU*error
+       if (colas=='S') {
+         criticojiU<-qchisq(alfa,v,lower.tail=F)
+         criticoXU<-criticojiU*error
+         r$PValor<-round(pchisq(estadistico,v,lower.tail=F),7)
+         r$Estadistico<-estadistico
+         r$RegionRechazoJi<-paste(">",round(criticojiU,7))
+         r$RegionRechazoX<-paste(">",round(criticoXU,7))
+       }else{
+         criticojiU<-round(qchisq(alfa/2,v,lower.tail=F),7)
+         criticoXU<-round(criticojiU*error,7)
+         criticojiL<-round(qchisq(alfa/2,v),7)
+         criticoXL<-round(criticojiL*error,7)
+         pvalor<-round(pchisq(estadistico,v),7)
+         r$Pvalor<-ifelse(var<varP,2*pvalor,2*(1-pvalor))
+         r$Estadistico<-estadistico
+         r$RegionRechazoJi<-paste("<",criticojiL,7,"y >",criticojiU,7)
+         r$RegionRechazoX<-paste("<",criticoXL,7,"y >",criticoXU,7)
+       }
+     }
+     return(r)
+   }
+ }
> if(is.null(alfa)){
+   Test<-TestVar(valores,var,colas=cola)
+ }else{
+   Test<-TestVar(valores,var,alfa=alfa,colas=cola)
+   resultado<-ifelse(Test[,"PValor"]>=alfa,
+                     "No se rechaza H0","Se rechaza H0")
+   Test$Resultado<-resultado
+   Test$Resultado[is.na(Test$Resultado)]<-"Pocos datos"
+ }
> identif<-data.frame(varInteres)
> names(identif)<-"var"
> Test<-data.frame(identif,Test)
> Test
          var  n   H0 var.muestral grados error.est alpha    Pvalor Estadistico
1 Contenido.l 10 0.03    0.0604444      9 0.0033333  0.05 0.0673156    18.13333
             RegionRechazoJi            RegionRechazoX
1 < 2.7003895 y > 19.0227678 < 0.0090013 y > 0.0634092
 \end{verbatim}
 El cual coincide con los c\'alculos obtenidos.
 N\'otese que en el resultado del c\'odigo se indica que el estad\'{\i}stico
 no est\'a en la regi\'on de rechazo de $\chi^2$,
 o lo que es an\'alogo, la varianza muestral no est\'a en la regi\'on de rechazo
 de $X$, y sin embargo se ha rechazado la hip\'otesis nula.
 Esto se debe a que no se consider\'o en la prueba un valor estricto de $\alpha$,
 mientras que en el c\'odigo, cuando no se da, asigna de forma predeterminada
 el valor $\alpha=0.05$; sin embargo, se consider\'o el valor $P$ para la prueba,
 el cual fue suficientemente bajo para el criterio de este autor.
 Que es a lo que se quer\'{\i}a llegar.${}_{\blacksquare}$
\end{solucion}
