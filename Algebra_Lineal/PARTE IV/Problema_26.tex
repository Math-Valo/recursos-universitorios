\begin{enunciado}
 Sea $\omega: \mathbb{R}^3 \to \mathbb{R}$ la forma cuadr\'atica dada por
 \begin{equation*}
  \omega(x,y,z) = y^2 - 2xy - 2xz - 2yz
 \end{equation*}
 Diagonalizar ortogonalmente la for forma cuadr\'atica $\omega$ (en $\mathbb{R}^3$ se considera el producto escalar can\'onico).
\end{enunciado}
 
\begin{solucion}
 N\'otese que la matriz de $\omega$ en la base can\'onica de $\mathbb{R}^3$ es
 \begin{equation*}
  A =
  \begin{bmatrix}
    0 & -1 & -1 \\
   -1 &  1 & -1 \\
   -1 & -1 &  0
  \end{bmatrix}
 \end{equation*}
 Luego entonces, se procede a diagonalizar esta matriz. Para ello se hallar\'an sus valores propios analizando las ra\'{\i}ces de su polinomio caracter\'{\i}stico $p(\lambda) = \det(A - \lambda I)$, el cual se calcula como sigue:
 \begin{eqnarray*}
  \det(A - \lambda I) & = & 
  \begin{vmatrix}
   -\lambda & -1 & -1 \\
   -1 & 1-\lambda & -1 \\
   -1 & -1 & - \lambda 
  \end{vmatrix}
  \\
  & = &
  -\lambda 
  \begin{vmatrix}
   1-\lambda & -1 \\
   -1 & -\lambda
  \end{vmatrix}
  +
  \begin{vmatrix}
   -1 & -1 \\
   -1 & -\lambda 
  \end{vmatrix}
  -
  \begin{vmatrix}
   -1 & -1 \\
   1-\lambda & -1
  \end{vmatrix} 
  \\
  & = & -\lambda(\lambda^2-\lambda -1) + (\lambda  - 1) - (1 + 1 - \lambda) \\ 
  & = & -\lambda^3 + \lambda^2 +\lambda + \lambda - 1 - 2 + \lambda  = -\lambda^3 + \lambda^2 + 3\lambda - 3 \\
  & = & -(\lambda^3 - \lambda^2 - 3\lambda + 3) \\
  & = & -(\lambda-1)(\lambda^2-3) \\ 
  & = & -(\lambda - 1)(\lambda - \sqrt{3})(\lambda + \sqrt{3})
 \end{eqnarray*}
 Luego entonces, los autovalores de $A$ son $\lambda_1 = 1$, $\lambda_2 = \sqrt{3}$ y $\lambda_3 = -\sqrt{3}$. Entonces, se procede a calcular los subespacios propios de $A$ y hallar la base de cada uno de estos, lo cual se har\'a a trav\'es de los vectores de coordenadas, $X=[x,y,z]^t$, en la base can\'onica, tales que $(A-\lambda_i I)X = O$, para $i \in \{ 1,2,3 \}$.
 \par 
 Para el caso $\lambda_1 = 1$, se tiene la ecuaci\'on $(A-I)X = O$, dada por
 \begin{equation*}
  \begin{bmatrix}
   -1 & -1 & -1 \\
   -1 &  0 & -1 \\
   -1 & -1 & -1
  \end{bmatrix}
  \begin{bmatrix}
   x \\ y \\ z
  \end{bmatrix}
  = 
  \begin{bmatrix}
   -x & -y & -z \\
   -x &    & -z \\
   -x & -y & -z
  \end{bmatrix}
  = 
  \begin{bmatrix}
   0 \\ 0 \\ 0
  \end{bmatrix}
 \end{equation*}
 lo cual, al igualar las entradas, da un sistema de tres ecuaciones. De la segunda ecuaci\'on, se tiene que $-x-z =0$, por lo que $x=-z$ y, restando la primera ecuaci\'on a la segunda ecuaci\'on, se tiene que $y=0$. Por lo tanto:
 \begin{equation*}
  E_{\lambda_1 = 1} = \left< (1,0,-1) \right>
 \end{equation*}
 Para el caso $\lambda_2 = \sqrt{3}$, se tiene la ecuaci\'on $(A-\sqrt{3}I)X = O$, dada por
 \begin{equation*}
  \begin{bmatrix}
   -\sqrt{3} & -1 & -1 \\
   -1 &  1-\sqrt{3} & -1 \\
   -1 & -1 & -\sqrt{3}
  \end{bmatrix}
  \begin{bmatrix}
   x \\ y \\ z
  \end{bmatrix}
  = 
  \begin{bmatrix}
   -\sqrt{3}x & -y & -z \\
   -x & (1-\sqrt{3})y & -z \\
   -x & -y & -\sqrt{3}z
  \end{bmatrix}
  = 
  \begin{bmatrix}
   0 \\ 0 \\ 0
  \end{bmatrix}
 \end{equation*}
 lo cual, al igualar las entradas, da un sistema de tres ecuaciones.
 Entonces, al restar la tercera ecuaci\'on a la primera, se tiene que $(1-\sqrt{3})x - (1-\sqrt{3})z = 0$, por lo que $x=z$ y, sustituyendo esto en la segunda ecuaci\'on, se tiene que $-2x + (1-\sqrt{3})y = 0$, por lo que $x = \frac{1-\sqrt{3}}{2}y$. Por lo tanto:
 \begin{equation*}
  E_{\lambda_2 = \sqrt{3}} = \left< (1,-1-\sqrt{3},1) \right>
 \end{equation*}
 Para el caso $\lambda_3 = -\sqrt{3}$, se tiene la ecuaci\'on $(A+\sqrt{3}I)X = O$, dada por
 \begin{equation*}
  \begin{bmatrix}
   \sqrt{3} & -1 & -1 \\
   -1 &  1+\sqrt{3} & -1 \\
   -1 & -1 & \sqrt{3}
  \end{bmatrix}
  \begin{bmatrix}
   x \\ y \\ z
  \end{bmatrix}
  = 
  \begin{bmatrix}
   \sqrt{3}x & -y & -z \\
   -x & (1+\sqrt{3})y & -z \\
   -x & -y & +\sqrt{3}z
  \end{bmatrix}
  = 
  \begin{bmatrix}
   0 \\ 0 \\ 0
  \end{bmatrix}
 \end{equation*}
 lo cual, al igualar las entradas, da un sistema de tres ecuaciones.
 Entonces, al restar la tercera ecuaci\'on a la primera, se tiene que $(1+\sqrt{3})x - (1+\sqrt{3})z = 0$, por lo que $x=z$ y, sustituyendo esto en la segunda ecuaci\'on, se tiene que $-2x + (1+\sqrt{3})y = 0$, por lo que $x = \frac{1+\sqrt{3}}{2}y$. Por lo tanto:
 \begin{equation*}
  E_{\lambda_3 = -\sqrt{3}} = \left< (1,\sqrt{3}-1,1) \right>
 \end{equation*}
 Por lo tanto, normalizando estos vectores generadores, se obtienen los siguientes: 
 \begin{eqnarray*}
  \bar{u}_1 & = & \frac{(1,0,-1)}{\lVert (1,0,-1) \rVert} = \frac{(1,0,-1)}{\sqrt{2}} \\
  & = & \frac{\sqrt{2}}{2}(1,0,-1) \\
  \bar{u}_2 & = & \frac{(1,-1-\sqrt{3},1)}{\lVert (1,-1-\sqrt{3},1) \rVert} = \frac{(1,-1-\sqrt{3},1)}{\sqrt{6 + 2\sqrt{3}}} \\
  & = & \frac{1}{\sqrt{6 + 2\sqrt{3}}}(1,-1-\sqrt{3},1) \\
  \bar{u}_3 & = & \frac{(1,\sqrt{3}-1,1)}{\lVert (1,\sqrt{3}-1,1) \rVert} = \frac{(1,\sqrt{3}-1,1)}{\sqrt{6 - 2\sqrt{3}}} \\
  & = & \frac{1}{\sqrt{6 - 2\sqrt{3}}}(1,\sqrt{3}-1,1)
 \end{eqnarray*}
 As\'{\i} que, por teorema, se tiene que en la base $\beta = \left( \bar{u}_1, \bar{u}_2, \bar{u}_3 \right)$ se diagonaliza ortogonalmente a $\omega$ y su matriz diagonal, $D$, en esta base est\'a dado por los autovalores correspondientes a los autovectores de $\beta$, en el mismo orden. De este modo, si $(x_1, x_2, x_3)$ representa las coordeandas de un vector gen\'erico en la base $\beta$, entonces
 \begin{equation*}
  \omega(x_1, x_2, x_3) = x_1^2 + \sqrt{3}x_2^2 - \sqrt{3}x_3^2
 \end{equation*}
 y la matriz de $\omega$ en esta base es:
 \begin{equation*}
  D = 
  \begin{bmatrix}
   1 & 0 & 0 \\
   0 & \sqrt{3} & 0 \\
   0 & 0 & -\sqrt{3}
  \end{bmatrix}
 \end{equation*}
 que es a lo que se quer\'{\i}a llegar.${}_{\blacksquare}$
\end{solucion}
