\begin{enunciado}
 Hallar los autovalores y los subespacios propios del endomorfismo $f:V \rightarrow V$ (donde $V$ es un espacio vectorial real de dimensi\'on $4$) que, en cierta base dada $(\overline{e}_1, \overline{e}_2, \overline{e}_3, \overline{e}_4)$ de $V$, tiene asociada la siguiente matriz $A$:
 \begin{equation*}
  A = 
  \begin{bmatrix}
    3 &  3 & 0 &  1 \\
   -1 & -1 & 0 & -1 \\
    1 &  2 & 1 &  1 \\
    2 &  4 & 0 &  3
  \end{bmatrix}
 \end{equation*}
 Anal\'{\i}cese si $f$ es diagonalizable.
\end{enunciado}

\begin{solucion}
 Se proceder\'a a calcular el polinomio caracter\'{\i}stico de $f$ a trav\'es de la matriz asociada a $f$ en la base $\beta = (\overline{e}_1, \overline{e}_2, \overline{e}_3, \overline{e}_4)$, $A$, como sigue:
 \begin{eqnarray*}
  \det(A-\lambda I) & = &
  \begin{vmatrix}
    3 - \lambda &  3           & 0     &          1           \\
   -1           & -1 - \lambda & 0     &          -1           \\
    1           &  2           & 1 - \lambda &  1           \\
    2           &  4           & 0     &          3 - \lambda 
  \end{vmatrix} \\
  & = & (1-\lambda)
  \begin{vmatrix}
    3 - \lambda &  3           &  1 \\
   -1           & -1 - \lambda & -1 \\
    2           &  4           &  3 - \lambda
  \end{vmatrix} \\ 
  & = & (1-\lambda)\left[ (3-\lambda)
  \begin{vmatrix}
   -1 - \lambda & -1           \\
    4           &  3 - \lambda
  \end{vmatrix}
  +
  \begin{vmatrix}
   3 & 1           \\
   4 & 3 - \lambda
  \end{vmatrix}
  + 2
  \begin{vmatrix}
    3           &  1 \\
   -1 - \lambda & -1
  \end{vmatrix}
  \right] \\
  & = & (1-\lambda)\left[ (3-\lambda)(\lambda^2 - 2 \lambda - 3 + 4) + (-3\lambda + 9 - 4) + 2(-3 + 1+\lambda) \right] \\ 
  & = & (1-\lambda)\left[ (3-\lambda)(\lambda^2 - 2\lambda +1 ) + (-3\lambda + 5) + 2(\lambda - 2) \right] \\
  & = & (1 - \lambda)\left[ (-\lambda^3 + 5\lambda^2 - 7\lambda + 3) + (-3\lambda + 5) + (2\lambda - 4) \right] \\
  & = & (1 - \lambda)(-\lambda^3 + 5\lambda^2 - 8\lambda + 4) \\
  & = & (\lambda - 1)(\lambda^3 - 5\lambda^2 + 8\lambda - 4) \\
  & = & (\lambda - 1)^2(\lambda^2 - 4\lambda + 4) \\ 
  & = & (\lambda - 1)^2(\lambda - 2)^2
 \end{eqnarray*}
 Por lo tanto, $\det (A - \lambda I) = 0$ para los autovalores, de $A$ que tambi\'en son de $f$: $\lambda_1 = 1$, con multiplicidad algebraica $m_1 = 2$, y $\lambda_2 = 2$, con multiplicidad algebraica $m_2 = 2$.
 \par 
 Para hallar los subespacios propios de $f$ se usar\'a nuevamente la matriz asociada a $f$ en la base $\beta$, calculando los vectores coordenadas $X = (x_1, x_2, x_3, x_4)^t$ de un vector gen\'erico $\overline{x}$ en la base $\beta$ que cumplan que $AX = \lambda_iX$ para cada $i \in \{ 1, 2 \}$. 
 \par 
 Para el caso $\lambda_1 = 1$, se tiene que $(A - I)X = 0$ es equivalente al sistema de ecuaciones:
 \begin{equation*}
  \begin{matrix}
   2x_1 & +3x_2 & + x_4 & = 0 \\
   -x_1 & -2x_2 & - x_4 & = 0 \\
    x_1 & +2x_2 & + x_4 & = 0 \\
   2x_1 & +4x_2 & +2x_4 & = 0
  \end{matrix}
 \end{equation*}
 en el cual la segunda, tercera y cuarta ecuaci\'on son equivalentes, por lo que se reduce a un sistema de dos ecuaciones el cual se resuelve como sigue:
 \begin{equation*}
  \left.
  \begin{matrix}
   2x_1 & +3x_2 & + x_4 & = 0 \\
   -x_1 & -2x_2 & - x_4 & = 0
  \end{matrix}
  \right\}
  \Leftrightarrow 
  \left. 
  \begin{matrix}
   2x_1 & +3x_2 & + x_4 & = 0 \\
   -x_1 & -2x_2 & - x_4 & = 0 \\
   x_1  & + x_2 &       & = 0
  \end{matrix}
  \right\}
  \Leftrightarrow 
  \left. 
  \begin{matrix}
   x_1 &       & = - x_2 \\
   x_2 & + x_4 & = 0 
  \end{matrix}
  \right\}
  \Leftrightarrow
  \left. 
  \begin{matrix}
   x_1 = \alpha \\
   x_2 = -\alpha \\
   x_3 = \beta \\
   x_4 = \alpha
  \end{matrix}
  \right\}
 \end{equation*}
 Por lo tanto, $V_{\lambda_1} = \left\{ \overline{x} \in V |\, \overline{x} = \alpha\overline{e}_1 - \alpha\overline{e}_2 + \beta\overline{e}_3 + \alpha\overline{e}_4, \text{ con } \alpha, \beta \in \mathbb{R} \right\}$.
 \par
 Para el caso $\lambda_2 = 2$, se tiene que $(A-2I)X = 0$ es equivalente al sistema de ecuaciones:
 \begin{equation*}
  \begin{matrix}
    x_1 & +3x_2 &      & +x_4 & = 0 \\
   -x_1 & -3x_2 &      & -x_4 & = 0 \\
    x_1 & +2x_2 & -x_3 & +x_4 & = 0 \\
   2x_1 & +4x_2 &      & +x_4 & = 0
  \end{matrix}
 \end{equation*}
 en el cual la segunda y tercera ecuaci\'on son equivalentes, por lo que se reduce al sistema de tres ecuaciones el cual se resuelve como sigue:
 \begin{equation*}
  \left.
  \begin{matrix}
    x_1 & +3x_2 &      & +x_4 & = 0 \\
    x_1 & +2x_2 & -x_3 & +x_4 & = 0 \\
   2x_1 & +4x_2 &      & +x_4 & = 0
  \end{matrix}
  \right\}
  \Leftrightarrow 
  \left. 
  \begin{matrix}
    x_1 & +3x_2 &      & +x_4 & = 0 \\
    x_1 & +2x_2 & -x_3 & +x_4 & = 0 \\
   2x_1 & +4x_2 &      & +x_4 & = 0 \\
    x_1 & + x_2 &      &      & = 0
  \end{matrix}
  \right\}
  \Leftrightarrow 
  \left. 
  \begin{matrix}
    x_1 &      &      & = -x_2 \\
   2x_2 &      & +x_4 & = 0 \\
    x_2 & -x_3 & +x_4 & = 0
  \end{matrix}
  \right\}
 \end{equation*}
 \begin{equation*}
  \Leftrightarrow 
  \left. 
  \begin{matrix}
    x_1 &      & = - x_2 \\
    x_4 &      & = -2x_2 \\
   -x_2 & -x_3 & = 0     
  \end{matrix}
  \right\}
  \Leftrightarrow 
  \left. 
  \begin{matrix}
   x_1 & = & - \alpha \\ 
   x_2 & = &   \alpha \\
   x_3 & = & - \alpha \\
   x_4 & = & -2\alpha 
  \end{matrix}
  \right\}
 \end{equation*}
 Por lo tanto, $V_{\lambda_2} = \left\{ \overline{x} \in V | \, \overline{x} = -\alpha\overline{e}_1 + \alpha\overline{e}_2 - \alpha\overline{e}_3 - 2\alpha\overline{e}_4, \text{ con } \alpha\in\mathbb{R} \right\}$.
 \par 
 Finalmente, aunque las multiplicidades algebraicas sumen la dimensi\'on del espacio, es decir $m_1 + m_2 = 2 + 2 = 4 = \dim V$, no ocurre que las dimensiones geom\'etricas sean iguales a sus correspondientes multiplicidades algebraicas, esto es $d_1 = 2 = m_1$, pero $d_2 = 1 \neq 2 = m_1$. 
 Por lo tanto, por teorema, se sigue que $f$ no es diagonalizable, que es a lo que se quer\'{\i}a llegar.${}_{\blacksquare}$
\end{solucion}

