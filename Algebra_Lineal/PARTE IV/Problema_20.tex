\begin{enunciado}
 En el espacio vectorial eucl\'{\i}deo can\'onico $\mathbb{R}^3$ y respecto de una base ortonormal $(\bar{e}_1, \bar{e}_2, \bar{e}_3)$, se considera un endomorfismo del que se sabe que:
 \begin{itemize}
  \item $f(\bar{e}_1) = 3\bar{e}_1 + 2\bar{e}_2 + 2\bar{e}_3$, $f(\bar{e}_2) = 2\bar{e}_1 + 2\bar{e}_2$.
  
  \item La matriz $A$ de $f$ es sim\'etrica.
  
  \item $\bar{a} = 2\bar{e}_1 - 2\bar{e}_2 - \bar{e}_3$ es autovector de $f$.
 \end{itemize}
 \begin{enumerate}[$a$)]
  \item Hallar $A$ y los autovalores y los autovectores de $f$.
  
  \item Diagonalizar ortogonalmente $f$, determinando una base en la que se obtenga dicha ortogonalizaci\'on.
 \end{enumerate}
\end{enunciado}
 
\begin{solucion}
 Antes de comenzar, se analizar\'an uno a uno los puntos del enunciado.
 \par 
 Del primer punto y dado que la $i-$\'esima columna de $A=[a_{ij}]_{3\times 3}$ est\'a conformada por el vector de coordenadas de $f(e_i)$, en la base $\beta = (\bar{e}_1, \bar{e}_2, \bar{e}_3)$, se tiene que:
 \begin{equation*}
  A =
  \begin{bmatrix}
   3 & 2 & a_{13} \\
   2 & 2 & a_{23} \\
   2 & 0 & a_{33}
  \end{bmatrix}
 \end{equation*}
 Luego, por el segundo punto, $A$ es sim\'etrica y, por lo tanto $a_{13} = a_{31} = 2$ y $a_{23} = a_{32} = 0$.
 \par 
 Y, del tercer punto, llamando $X$ el vector de coordenadas de $\bar{a}$ en la base $\beta$, se tiene que, como $\bar{a}$ es autovector, existe $\lambda$ tal que:
 \begin{equation*}
  \begin{bmatrix}
   3 & 2 & 2 \\
   2 & 2 & 0 \\
   2 & 0 & a_{33}
  \end{bmatrix}
  \begin{bmatrix}
   2 \\ -2 \\ -1
  \end{bmatrix}
  = 
  \begin{bmatrix}
   0 \\ 0 \\ 4 - a_{33}
  \end{bmatrix}
  = 
  \begin{bmatrix}
   2\lambda \\ -2\lambda \\ -\lambda
  \end{bmatrix}
 \end{equation*}
 luego entonces $\bar{a}$ es autovector de $f$ correspondiente al autovalor $\lambda_1 = 0$, y, como $4-a_{33} = 0$, se tiene que $a_{33} = 4$.
 \begin{enumerate}[$a$)]
  \item De todo lo anterior se tiene por resultado inmediato que la matriz $A$ de $f$ en la base $\beta$ es:
  \begin{equation*}
   \begin{bmatrix}
    3 & 2 & 2 \\
    2 & 2 & 0 \\
    2 & 0 & 4
   \end{bmatrix}
  \end{equation*}
  Luego, como $0$ es un autovalor de $A$, se tiene que $A$ es singular y por lo tanto $\det(A) = 0$, adem\'as, $\text{tr} A = 9$, y calculando la suma de los 3 menores diagonales de orden $2$ de $A$, esto es:
  \begin{equation*}
   \begin{vmatrix}
    3 & 2 \\
    2 & 2 
   \end{vmatrix}
   +
   \begin{vmatrix}
    3 & 2 \\
    2 & 4 
   \end{vmatrix}
   + 
   \begin{vmatrix}
    2 & 0 \\
    0 & 4
   \end{vmatrix}
   = (6-4) + (12 - 4) + (8) = 2+8+8 = 18
  \end{equation*}
  se tiene entonces que el polinomio caracter\'{\i}stico de $f$ es: $-\lambda^3 + 9\lambda - 18\lambda = -\lambda(\lambda^2 - 9\lambda +18) = -\lambda(\lambda -3)(\lambda - 6)$. Por lo tanto, los autovalores de $f$, que son las ra\'{\i}ces de dicho polinomio, son: $\lambda_1 = 0$, $\lambda_2 = 3$ y $\lambda_3 = 6$.
  \par 
  Para calcular los autovectores de $f$, se procede a calcular sus vectores de coordenadas en la base $\beta$, los cuales son aquellos que cumplen que $(A-\lambda_i I) X = O$, para cada $i\in \{ 1, 2, 3\}$. N\'otese que como las multiplicidades geom\'etricas, $d_i$, y las multiplicidades algebraicas, $m_i$, cumplen que $1\leq d_i \leq m_i = 1$, entonces $\dim E_{\lambda_i} = 1$, para cada $i\in\{1,2,3\}$, o lo que es equivalente, cada $E_{\lambda_i}$ est\'a generado por un \'unico vector de $\mathbb{R}^3$.
  \par 
  Para el caso $\lambda_1 = 0$, ya se sabe que $\bar{a}$ es autovector de $f$. Por lo tanto:
  \begin{equation*}
   E_{\lambda_1 = 0} = \left< 2\bar{e}_1 - 2\bar{e}_2 - \bar{e}_3 \right>
  \end{equation*}
  Para el caso $\lambda_2 = 3$, se resuelve la ecuaci\'on $\det(A-3I)X = O$, para el vector de coordenadas $X=[x,y,z]^t$, como sigue:
  \begin{equation*}
   \begin{bmatrix}
    0 &  2 & 2 \\
    2 & -1 & 0 \\
    2 &  0 & 1
   \end{bmatrix}
   \begin{bmatrix}
    x \\ y \\ z
   \end{bmatrix}
   =
   \begin{bmatrix}
       &   & 2y & + & 2z \\
    2x & - &  y          \\
    2x &   &    & + & z
   \end{bmatrix}
   =
   \begin{bmatrix}
    0 \\ 0 \\ 0
   \end{bmatrix}
  \end{equation*}
  Por lo tanto, como $2x-y=0$ y $y+z = 0$, se tiene que $y = 2x = -z$, por lo que $X=[1, 2,-2]$ es el vector de coordenadas de un autovector de $E_{\lambda_2 = 3}$. Por lo tanto:
  \begin{equation*}
   E_{\lambda_2 = 3} = \left< \bar{e}_1 + 2\bar{e}_2 - 2\bar{e}_3 \right>
  \end{equation*}
  Y, para el caso $\lambda_3 = 6$, se resuelve la ecuaci\'on $\det(A - 6I)X = O$, para el vector de coordenadas $X = [x,y,z]^t$, como sigue:
  \begin{equation*}
   \begin{bmatrix}
    -3 &  2 &  2 \\
     2 & -4 &  0 \\
     2 &  0 & -2
   \end{bmatrix}
   \begin{bmatrix}
    x \\ y \\ z
   \end{bmatrix}
   = 
   \begin{bmatrix}
    -3x & + & 2y & + & 2z \\
     2x & - & 4y          \\
     2x &   &    & - & 2z
   \end{bmatrix}
   =
   \begin{bmatrix}
    0 \\ 0 \\ '
   \end{bmatrix}
  \end{equation*}
  Por lo tanto, como $2x - 4y = 0$ y $2x - 2z = 0$, se tiene que $x = 2y = z$, por lo que $X = [2, 1, 2]$ es el vector de coordenadas de un autovector de $E_{\lambda_3 = 6}$. Por lo tanto:
  \begin{equation*}
   E_{\lambda_3 = 6} = \left< 2\bar{e}_1 + \bar{e}_2 + 2\bar{e}_3 \right>
  \end{equation*}
  
  \item Finalmente, usando todo lo mencionado anteriormente, se tiene que los vectores que se dieron que generan los subespacios propios de $f$ son ortogonales, puesto que $\left( 2\bar{e}_1 - 2\bar{e}_2 - \bar{e}_3 \right) \cdot \left( \bar{e}_1 + 2\bar{e}_2 - 2\bar{e}_3 \right) = 2-4+2 = 0$, $\left( 2\bar{e}_1 - 2\bar{e}_2 - \bar{e}_3 \right) \cdot \left( 2\bar{e}_1 + \bar{e}_2 +2\bar{e}_3 \right) = 4 - 2 - 2 = 0$ y $\left( \bar{e}_1 + 2\bar{e}_2 - 2\bar{e}_3 \right) \cdot \left( 2\bar{e}_1 + \bar{e}_2 +2\bar{e}_3 \right) = 2 + 2 - 4 = 0$, entonces al normalizar estos vectores, esto es, tomando $\bar{u}_1 = \frac{2\bar{e}_1 - 2\bar{e}_2 - \bar{e}_3}{\left\lVert 2\bar{e}_1 - 2\bar{e}_2 - \bar{e}_3 \right\rVert} = \frac{2\bar{e}_1 - 2\bar{e}_2 - \bar{e}_3}{\sqrt{4+4+1}} = \frac{1}{3}\left(2\bar{e}_1 - 2\bar{e}_2 - \bar{e}_3 \right)$,
  $\bar{u}_2 = \frac{\bar{e}_1 + 2\bar{e}_2 - 2\bar{e}_3}{\left\lVert \bar{e}_1 + 2\bar{e}_2 - 2\bar{e}_3 \right\rVert} = \frac{\bar{e}_1 + 2\bar{e}_2 - 2\bar{e}_3}{\sqrt{1+4+4}} = \frac{1}{3}\left(\bar{e}_1 + 2\bar{e}_2 - 2\bar{e}_3 \right)$
  y $\bar{u}_3 = \frac{2\bar{e}_1 + \bar{e}_2 +2\bar{e}_3}{\left\lVert 2\bar{e}_1 + \bar{e}_2 +2\bar{e}_3 \right\rVert} = \frac{2\bar{e}_1 + \bar{e}_2 +2\bar{e}_3}{\sqrt{4 + 1+ 4}} = \frac{1}{3} \left( 2\bar{e}_1 + \bar{e}_2 +2\bar{e}_3 \right)$, se tiene una base ortonormal $\beta' = \left( \bar{u}_1, \bar{u}_2, \bar{u}_3 \right)$ en donde $f$ es diagonal. La matriz diagonal, $D$, de $f$ en esta base y la matriz de cambio de coordenadas, $P$, son las siguientes y adem\'as cumplen que $D = P^tAP$.
  \begin{equation*}
   D = 
   \begin{bmatrix}
    0 & 0 & 0 \\
    0 & 3 & 0 \\
    0 & 0 & 6
   \end{bmatrix}
   \qquad \text{ y } \qquad
   P = \frac{1}{3}
   \begin{bmatrix}
     2 &  1 & 2 \\
    -2 &  2 & 1 \\
    -1 & -2 & 2
   \end{bmatrix}
  \end{equation*}
  que es a lo que se quer\'{\i}a llegar.${}_{\blacksquare}$
 \end{enumerate}
\end{solucion}
