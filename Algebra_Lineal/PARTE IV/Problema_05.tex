\begin{enunciado}
 Hallar los autovalores y los subespacios propios del endomorfismo $f:\mathbb{R}^3 \rightarrow \mathbb{R}^3$ que, en la base can\'onica, tiene asociada la matriz $A$:
 \begin{equation*}
  A = 
  \begin{bmatrix}
   1 &  1 & 0 \\
   3 & -1 & 6 \\
   1 & -1 & 3
  \end{bmatrix}
 \end{equation*}
 Anal\'{\i}cese si $f$ es diagonalizable.
\end{enunciado}

\begin{solucion}
 Calculando $\det (A - \lambda I)$, se tiene que:
 \begin{eqnarray*}
  \det (A - \lambda I) & = & 
  \begin{vmatrix}
   1-\lambda &  1         & 0 \\
   3         & -1-\lambda & 6 \\
   1         & -1         & 3-\lambda
  \end{vmatrix} \\ 
  & = & (1-\lambda)
  \begin{vmatrix}
   -1-\lambda & 6         \\
   -1         & 3-\lambda
  \end{vmatrix}
  - 3
  \begin{vmatrix}
    1 & 0         \\
   -1 & 3-\lambda
  \end{vmatrix}
  + 
  \begin{vmatrix}
    1         & 0 \\
   -1-\lambda & 6
  \end{vmatrix} \\ 
  & = & (1-\lambda)\left[ (-1-\lambda)(3-\lambda) + 6 \right] - 3(3-\lambda) + (6) \\ 
  & = & (1-\lambda)(-3-2\lambda + \lambda^2 + 6) - 9 + 3\lambda + 6 = (1-\lambda)(\lambda^2 - 2\lambda + 3) + 3\lambda - 3 \\
  & = & (-\lambda^3 + 3\lambda^2 -5\lambda + 3) + 3\lambda - 3 = -\lambda^3 +3\lambda^2 -2\lambda \\
  & = & -\lambda(\lambda^2 - 3\lambda + 2) = -\lambda(\lambda - 2)(\lambda - 1) 
 \end{eqnarray*}
 Por lo tanto, $\det(A-\lambda I) = 0$ si y s\'olo si $\lambda \in \{ 0, 1, 2 \}$. Luego, como los autovalores de $A$ son los mismo que los de $f$, sin importar la base, se tiene que $\lambda_0 = 0$, $\lambda_1 = 1$ y $\lambda_2 = 2$ son los autovalores de $f$.
 \par 
 Para hallar los subespacios propios de $f$, basta con encontrar los vectores $\overline{x}$ que cuyo vector de coordenadas en la base can\'onica, $X = (x, y, z)^t$, cumplan que $AX = \lambda_iX$ para cada $i \in \{ 0, 1, 2\}$.
 \par 
 Para el caso $\lambda = 0$, se tiene que
 \begin{equation*}
  AX = O \Leftrightarrow 
  \left.
  \begin{matrix}
    x & +y &     & = 0 \\
   3x & -y & +6z & = 0 \\
  \end{matrix}
  \right\} \Leftrightarrow
  \left.
  \begin{matrix}
   x  & = -y \\
   4x & +6z = 0
  \end{matrix}
  \right\} \Leftrightarrow
  \left.
  \begin{matrix}
   x  & = -y \\
   x & = -\frac{3}{2}z
  \end{matrix}
  \right\} \Leftrightarrow
  \left.
  \begin{matrix}
   x & = -3\alpha \\
   y & =  3\alpha \\
   z & =  2\alpha
  \end{matrix}
  \right\}
 \end{equation*}
 Por lo tanto $V_{\lambda_0} = \{ (-3\alpha, 3\alpha, 2\alpha) \in \mathbb{R}^3| \alpha \in \mathbb{R} \}$. \par 
 Para el caso $\lambda = 1$, se tiene que:
 \begin{equation*}
  AX = X \Leftrightarrow 
  \left.
  \begin{matrix}
    x & +y &     & = x \\
   3x & -y & +6z & = y \\
  \end{matrix}
  \right\} \Leftrightarrow
  \left.
  \begin{matrix}
   y  & = 0 \\
   3x & +6z = 0
  \end{matrix}
  \right\} \Leftrightarrow
  \left.
  \begin{matrix}
   y & = 0 \\
   x & = -2z
  \end{matrix}
  \right\} \Leftrightarrow
  \left.
  \begin{matrix}
   x & = -2\alpha \\
   y & = 0 \\
   z & = \alpha
  \end{matrix}
  \right\}
 \end{equation*}
 Por lo tanto $V_{\lambda_1} = \{ (-2\alpha, 0, \alpha) \in \mathbb{R}^3 | \alpha \in \mathbb{R} \}$.
 \par 
 Y, para el caso $\lambda = 2$, se tiene que: 
 \begin{equation*}
  AX = 2X \Leftrightarrow 
  \left.
  \begin{matrix}
    x & +y &     & = 2x \\
   3x & -y & +6z & = 2y \\
  \end{matrix}
  \right\} \Leftrightarrow
  \left.
  \begin{matrix}
   y  & = x \\
   2x & +6z = 2x
  \end{matrix}
  \right\} \Leftrightarrow
  \left.
  \begin{matrix}
   x & = y \\
   z & = 0
  \end{matrix}
  \right\} \Leftrightarrow
  \left.
  \begin{matrix}
   x & = \alpha \\
   y & = \alpha \\
   z & = 0
  \end{matrix}
  \right\}
 \end{equation*}
 Por lo tanto, $V_{\lambda_2} = \{(\alpha, \alpha, 0) \in \mathbb{R} | \alpha \in \mathbb{R} \}$.
 \par 
 Finalmente, como se tiene que la dimensi\'on del espacio es $3$ y $f$ tiene $3$ autovalores reales distintos, se concluye que $f$ es diagonalizable; a saber, la matriz diagonal $D = P^{-1}AP$ de $f$ se obtiene respecto a la nueva base $\left( (-3, 3, 2), (2, 0, -1), (1, 1, 0) \right)$ donde $D$ y $P$ son:
 \begin{equation*}
  D = 
  \begin{pmatrix}
   0 & 0 & 0 \\
   0 & 1 & 0 \\
   0 & 0 & 2
  \end{pmatrix}
  \qquad \text{y} \qquad 
  P = 
  \begin{pmatrix}
   -3 &  2 & 1 \\
    3 &  0 & 1 \\
    2 & -1 & 0
  \end{pmatrix}
 \end{equation*}
 que es a lo que se quer\'{\i}a llegar.${}_{\blacksquare}$ 
\end{solucion}

