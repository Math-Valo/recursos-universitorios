\begin{enunciado}
 En el espacio vectorial eucl\'{\i}deo can\'onico $\mathbb{R}^3$ se considera un endomorfismo $f:\mathbb{R}^3 \to \mathbb{R}^3$ del que se sabe que:
 \begin{itemize}
  \item La matriz $A$ de $f$ en la base can\'onica es sim\'etrica.
  
  \item El subespacio $\left< (2, -2, -1) \right>$ es un subespacio propio de $f$.
  
  \item Los vectores $(1,0,0)$ y $(0,1,0)$ se transforman, respectivamente, en los vectores $(3,2,2)$ y $(2,2,0)$.
 \end{itemize}
 \begin{enumerate}[$a$)]
  \item Razonar si $f$ es ortogonalmente diagonalizable y si es una transformaci\'on ortogonal.
  
  \item Hallar los autovalores y los subespacios propios de $f$.
  
  \item Hallar la forma diagonal de $f$ y obtener, si es posible, una base ortonormal de $\mathbb{R}^3$ en la que $f$ tome dicha forma.
 \end{enumerate}
\end{enunciado}
 
\begin{solucion}
 Antes de comenzar, se analizar\'an los puntos en el enunciado para hallar $A=[a_{ij}]_{3\times 3}$, la matriz de $f$ en la base can\'onica.
 \par 
 Del tercer punto, como se sabe que los vectores coordenadas, en la base de $A$, de los vectores abstractos a los que $f$ transforma los elementos de esta base son las columnas de $A$ (correspondiendo cada columna a la posici\'on en el que el elemento de la base aparece en la base), se tiene que de $[3,2,2]^t$ y $[2,2,0]^t$ corresponden a la primera y segunda columna, respectivamente, de $A$.
 \par 
 Del primer punto, como $A$ es sim\'etrica, se cumple que $a_{13} = a_{31} = 2$ y $a_{23} = a_{32} = 0$.
 \par 
 Finalmente, del segundo punto como se sabe que existe un escalar, $\lambda$, tal que $A[2,-2,-1]^t = \lambda[2,-2,-1]^t$, se procede a encontrar $\lambda$ y, con ello, al valor $a_{33}$ que falta. Esto es:
 \begin{equation*}
  \begin{bmatrix}
   3 & 2 & 2 \\
   2 & 2 & 0 \\
   2 & 0 & a_{33}
  \end{bmatrix}
  \begin{bmatrix}
   2 \\ -2 \\ -1
  \end{bmatrix}
  = 
  \begin{bmatrix}
   0 \\ 0 \\ 4 - a_{33}
  \end{bmatrix}
  = 
  \begin{bmatrix}
   \lambda x \\ \lambda y \\ \lambda z
  \end{bmatrix}
 \end{equation*}
 se tiene entonces que $\lambda = 0$ y, por lo tanto, $a_{33} = 4$.
 Por lo tanto
 \begin{equation*}
  A = 
  \begin{bmatrix}
   3 & 2 & 2 \\
   2 & 2 & 0 \\
   2 & 0 & 4
  \end{bmatrix}
 \end{equation*}

 \begin{enumerate}[$a$)]
  \item Como la matriz $A$ de $f$, en la base can\'onica que es ortonormal, es sim\'etrica, entonces $f$ es sim\'etrico, entonces $f$ es ortogonalmente diagonalizable. Como $A$ no es ortogonal, puesto que sus filas no son ortonormales, lo cual se puede verificar viendo que el producto punto de la segunda con la tercera columna da $4$ y no $0$, entonces $f$ no es una transformaci\'on ortogonal.
  
  \item Para hallar los autovalores de $f$ se calcular\'a el polinomio caracter\'{\i}stico de $f$ a trav\'es el c\'alculo de $\det(A - \lambda I)$, como sigue:
  \begin{eqnarray*}
   \det(A - \lambda I) & = & 
   \begin{vmatrix}
    3 - \lambda & 2 & 2 \\
    2 & 2 - \lambda & 0 \\
    2 & 0 & 4 - \lambda
   \end{vmatrix}
   \\
   & = & 
   3
   \begin{vmatrix}
    2 - \lambda & 0 \\
    0 & 4 - \lambda 
   \end{vmatrix}
   -2
   \begin{vmatrix}
    2 & 2 \\
    0 & 4 - \lambda 
   \end{vmatrix}
   +2
   \begin{vmatrix}
    2 & 2 \\
    2 - \lambda & 0
   \end{vmatrix}
   \\
   & = & 
   (3-\lambda)(2-\lambda)(4-\lambda) -2(8 - 2\lambda) +2(-4 + 2\lambda) \\
   & = & 
   (3-\lambda)(\lambda^2 - 6\lambda + 8) - 16 + 4\lambda -8 + 4\lambda \\
   & = & 
   -\lambda^3 +9\lambda^2 - 26\lambda + 24 - 24 + 8\lambda \\
   & = & 
   -\lambda^3 + 9\lambda^2 - 18\lambda \\
   & = & -\lambda(\lambda^2 - 9\lambda + 18) \\
   & = & -\lambda(\lambda - 3)(\lambda - 6)
  \end{eqnarray*}
  Por lo tanto, los autovalores de $f$, que son las ra\'{\i}ces de su polinomio caracter\'{\i}stico, son $\lambda_1 = 0$, $\lambda_2 = 3$ y $\lambda_3 = 6$.
  \par 
  Del enunciado junto con la primera parte de esta soluci\'on ya se sabe que el subespacio propio de $\lambda_1 = 0$ es:
  \begin{equation*}
   E_{\lambda_1 = 0} = \left< (2, -2, -1) \right>
  \end{equation*}
  Y para el c\'alculo del resto de los subespacios propios, se buscaran los vectores coordenadas $X = [x,y,z]^t$ tales que $(A - \lambda I)X = O$, para $i \in \{2,3 \}$.
  \par 
  Para el caso $\lambda_2 = 3$, se tiene que $(A-3\lambda)X = O$ se representa como
  \begin{equation*}
   \begin{bmatrix}
    0 &  2 & 2 \\
    2 & -1 & 0 \\
    2 &  0 & 1
   \end{bmatrix}
   \begin{bmatrix}
    x \\ y \\ z
   \end{bmatrix}
   =
   \begin{bmatrix}
       &   & 2y & + & 2z \\
    2x & - &  y &   &    \\
    2x &   &    & + & z
   \end{bmatrix}
   = 
   \begin{bmatrix}
    0 \\ 0 \\ 0
   \end{bmatrix}
  \end{equation*}
  Por lo que $2x-y = 0$ y $2x + z = 0$, entonces $2x = y = -z$, y, por lo tanto, el subespacio propio es correspondiente es
  \begin{equation*}
   E_{\lambda_2 = 3} \left< (1, 2, -2) \right>
  \end{equation*}
  Finalmente, para el caso $\lambda_3 = 6$, se tiene que $(A-6\lambda)X = O$ se representa como
  \begin{equation*}
   \begin{bmatrix}
    -3 &  2 &  2 \\
     2 & -4 &  0 \\
     2 &  0 & -2
   \end{bmatrix}
   \begin{bmatrix}
    x \\ y \\ z
   \end{bmatrix}
   =
   \begin{bmatrix}
    -3x & + & 2y & + & 2z \\
     2x & - & 4y &   &    \\
     2x &   &    & - & 2z
   \end{bmatrix}
   = 
   \begin{bmatrix}
    0 \\ 0 \\ 0
   \end{bmatrix}
  \end{equation*}
  Por lo que $2x - 4y = 0$ y $2x - 2z = 0$, entonces $2y = x = z$ y, por lo tanto, el subespacio propio correspondiente es
  \begin{equation*}
   E_{\lambda_3  = 6} = \left< (2, 1, 2) \right>
  \end{equation*}
  
  \item Finalmente, como se mencion\'o al principio, $f$ es ortogonalmente diagonalizable, por lo que sus subespacios propios son ortogonales y sus vectores generadores, por lo tanto, tambi\'en son ortogonales. Por lo tanto, tomando como base la normalizaci\'n de estos vectores, es decir, es decir $\beta = \left( \bar{u}_1 = \frac{1}{3}(2,-2,-1), \bar{u}_2 = \frac{1}{3}(1,2,-2), \bar{u}_3 = \frac{1}{3}(2,1,2) \right)$, se tiene una matriz diagonal, $D$, formada en su diagonal por los autovalores a los que corresponden los autovectores de la base, en el mismo orden, esto es:
  \begin{equation*}
   D = 
   \begin{bmatrix}
    0 & 0 & 0 \\
    0 & 3 & 0 \\
    0 & 0 & 6
   \end{bmatrix}
  \end{equation*}
  que es a lo que se quer\'{\i}a llegar.${}_{\blacksquare}$
 \end{enumerate}
\end{solucion}
