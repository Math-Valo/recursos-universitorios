\begin{enunciado}
 De la siguiente matriz $A$ se sabe que $\lambda_1 = 1$ es uno de sus autovalores y que $(1,1,1)$ es un vector prpopio de $A$ asociado al autovalor $\lambda_1$:
 \begin{equation*}
  A = 
  \begin{bmatrix}
   1 & 2 & \alpha \\
   2 & 1 & \beta  \\
   2 & 2 & \gamma
  \end{bmatrix}
 \end{equation*}
 \begin{enumerate}[$a$)]
  \item Hallar $\alpha$, $\beta$ y $\gamma$.
  
  \item Hallar los autovalores y los subespacios propios de $A$.
  
  \item Comprobar que $A$ es diagonalizable y hallar su matriz diagonal $D$.
  
  \item Analizar si $A$ es ortogonalmente diagonalizable y, en caso afirmativo, hallar una matriz ortogonal $P$ tal que $D = P^{-1}AP$
 \end{enumerate}
\end{enunciado}
 
\begin{solucion}
 $\phantom{0}$
 \begin{enumerate}[$a$)]
  \item Sea $X_1 = (1,1,1)^t$, como $AX=X$, entonces se procede a multiplicar como sigue:
  \begin{equation*}
   \begin{bmatrix}
    1 & 2 & \alpha \\
    2 & 1 & \beta  \\
    2 & 2 & \gamma 
   \end{bmatrix}
   \begin{bmatrix}
    1 \\ 1 \\ 1
   \end{bmatrix}
   =
   \begin{bmatrix}
    1 + 2 + \alpha \\
    2 + 1 + \beta  \\
    2 + 2 + \gamma
   \end{bmatrix}
   =
   \begin{bmatrix}
    1 \\ 1 \\ 1
   \end{bmatrix}
  \end{equation*}
  Entonces $1+2 + \alpha = 1$, $2+1+\beta = 1$ y $2 + 2 + \gamma = 1$, por lo que $\alpha = \beta = -2$ y $\gamma = -3$.
  
  \item Se procede a calcular el polinomio caracter\'{\i}stico de $A$ calculando $\det(A - \lambda I)$. Esto es:
  \begin{eqnarray*}
   \det(A - \lambda I) & = &
   \begin{vmatrix}
    1-\lambda & 2 & -2 \\
    2 & 1-\lambda & -2 \\
    2 & 2 & -3-\lambda
   \end{vmatrix} \\
   & = & 
   (1-\lambda)
   \begin{vmatrix}
    1-\lambda & -2 \\
    2 & -3-\lambda 
   \end{vmatrix}
   -2
   \begin{vmatrix}
    2 & -2 \\
    2 & -3-\lambda
   \end{vmatrix}
   +2
   \begin{vmatrix}
    2 & -2 \\
    1-\lambda & -2
   \end{vmatrix} \\
   & = & (1-\lambda)\left[ (1-\lambda)(-3-\lambda) + 4 \right] -2(-6-2\lambda + 4) +2(-4 + 2 -2\lambda) \\
   & = & (1-\lambda)(-3 + 2\lambda + \lambda^2 + 4) - \cancel{2(-2\lambda - 2)} + \cancel{2(-2\lambda - 2)} \\ 
   & = & -(\lambda-1)(\lambda^2 + 2\lambda + 1) \\ 
   & = & -(\lambda-1)(\lambda+1)^2
  \end{eqnarray*}
  Por lo tanto, los valores propios de $A$, que son las ra\'{\i}ces de su polinomio caracter\'{\i}stico, son $\lambda_1 = 1$, con multiplicidad algebraica $m_1 = 1$, y $\lambda_2 = -1$, con multiplicidad algebraica $m_2 = 2$.
  \par 
  Para el c\'alculo de los subespacios propios de $A$, se calcular\'a, caso por caso, el sistema de ecuaciones que generan $(A-\lambda_i I)X = O$, para $i \in \{ 1,2 \}$, usando el vector gen\'ericos $X = [x,y,z]^t$, como se muestra a continuaci\'on.
  \par 
  Para el caso $\lambda_1 = 1$, se tiene que 
  \begin{equation*}
   \begin{bmatrix}
    0 & 2 & -2 \\
    2 & 0 & -2 \\
    2 & 2 & -4
   \end{bmatrix}
   \begin{bmatrix}
    x \\ y \\ z
   \end{bmatrix}
   =
   \begin{bmatrix}
    & & 2y & - & 2z \\
    2x & & & - & 2z \\
    2x & + & 2y & - & 4z
   \end{bmatrix}
   =
   \begin{bmatrix}
    0 \\ 0 \\ 0
   \end{bmatrix}
  \end{equation*}
  Con lo que resulta que $2y - 2z = 0$ y $2x - 2z = 0$, entonces $x = y = z$. Por lo tanto:
  \begin{equation*}
   E_{\lambda_1 = 1} = \left< (1,1,1) \right>
  \end{equation*}
  Para el caso $\lambda_2 = -1$, se tiene que
  \begin{equation*}
   \begin{bmatrix}
    2 & 2 & -2 \\ 
    2 & 2 & -2 \\
    2 & 2 & -2
   \end{bmatrix}
   \begin{bmatrix}
    x \\ y \\ z
   \end{bmatrix}
   = 
   \begin{bmatrix}
    2x + 2y - 2z \\
    2x + 2y - 2z \\
    2x + 2y - 2z    
   \end{bmatrix}
   =
   \begin{bmatrix}
    0 \\ 0 \\ 0
   \end{bmatrix}
  \end{equation*}
  Por lo que $2x + 2y - 2z = 0$ y $E_{\lambda_2}$ lo conforma todo vector $(x,y,z)$ tal que $x+y-z=0$, es decir:
  \begin{equation*}
   E_{\lambda_2 = -1} = \left\{ \left. (x,y,z) \in \mathbb{R}^3 \right| x+y+z = 0 \right\} = \left< (1,0,1), (1,-2,-1) \right>
  \end{equation*}
  
  \item Dado que $m_1+m_2 = 3 = \dim A$ y las multiplicidades geom\'etricas, $d_i$, cumplen que $d_1 = m_1 = 1$ y $d_2 = m_2 = 2$, entonces, por teorema $A$ es diagonalizable a una matriz diagonal $D$, y una matriz $P$ tal que $D = P^{-1}AP$ se conforma por los vectores que generan los subespacios propios de $A$, esto es, $D$ y $P$ son:
  \begin{equation*}
   D = 
   \begin{bmatrix}
    1 &  0 &  0 \\
    0 & -1 &  0 \\
    0 &  0 & -1
   \end{bmatrix}
   \qquad \text{ y } \qquad 
   P = 
   \begin{bmatrix}
    1 & 1 &  1 \\
    1 & 0 & -2 \\
    1 & 1 & -1
   \end{bmatrix}
  \end{equation*}
  
  \item Finalmente, como se vio anteriormente, $A = [a_{ij}]_{3\times 3}$ es ortogonalmente diagonalizable si y s\'olo si es sim\'etrica pero $a_{13}\neq a_{31}$ ya que $a_{13} = -2$ y $a_{31} = 2$. Por lo tanto $A$ no es ortogonalmente diagonalizable, que es a lo que se quer\'{\i}a llegar.${}_{\blacksquare}$
 \end{enumerate}
\end{solucion}
