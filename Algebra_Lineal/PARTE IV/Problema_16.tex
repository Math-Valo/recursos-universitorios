\begin{enunciado}
 Sea $A$ la matriz real, dependiente del par\'ametro $\alpha$:
 \begin{equation*}
  A = 
  \begin{bmatrix}
   2\alpha+4 & 1-\alpha & -2\alpha-\alpha^2 \\
   0 & 4 - \alpha & 0 \\
   0 & 0 & 4 - \alpha^2
  \end{bmatrix}
 \end{equation*}
 \begin{enumerate}[$a$)]
  \item Obtener los valores de $\alpha$ para los que $A$ es diagonalizable por semejanza.
  \item Diagonalizar $A$ para $\alpha = 1$ y para $\alpha = 2$.
 \end{enumerate}

\end{enunciado}
 
\begin{solucion}
 $\phantom{0}$
 \begin{enumerate}[$a$)]
  \item Dado que el determinante de una matriz triangular es igual al producto de sus elementos en la diagonal, se tiene que 
  \begin{equation*}
   \det (A - \lambda I) = 
   \begin{vmatrix}
    2\alpha+4-\lambda & 1-\alpha & -2\alpha-\alpha^2 \\
    0 & 4 - \alpha - \lambda & 0 \\
    0 & 0 & 4 - \alpha^2 - \lambda 
   \end{vmatrix}
   = (2\alpha+4-\lambda)(4-\alpha-\lambda)(4-\alpha^2-\lambda)
  \end{equation*}
  entonces los autovalores de $\lambda$, que son obtenidos por las ra\'{\i}ces de la ecuaci\'on caracter\'{\i}stica $\det (A - \lambda I) = 0$, son $\lambda_1 = 2\alpha + 4$, $\lambda_2 =  4 - \alpha$ y $\lambda_3 = 4 - \alpha^2$.
  \par 
  Luego, se sabe que $A$ es diagonalible por semejanza si y s\'olo si las multiplicidades algebraicas de los autovalores distintos suman $n=3$, el orden de la matriz, y la multiplicidad geom\'etrica, que se sabe siempre es al menos 1, es igual a la multiplicidad algebraica, para cada autovalor distinto.
  \par 
  En particular, si los tres valores de $\lambda$ antes mencionados son distintos, entonces $A$ es diagonalizable por semejanza. Se va a buscar ahora, por lo tanto, cu\'ando es $A$ diagonalizable por semejanza si los autovalores no son todos distintos, esto es: $\lambda_1 = \lambda_2$, $\lambda_1 = \lambda_3$ o $\lambda_2 = \lambda_3$. En cada caso, se tiene que $2\alpha + 4 = 4-\alpha$, $2\alpha + 4 = 4-\alpha^2$ y $4-\alpha = 4-\alpha^2$, respectivamente. El primero caso se resuelve como: $3\alpha= 0$, entonces $\alpha = 0$; el segundo caso se resuelve como $\alpha^2 + 2\alpha = 0$, entonces $\alpha = 0$ o $\alpha = -2$; y, el tercer caso se resuelve como: $\alpha^2 - \alpha = 0$, entonces $\alpha = 0$ o $\alpha = 1$
  \par 
  Para el caso $\alpha = 0$, entonces $\lambda_1 = \lambda_2 = \lambda_3 = 4$ y se tiene que el subespacio propio relacionado $\lambda=4$ est\'a formado por los vectores $X=[x,y,z]^t\in\mathbb{R}^3$ tales que $(A- 4I)X = O$, esto es, tales que:
  \begin{equation*}
   \begin{bmatrix}
    0 & 1 & 0 \\
    0 & 0 & 0 \\
    0 & 0 & 0 
   \end{bmatrix}
   \begin{bmatrix}
    x \\ y \\ z
   \end{bmatrix}
   =
   \begin{bmatrix}
    0 \\ 0 \\ 0
   \end{bmatrix}
  \end{equation*}
  Entonces, cuando $\alpha=0$,
  \begin{equation*}
   E_{\lambda=4} = \left\{ [x,y,z]^t\in\mathbb{R}^3 |\, [x,y,z]^t = [a, 0, b]^t, \text{ con } a,b\in\mathbb{R} \right\} = \left< (1,0,0), (0,0,1) \right>
  \end{equation*}
  el cual tiene dimensi\'on 2, mientras que la multiplicidad algebraica de $\lambda=4$ es 3. Por lo que si $\alpha = 0$, entonces $A$ no es diagonalizable por semejanza.
  \par 
  Para el caso $\alpha =  -2$, entonces $\lambda_1 = \lambda_3 = 0$ y $\lambda_2 = 6$ y se tiene que el subespacio propio relacionado con $\lambda = -2$ est\'a formado por los vectores $X = [x,y,z]^t \in \mathbb{R}^3$ tales que $AX = O$, esto es, tales que:
  \begin{equation*}
   \begin{bmatrix}
    0 & 3 & 0 \\
    0 & 6 & 0 \\
    0 & 0 & 0 
   \end{bmatrix}
   \begin{bmatrix}
    x \\ y \\ z
   \end{bmatrix}
   =
   \begin{bmatrix}
    0 \\ 0 \\ 0
   \end{bmatrix}
  \end{equation*}
  Entonces, cuando $\alpha=-2$,
  \begin{equation*}
   E_{\lambda= 0} = \left\{ [x,y,z]^t \in \mathbb{R}^3 |\, [x,y,z]^t = [a,0,b]^t, \text{ con } a,b\in\mathbb{R} \right\} = \left<(1,0,0), (0,0,1)\right>
  \end{equation*}
  el cual tiene dimensi\'on 2. Por lo tanto, como la multiplicidad geom\'etrica es igual a la multiplicidad algebraica tanto para $\lambda = -2$ como para $\lambda = 3$, entonces $A$ s\'{\i} es diagonalizable por semejanza.
  \par
  Finalmente, para el caso $\alpha=1$, en tal caso $\lambda_2 = \lambda_3 = 3$ y $\lambda_1 = 6$ y se tiene que el subespacio propio relacionado a $\lambda = 3$ est\'a formado por los vectores $X = [x,y,z]^t \in\mathbb{R}^3$ tales que $(A-3I)X = 0$, esto es, tales que:
  \begin{equation*}
   \begin{bmatrix}
    3 & 0 & -3 \\
    0 & 0 &  0 \\
    0 & 0 &  0
   \end{bmatrix}
   \begin{bmatrix}
    x \\ y \\ z
   \end{bmatrix}
   =
   \begin{bmatrix}
    0 \\ 0 \\ 0
   \end{bmatrix}
  \end{equation*}
  entonces $x = z$ e $y$ es variable libre, es decir, cuando $\alpha = 1$,
  \begin{equation*}
   E_{\lambda= 3} = \left\{ [x,y,z]^t \in\mathbb{R}^3 |\, [x,y,z]^t = [a,b,a]^t, \text{ con } a,b\in\mathbb{R} \right\} = \left< (1,0,1), (0,1,0) \right>
  \end{equation*}
  el cual tiene dimensi\'on 2. Por lo tanto, como la multiplicidad geom\'etrica es igual a la multiplicidad algebraica tanto para $\lambda = 3$ como para $\lambda = 6$, entonces $A$ s\'{\i} es diagonalizable por semejanza.
  \par 
  Por lo tanto $A$ es diagonalizable por semejanza si y s\'olo si $\alpha \neq 0$.
  
  \item Dado que ya se conocen los autovalores en cada caso, lo \'unico que queda por calcular es la matriz de cambio de coordenadas para obtener la matriz diagonal.
  \par 
  Para el caso $\alpha = 1$, ya se encontr\'o que los autovalores son: $\lambda_1 = 3$ y $\lambda_2 = 6$; adem\'as, se sabe que $E_{\lambda=3} = \left< (1,0,1), (0,1,0) \right>$. Nada m\'as queda encontrar un vector en $E_{\lambda=6}$ y entonces se tendr\'an tres vectores cuyas columnas formar\'an la matriz $P$ con el que se obtiene la matriz diagonal. 
  Para hallar un vector en $E_{\lambda=6}$, basta con encontrar un vector $[x,y,z]^t$ tal que $(A-6I)X = O$, esto es:
  \begin{equation*}
   \begin{bmatrix}
    0 &  0 & -3 \\ 
    0 & -3 &  0 \\
    0 &  0 & -3 
   \end{bmatrix}
   \begin{bmatrix}
    x \\ y \\ z
   \end{bmatrix}
   =
   \begin{bmatrix}
    0 \\ 0 \\ 0
   \end{bmatrix}
  \end{equation*}
  luego entonces, $z = 0$, $y= 0$ y $x$ es un par\'ametro libre, por lo que $E_{\lambda=6} = \left< (1,0,0)\right>$ y, por lo tanto, la diagonalizaci\'on de $A$ cuando $\alpha=1$ est\'a dado por $D = P^{-1}AP$ donde
  \begin{equation*}
   D =
   \begin{bmatrix}
    3 & 0 & 0 \\
    0 & 3 & 0 \\
    0 & 0 & 6
   \end{bmatrix}
   \qquad \text{ y } \qquad 
   P = 
   \begin{bmatrix}
    1 & 0 & 1 \\
    0 & 1 & 0 \\
    1 & 0 & 0
   \end{bmatrix}
  \end{equation*}
  Para el caso $\alpha = 2$, los autovalores son: $\lambda_1 = 8$, $\lambda_2 = 2$ y $\lambda_3 = 0$.
  Bastar\'a encontrar un elemento en cada espacio $E_{\lambda}$, para $\lambda \in \{0,2,8\}$ para que, us\'andolo como columnas, se construya la matriz $P$ con la que se obtiene la forma diagonal. 
  Si $X=[x,y,z]^t \in E_{\lambda=8}$, entonces debe cumplirse que $(A-8I)X = O$. Esto es:
  \begin{equation*}
   \begin{bmatrix}
    0 & -1 & -8 \\
    0 & -6 &  0 \\
    0 &  0 & -8 
   \end{bmatrix}
   \begin{bmatrix}
    x \\ y \\ z
   \end{bmatrix}
   =
   \begin{bmatrix}
    0 \\ 0 \\ 0
   \end{bmatrix}
  \end{equation*}
  por lo que $y = z = 0$ y $x$ es un par\'ametro libre, es decir $E_{\lambda=8} = \left< (1, 0, 0) \right>$. Si $X=[x,y,z]^t \in E_{\lambda=2}$, entonces se debe de cumplir que $(A-2I)X = O$. Esto es:
  \begin{equation*}
   \begin{bmatrix}
    6 & -1 & -8 \\
    0 &  0 &  0 \\
    0 &  0 & -2
   \end{bmatrix}
   \begin{bmatrix}
    x \\ y \\ z
   \end{bmatrix}
   =
   \begin{bmatrix}
    0 \\ 0 \\ 0
   \end{bmatrix}
  \end{equation*}
  Por lo tanto $z = 0$ y $y = 6x$, entonces $E_{\lambda=2} = \left< (1, 6, 0) \right>$. Y, si $X = [x, y, z]^t \in E_{\lambda=0}$, entonces se debe de cumplir que $(A - 0I)X = AX = 0$. Esto es:
  \begin{equation*}
   \begin{bmatrix}
    8 & -1 & -8 \\
    0 &  2 &  0 \\
    0 &  0 &  0
   \end{bmatrix}
   \begin{bmatrix}
    x \\ y \\ z
   \end{bmatrix}
   =
   \begin{bmatrix}
    0 \\ 0 \\ 0
   \end{bmatrix}
  \end{equation*}
  Por lo tanto $y = 0$ y $x=z$, entonces $E_{\lambda= 0} = \left< (1, 0, 1) \right>$, entonces la diagonalizaci\'on de $A$ est\'a dada por $D = P^{-1}AP$ donde
  \begin{equation*}
   D =
   \begin{bmatrix}
    8 & 0 & 0 \\
    0 & 2 & 0 \\
    0 & 0 & 0
   \end{bmatrix}
   \qquad \text{ y } \qquad 
   P = 
   \begin{bmatrix}
    1 & 1 & 1 \\
    0 & 6 & 0 \\
    0 & 0 & 1
   \end{bmatrix}
  \end{equation*}
  que es a lo que se quer\'{\i}a llegar.${}_{\blacksquare}$
 \end{enumerate}
\end{solucion}
