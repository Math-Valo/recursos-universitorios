\begin{enunciado}
 Sea $A$ una matriz sim\'etrica real y sean $\alpha$ y $\beta$ el menor y mayor de los autovalores de $A$. Llamando $A'(h) = A - hI$, hallar los valores de $h$ para los que $A'(h)$ es definida positiva, definida negativa y no definida.
\end{enunciado}

\begin{solucion}
 Por teorema, como $A$ es una matriz sim\'etrica, entonces se puede diagonalizar ortogonalmente a una matriz $D$; esto es, existe una matriz ortogonal $P$ tal que $D=P^tAP$. La matriz $D$ est\'a conformada en su diagonal por los autovalores de $A$ y la matriz $P$ est\'a conformado por columnas de vectores ortonormales, que son autovectores de $A$ correspondientes a los autovalores de $A$, en el mismo orden en el que aparecen en la matriz $D$.
 \par 
 De todo esto, se tiene que si se intercambian las columnas de $P$, se intercambian tambi\'en los autovalores que aparecen en la diagonal de la matriz $D$. Por lo tanto, se puede suponer que la matriz diagonal $D$, congruente a $A$, tiene en su diagonal, en este orden, a los elementos $(\lambda_1, \lambda_2, \lambda_3, \ldots, \lambda_n)$, donde $\lambda_1 = \alpha \leq \lambda_2 \leq \lambda_3 \leq \cdots \leq \lambda_n = \beta$. Adem\'as, $D'(h) = P^tA'(h)P = P^t(A-hI)P = P^tAP - h(P^tIP) = D - hI$, es matriz diagonal y congruente a $A'(h)$, el cual tiene en su diagonal, en este orden, a los elementos $(\lambda_1 - h, \lambda_2 - h, \ldots, \lambda_n - h)$, los cuales cumplen que $\lambda_1 -h \leq \lambda_2 - h \leq \cdots \leq \lambda_n - h$.
 \par 
 Por lo tanto, como $A'(h)$ es definida positiva, definida negativa o no definida si y s\'olo si es definida positiva, definida negativa o no definida cualquiera de sus matriz congruentes, respectivamente, entonces el estudio de esto se reduce estudiarlo en $D'(h)$. Por lo tanto, como una matriz diagonal es definida positiva, definida negativa o no definida si y s\'olo si los elementos de su diagonal son todos positivos, todos negativos o ninguno de estos dos casos, respectivamente, se tiene que para que $D'(h)$ sea definida positiva se debe cumplir que $\lambda_1-h > 0$, es decir $h <\lambda_1 = \alpha$; para que $D'(h)$ sea definida negativa se debe cumplir que $\lambda_n-h < 0$, es decir $h > \lambda_n = \beta$; y, para que $D'(h)$ sea no definida se debe cumplir que $h\geq \alpha$ y $h\leq\beta$. Por lo tanto, $A'(h)$ es definida positiva si y s\'olo si $h\in(-\infty, \alpha)$, $A'(h)$ es no definida si y s\'olo si $h\in[\alpha,\beta]$ y $A'(h)$ es definida negativa si y s\'olo si $h\in(\beta,\infty)$, que es a lo que se quer\'{\i}a llegar.${}_{\blacksquare}$
\end{solucion}
