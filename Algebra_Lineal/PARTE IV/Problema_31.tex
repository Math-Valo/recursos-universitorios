\begin{enunciado}
 Sea $V$ un espacio vectorial real de dimensi\'on finita y sea $f: V \to V$ un endomorfismo. Obtener una condici\'on necesaria y suficiente para que exista un producto escalar en $V$ con el que $f$ sea un endomorfismo sim\'etrico.
\end{enunciado}

\begin{solucion}
 Se ha probado ya que si $f$ es un endomorfismo sim\'etrico, entonces existe una base ortonormal de $V$ formado por vectores propios de $f$, para cierto producto escalar. Se va a querer probar ahora que el regreso tambi\'en es cierto, es decir, que, para cierto producto escalar, si existe una base ortonormal de $V$ formado por vectores propios de $f$, entonces $f$ es un endomorfismo sim\'etrico.
 \par 
 Si existe tal base, entonces cualesquiera dos vectores $\bar{x}$ y $\bar{y}$ tienen una representaci\'on \'unica como combianci\'on lineal de dichos vectores. Es decir, sea $\beta = \{ \bar{e}_1, \bar{e}_2, \ldots, \bar{e}_n \}$ dicha base (se est\'a suponiendo que la dimensi\'on de $V$ es $n$), entonces existen conjuntos de escalares, $\{\alpha_1, \alpha_2, \ldots, \alpha_n \}$ y $\{ \beta_1, \beta_2, \ldots, \beta_n \}$, tales que
 \begin{equation*}
  \bar{x} = \sum_{i=1}^n \alpha_i \bar{e}_i
  \qquad \text{ y } \qquad 
  \bar{y} = \sum_{j=1}^n \beta_j \bar{e}_j
 \end{equation*}
 Luego, como $\beta$ es base ortonormal, se cumple que $\bar{e}_i\cdot\bar{e}_i = 1$ y $\bar{e}_i\cdot \bar{e}_j = 0$ para cualesquiera $i,j\in\mathbb{N}\cap[1,n]$ con $i\neq j$, y, como son vectores propios tambi\'en, se cumple tambi\'en que existen un conjunto de escalares. $\{ \lambda_1, \lambda_2, \ldots, \lambda_n \}$, tales que $f(\bar{e}_i) = \lambda_i \bar{e}_i$, para toda $i\in\mathbb{N}\cap[1,n]$. Por lo tanto:
 \begin{eqnarray*}
  \bar{x}\cdot f(\bar{y}) & = & \left( \sum_{i=1}^n \alpha_i \bar{e}_i \right) \cdot f\left( \sum_{j=1}^n \beta_j \bar{e}_j \right) = \left( \sum_{i=1}^n \alpha_i \bar{e}_i \right) \cdot \left[ \sum_{j=1}^n \beta_j f(\bar{e}_j) \right] \\
  & = & \left( \sum_{i=1}^n \alpha_i \bar{e}_i \right) \cdot \left[ \sum_{j=1}^n \beta_j (\lambda_j \bar{e}_j) \right] = \sum_{i,j=1}^n (\alpha_i \bar{e}_i) \cdot (\lambda_j \beta_j \bar{e}_j) = \sum_{i,j=1}^n \lambda_j \alpha_i \beta_j (\bar{e}_i \cdot \bar{e}_j) \\
  & = & \sum_{i=1}^n \lambda_i \alpha_i \beta_i
 \end{eqnarray*}
 y. por otro lado,
 \begin{eqnarray*}
  f(\bar{x})\cdot \bar{y} & = & f\left( \sum_{i=1}^n \alpha_i \bar{e}_i \right) \cdot \left( \sum_{j=1}^n \beta_j \bar{e}_j \right) = \left[ \sum_{i=1}^n \alpha_i f(\bar{e}_i) \right] \cdot \left( \sum_{j=1}^n \beta_j \bar{e}_j \right) \\
  & = & \left[ \sum_{i=1}^n \alpha_i (\lambda_i \bar{e}_i) \right] \cdot \left( \sum_{j=1}^n \beta_j \bar{e}_j \right) = \sum_{i,j=1}^n (\lambda_i \alpha_i \bar{e}_i) \cdot (\beta_j \bar{e}_j) = \sum_{i,j=1}^n \lambda_i \alpha_i \beta_j (\bar{e}_i \cdot \bar{e}_j) \\
  & = & \sum_{i=1}^n \lambda_i \alpha_i \beta_i
 \end{eqnarray*}
 Por lo que $\bar{x}\cdot f(\bar{y}) = f(\bar{x})\cdot \bar{y}$. Por lo tanto, bajo la supuesta existencia de un producto escalar apropiado, $f$ es sim\'etrica si y s\'olo si existe una base ortonormal de vectores propios de $f$. Desglozando el hecho de que existe una base ortonormal de vectores propios $f$, se tiene que que $f$ tiene $n$ autovectores linealmente independientes y, de ello, al separarlos en tantos conjuntos como valores propios distintos haya, estos conjuntos generan los subespacios propios de $f$, llamando $d_i$ a la multiplicidad geom\'etricas del $i-$\'esimo autovalor, es decir la dimensiones del subespacio propio $i-$\'esimo, y $m_i$ a la multiplicidad algebraica del $i-$\'esimo autovalor, se tiene que $d_1+d_2+\cdots +d_p = n$, donde $p$ representa la cantidad de autovalores propios distintos, entonces, al ser $d_i\leq m_i$ y $n=d_1 + d_2 + \cdot + d_n \leq m_1 + m_2 + \cdots + m_p \leq n$, se sigue que $m_1 + m_2 + \cdots m_p = n$ y $d_i = m_i$ para toda $i\in \mathbb{N}\cap[1,p]$, por lo tanto $f$ es diagonalizable; por otro lado, sea $D$ la matriz diagonal de $f$ en cierta base compuesta por los autovalores de $f$, se debe cumplir que en la base formada por autovectores propios sea ortonormal, es decir, el producto escalar debe de hacer que los autovectores sean ortonormales.
 \par 
 N\'otese que estas dos resultados a partir de que existe una base ortonormal de vectores propios de $f$, es decir, que esto implica que $f$ es diagonalizable y que el producto escalar hace que una base en la que la matriz de $f$ es diagonal es base ortonormal, es una equivalencia. Esto es, si $f$ es diagonalizable y el producto escalar hace que una base en la que la matriz de $f$ es diagonal, compuesta por los autovalores de $f$, sea base ortonormal, implica que existe una base ortonormal de vectores propios de $f$.
 \par 
 Resumiendo, existe un producto escalar en $V$ con el que $f$ es un endomorfismo sim\'etrico si y s\'olo si aquel producto escalar hace que $V$ tenga una base ortonormal de vectores propios de $f$ si y s\'olo si $f$ es diagonalizable y el producto escalar hace que una base en la que la matriz de $f$ es diagonal, compuesta por los autovalores de $f$, sea base ortonormal. Por lo tanto, la condici\'on necesaria y suficiente para que exista un producto escalar en $V$ con el que $f$ es un endomorfismo sim\'etrico es que $f$ sea diagonalizable y que el producto escalar haga ortonormal una base en la que se obtiene la matriz $D$ de la diagonalizaci\'on de $f$, que es a lo que se quer\'{\i}a llegar.${}_{\blacksquare}$
\end{solucion}
