\documentclass[a4paper,11pt]{article}
%%\documentclass[a4paper,12pt]{amsart}
\usepackage[cp1252]{inputenc}
\usepackage[spanish]{babel}
\usepackage{amsmath}
\usepackage{amsthm}
\usepackage{amssymb}
\usepackage{amsfonts}
\usepackage{graphicx}
\usepackage{cancel}
\usepackage{color}
\usepackage{multirow}
\usepackage{colortbl}

\setlength{\textheight}{23.5cm} \setlength{\evensidemargin}{0cm}
\setlength{\oddsidemargin}{-0.8cm} \setlength{\topmargin}{-2.5cm}
\setlength{\textwidth}{17.5cm} \setlength{\parskip}{0.25cm}

\hyphenation{pro-ba-bi-li-dad}
\spanishdecimal{.}

\newtheoremstyle{teoremas}{\topsep}{\topsep}%
     {}% Body font
     {}% Indent amount (empty = no indent, \parindent = para indent)
     {}% Thm head font
     {}% Punctuation after thm head
     {0.5em}% Space after thm head (\newline = linebreak)
     {\thmname{{\bfseries#1}}\thmnumber{ {\bfseries#2}.}\thmnote{ {\itshape#3}.}}% Thm head spec
\theoremstyle{teoremas}

\newtheorem{teorema}{Teorema}[section]
\newtheorem{corolario}[teorema]{Corolario}


\newtheoremstyle{ejemplos}{\topsep}{\topsep}%
     {}%         Body font
     {}%         Indent amount (empty = no indent, \parindent = para indent)
     {}%         Thm head font
     {.}%        Punctuation after thm head
     {0.5em}%     Space after thm head (\newline = linebreak)
     {\thmname{{\bfseries#1}}\thmnumber{ {\bfseries#2}}\thmnote{ {\itshape#3}}}%         Thm head spec
\theoremstyle{ejemplos}

\newtheoremstyle{definiciones}{\topsep}{\topsep}%
     {}%         Body font
     {}%         Indent amount (empty = no indent, \parindent = para indent)
     {}%         Thm head font
     {.}%        Punctuation after thm head
     {0.5em}%     Space after thm head (\newline = linebreak)
     {\thmname{{\bfseries#1}}\thmnumber{ {\bfseries#2}}\thmnote{ {\itshape#3}}}%         Thm head spec
\theoremstyle{definiciones}

\newtheoremstyle{lemas}{\topsep}{\topsep}%
     {}%         Body font
     {}%         Indent amount (empty = no indent, \parindent = para indent)
     {}%         Thm head font
     {.}%        Punctuation after thm head
     {0.5em}%     Space after thm head (\newline = linebreak)
     {\thmname{{\bfseries#1}}\thmnumber{ {\bfseries#2}}\thmnote{ {\itshape#3}}}%         Thm head spec
\theoremstyle{lemas}


\newtheorem*{definicion}{Definici\'on}
\newtheorem*{enunciado}{Enunciado}
\newtheorem*{solucion}{Soluci\'on}
\newtheorem*{demostracion}{Demostraci\'on}
\newtheorem*{lema}{Lema}
\newtheorem*{hipotesis}{Hip\'otesis}
\newtheorem*{estadistico}{Estad\'{\i}stico de Prueba}
\newtheorem*{region}{Regi\'on de Rechazo}
\newtheorem*{conclusion}{Conclusi\'on}
\newtheorem*{hallar}{Por hallar}
\newtheorem*{observaciones}{Observaciones}
\newtheorem*{propiedades}{Propiedades}


\title{Valor del determinante de Cauchy}
\author{\'Alvaro J. Carde\~na Mej\'{\i}a}
\date{\today}

\begin{document}

\maketitle

\section{Definici\'on de la matriz de Cauchy}

S\'upongase que se tienen $2n$ elementos $a_1, a_2, \ldots, a_n, b_1, b_2, \ldots, b_n$ con $a_i \neq b_j$ para todo $i$ y $j$. Entonces la matriz de $n\times n$
\begin{equation}
 \begin{pmatrix}
  \displaystyle{ \frac{1}{a_1 - b_1} } & \displaystyle{ \frac{1}{a_1 - b_2} } & \cdots & \displaystyle{ \frac{1}{a_1 - b_n} } \\
  \displaystyle{ \frac{1}{a_2 - b_1} } & \displaystyle{ \frac{1}{a_2 - b_2} } & \cdots & \displaystyle{ \frac{1}{a_2 - b_2} } \\
  \vdots & \vdots & \ddots & \vdots \\
  \displaystyle{ \frac{1}{a_n - b_1} } & \displaystyle{ \frac{1}{a_n - b_2} } & \cdots & \displaystyle{ \frac{1}{a_n - b_n} }
 \end{pmatrix}
\end{equation}
es llamada una matriz de Cauchy.

\section{Teorema sobre el determinante de una matriz de Cauchy}

\begin{teorema}
 El determinante de una matriz de Cauchy, $C$, es:
 \begin{equation*}
  \det C = \frac{\displaystyle{ \prod_{1 \leq i < j \leq n} (a_j - a_i)(b_i - b_j) }}{\displaystyle{ \prod_{1 \leq i,j \leq n} (a_i - b_j) }} 
 \end{equation*}
\end{teorema}

\section{Demostraci\'on del teorema}

\begin{demostracion}
 Dado que el determinante de una matriz es invariante si a una columna (o rengl\'on) se le suma o resta un m\'ultiplo de otra, entonces, el determinante de la matriz de Cauchy de orden $n$ (al que se le denotar\'a en lo sucesivo como $D_n$) permanecer\'a igual si se le resta la primera columna a cada columna $j$, con $j$ desde 2 hasta $n$. Al hacer esta operaci\'on, los valores de las sentradas en cada columna desde la segunda hasta la $n$-\'esima ser\'an:
 \begin{equation*}
  a_{ij} = \frac{1}{a_i-b_j} - \frac{1}{a_i - b_1} = \frac{(a_i - b_1) - (a_i - b_j)}{(a_i - b_j)(a_i - b_1)} = \left( \frac{b_j - b_1}{a_i - b_1} \right) \left( \frac{1}{a_i - b_j} \right)
 \end{equation*}
 Ahora, extrayendo el factor $\frac{1}{a_i - b_1}$ para cada rengl\'on $1\leq i \leq n$ y luego extrayendo el factor $b_j - b_1$ de cada columna $2 \leq j \leq n$, por la propiedad de que un determinante es igual al producto del determinante entre el factor constante de un rengl\'on o columna por el factor constante, se tiene que el determinante de la matriz de orden $n$ es:
 \begin{equation*}
  D_n = \left( \prod_{i=1}^{n} \frac{1}{a_i - b_1} \right) \left( \prod_{j=2}^n b_j - b_1 \right)
  \begin{vmatrix}
   1 & \displaystyle{\frac{1}{a_1 - b_2}} & \displaystyle{\frac{1}{a_1 - b_3}} & \cdots & \displaystyle{\frac{1}{a_1 - b_n}} \\ 
   1 & \displaystyle{\frac{1}{a_2 - b_2}} & \displaystyle{\frac{1}{a_2 - b_3}} & \cdots & \displaystyle{\frac{1}{a_2 - b_2}} \\ 
   1 & \displaystyle{\frac{1}{a_3 - b_2}} & \displaystyle{\frac{1}{a_3 - b_3}} & \cdots & \displaystyle{\frac{1}{a_3 - b_n}} \\
   \vdots & \vdots & \vdots & \ddots & \vdots \\
   1 & \displaystyle{\frac{1}{a_n - b_2}} & \displaystyle{\frac{1}{a_n - b_3}} & \cdots & \displaystyle{\frac{1}{a_n - b_n}}
  \end{vmatrix}
 \end{equation*}
 Ahora, restando el rengl\'on 1 a cada uno de los renglones desde $2$ hasta $n$, la columna 1 se vuelve $0$ para todos excepto el primer rengl\'on y de la columna $2$ hasta la $n$, se vuelven los elementos:
 \begin{equation*}
  a_{ij} = \frac{1}{a_i - b_j} - \frac{1}{a_1 - b_j} = \frac{(a_1 - b_j) - (a_i - b_j)}{(a_i - b_j)(a_1 - b_j)} = \left( \frac{a_1 - a_i}{a_1 - b_j} \right) \left( \frac{1}{a_i - b_j} \right)
 \end{equation*}
 Esta operaci\'on fue la resta de cada rengl\'on menos un rengl\'on, por lo que, por lo antes mencionado, el determinante no cambia.
 Luego, extrayendo el factor $a_1 - a_i$ de cada uno de los renglones $2 \leq i \leq n$ y, luego, extrayendo el factor $\frac{1}{a_1 - b_j}$ de cada una de las columnas $2 \leq j \leq n$, usando nuevamente la propiedad de determinantes de extracci\'on de valores constantes en filas y columnas, se tiene lo siguiente:
 \begin{equation*}
  D_n = \left( \prod_{i=1}^n \frac{1}{a_i - b_1} \right) \left( \prod_{j=2}^n \frac{1}{a_1 - b_j} \right) \left( \prod_{i=2}^n a_1 - a_i \right) \left( \prod_{j=2}^n b_j - b_1 \right)
  \begin{vmatrix}
   1 & 1 & 1 & \cdots & 1 \\ 
   0 & \displaystyle{\frac{1}{a_2 - b_2}} & \displaystyle{\frac{1}{a_2 - b_3}} & \cdots & \displaystyle{\frac{1}{a_2 - b_n}} \\ 
   0 & \displaystyle{\frac{1}{a_3 - b_2}} & \displaystyle{\frac{1}{a_3 - b_3}} & \cdots & \displaystyle{\frac{1}{a_3 - b_n}} \\
   \vdots & \vdots & \vdots & \ddots & \vdots \\
   0 & \displaystyle{\frac{1}{a_n - b_2}} & \displaystyle{\frac{1}{a_n - b_3}} & \cdots & \displaystyle{\frac{1}{a_n - b_n}}
  \end{vmatrix}
 \end{equation*}
 Entonces, de las propiedades de determinantes, esto se reduce al bloque de $(n-1) \times (n-1)$ y se tiene que:
 \begin{equation}
  D_n = \frac{\displaystyle{ \prod_{i=2}^n (a_i - a_1)(b_1 - b_i) }}{\displaystyle{ (a_1 - b_1) \prod_{2 \leq i,j \leq n} (a_i - b_1)(a_1 - b_j) }}
  \begin{vmatrix}
   \displaystyle{\frac{1}{a_2 - b_2}} & \displaystyle{\frac{1}{a_2 - b_3}} & \cdots & \displaystyle{\frac{1}{a_2 - b_n}} \\ 
   \displaystyle{\frac{1}{a_3 - b_2}} & \displaystyle{\frac{1}{a_3 - b_3}} & \cdots & \displaystyle{\frac{1}{a_3 - b_n}} \\
   \vdots & \vdots & \ddots & \vdots \\
   \displaystyle{\frac{1}{a_n - b_2}} & \displaystyle{\frac{1}{a_n - b_3}} & \cdots & \displaystyle{\frac{1}{a_n - b_n}}
  \end{vmatrix}
 \end{equation}
 Repitiendo el proceso para el resto de los renglones y columnas de $2$ a $n$, se llega al resultado deseado.${}_{\blacksquare}$
 
\end{demostracion}


\end{document}
