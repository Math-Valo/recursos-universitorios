\begin{enunciado}
 Dados tres puntos alineados $P$, $Q$ y $R$ (de $E_1$, de $E_2$ o de $E_3$), se llama <<raz\'on simple>> $(PQR)$ al n\'umero real $\rho$ que permite poner $\overrightarrow{PR} = \rho\overrightarrow{PQ}$. \\
 Sean $A_1$, $A_2$ y $A_3$ tres puntos no alineados del plano af\'{\i}n $E_2$ y sea $r$ una recta que no pasa por ninguno de dichos puntos. Se llaman $B_1$, $B_2$ y $B_3$ a los puntos de intersecci\'on de $r$ con las rectas $A_2A_3$, $A_3A_1$ y $A_1A_2$, respectivamente. Pru\'ebese que (teorema de Menelao):
 \begin{equation*}
  \left( B_1A_2A_3 \right)\left( B_2A_3A_1 \right)\left( B_3A_1A_2 \right) = 1
 \end{equation*}
\end{enunciado}

\begin{solucion}
 Para resolver el problema, consid\'erense coordenadas rectangulares 
 y se nombrar\'an de la siguiente forma a las rectas que
 se obtienen de 3 puntos alineados conocidos por el enunciado.
 \begin{eqnarray*}
  \ell   & = & \overline{B_1B_2B_3} \\
  \ell_1 & = & \overline{B_1A_2A_3} \\
  \ell_2 & = & \overline{A_1B_2A_3} \\
  \ell_3 & = & \overline{A_1A_2B_3}
 \end{eqnarray*}
 Sean $p$, $q$ y $r$ las alturas sobre $r$ de $A_1$, $A_2$ y $A_3$,
 respectivamente, entonces se cumplen las siguientes igualdades:
 \begin{eqnarray*}
  p & = &
  \left\lVert \overrightarrow{B_2A_1} \right\rVert
  \cos\left( \ell, \ell_2 \right)
  = \left\lVert \overrightarrow{B_3A_1} \right\rVert
  \cos\left( \ell, \ell_3 \right)
  \\
  q & = & 
  \left\lVert \overrightarrow{B_1A_2} \right\rVert
  \cos\left( \ell, \ell_1 \right)
  = \left\lVert \overrightarrow{B_3A_2} \right\rVert
  \cos\left( \ell, \ell_3 \right)
  \\
  r & = & 
  \left\lVert \overrightarrow{B_1A_3} \right\rVert
  \cos\left( \ell, \ell_1 \right)
  = \left\lVert \overrightarrow{B_2A_3} \right\rVert
  \cos\left( \ell, \ell_2 \right)
  \\
 \end{eqnarray*}
 Y, nombrando de la siguiente forma a las razones simples que aparecen
 en el teorema, $\left( B_1A_2A_3 \right) = \rho_1$,
 $\left( B_2A_3A_1 \right) = \rho_2$
 y $\left( B_3A_1A_2 \right) = \rho_3$,
 se tienen las siguiente igualdades:
 \begin{equation*}
  \lvert \rho_1 \rvert =
  \frac{
  \left\lVert \overrightarrow{B_1A_3} \right\rVert
  }{
  \left\lVert \overrightarrow{B_1A_2} \right\rVert
  } 
  \hspace{2cm}
  \lvert \rho_2 \rvert =
  \frac{
  \left\lVert \overrightarrow{B_2A_1} \right\rVert
  }{
  \left\lVert \overrightarrow{B_2A_3} \right\rVert
  }
  \hspace{2cm}
  \lvert \rho_3 \rvert =
  \frac{
  \left\lVert \overrightarrow{B_3A_2} \right\rVert
  }{
  \left\lVert \overrightarrow{B_3A_1} \right\rVert
  }
 \end{equation*}
 Por lo tanto:
 \begin{eqnarray*}
  \frac{r}{q} & = &
  \frac{
  \left\lVert \overrightarrow{B_1A_3} \right\rVert
  \cancel{ \cos\left( \ell, \ell_1 \right) }
  }{
  \left\lVert \overrightarrow{B_1A_2} \right\rVert
  \cancel{ \cos\left( \ell, \ell_1 \right) }
  }
  =
  \frac{
  \left\lVert \overrightarrow{B_1A_3} \right\rVert
  }{
  \left\lVert \overrightarrow{B_1A_2} \right\rVert
  }
  = \lvert \rho_1 \rvert
  \\
  \frac{p}{r} & = &
  \frac{
  \left\lVert \overrightarrow{B_2A_1} \right\rVert
  \cancel{ \cos\left( \ell, \ell_2 \right) }
  }{
  \left\lVert \overrightarrow{B_2A_3} \right\rVert
  \cancel{ \cos\left( \ell, \ell_2 \right) }
  }
  =
  \frac{
  \left\lVert \overrightarrow{B_2A_1} \right\rVert
  }{
  \left\lVert \overrightarrow{B_2A_3} \right\rVert
  }
  = \lvert \rho_2 \rvert
  \\
  \frac{q}{p} & = &
  \frac{
  \left\lVert \overrightarrow{B_3A_2} \right\rVert
  \cancel{ \cos\left( \ell, \ell_3 \right) }
  }{
  \left\lVert \overrightarrow{B_3A_1} \right\rVert
  \cancel{ \cos\left( \ell, \ell_3 \right) }
  }
  =
  \frac{
  \left\lVert \overrightarrow{B_3A_2} \right\rVert
  }{
  \left\lVert \overrightarrow{B_3A_1} \right\rVert
  }
  = \lvert \rho_3 \rvert
 \end{eqnarray*}
 Por lo tanto:
 \begin{equation*}
  \left\lvert \rho_1 \rho_2 \rho_3 \right\rvert
  = 
  \left\lvert \rho_1 \right\rvert
  \left\lvert \rho_2 \right\rvert
  \left\lvert \rho_3 \right\rvert
  = \frac{\cancel{r}}{\cancel{q}} \cdot
  \frac{p}{\cancel{\cancel{r}}} \cdot \frac{\cancel{q}}{\cancel{p}}
  = 1
 \end{equation*}
 Por lo tanto,
 $\left( B_1A_2A_3 \right)\left( B_2A_3A_1 \right)\left( B_3A_1A_2 \right) = \pm 1$
\end{solucion}
