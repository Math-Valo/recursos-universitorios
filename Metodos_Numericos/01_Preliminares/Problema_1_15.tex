\begin{enunciado}
 Halle el \'area media de todos los c\'{\i}rculos centrados en el origen cuyo radio est\'a comprendido entre $1$ y $3$.
\end{enunciado}

\begin{solucion}
 Sea $A(r)$ el \'area del c\'{\i}rculo centrado en el origen y cuyo radio, $r$, cumple que $1\leq r \leq 3$, entonces $A(r) = \pi r^2$. Luego entonces, el \'area media de todos los c\'{\i}rculos centrados en el origen cuyo radio est\'a comprendido entre $1$ y $3$ estar\'a dado por la altura media de la curva $A(r)$ en el intervalo $[1,3]$. Esto se calcula por medio de la integral como sigue:
 \begin{equation*}
  \frac{1}{3-1} \int_{1}^{3} A(r) \, dr = \frac{1}{2} \int_{1}^{3} \pi r^2 \, dr = \frac{\pi}{2} \left[ \frac{r^3}{3} \right]_{1}^{3} = \frac{\pi}{2}\left( \frac{(3)^3}{3} - \frac{(1)^3}{3}\right) = \frac{\pi(27-1)}{6} = \frac{26\pi}{6} = \frac{13\pi}{3}
 \end{equation*}
 Por lo tanto, $\displaystyle{ \frac{13\pi}{3} }$ es el valor medio de las \'areas mencionadas, que es a lo que se quer\'{\i}a llegar.${}_{\blacksquare}$
\end{solucion}
