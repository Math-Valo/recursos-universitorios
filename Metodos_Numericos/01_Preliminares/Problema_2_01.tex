\begin{enunciado}
 Use un computador para realizar las siguientes operaciones de forma acumulada; la intenci\'on es que el computador vaya haciendo las substracciones de forma repetida; sin emplear el atajo de la multiplicaci\'on.
 \begin{multicols}{2}
  \begin{enumerate}[(a)]
   \item $10\,000 - \sum_{k=1}^{100\,000} 0.1$
   \item $10\,000 - \sum_{k=1}^{80\,000} 0.125$
  \end{enumerate}
 \end{multicols}
\end{enunciado}

\begin{solucion}
 Las siguientes c\'alculos fueron realizadas con el programa Octave usando 64 cifras para representar n\'umeros reales, y maneja, aproximadamente entre 14 y 15 d\'{\i}gitos decimales de exactitud y que pueden ser observados hasta 14 de ellos ingresando la l\'{\i}nea de c\'odigo siguiente:
 \begin{verbatim}
> format long
 \end{verbatim}
  \vspace{-0.5cm}
 En cada inciso se muestra tambi\'en el c\'odigo en Octave con el que se hizo el c\'alculo.
 \begin{enumerate}[(a)]
  \item Usando las siguientes
  \begin{verbatim}
> a = 0;
> for i = 1:100000
>     a = a + 0.1;
> end
> 10000 - a
  \end{verbatim}
  \vspace{-0.5cm}
  de este modo se obtiene un resultado en pantalla como se muestra a continuaci\'on:
  \begin{verbatim}
-1.88483681995422e-08
  \end{verbatim}
  \vspace{-0.5cm}
  Es decir, seg\'un esto se tiene que
  \begin{equation*}
   10\,000 - \sum_{k=1}^{100\,000} 0.1 =
   -0.0000000188483681995422
  \end{equation*}
  
  \item Usando las siguientes
  \begin{verbatim}
> a = 0;
> for i = 1:80000
>     a = a + 0.125;
> end
> 10000 - a
  \end{verbatim}
  \vspace{-0.5cm}
  de este modo se obtiene un resultado en pantalla como se muestra a continuaci\'on:
  \begin{verbatim}
0
  \end{verbatim}
  \vspace{-0.5cm}
  Es decir, la resta dio exactamente $0$, que es  lo que se quer\'{\i}a llegar.${}_{\blacksquare}$
 \end{enumerate}
\end{solucion}
