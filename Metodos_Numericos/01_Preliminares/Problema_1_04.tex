\begin{enunciado}
 Halla la cota superior e inferior cuya existencia garantiza el teorema de los valores extremos para cada una de las siguientes funciones en el intervalo que se indica.
 \begin{enumerate}[(a)]
  \item $f(x) = x^2 - 3x + 1$ en $[-1, 2]$.
  \item $f(x) = \cos^2(x) - \sin(x)$ en $[0,2\pi]$.
 \end{enumerate}
\end{enunciado}

\begin{solucion}
 Para resolver los incisos, se recordar\'a que un punto cr\'{\i}tico es un m\'aximo o m\'{\i}nimo (o punto indefinido, que en estas funciones continuas no puede ser el caso) de un funci\'on. Los puntos cr\'{\i}ticos se pueden encontrar derivando una funci\'on e igualando a cero. Por lo tanto, se proceder\'a a obtener los valores $c$ tales que $f'(c) = 0$ en cada caso y determinar si se encuentran en el intervalo dado. Adem\'as, como el se trata de m\'aximos y m\'{\i}nimos en un intervalo, no se descarta la posibilidad de que estos se encuentren en los extremos del intervalo, por lo que al final se evaluar\'an en $f$ los valores $c$ previamente encontrados y los extremos de los intervalos.
 \begin{enumerate}[(a)]
  \item Dado que $f(x) = x^2 - 3x +1$, entonces $f'(x) = 2x - 3$, entonces, si $c$ es un punto cr\'{\i}tico, se cumple que $f'(c) = 2c - 3 = 0$; luego, $c = \frac{3}{2}$, el cual, en efecto, se encuentra en el intervalo $[-1,2]$. Finalmente, se eval\'uan los extremos y este valor $c$:
  \begin{eqnarray*}
   f(-1) & = & (-1)^2 - 3(-1) + 1 = 1 + 3 + 1 = 5 \\
   f\left( \frac{3}{2} \right) & = & \left( \frac{3}{2} \right)^2 - 3\left( \frac{3}{2} \right) + 1 = \frac{9}{4} - \frac{9}{2} + 1 = \frac{9-18+4}{4} = \frac{-5}{4} = -1.25 \\
   f(2) & = & (2)^2 - 3(2) + 1 = 4 - 6 + 1 = -1
  \end{eqnarray*}
  Por lo tanto, esta funci\'on en el intervalo $[-1,2]$ tiene como cota inferior a $M_1 = -1.25 = f(3/2)$ y tiene como cota superior a $M_2 = 5 = f(-1)$.
  
  \item Dado que $f(x) = \cos^2(x) - \sin(x)$, entonces, al derivar $f$, se tiene que $f'(x) = -2\cos(x)\sin(x) - \cos(x) = -\cos(x)\left[ 2\sin(x) + 1 \right]$, entonces, si $c$ es un punto cr\'{\i}tico, se cumple que $f(c) = -\cos(c)\left[ 2\sin(c) + 1 \right] = $, lo cual ocurre si y s\'olo si $\cos(c) = 0$ o $2\sin(c) = 1$, donde $\cos(c) = 0$ si y s\'olo si $c = 2\pi n \pm \frac{\pi}{2}$ y $\sin(c) = -1/2$ si y s\'olo si $c = \frac{3\pi}{2} \pm \frac{\pi}{3}$; luego, los puntos cr\'{\i}ticos en el intervalo $[0,2\pi]$ son: $c_1 = \frac{\pi}{2}$, $c_2 = \frac{3\pi}{2}$, $c_3 = \frac{7\pi}{6}$ y $c_4 = \frac{11\pi}{6}$. Finalmente, se eval\'uan los extremos y estos valores de $c$:
  \begin{eqnarray*}
   f(0) & = & \cos^2(0) - \sin(0) = 1 - 0 = 1 \\
   f\left( \frac{\pi}{2} \right) & = & \cos^2 \left( \frac{\pi}{2} \right) - \sin\left( \frac{\pi}{2} \right) = 0 - 1 = -1 \\ 
   f\left( \frac{3\pi}{2} \right) & = & \cos^2 \left( \frac{3\pi}{2} \right) - \sin\left( \frac{3\pi}{2} \right) = 0 - (-1) = 1 \\ 
   f\left( \frac{7\pi}{6} \right) & = & \cos^2 \left( \frac{7\pi}{6} \right) - \sin\left( \frac{7\pi}{6} \right) = \left( -\frac{ \sqrt{3} }{2} \right)^2 - \left( -\frac{1}{2} \right) = \frac{3}{4} + \frac{1}{2} = \frac{5}{4} \\ 
   f\left( \frac{11\pi}{6} \right) & = & \cos^2 \left( \frac{11\pi}{6} \right) - \sin\left( \frac{11\pi}{6} \right) = \left( \frac{ \sqrt{3} }{2} \right)^2 - \left( -\frac{1}{2} \right) = \frac{3}{4} + \frac{1}{2} = \frac{5}{4} \\ 
   f(2\pi) & = & \cos^2(2\pi) - \sin(2\pi) = 1 - 0 = 1
  \end{eqnarray*}
  Por lo tanto, esta funci\'on en el intervalo $[-1,2]$ tiene como cota inferior $M_1 = -1 = f(\pi/2)$ y tiene como cota superior a $M_2 = 5/4 = f(7\pi/6) = f(11\pi/6)$, que es a lo que se quer\'{\i}a llegar.${}_{\blacksquare}$

 \end{enumerate}
\end{solucion}

