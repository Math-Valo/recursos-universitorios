\begin{enunciado}
 Halle el polinomio de Taylor de grado $n=4$ para cada una de las siguientes funciones alrededor del punto $x_0$ dado.
 \begin{enumerate}[(a)]
  \item $f(x) = \sqrt{x}$, $x_0 = 1$.
  \item $f(x) = x^5 + 4x^2 + 3x + 1$, $x_0 = 0$.
  \item $f(x) = \cos(x)$, $x_0 = 0$.
 \end{enumerate}
\end{enunciado}

\begin{solucion}
 Para obtener los polinomios de Taylor de grado $n=4$, n\'otese que son necesarias las primeras cuatro derivadas. Por lo tanto, en cada soluci\'on, lo primero que se realizar\'a ser\'a la obtenci\'on de las funciones derivadas requeridas y la evaluaci\'on de cada una de \'estas en el punto $x_0$ alrededor de donde se piden los polinomios.
 \begin{enumerate}[(a)]
  \item Dado que
  \begin{center}
   \begin{tabular}{rclcrcl}
    $f(x)$ & $=$ & $\sqrt{x}$ & \hspace{2cm} & $f(1)$ & $=$ & $1$ \\
    \\
    $f'(x)$ & $=$ & $\displaystyle{ \frac{1}{2\sqrt{x}} }$ & & $f'(1)$ & $=$ & $\displaystyle{ \frac{1}{2} }$ \\
    \\
    $f''(x)$ & $=$ & $\displaystyle{ -\frac{1}{4x^{3/2}} }$ & & $f'(1)$ & $=$ & $\displaystyle{ - \frac{1}{4} }$ \\
    \\
    $f^{(3)}(x)$ & $=$ & $\displaystyle{ \frac{3}{8x^{5/2}} }$ & & $f(1)$ & $=$ & $\displaystyle{ \frac{3}{8} }$ \\
    \\
    $f^{(4)}(x)$ & $=$ & $\displaystyle{ -\frac{15}{16x^{7/2}} }$ & & $f(1)$ & $=$ & $\displaystyle{ - \frac{15}{16} }$
   \end{tabular}
  \end{center}
  Entonces el polinomio de Taylor, de grado $4$, para esta funci\'on alrededor de $x_0 = 1$ es
  \begin{eqnarray*}
   P_4(x) & = & \sum_{k=0}^{4} \frac{f^{(k)}(x_0)}{k!} (x-x_0)^k \\
   & = & \frac{1}{0!} (x-1)^{0} + \frac{1/2}{1!} (x-1)^{1} - \frac{1/4}{2!} (x-1)^{2} + \frac{3/8}{3!} (x-1)^{3} - \frac{15/16}{4!} (x-1)^{4} \\
   & = & 1 + \frac{1}{2}(x-1) - \frac{1}{8}(x-1)^2 + \frac{1}{16}(x-1)^3 - \frac{5}{128}(x-1)^4
%    & = & 1 + \frac{x-1}{2} + \frac{x^2 - 2x + 1}{8} + \frac{x^3 - 3x^2 + 3x - 1}{16} + \frac{5x^4 - 20x^3 + 30x^2 - 20x + 5}{128} \\
%    & = & \frac{128 + 64x - 64  + 16x^2 - 32x + 16  + 8x^3 - 24x^2 + 24x - 8 + 5x^4 - 20x^3 + 30x^2 - 20x + 5}{128} \\
%    & = & \frac{77 + 36x + 22x^2 - 12x^3 + 5x^4}{128} \\ 
%    & = & \frac{77}{128} + \frac{9x}{32} + \frac{11x^2}{64} - \frac{3x^3}{32} + \frac{5x^4}{128}
  \end{eqnarray*}

  
  \item Dado que 
  \begin{center}
   \begin{tabular}{rclcrcl}
    $f(x)$ & $=$ & $x^5 + 4x^2 + 3x + 1$ & \hspace{2cm} & $f(0)$ & $=$ & $1$ \\
    $f'(x)$ & $=$ & $5x^4 + 8x + 3$ & & $f'(0)$ & $=$ & $3$ \\
    $f''(x)$ & $=$ & $20x^3 + 8$ & & $f''(0)$ & $=$ & $8$ \\
    $f^{(3)}(x)$ & $=$ & $60x^2$ & & $f^{(3)}(0)$ & $=$ & $0$ \\
    $f^{(4)}(x)$ & $=$ & $120x$ & & $f^{(4)}(0)$ & $=$ & $0$
   \end{tabular}
  \end{center}
  Entonces el polinomio de Taylor, de grado $4$, para esta funci\'on alrededor de $x_0 = 0$ es
  \begin{eqnarray*}
   P_4(x) & = & \sum_{k=0}^{4} \frac{f^{(k)}(x_0)}{k!}(x-x_0)^k \\
   & = & \frac{1}{0!}(x-0)^0 + \frac{3}{1!}(x-0)^1 + \frac{8}{2!}(x-0)^2 + \frac{0}{3!}(x-0)^3 + \frac{0}{4!}(x-0)^4 \\
   & = & 1 + 3x + 4x^2
  \end{eqnarray*}
  
  \item Finalmente, dado que
  \begin{center}
   \begin{tabular}{rclcrcl}
    $f(x)$ & $=$ & $\cos(x)$ & \hspace{2cm} & $f(0)$ & $=$ & $1$ \\
    $f'(x)$ & $=$ & $-\sin(x)$ & & $f'(0)$ & $=$ & $0$ \\
    $f''(x)$ & $=$ & $-\cos(x)$ & & $f''(0)$ & $=$ & $-1$ \\
    $f^{(3)}(x)$ & $=$ & $\sin(x)$ & & $f^{(3)}(0)$ & $=$ & $0$ \\
    $f^{(4)}(x)$ & $=$ & $\cos(x)$ & & $f^{(4)}(0)$ & $=$ & $1$
   \end{tabular}
  \end{center}
  Entonces el polinomio de Taylor, de grado $4$, para esta funci\'on alrededor de $x_0 = 1$ es
  \begin{eqnarray*}
   P_4(x) & = & \sum_{k=0}^{4} \frac{f^{(k)}(x_0)}{k!}(x-x_0)^k \\
   & = & \frac{1}{0!}(x-0)^0 + \frac{0}{1!}(x-0)^1 + \frac{-1}{2!}(x-0)^2 + \frac{0}{3!}(x-0)^3 + \frac{1}{4!}(x-0)^4 \\
   & = & 1 - \frac{x^2}{2!} + \frac{x^4}{4!}
  \end{eqnarray*}
  que es a lo que se quer\'{\i}a llegar.${}_{\blacksquare}$
 \end{enumerate}
\end{solucion}
