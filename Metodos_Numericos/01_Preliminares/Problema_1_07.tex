\begin{enunciado}
 Aplique el teorema de Rolle generalizado a la funci\'on $f(x) = x(x-1)(x-3)$ en el intervalo $[0,3]$.
\end{enunciado}

\begin{solucion}
 Dado que $f(0)=f(1)=f(3)=0$ y existen $f'$ y $f''$, entonces se puede aplicar el teorema de Rolle generalizado, el cual garantiza que existe un valor $c \in [0,3]$ tal que $f''(c)=0$. Como $f''(x) = \left[x(x-1)(x-3)\right]'' = \left( x^3-4x^2+3x \right)'' = \left( 3x^2 - 8x + 3 \right)' = 6x - 8$, entonces, igualando $f''(c) = 0$ y despejando, se tiene que $6c - 8 = 0$ si y s\'olo si $c = \frac{8}{6} = \frac{4}{3}$. Por lo tanto, el valor $c$ que garantiza el teorema de Rolle generalizado es $c = \frac{4}{3} \approx 1.333333$, que es a lo que se quer\'{\i}a llegar.${}_{\blacksquare}$
\end{solucion}

