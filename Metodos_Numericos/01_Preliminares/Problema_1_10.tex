\begin{enunciado}
 Halle los n\'umeros $c$ cuya existencia garantiza el teorema del valor medio para integrales para cada una de las siguientes funciones en el intervalo que se indica.
 \begin{enumerate}[(a)]
  \item $f(x) = 6x^2$ en $[-3,4]$.
  
  \item $f(x) = x\cos(x)$ en $[0,3\pi/2]$.
 \end{enumerate}
\end{enunciado}

\begin{solucion}
 $\phantom{0}$
 \begin{enumerate}[(a)]
  \item Dado que
  \begin{equation*}
   \int_{-3}^{4} 6x^2 \, dx = \left. 2x^3 \right|_{-3}^{4} = 2(4)^3 - 2(-3)^3 = 2(64) - 2(-27) = 128 + 54 = 182
  \end{equation*}
  entonces 
  \begin{equation*}
   \frac{1}{4-(-3)} \int_{-3}^{4} 6x^2 \, dx = \frac{182}{7} = 26
  \end{equation*}
  por lo que, al igualar $6c^2$ a $26$, se tiene que $c^2 = \frac{26}{6} = \frac{13}{3}$, entonces $c = \pm \sqrt{\frac{13}{3}}$. Como ambas soluciones pertenecen al intervalo $[-3,4]$, entonces ambas son v\'alidas. Por lo tanto, los valores $c$ buscados son:
  \begin{eqnarray*}
   c_1 & = & \sqrt{\frac{13}{3}} \approx 2.0816659994661 \\
   c_2 & = & - \sqrt{\frac{13}{3}} \approx - 2.0816659994661
  \end{eqnarray*}

  \item Por el m\'etodo de integraci\'on por partes y haciendo $u = x$ y $v = \sin(x)$, por lo que $du = dx$ y $dv = \cos(x) dx$, se sigue entonces que
  \begin{equation*}
   \int x\cos(x)\, dx = \int udv = uv - \int vdu = x\sin(x) - \int \sin(x) \, dx = x\sin(x) + \cos(x)
  \end{equation*}
  Luego entonces se tiene que
  \begin{eqnarray*}
   \int_{0}^{3\pi/2} x\cos(x) \, dx & = & \left[ x\sin(x) + \cos(x) \right]_{0}^{3\pi/2} \\
   & = & \left[ \left( \frac{3\pi}{2} \right)\sin\left( \frac{3\pi}{2} \right) + \cos\left( \frac{3\pi}{2} \right) \right] - \left[ (0)\sin(0) + \cos(0) \right] \\
   & = & \left[ \left( \frac{3\pi}{2} \right)(-1)+0\right]- \left[(0)(0) + (1) \right] \\
   & = & -\frac{3\pi}{2} - 1 = - \frac{2 + 3\pi}{2}
  \end{eqnarray*}
  Por lo que
  \begin{equation*}
   \frac{1}{3\pi/2 - 0} \int_{0}^{3\pi/2} x\cos(x) \, dx = \left( \frac{\cancel{2}}{3\pi} \right) \left( -\frac{2+3\pi}{\cancel{2}}  \right) = - \frac{2+3\pi}{3\pi} = -1- \frac{2}{3\pi}
  \end{equation*}
  Luego entonces, hay dos valores $c$ tales que $c\cos(c) = -\frac{2+3\pi}{3\pi}$ en el intervalo $[0,3\pi/2]$ y estos son:
  \begin{eqnarray*}
   c_1 & \approx & 2.1650566 \\
   c_2 & \approx & 4.435575
  \end{eqnarray*}
  que es a lo que se quer\'{\i}a llegar.${}_{\blacksquare}$
 \end{enumerate}
\end{solucion}
