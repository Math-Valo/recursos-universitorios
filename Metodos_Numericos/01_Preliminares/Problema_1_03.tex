\begin{enunciado}
 Halle los n\'umeros $c$ cuya existencia garantiza el teorema del valor intermedio para cada una de las siguientes funciones, en el intervalo que se indica y para el valor de $L$ dado.
 \begin{enumerate}[(a)]
  \item $f(x) = -x^2 +2x +3$ en $[-1,0]$ para $L=2$.
  
  \item $f(x)=\sqrt{x^2 -5x -2}$ en $[6,8]$ para $L=3$.
 \end{enumerate}
\end{enunciado}

\begin{solucion}
 $\phantom{0}$
 \begin{enumerate}[(a)]
  \item Dado que $f(-1) = -(-1)^2 + 2(-1) + 3 = -1-2+3 = 0$ y $f(0) = -(0)^2 + 2(0) + 3 = 3$, entonces, en efecto, $L = 2 \in [f(-1), f(0)]$. Luego, por el teorema del valor intermedio, $\exists c \in [-1,0]$ tal que $f(c) = 2$. Igualando, y despejando se tiene lo siguiente:
  \begin{eqnarray*}
   & & -c^2 + 2c + 3 = 2 \\
   \Leftrightarrow & & c^2 -2c - 1 = 0 \\ 
   \Leftrightarrow & & (c-1)^2 -2 = 0 \\
   \Leftrightarrow & & (c-1)^2 = 2 \\
   \Leftrightarrow & & c-1 = \pm \sqrt{2} \\
   \Leftrightarrow & & c = 1 \pm \sqrt{2}
  \end{eqnarray*}
  Luego, los posibles valores de $c$ son $c = 1+\sqrt{2} > 0$ y $c = 1 - \sqrt{2}$. Como $c$ debe pertenecer al intervalo $[-1, 0]$, entonces el \'unico valor de $c$ que cumple lo pedido es $c = 1-\sqrt{2}\approx -0.4142$.
  
  \item Dado que $f(6) = \sqrt{(6)^2 - 5(6) - 2} = \sqrt{36 - 30 - 2} = \sqrt{4} = 2$ y $f(8) = \sqrt{(8)^2 - 5(8) - 2} = \sqrt{64 - 40 - 2} = \sqrt{22} > \sqrt{16} = 4$, entonces, en efecto, $L = 3 \in [f(6),f(8)]$. Luego, por el teorema del valor intermedio, $\exists c \in [6,8]$ tal que $f(c) = 3$. Igualando, y despejando se tiene lo siguiente:
  \begin{eqnarray*}
   & & \sqrt{c^2 - 5c - 2} = 3 \\ 
   \Leftrightarrow & & c^2 - 5c -2 = 3^2 = 9 \\
   \Leftrightarrow & & c^2 - 5x - 11 = 0 \\ 
   \Leftrightarrow & & \left( c - \frac{5}{2} \right)^2 - \frac{25}{4} - \frac{44}{4} = 0 \\
   \Leftrightarrow & & \left( c - \frac{5}{2} \right)^2 = \frac{69}{4} \\
   \Leftrightarrow & & c- \frac{5}{2} = \pm  \sqrt{\frac{69}{4}} = \frac{ \pm \sqrt{69}}{2} \\ 
   \Leftrightarrow & & c = \frac{5 \pm \sqrt{69}}{2}
  \end{eqnarray*}
  Luego, los valores de $c$ para los que $f(c) = 3$ son $c = \frac{5-\sqrt{69}}{2} < 0$ y $c = \frac{5+\sqrt{69}}{2} \approx 6.65331$. Como $c$ debe pertenecer al intervalo $[6, 8]$, entonces el \'unico valor de $c$ que cumple lo pedido es $c = \frac{5+\sqrt{69}}{2} \approx 6.65331$, que es a lo que se quer\'{\i}a llegar.${}_{\blacksquare}$
 \end{enumerate}
\end{solucion}

