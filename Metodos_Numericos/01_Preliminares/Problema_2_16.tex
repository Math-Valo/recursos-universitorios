\begin{enunciado}
 Pruebe que si sustituimos $2$ por $5$ en todas las f\'ormulas de (8), el resultado es un m\'etodo para hallar la expresi\'on en base $5$ de un n\'umero natural. Utilice esto para expresar los siguientes n\'umeros en base $5$.
 \begin{multicols}{4}
  \begin{enumerate}
   \item $10$
   \item $35$
   \item $721$
   \item $734$
  \end{enumerate}
 \end{multicols}
\end{enunciado}

\begin{solucion}
 La demostraci\'on ya se hizo en el enunciado 14 de esta secci\'on de ejercicios para el caso general $n\in\mathbb{N}\backslash\{ 1 \}$. Por lo que se procede hacer los c\'alculos cuando $n = 5$.
 \begin{enumerate}
  \item Dividiendo iteradamente $N=10$ entre $n=5$, se tiene que
  \begin{center}
   \begin{tabular}{rclrr}
    $10$ & $=$ & $5\times$ & $2 + 0$, & $b_0 = 0$ \\
    $2$ & $=$ & $5\times$ & $0 + 2$, & $b_1 = 2$
   \end{tabular}
  \end{center}
  As\'{\i} que la representaci\'on en base $5$ de $10$ es
  \begin{equation*}
   10 = b_1b_0{}_{\text{cinco}} = 20_{\text{cinco}}
  \end{equation*}

  \item Dividiendo iteradamente $N=35$ entre $n=5$, se tiene que
  \begin{center}
   \begin{tabular}{rclrr}
    $35$ & $=$ & $5\times$ & $7 + 0$, & $b_0 = 0$ \\
    $7$ & $=$ & $5\times$ & $1 + 2$, & $b_1 = 2$ \\
    $1$  & $=$ & $5\times$ & $0 + 1$, & $b_2 = 1$
   \end{tabular}
  \end{center}
  As\'{\i} que la representaci\'on en base $5$ de $35$ es
  \begin{equation*}
   35 = b_2b_1b_0{}_{\text{cinco}} = 120_{\text{cinco}}
  \end{equation*}

  \item Dividiendo iteradamente $N=721$ entre $n=5$, se tiene que
  \begin{center}
   \begin{tabular}{rclrr}
    $721$ & $=$ & $5\times$ & $144 + 1$, & $b_0 = 1$ \\
    $144$ & $=$ & $5\times$ & $28 + 4$, & $b_1 = 4$ \\
    $28$  & $=$ & $5\times$ & $5 + 3$, & $b_2 = 3$ \\
    $5$  & $=$ & $5\times$ & $1 + 0$, & $b_3 = 0$ \\
    $1$  & $=$ & $5\times$ & $0 + 1$, & $b_4 = 1$ 
   \end{tabular}
  \end{center}
  As\'{\i} que la representaci\'on en la base $5$ de $721$ es
  \begin{equation*}
   721 = b_4b_3b_2b_1b_0{}_{\text{cinco}} = 10341_{\text{cinco}}
  \end{equation*}

  \item Dividiendo iteradamente $N=734$ entre $n=5$, se tiene que
  \begin{center}
   \begin{tabular}{rclrr}
    $734$ & $=$ & $5\times$ & $146 + 4$, & $b_0 = 4$ \\
    $146$ & $=$ & $5\times$ & $29 + 1$, & $b_1 = 1$ \\
    $29$  & $=$ & $5\times$ & $5 + 4$, & $b_2 = 4$ \\
    $5$  & $=$ & $5\times$ & $1 + 0$, & $b_3 = 0$ \\
    $1$  & $=$ & $5\times$ & $0 + 1$, & $b_4 = 1$ 
   \end{tabular}
  \end{center}
  As\'{\i} que la representaci\'on en la base $5$ de $734$ es
  \begin{equation*}
   734 = b_4b_3b_2b_1b_0{}_{\text{cinco}} = 10414_{\text{cinco}}
  \end{equation*}
  que es a lo que se quer\'{\i}a llegar.${}_{\blacksquare}$
 \end{enumerate}
\end{solucion}
