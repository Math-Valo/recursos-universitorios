\documentclass[a4paper,11pt]{article}
%%\documentclass[a4paper,12pt]{amsart}
\usepackage[cp1252]{inputenc}
\usepackage[spanish]{babel}
\usepackage{amsmath}
\usepackage{amsthm}
\usepackage{amssymb}
\usepackage{amsfonts}
\usepackage{graphicx}
\usepackage{cancel}
\usepackage{color}
\usepackage{multicol}
\usepackage{multirow}
\usepackage{colortbl}
\usepackage{enumerate}
\usepackage{hhline}  % Para el doble hhline (doble cline)

\setlength{\textheight}{23.5cm} \setlength{\evensidemargin}{0cm}
\setlength{\oddsidemargin}{-0.8cm} \setlength{\topmargin}{-2.5cm}
\setlength{\textwidth}{17.5cm} \setlength{\parskip}{0.25cm}

\hyphenation{pro-ba-bi-li-dad}
\spanishdecimal{.}

\newtheoremstyle{teoremas}{\topsep}{\topsep}%
     {}% Body font
     {}% Indent amount (empty = no indent, \parindent = para indent)
     {}% Thm head font
     {}% Punctuation after thm head
     {0.5em}% Space after thm head (\newline = linebreak)
     {\thmname{{\bfseries#1}}\thmnumber{ {\bfseries#2}.}\thmnote{ {\itshape#3}.}}% Thm head spec
\theoremstyle{teoremas}

\newtheorem{teorema}{Teorema}[section]
\newtheorem{corolario}[teorema]{Corolario}


\newtheoremstyle{ejemplos}{\topsep}{\topsep}%
     {}%         Body font
     {}%         Indent amount (empty = no indent, \parindent = para indent)
     {}%         Thm head font
     {.}%        Punctuation after thm head
     {0.5em}%     Space after thm head (\newline = linebreak)
     {\thmname{{\bfseries#1}}\thmnumber{ {\bfseries#2}}\thmnote{ {\itshape#3}}}%         Thm head spec
\theoremstyle{ejemplos}

\newtheoremstyle{definiciones}{\topsep}{\topsep}%
     {}%         Body font
     {}%         Indent amount (empty = no indent, \parindent = para indent)
     {}%         Thm head font
     {.}%        Punctuation after thm head
     {0.5em}%     Space after thm head (\newline = linebreak)
     {\thmname{{\bfseries#1}}\thmnumber{ {\bfseries#2}}\thmnote{ {\itshape#3}}}%         Thm head spec
\theoremstyle{definiciones}

\newtheoremstyle{lemas}{\topsep}{\topsep}%
     {}%         Body font
     {}%         Indent amount (empty = no indent, \parindent = para indent)
     {}%         Thm head font
     {.}%        Punctuation after thm head
     {0.5em}%     Space after thm head (\newline = linebreak)
     {\thmname{{\bfseries#1}}\thmnumber{ {\bfseries#2}}\thmnote{ {\itshape#3}}}%         Thm head spec
\theoremstyle{lemas}


\newtheorem*{enunciado}{Enunciado}
\newtheorem*{solucion}{Soluci\'on}
\newtheorem*{demostracion}{Demostraci\'on}
\newtheorem*{lema}{Lema}
\newtheorem*{hipotesis}{Hip\'otesis}

\title{Preliminares}
\author{\'Alvaro J. Carde\~na Mej\'{\i}a}

\begin{document}

\maketitle

\section{Un repaso al c\'alculo infinitesimal}

\begin{enumerate}
 \item \begin{enunciado}
 $\phantom{0}$
 \begin{enumerate}[(a)]
  \item Halle $L = \lim_{n\to\infty} (4n+1)/(2n+1)$. Despu\'es determine $\{ \varepsilon_n \} = \{ L - x_n \}$ y halle $\lim_{n\to\infty} \varepsilon_n$.
  
  \item Halle $L = \lim_{n\to\infty} (2n^2 + 6n -1)/(4n^2 + 2n + 1)$. Despu\'es determine $\{ \varepsilon_n \} = \{ L - x_n \}$ y halle $\lim_{n\to\infty} \varepsilon_n$.
 \end{enumerate}
\end{enunciado}

\begin{solucion}
 A partir de los conocimientos de l\'{\i}mites, se sabe que el l\'{\i}mite de un cociente de polinomios converge si y s\'olo si el grado del polinomio del numerados es menor o igual al grado del polinomio del denominador, y que, en caso de converger, el l\'{\i}mite es igual al cociente de los coeficientes en el t\'ermino que corresponde al de mayor grado del denominador, y, en caso de que el numerador tenga menor grado, entonces el l\'{\i}mite es cero. Por lo que $\lim_{x\to \infty} (4x+1)/(2x+1) = 4/2 = 2$ y $\lim_{x\to \infty} (2x^2+6x-1)/(4x^2+2x+1) = 2/4 = 1/2$. Lo que se har\'a a continuaci\'on es comprobar que estos dos resultados son tambi\'en, en efecto, los l\'{\i}mites de las sucesiones, usando la definici\'on formal. Para ello se usar\'a la propiedad arquimediana, la cual dice:
 \begin{equation*}
  \text{Sea } \varepsilon \text{ un n\'umero real cualquiera tal que } \varepsilon>0 \; \text{ entonces } \exists N \in \mathbb{N} \text{ tal que } n\varepsilon > 1
 \end{equation*}
 \begin{enumerate}[(a)]
  \item Se tiene que 
  \begin{eqnarray*}
   \lim_{n\to\infty} \frac{4n+1}{2n+1} = 2 & \Leftrightarrow & \forall \varepsilon > 0 \; \exists N\in\mathbb{N} \text{ tal que } \forall n > N: \; \left\lvert \frac{4n+1}{2n+1} - 2 \right\rvert < \varepsilon \\
   & \Leftrightarrow & \forall \varepsilon > 0 \; \exists N\in\mathbb{N} \text{ tal que } \forall n > N: \; \left\lvert \frac{-1}{2n+1} \right\rvert < \varepsilon \\
   & \Leftrightarrow & \forall \varepsilon > 0 \; \exists N\in\mathbb{N} \text{ tal que } \forall n > N: \; \frac{1}{2n+1} < \varepsilon \\
   & \Leftrightarrow & \forall \varepsilon > 0 \; \exists N\in\mathbb{N} \text{ tal que } \forall n > N: \; 1 < \varepsilon(2n+1) = 2\varepsilon n + \varepsilon \\
   & \Leftrightarrow & \forall \varepsilon > 0 \; \exists N\in\mathbb{N} \text{ tal que } \forall n > N: \; \frac{1-\varepsilon}{2\varepsilon} <  n
  \end{eqnarray*}
  Luego, como $\varepsilon > 0$, entonces $\frac{1-\varepsilon}{2\varepsilon} \leq 0$ cuando $\varepsilon \geq 1$, en ese caso $\forall n \in \mathbb{N}$ cumple que $\frac{1-\varepsilon}{2\varepsilon} <  n$. Y si $\varepsilon < 1$, entonces 
  \begin{eqnarray*}
   \lim_{n\to\infty} \frac{4n+1}{2n+1} = 2 & \Leftrightarrow & \forall \varepsilon > 0 \; \exists N\in\mathbb{N} \text{ tal que } \forall n > N: \; \frac{1-\varepsilon}{2\varepsilon} <  n \\
   & \Leftrightarrow & \forall \varepsilon > 0 \; \exists N\in\mathbb{N} \text{ tal que } \forall n > N: \; 1 < \left(  \frac{2\varepsilon}{1-\varepsilon} \right) n
  \end{eqnarray*}
  Luego entonces por la propiedad arquimediana se garantiza que existe un valor $N \in \mathbb{N}$ tal que $N\varepsilon' > 1$, donde $\varepsilon' = \frac{2\varepsilon}{1-\varepsilon} > 0$. Por lo tanto, sea $n \geq N$, se cumple que $1 < \left( \frac{2\varepsilon}{1-\varepsilon} \right) n$, por lo que $\lim_{n\to\infty} \frac{4n+1}{2n+1} = 2$, que es a lo que se quer\'{\i}a llegar.
  \par 
  Hacer notar que aqu\'{\i} s\'{\i} se puede expresar el valor de $N$ m\'{\i}nimo, tal que $N\varepsilon' > 1$, en t\'ermino de $\varepsilon$ como sigue:
  \begin{equation*}
   N = \left\{
   \begin{tabular}{ll}
    $1$ & si $\varepsilon \geq 1$ \\
    $\phantom{0}$
    \\
    $\left\lceil \frac{2\varepsilon}{1-\varepsilon} \right\rceil$ & si $\varepsilon < 1$ y $\frac{2\varepsilon}{1-\varepsilon} \neq \left\lceil \frac{2\varepsilon}{1-\varepsilon} \right\rceil$ \\
    $\phantom{0}$
    \\
    $\left\lceil \frac{2\varepsilon}{1-\varepsilon} \right\rceil + 1$ & si $\varepsilon < 1$ y $\frac{2\varepsilon}{1-\varepsilon} = \left\lceil \frac{2\varepsilon}{1-\varepsilon} \right\rceil$
   \end{tabular}
   \right.
  \end{equation*}
  Luego
  \begin{equation*}
   \varepsilon_n = L-x_n = 2 - \frac{4n+1}{2n+1} = \frac{(4n+2)-(4n+1)}{2n+1} = \frac{1}{2n+1}
  \end{equation*}
  y, por lo tanto
  \begin{eqnarray*}
   \lim_{n\to\infty} \varepsilon_n = \lim_{n\to \infty} \frac{1}{2n+1}
  \end{eqnarray*}
  y como ya se vio, $\forall \varepsilon > 0 \; \exists N \in \mathbb{N}$ tal que $\forall n > N$, se cumple que $\varepsilon > \frac{1}{2n+1} = \left\lvert \frac{1}{2n+1} - 0 \right\rvert = \left\lvert \frac{1}{2n+1} - L \right\rvert$, por lo que el l\'{\i}mite buscado, $L$, es $L = 0$, es decir
  \begin{equation*}
   \lim_{n\to\infty} \varepsilon_n = 0
  \end{equation*}

  \item Se tiene que
  \begin{eqnarray*}
   \lim_{n \to \infty} \frac{2n^2 + 6n - 1}{4n^2 + 2n + 1} = \frac{1}{2} & \Leftrightarrow & \forall \varepsilon > 0 \; \exists N\in\mathbb{N} \text{ tal que } \forall n > N: \; \left\lvert \frac{2n^2 + 6n - 1}{4n^2 + 2n + 1} - \frac{1}{2} \right\rvert < \varepsilon \\
   & \Leftrightarrow & \forall \varepsilon > 0 \; \exists N\in\mathbb{N} \text{ tal que } \forall n > N: \; \left\lvert \frac{5n - 3/2}{4n^2 + 2n + 1}\right\rvert < \varepsilon \\
  \end{eqnarray*}
  Luego, como $n$ es un entero positivo, se tiene que $5n-3/2 > 0$ y $4n^2 + 2n + 1 > 0$, por lo que $|(5n-3/2)/(4n^2 + 2n + 1)| = (5n-3/2)/(4n^2 + 2n + 1)$; adem\'as, $5n - 3/2 \leq 5n - (3/2) n = (13/2) n  < 7n$ y $4n^2 + 2n + 1 > 4n^2$. Por lo tanto
  \begin{equation*}
   \left\lvert \frac{5n - 3/2}{4n^2 + 2n + 1}\right\rvert = \frac{5n - 3/2}{4n^2 + 2n + 1} < \frac{7n}{4n^2} = \frac{7}{4n}
  \end{equation*}
  Por lo tanto, si $\forall \varepsilon > 0$ se encuentra un valor $N \in \mathbb{N}$ tal que $\forall n > N$ se cumple que $\frac{7}{4n} < \varepsilon$, entonces, en particular se cumplir\'a que 
  $\left\lvert \frac{5n - 3/2}{4n^2 + 2n + 1}\right\rvert < \varepsilon$. Luego
  \begin{equation*}
   \frac{7}{4n} < \varepsilon \Leftrightarrow 1 < \left(  \frac{4\varepsilon}{7} \right) n
  \end{equation*}
  Luego entonces por la propiedad arquimediana se garantiza que existe un valor $N \in \mathbb{N}$ tal que $N\varepsilon' > 1$, donde $\varepsilon' = \frac{4\varepsilon}{7} > 0$. Por lo tanto, sea $n \geq N$, se cumple que $1 < \left( \frac{4\varepsilon}{7} \right) n$ y, por ello $\left\lvert \frac{5n-3/2}{4n^2 + 2n + 1} \right\rvert < \varepsilon$.
  \par 
  N\'otese que el m\'{\i}nimo valor de $N$ tal que $N\varepsilon' > 1$ es 
  \begin{equation*}
   N = \left\{
   \begin{tabular}{ll}
    $\left\lceil \frac{7}{4n} \right\rceil$ & si $\frac{7}{4n} \neq \left\lceil \frac{7}{4n} \right\rceil$ \\
    $\phantom{0}$
    \\
    $\left\lceil \frac{7}{4n} \right\rceil + 1$ & si $\frac{7}{4n} = \left\lceil \frac{7}{4n} \right\rceil$
   \end{tabular}
   \right.
  \end{equation*}
  aunque no necesariamente \'este ser\'a el m\'{\i}nimo valor de $N$ para el que $\forall n > N$ se cumpla que $\left\lvert \frac{5n-3/2}{4n^2 + 2n + 1} \right\rvert < \varepsilon$.
  \par 
  Por lo tanto 
  \begin{equation*}
   \forall \varepsilon > 0 \; \exists N = \left\lceil \frac{7}{4\varepsilon} \right\rceil + 1 \in \mathbb{N} \text{ tal que } \forall n > N: \; \left\lvert \frac{2n^2 + 6n -1 }{4n^2 + 2n + 1} - \frac{1}{2} \right\rvert < \varepsilon 
  \end{equation*}
  Es decir, como se hab\'{\i}a dicho al principio, en efecto se tiene que 
  \begin{equation*}
   \lim_{n\to\infty} \frac{2n^2 + 6n -1 }{4n^2 + 2n + 1} = \frac{1}{2}
  \end{equation*}
  Luego
  \begin{equation*}
   \varepsilon_n = L - x_n = \frac{1}{2} - \frac{2n^2 + 6n -1 }{4n^2 + 2n + 1} = \frac{(2n^2 + n + 1/2) - (2n^2 + 6n - 1)}{4n^2 + 2n + 1} = \frac{-5n + 3/2}{4n^2 + 2n + 1}
  \end{equation*}
  y, por lo tanto
  \begin{equation*}
   \lim_{n\to\infty} \varepsilon_n = \lim_{n\to\infty} \frac{-5n + 3/2}{4n^2 + 2n + 1}
  \end{equation*}
  y, como ya se vio, $\forall\varepsilon > 0$ $\exists N \in \mathbb{N}$ tal que $\forall n > N$, se cumple que $\varepsilon > \left\lvert \frac{5n - 3/2}{4n^2 + 2n + 1} \right\rvert = \left\lvert \frac{-5n + 3/2}{4n^2 + 2n + 1} \right\rvert = \left\lvert \frac{-5n + 3/2}{4n^2 + 2n + 1} - 0 \right\rvert = \left\lvert \frac{-5n + 3/2}{4n^2 + 2n + 1} - L \right\rvert$, por lo que el l\'{\i}mite buscado, $L$, es $L = 0$, es decir
  \begin{equation*}
   \lim_{n\to\infty} \varepsilon_n = 0
  \end{equation*}
  que es a lo que se quer\'{\i}a llegar.${}_{\blacksquare}$
 \end{enumerate}
\end{solucion}

 % Si escribo \include{Problema_1}, se escribirá el contenido en una hoja a parte.
 \item \begin{enunciado}
 Sea $\{ x_n \}_{n=1}^\infty$ una sucesi\'on tal que $\lim_{n \to \infty} x_n = 2$.
 \begin{multicols}{2}
  \begin{enumerate}[(a)]
   \item Halle $\lim_{n\to\infty} \sin(x_n)$.
   
   \item Halle $\lim_{n\to\infty} \ln(x_n^2)$
  \end{enumerate}
 \end{multicols}
\end{enunciado}

\begin{solucion}
 Para resolver esto se usar\'a la propiedad de continuidad que dice que si $f$ es continua en $S$ y si $\{ x_n \}_{n=1}^{\infty} \subset S$ y $\lim_{n\to\infty} x_n = x_0$, entonces $\lim_{n\to\infty} f(x_n) = f(x_0)$.
 \begin{enumerate}[(a)]
  \item Como $\sin(x)$ es continuo en $S = \mathbb{R}$, entonces es claro que $\{ x_n \}_{n=1}^{\infty} \subset S$, y como $\lim_{n\to\infty} x_n = 2$, entonces 
  \begin{equation*}
   \lim_{n\to\infty} \sin(x_n) = \sin(2) \approx 0.909297
  \end{equation*}

  \item Como $x^2$ es continuo en $S = \mathbb{R}$, entonces es claro que $\{ x_{n} \}_{n=1}^{\infty} \subset S$, y como $\lim_{n\to\infty} x_n = 2$, entonces
  \begin{equation*}
   \lim_{n\to\infty} x_n^2 = (2)^2 = 4
  \end{equation*}
  Luego, se tiene que $\ln(x)$ es continuo en $S = \{ x\in \mathbb{R} | \, x > 0 \}$ y $x_n^2 \geq 0$ para todo $n \in \mathbb{N}$, por lo que queda por revisar los casos cuando $x_n^2 = 0$. Dado que el $\lim_{n\to\infty} x_n^2 = 4$, entonces se tiene que para todo $\varepsilon > 0$, existe un n\'umero natural $N(\varepsilon)$ tal que para todo $n>N$ se cumple que $|x_n^2 - 4| < \varepsilon$. Para el caso en particular que $\varepsilon = 4$, entonces $x_n^2$ es mayor a $0$ para todo $n > N$; luego entonces, sea $\{ x_n' \}_{n=1}^{\infty}$ la sucesi\'on id\'entica a $\{ x_n \}_{n=1}^{\infty}$, salvo que para todo $n \leq N$,  $x_n' = 1$, entonces $(x_n')^2$ es positivo para todo $n$ y, como \'unicamente se cambiaron una cantidad finita de valores, se tiene por teorema que $\lim_{n\to\infty} f(x_n) = \lim_{n\to\infty} f(x_n')$, para cualquiera que sea la funci\'on $f$.
  \par 
  Finalmente, como $\lim_{n\to\infty} \left( x_n' \right)^2 = 4$, $\ln(x)$ es continua en $S=\{ x\in\mathbb{R}| \, x> 0\}$ y $\{ (x_n')^2 \}_{n=1}^{\infty} \subset S$, entonces:
  \begin{equation*}
   \lim_{n\to\infty} \ln\left( x_n^2 \right) = \lim_{n\to\infty} \ln\left[ \left( x_n' \right)^2 \right] = \ln(4) \approx 1.386294
  \end{equation*}
 \end{enumerate}

\end{solucion}


 \item \begin{enunciado}
 Halle los n\'umeros $c$ cuya existencia garantiza el teorema del valor intermedio para cada una de las siguientes funciones, en el intervalo que se indica y para el valor de $L$ dado.
 \begin{enumerate}[(a)]
  \item $f(x) = -x^2 +2x +3$ en $[-1,0]$ para $L=2$.
  
  \item $f(x)=\sqrt{x^2 -5x -2}$ en $[6,8]$ para $L=3$.
 \end{enumerate}
\end{enunciado}

\begin{solucion}
 $\phantom{0}$
 \begin{enumerate}[(a)]
  \item Dado que $f(-1) = -(-1)^2 + 2(-1) + 3 = -1-2+3 = 0$ y $f(0) = -(0)^2 + 2(0) + 3 = 3$, entonces, en efecto, $L = 2 \in [f(-1), f(0)]$. Luego, por el teorema del valor intermedio, $\exists c \in [-1,0]$ tal que $f(c) = 2$. Igualando, y despejando se tiene lo siguiente:
  \begin{eqnarray*}
   & & -c^2 + 2c + 3 = 2 \\
   \Leftrightarrow & & c^2 -2c - 1 = 0 \\ 
   \Leftrightarrow & & (c-1)^2 -2 = 0 \\
   \Leftrightarrow & & (c-1)^2 = 2 \\
   \Leftrightarrow & & c-1 = \pm \sqrt{2} \\
   \Leftrightarrow & & c = 1 \pm \sqrt{2}
  \end{eqnarray*}
  Luego, los posibles valores de $c$ son $c = 1+\sqrt{2} > 0$ y $c = 1 - \sqrt{2}$. Como $c$ debe pertenecer al intervalo $[-1, 0]$, entonces el \'unico valor de $c$ que cumple lo pedido es $c = 1-\sqrt{2}\approx -0.4142$.
  
  \item Dado que $f(6) = \sqrt{(6)^2 - 5(6) - 2} = \sqrt{36 - 30 - 2} = \sqrt{4} = 2$ y $f(8) = \sqrt{(8)^2 - 5(8) - 2} = \sqrt{64 - 40 - 2} = \sqrt{22} > \sqrt{16} = 4$, entonces, en efecto, $L = 3 \in [f(6),f(8)]$. Luego, por el teorema del valor intermedio, $\exists c \in [6,8]$ tal que $f(c) = 3$. Igualando, y despejando se tiene lo siguiente:
  \begin{eqnarray*}
   & & \sqrt{c^2 - 5c - 2} = 3 \\ 
   \Leftrightarrow & & c^2 - 5c -2 = 3^2 = 9 \\
   \Leftrightarrow & & c^2 - 5x - 11 = 0 \\ 
   \Leftrightarrow & & \left( c - \frac{5}{2} \right)^2 - \frac{25}{4} - \frac{44}{4} = 0 \\
   \Leftrightarrow & & \left( c - \frac{5}{2} \right)^2 = \frac{69}{4} \\
   \Leftrightarrow & & c- \frac{5}{2} = \pm  \sqrt{\frac{69}{4}} = \frac{ \pm \sqrt{69}}{2} \\ 
   \Leftrightarrow & & c = \frac{5 \pm \sqrt{69}}{2}
  \end{eqnarray*}
  Luego, los valores de $c$ para los que $f(c) = 3$ son $c = \frac{5-\sqrt{69}}{2} < 0$ y $c = \frac{5+\sqrt{69}}{2} \approx 6.65331$. Como $c$ debe pertenecer al intervalo $[6, 8]$, entonces el \'unico valor de $c$ que cumple lo pedido es $c = \frac{5+\sqrt{69}}{2} \approx 6.65331$, que es a lo que se quer\'{\i}a llegar.${}_{\blacksquare}$
 \end{enumerate}
\end{solucion}


 \item \begin{enunciado}
 Halla la cota superior e inferior cuya existencia garantiza el teorema de los valores extremos para cada una de las siguientes funciones en el intervalo que se indica.
 \begin{enumerate}[(a)]
  \item $f(x) = x^2 - 3x + 1$ en $[-1, 2]$.
  \item $f(x) = \cos^2(x) - \sin(x)$ en $[0,2\pi]$.
 \end{enumerate}
\end{enunciado}

\begin{solucion}
 Para resolver los incisos, se recordar\'a que un punto cr\'{\i}tico es un m\'aximo o m\'{\i}nimo (o punto indefinido, que en estas funciones continuas no puede ser el caso) de un funci\'on. Los puntos cr\'{\i}ticos se pueden encontrar derivando una funci\'on e igualando a cero. Por lo tanto, se proceder\'a a obtener los valores $c$ tales que $f'(c) = 0$ en cada caso y determinar si se encuentran en el intervalo dado. Adem\'as, como el se trata de m\'aximos y m\'{\i}nimos en un intervalo, no se descarta la posibilidad de que estos se encuentren en los extremos del intervalo, por lo que al final se evaluar\'an en $f$ los valores $c$ previamente encontrados y los extremos de los intervalos.
 \begin{enumerate}[(a)]
  \item Dado que $f(x) = x^2 - 3x +1$, entonces $f'(x) = 2x - 3$, entonces, si $c$ es un punto cr\'{\i}tico, se cumple que $f'(c) = 2c - 3 = 0$; luego, $c = \frac{3}{2}$, el cual, en efecto, se encuentra en el intervalo $[-1,2]$. Finalmente, se eval\'uan los extremos y este valor $c$:
  \begin{eqnarray*}
   f(-1) & = & (-1)^2 - 3(-1) + 1 = 1 + 3 + 1 = 5 \\
   f\left( \frac{3}{2} \right) & = & \left( \frac{3}{2} \right)^2 - 3\left( \frac{3}{2} \right) + 1 = \frac{9}{4} - \frac{9}{2} + 1 = \frac{9-18+4}{4} = \frac{-5}{4} = -1.25 \\
   f(2) & = & (2)^2 - 3(2) + 1 = 4 - 6 + 1 = -1
  \end{eqnarray*}
  Por lo tanto, esta funci\'on en el intervalo $[-1,2]$ tiene como cota inferior a $M_1 = -1.25 = f(3/2)$ y tiene como cota superior a $M_2 = 5 = f(-1)$.
  
  \item Dado que $f(x) = \cos^2(x) - \sin(x)$, entonces, al derivar $f$, se tiene que $f'(x) = -2\cos(x)\sin(x) - \cos(x) = -\cos(x)\left[ 2\sin(x) + 1 \right]$, entonces, si $c$ es un punto cr\'{\i}tico, se cumple que $f(c) = -\cos(c)\left[ 2\sin(c) + 1 \right] = $, lo cual ocurre si y s\'olo si $\cos(c) = 0$ o $2\sin(c) = 1$, donde $\cos(c) = 0$ si y s\'olo si $c = 2\pi n \pm \frac{\pi}{2}$ y $\sin(c) = -1/2$ si y s\'olo si $c = \frac{3\pi}{2} \pm \frac{\pi}{3}$; luego, los puntos cr\'{\i}ticos en el intervalo $[0,2\pi]$ son: $c_1 = \frac{\pi}{2}$, $c_2 = \frac{3\pi}{2}$, $c_3 = \frac{7\pi}{6}$ y $c_4 = \frac{11\pi}{6}$. Finalmente, se eval\'uan los extremos y estos valores de $c$:
  \begin{eqnarray*}
   f(0) & = & \cos^2(0) - \sin(0) = 1 - 0 = 1 \\
   f\left( \frac{\pi}{2} \right) & = & \cos^2 \left( \frac{\pi}{2} \right) - \sin\left( \frac{\pi}{2} \right) = 0 - 1 = -1 \\ 
   f\left( \frac{3\pi}{2} \right) & = & \cos^2 \left( \frac{3\pi}{2} \right) - \sin\left( \frac{3\pi}{2} \right) = 0 - (-1) = 1 \\ 
   f\left( \frac{7\pi}{6} \right) & = & \cos^2 \left( \frac{7\pi}{6} \right) - \sin\left( \frac{7\pi}{6} \right) = \left( -\frac{ \sqrt{3} }{2} \right)^2 - \left( -\frac{1}{2} \right) = \frac{3}{4} + \frac{1}{2} = \frac{5}{4} \\ 
   f\left( \frac{11\pi}{6} \right) & = & \cos^2 \left( \frac{11\pi}{6} \right) - \sin\left( \frac{11\pi}{6} \right) = \left( \frac{ \sqrt{3} }{2} \right)^2 - \left( -\frac{1}{2} \right) = \frac{3}{4} + \frac{1}{2} = \frac{5}{4} \\ 
   f(2\pi) & = & \cos^2(2\pi) - \sin(2\pi) = 1 - 0 = 1
  \end{eqnarray*}
  Por lo tanto, esta funci\'on en el intervalo $[-1,2]$ tiene como cota inferior $M_1 = -1 = f(\pi/2)$ y tiene como cota superior a $M_2 = 5/4 = f(7\pi/6) = f(11\pi/6)$, que es a lo que se quer\'{\i}a llegar.${}_{\blacksquare}$

 \end{enumerate}
\end{solucion}


 \item \begin{enunciado}
 Halle los n\'umeros $c$ cuya existencia garantiza el teorema de Rolle para cada una de las siguientes funciones en el intervalo que se indica.
 \begin{enumerate}[(a)]
  \item $f(x) = x^4 - 4x^2$ en $[-2,2]$.
  \item $f(x) = \sin(x) + \sin(2x)$ en $[0,2\pi]$.
 \end{enumerate}
\end{enunciado}

\begin{solucion}
 Antes de resolver cada inciso, se verificar\'an las condiciones del teorema de Rolle; es decir, en efecto, $f(a) = f(b) = 0$ para cada funci\'on $f$ y su respectivo intervalo $[a,b]$ de los incisos. La continuidad de $f$ y $f'$ se est\'a obviando, puesto que se sabe los polinomios y las funciones $\sin(x)$ y $\cos(x)$ cumplen que son continuas en cualquiera de sus derivadas.
 \begin{enumerate}[(a)]
  \item Dado que $f(-2) = (-2)^4 - 4(-2)^2 = 16 - 16 = 0$ y $f(2) = (2)^4 - 4(2)^2 = 16 - 16 = 0$, entonces se cumplen las hip\'otesis para aplicar el teorema de Rolle. Luego, $f'(x) = 4x^3 - 8x$, por lo que, igualando a $f'(c)$ a $0$, se obtiene que:
  \begin{eqnarray*}
   & & 4c^3 - 8c = 0 \\
   \Leftrightarrow & & 4c(c^2 - 2) = 0 \\
   \Leftrightarrow & & 4c(c-\sqrt{2})(c+\sqrt{2}) = 0
  \end{eqnarray*}
  Por lo tanto, los valores $c$ tales $f'(c) = 0$ son $c = -\sqrt{2}$, $c= 0$ y $c = \sqrt{2}$, los cuales se encuentran en el intervalo $[-2,2]$ por lo que son todos estos valores v\'alidos.
  Es decir, los valores buscados son: $c_1 = -\sqrt{2}$, $c_2 = 0$ y $c_3 = \sqrt{2}$.
  
  \item Dado que $f(0) = \sin(0) + \sin[2(0)] = 0 + 0 = 0$ y $f(2\pi) = \sin(2\pi) + \sin[2(2\pi)] = 0 + 0 = 0$, entonces se cumplen las hip\'otesis para aplicar el teorema de Rolle. Luego, $f'(x) = \cos(x) + 2\cos(2x)$, por lo que, igualando $f'(c)$ a $0$, se obtiene que:
  \begin{eqnarray*}
   & & \cos(c) + 2\cos(2c) = 0 \\
   \Leftrightarrow & & \cos(c) + 2\left[ \cos^2(c) - \sin^2(c) \right] = 0 \\
   \Leftrightarrow & & \cos(c) + 2\cos^2(c) - 2\sin^2(c) = 0 \\ 
   \Leftrightarrow & & \cos(c) + 2\cos^2(c) - 2\left[ 1 - \cos^2(c) \right] = 0 \\
   \Leftrightarrow & & \cos(c) + 2\cos^2(c) - 2 + 2\cos^2(c) = 0 \\
   \Leftrightarrow & & 4\cos^2(c) + \cos(c) - 2 = 0
  \end{eqnarray*}
  Luego entonces, sea $u = \cos(c)$, se tiene la ecuaci\'on cuadr\'atica $4u^2 + u - 2 = 0$ cuya soluci\'on es
  \begin{equation*}
   u = \frac{-b\pm\sqrt{b^2 - 4ac}}{2a} = \frac{-1 \pm \sqrt{1 + 32}}{8} = \frac{1 \pm \sqrt{33}}{8}
  \end{equation*}
  Por lo tanto, $\cos(c) = \frac{1\pm \sqrt{33}}{8}$ y $c = \arccos\left( \frac{1\pm \sqrt{33}}{8} \right)$, adem\'as, tambi\'en hay que considerar que la funci\'on $\arccos(x)$ da las soluciones en el intervalo $[0,\pi]$, pero como $\cos(x)$ es una funci\'on par y peri\'odica con peri\'odo de $2\pi$, se tiene adem\'as otras dos soluciones, tomando en cuenta que:
  \begin{eqnarray*}
   \cos\left[ \arccos\left( \frac{1+ \sqrt{33}}{8} \right) \right] & = & \cos \left[ - \arccos\left( \frac{1+ \sqrt{33}}{8} \right)  \right] = \cos\left[ 2\pi - \arccos\left( \frac{1+ \sqrt{33}}{8} \right) \right] \\
   \cos\left[ \arccos\left( \frac{1- \sqrt{33}}{8} \right) \right] & = & \cos\left[- \arccos\left( \frac{1- \sqrt{33}}{8} \right) \right] = \cos \left[2\pi - \arccos\left( \frac{1- \sqrt{33}}{8} \right) \right]
  \end{eqnarray*}
  Por lo tanto, los valores $c$ tales que $f'(c) = 0$ son:
  \begin{eqnarray*}
   c_1 & = & \arccos\left( \frac{1+ \sqrt{33}}{8} \right) \approx 0.935929 \\
   c_2 & = & \arccos\left( \frac{1- \sqrt{33}}{8} \right) \approx 2.573763 \\
   c_3 & = & 2\pi - \arccos\left( \frac{1- \sqrt{33}}{8} \right) \approx 3.709422 \\
   c_4 & = & 2\pi - \arccos\left( \frac{1+ \sqrt{33}}{8} \right) \approx 5.3472585 
  \end{eqnarray*}
  que es a lo que se quer\'{\i}a llegar.${}_{\blacksquare}$
 \end{enumerate}
\end{solucion}




 \item \begin{enunciado}
 Halle los n\'umeros $c$ cuya existencia garantiza el teorema del valor medio para cada una de las siguientes funciones en el intervalo que se indica.
 \begin{enumerate}[(a)]
  \item $f(x) = \sqrt{x}$ en $[0,4]$.
  \item $f(x) = \displaystyle{ \frac{x^2}{x+1} }$ en $[0,1]$.
 \end{enumerate}
\end{enunciado}

\begin{solucion}
 $\phantom{0}$
 \begin{enumerate}[(a)]
  \item Dado que $f(4) = \sqrt{4} = 2$ y $f(0) = \sqrt{0} = 0$, entonces se quiere encontrar un valor $c\in[0,4]$ tal que $f'(c) = \frac{f(4)-f(0)}{4-0} = \frac{2-0}{4-0} = \frac{2}{4}=\frac{1}{2}$. Como $f'(x) = \frac{1}{2\sqrt{x}}$, entonces igualando $f'(c) =\frac{1}{2}$, se obtiene que:
  \begin{eqnarray*}
   & & \frac{1}{2\sqrt{c}} = \frac{1}{2} \\
   \Leftrightarrow & & 2\sqrt{c} = 2 \\
   \Leftrightarrow & & \sqrt{c} = 1 \\
   \Leftrightarrow & & c = 1
  \end{eqnarray*}
  Por lo tanto, el valor $c\in[0,4]$ buscado es $c=1$.
  
  \item Dado que $f(1) = \frac{1^2}{1+1} = \frac{1}{2}$ y $f(0) = \frac{0^2}{0+1} = \frac{0}{1}=0$, entonces se quiere encontrar un valor $c\in[0,1]$ tal que $f'(c) = \frac{1/2 - 0}{1-0} = \frac{1/2}{1} = \frac{1}{2}$. Como $f'(x)$ es igual a:
  \begin{equation*}
   \left( \frac{x^2}{x+1} \right)' = \frac{(x^2)'(x+1) - (x^2)(x+1)'}{(x+1)^2} = \frac{2x(x+1) - x^2}{(x+1)^3} = \frac{2x^2 + 2x - x^2}{x+1} = \frac{x^2 + 2x}{(x+1)^2}
  \end{equation*}
  Entonces, igualando $f'(c) =\frac{1}{2}$, se obtiene que:
  \begin{eqnarray*}
   & & \frac{c^2+2c}{(c+1)^2} = \frac{1}{2} \\
   \Leftrightarrow & & \frac{2(c^2+2c)}{2(c+1)^2} - \frac{(c+1)^2}{2(c+1)^2} = 0\\
   \Leftrightarrow & & \frac{2c^2 + 4c - c^2 - 2c - 1}{2(c+1)^2} = 0 \\
   \Leftrightarrow & & c^2 + 2c - 1 = 0 \\ 
   \Leftrightarrow & & (c+1)^2 -2 = 0 \\
   \Leftrightarrow & & c = -1 \pm\sqrt{2}
  \end{eqnarray*}
  Como $c\in[0,1]$, entonces se descarta la soluci\'on $c=-1-\sqrt{2}$ y, por lo tanto, el \'unico valor $c\in [0,1]$ que cumple lo pedido es $c= \sqrt{2}-1 \approx 0.41421356$, que es a lo que se quer\'{\i}a llegar.${}_{\blacksquare}$
 \end{enumerate}
\end{solucion}


 \item \begin{enunciado}
 Aplique el teorema de Rolle generalizado a la funci\'on $f(x) = x(x-1)(x-3)$ en el intervalo $[0,3]$.
\end{enunciado}

\begin{solucion}
 Dado que $f(0)=f(1)=f(3)=0$ y existen $f'$ y $f''$, entonces se puede aplicar el teorema de Rolle generalizado, el cual garantiza que existe un valor $c \in [0,3]$ tal que $f''(c)=0$. Como $f''(x) = \left[x(x-1)(x-3)\right]'' = \left( x^3-4x^2+3x \right)'' = \left( 3x^2 - 8x + 3 \right)' = 6x - 8$, entonces, igualando $f''(c) = 0$ y despejando, se tiene que $6c - 8 = 0$ si y s\'olo si $c = \frac{8}{6} = \frac{4}{3}$. Por lo tanto, el valor $c$ que garantiza el teorema de Rolle generalizado es $c = \frac{4}{3} \approx 1.333333$, que es a lo que se quer\'{\i}a llegar.${}_{\blacksquare}$
\end{solucion}


 \item \begin{enunciado}
 Aplique el primer teorema fundamental del c\'alculo a cada una de las siguientes funciones en el intervalo que se indica.
 \begin{enumerate}[(a)]
  \item $f(x) = xe^x$ en $[0,2]$.
  
  \item $f(x) = \displaystyle{ \frac{3x}{x^2 + 1} }$ en $[-1,1]$
 \end{enumerate}
\end{enunciado}

\begin{solucion}
 $\phantom{0}$
 \begin{enumerate}[(a)]
  \item Dado que $x$ y $e^x$ son funciones continuas en $\mathbb{R}$, entonces $xe^x$ es continua en $\mathbb{R}$, en particular, la funci\'on es continua en $[0,2]$. Luego entonces, se buscar\'a una de sus primitivas, para lo cual se usar\'a el m\'etodo de la antidervaci\'on por partes, haciendo $u = x$ y $v = e^x$, con lo que $du = dx$ y $dv = e^x dx$. Entonces:
  \begin{equation*}
   \int xe^x \, dx = \int u\,dv = uv - \int v\,du = xe^x - \int e^x \, dx = xe^x - e^x + C
  \end{equation*}
  Por lo tanto, $xe^x - e^x$ es una primitiva de $xe^x$ y por lo tanto, usando el primer teorema fundamental del c\'alculo, se tiene que:
  \begin{equation*}
   \int_{0}^{2} xe^x \, dx = \left[ (2)e^{(2)} - e^{(2)} \right] - \left[ (0)e^{(0)} - e^{(0)} \right] = 2e^2 - e^2 - 0 + 1 = 2e^2 + 1 \approx 15.778112.
  \end{equation*}

  \item Dado que $\frac{3x}{x^2 + 1}$ es un cociente de polinomios donde $x^2 + 1 \neq 0$ para todo $x$, entonces la funci\'on es continua en $\mathbb{R}$, particularmente en $[-1,1]$. Luego entonces, se buscar\'a una de sus primitivas, para lo cual se usar\'a el m\'etodo de sustituci\'on, haciendo $u = x^2 + 1$, con lo que $du = 2x \, dx$. Entonces:
  \begin{equation*}
   \int \frac{3x}{x^2 + 1} dx = \frac{3}{2} \int \frac{2x}{x^2} dx = \frac{3}{2} \int \frac{du}{u} = \frac{3}{2} \ln(u) + C = \frac{3}{2} \ln(x^2 + 1) + C
  \end{equation*}
  Por lo tanto, $\frac{3}{2} \ln\left(x^2 + 1\right)$ es una primitiva de $\frac{3x}{x^2 + 1}$ y por lo tanto, usando el primer teorema fundamental del c\'alculo, se tiene que:
  \begin{equation*}
   \int_{-1}^{1} \frac{3x}{x^2+1}\, dx = \frac{3}{2}\ln\left[ (1)^2 + 1 \right] - \frac{3}{2}\ln\left[ (-1)^2 + 1 \right] = \frac{3}{2}\ln(2) - \frac{3}{2}\ln(2) = 0
  \end{equation*}
  que es a lo que se quer\'{\i}a llegar.${}_{\blacksquare}$
 \end{enumerate}
\end{solucion}

 \item \begin{enunciado}
 Aplique el segundo teorema fundamental del c\'alculo a cada una de las siguientes funciones:
 \par 
 \begin{multicols}{2}
  \begin{enumerate}[(a)]
   \item $\frac{d}{dx} \int_{0}^{x} t^2\cos(t) \, dt$.
   
   \item $\frac{d}{dx} \int_{1}^{x^3} e^{t^2}\, dt$.
  \end{enumerate}
 \end{multicols}
\end{enunciado}

\begin{solucion}
 N\'otese que si $F$ es una funci\'on primitiva de la funci\'on $f$ continua en el rango entre $a(x)$ y $b(x)$ (independientemente de quien sea mayor), entonces, por el primer teorema fundamental del c\'alculo, se tiene que:
 \begin{equation*}
  \int_{a(x)}^{b(x)} f(t) \,dt = F\left( b(x) \right) - F\left( a(x) \right)
 \end{equation*}
 por lo que, derivando con respecto a $x$ y usando la regla de la cadena, se tiene que
 \begin{equation*}
  \frac{d}{dx} \int_{a(x)}^{b(x)} f(t) \,dt = \frac{d}{dx}\left[ F\left( b(x) \right) - F\left( a(x) \right) \right] = f\left( b(x) \right)b'(x) - f\left( a(x) \right) a'(x)
 \end{equation*}
 Por lo tanto, la hip\'otesis del segundo teorema fundamental del c\'alculo que dice que $x\in(a,b)$, donde $a$ es el l\'{\i}mite inferior de la integral y $b$ es un punto hasta donde es continua $f$, desde $a$, puede ampliarse para cualquier $x$ en donde es continua $f$, incluyendo si $x<a$. Esto se usar\'a en las soluciones siguientes.
 \begin{enumerate}[(a)]
  \item Usando la previa aclaraci\'on y dado que $t^2$ y $\cos(t)$ son continuas en $\mathbb{R}$, entonces $f(t) = t^2\cos(t)$ es continua en $\mathbb{R}$, entonces, para cualquier valor de $x$, se cumple que:
  \begin{equation*}
   \frac{d}{dx} \int_0^x t^2 \cos(t)\, dt = x^2\cos(x)
  \end{equation*}
  
  \item Usando la previa aclaraci\'on y dado que $e^t$ es continua para toda $t \in \mathbb{R}$, entonces $f(t) = e^{t^2}$ es continua en $\mathbb{R}$, por lo que, independientemente del valor de $x$ y usando la regla de la cadena, se tiene que
  \begin{equation*}
   \frac{d}{dx} \int_{1}^{x^3} e^{t^2} \, dt = e^{\left( x^3 \right)^2}  \left( x^3 \right)' = e^{x^6} (3x^2) = 3x^2 e^{x^6}.
  \end{equation*}
  Que es a lo que se quer\'{\i}a llegar.${}_{\blacksquare}$
 \end{enumerate}
\end{solucion}

 \newpage
 \item \begin{enunciado}
 Halle los n\'umeros $c$ cuya existencia garantiza el teorema del valor medio para integrales para cada una de las siguientes funciones en el intervalo que se indica.
 \begin{enumerate}[(a)]
  \item $f(x) = 6x^2$ en $[-3,4]$.
  
  \item $f(x) = x\cos(x)$ en $[0,3\pi/2]$.
 \end{enumerate}
\end{enunciado}

\begin{solucion}
 $\phantom{0}$
 \begin{enumerate}[(a)]
  \item Dado que
  \begin{equation*}
   \int_{-3}^{4} 6x^2 \, dx = \left. 2x^3 \right|_{-3}^{4} = 2(4)^3 - 2(-3)^3 = 2(64) - 2(-27) = 128 + 54 = 182
  \end{equation*}
  entonces 
  \begin{equation*}
   \frac{1}{4-(-3)} \int_{-3}^{4} 6x^2 \, dx = \frac{182}{7} = 26
  \end{equation*}
  por lo que, al igualar $6c^2$ a $26$, se tiene que $c^2 = \frac{26}{6} = \frac{13}{3}$, entonces $c = \pm \sqrt{\frac{13}{3}}$. Como ambas soluciones pertenecen al intervalo $[-3,4]$, entonces ambas son v\'alidas. Por lo tanto, los valores $c$ buscados son:
  \begin{eqnarray*}
   c_1 & = & \sqrt{\frac{13}{3}} \approx 2.0816659994661 \\
   c_2 & = & - \sqrt{\frac{13}{3}} \approx - 2.0816659994661
  \end{eqnarray*}

  \item Por el m\'etodo de integraci\'on por partes y haciendo $u = x$ y $v = \sin(x)$, por lo que $du = dx$ y $dv = \cos(x) dx$, se sigue entonces que
  \begin{equation*}
   \int x\cos(x)\, dx = \int udv = uv - \int vdu = x\sin(x) - \int \sin(x) \, dx = x\sin(x) + \cos(x)
  \end{equation*}
  Luego entonces se tiene que
  \begin{eqnarray*}
   \int_{0}^{3\pi/2} x\cos(x) \, dx & = & \left[ x\sin(x) + \cos(x) \right]_{0}^{3\pi/2} \\
   & = & \left[ \left( \frac{3\pi}{2} \right)\sin\left( \frac{3\pi}{2} \right) + \cos\left( \frac{3\pi}{2} \right) \right] - \left[ (0)\sin(0) + \cos(0) \right] \\
   & = & \left[ \left( \frac{3\pi}{2} \right)(-1)+0\right]- \left[(0)(0) + (1) \right] \\
   & = & -\frac{3\pi}{2} - 1 = - \frac{2 + 3\pi}{2}
  \end{eqnarray*}
  Por lo que
  \begin{equation*}
   \frac{1}{3\pi/2 - 0} \int_{0}^{3\pi/2} x\cos(x) \, dx = \left( \frac{\cancel{2}}{3\pi} \right) \left( -\frac{2+3\pi}{\cancel{2}}  \right) = - \frac{2+3\pi}{3\pi} = -1- \frac{2}{3\pi}
  \end{equation*}
  Luego entonces, hay dos valores $c$ tales que $c\cos(c) = -\frac{2+3\pi}{3\pi}$ en el intervalo $[0,3\pi/2]$ y estos son:
  \begin{eqnarray*}
   c_1 & \approx & 2.1650566 \\
   c_2 & \approx & 4.435575
  \end{eqnarray*}
  que es a lo que se quer\'{\i}a llegar.${}_{\blacksquare}$
 \end{enumerate}
\end{solucion}

 \item \begin{enunciado}
 Halle la suma de cada una de las siguientes sucesiones o series.
 \begin{multicols}{2}
  \begin{enumerate}[(a)]
   \item $\displaystyle{ \left\{ \frac{1}{2^n} \right\}_{n=0}^{\infty} }$.
   \item $\displaystyle{ \left\{ \frac{2}{3^n} \right\}_{n=1}^{\infty} }$.
  \end{enumerate}
 \end{multicols}
 \begin{multicols}{2}
  \begin{enumerate}[(a)]
   \setcounter{enumii}{2}
   \item $\displaystyle{ \sum_{n=1}^{\infty} \frac{3}{n(n+1)} }$.
   \item $\displaystyle{ \sum_{k=1}^{\infty} \frac{1}{4k^2 - 1} }$.
  \end{enumerate}
 \end{multicols}
\end{enunciado}

\begin{solucion}
 Recordar que las series de potencias convergen absolutamente si $\left\lvert \frac{1}{p} \right\rvert < 1$ y $\sum_{n=1}^{\infty} \left( \frac{1}{p} \right)^n = \frac{1}{p-1}$. Entonces se tiene lo siguiente:
 \begin{enumerate}[(a)]
  \item Usando directamente el resultado de series de potencia, se tiene que
  \begin{equation*}
   \sum_{n=0}^{\infty} \frac{1}{2^n} = 1 + \sum_{n=1}^{\infty} \frac{1}{2^n} = 1 + \frac{1}{2-1} = 1 + 1 = 2.
  \end{equation*}

  \item Dado que la serie converge, se puede sacar la constante $2$ del numerador y se tiene que
  \begin{equation*}
   \sum_{n=1}^{\infty} \frac{2}{3^n} = 2 \sum_{n=1}^{\infty} \frac{1}{3^n} = 2\left( \frac{1}{3-1} \right) = \frac{2}{2} = 1
  \end{equation*}

  \item Usando el resultado del ejemplo del libro, se tiene que la serie $\sum_{n=1}^{\infty} \frac{1}{n(n+1)}$ converge a $1$, por lo que se tiene que
  \begin{equation*}
   \sum_{n=1}^{\infty} \frac{3}{n(n+1)} = 3\sum_{n=1}^{\infty} \frac{1}{n(n+1)} = 3(1) = 3.
  \end{equation*}

  \item Finalmente, por fracciones parciales, se va a separar un t\'ermino gen\'erico de la serie como sigue:
  \begin{equation*}
   \frac{1}{4k^2 - 1} = \frac{1}{(2k-1)(2k+1)} = \frac{A}{2k-1} + \frac{B}{2k+1} = \frac{2k(A+B) + (A-B)}{4k^2 - 1}
  \end{equation*}
  Por lo que $A+B = 0$ y $A-B = 1$, de ello entonces que $A = 1/2$ y $B = -1/2$. Por lo tanto
  \begin{equation*}
   \frac{1}{4k^2 - 1} = \frac{1}{2(2k-1)} - \frac{1}{2(2k+1)} = \frac{1}{4k-2} - \frac{1}{4k+2}
  \end{equation*}
  N\'otese que para el siguiente valor de $k$, se tiene que
  \begin{equation*}
   \frac{1}{4(k+1)-2} - \frac{1}{4(k+1)+2} = \frac{1}{4k + 2} - \frac{1}{4(k+1)+2}
  \end{equation*}
  por lo que se cancelan t\'erminos y por la propiedad telesc\'opica se tiene que, como 
  \begin{equation*}
   \sum_{k=1}^{n} \frac{1}{4k^2-1} = \sum_{k=1}^{n} \left( \frac{1}{4k-2} - \frac{1}{4k+2} \right) = \frac{1}{4(1)-2} - \frac{1}{4(n)+2} = \frac{1}{2} - \frac{1}{4n+2}
  \end{equation*}
  entonces
  \begin{equation*}
   \sum_{k=1}^{\infty} \frac{1}{4k^2 - 1} = \frac{1}{2} - \lim_{n\to \infty} \frac{1}{4n+2} = \frac{1}{2} - 0 = \frac{1}{2}
  \end{equation*}
  que es a lo que se quer\'{\i}a llegar.${}_{\blacksquare}$
 \end{enumerate}

\end{solucion}

 \item \begin{enunciado}
 Halle el polinomio de Taylor de grado $n=4$ para cada una de las siguientes funciones alrededor del punto $x_0$ dado.
 \begin{enumerate}[(a)]
  \item $f(x) = \sqrt{x}$, $x_0 = 1$.
  \item $f(x) = x^5 + 4x^2 + 3x + 1$, $x_0 = 0$.
  \item $f(x) = \cos(x)$, $x_0 = 0$.
 \end{enumerate}
\end{enunciado}

\begin{solucion}
 Para obtener los polinomios de Taylor de grado $n=4$, n\'otese que son necesarias las primeras cuatro derivadas. Por lo tanto, en cada soluci\'on, lo primero que se realizar\'a ser\'a la obtenci\'on de las funciones derivadas requeridas y la evaluaci\'on de cada una de \'estas en el punto $x_0$ alrededor de donde se piden los polinomios.
 \begin{enumerate}[(a)]
  \item Dado que
  \begin{center}
   \begin{tabular}{rclcrcl}
    $f(x)$ & $=$ & $\sqrt{x}$ & \hspace{2cm} & $f(1)$ & $=$ & $1$ \\
    \\
    $f'(x)$ & $=$ & $\displaystyle{ \frac{1}{2\sqrt{x}} }$ & & $f'(1)$ & $=$ & $\displaystyle{ \frac{1}{2} }$ \\
    \\
    $f''(x)$ & $=$ & $\displaystyle{ -\frac{1}{4x^{3/2}} }$ & & $f'(1)$ & $=$ & $\displaystyle{ - \frac{1}{4} }$ \\
    \\
    $f^{(3)}(x)$ & $=$ & $\displaystyle{ \frac{3}{8x^{5/2}} }$ & & $f(1)$ & $=$ & $\displaystyle{ \frac{3}{8} }$ \\
    \\
    $f^{(4)}(x)$ & $=$ & $\displaystyle{ -\frac{15}{16x^{7/2}} }$ & & $f(1)$ & $=$ & $\displaystyle{ - \frac{15}{16} }$
   \end{tabular}
  \end{center}
  Entonces el polinomio de Taylor, de grado $4$, para esta funci\'on alrededor de $x_0 = 1$ es
  \begin{eqnarray*}
   P_4(x) & = & \sum_{k=0}^{4} \frac{f^{(k)}(x_0)}{k!} (x-x_0)^k \\
   & = & \frac{1}{0!} (x-1)^{0} + \frac{1/2}{1!} (x-1)^{1} - \frac{1/4}{2!} (x-1)^{2} + \frac{3/8}{3!} (x-1)^{3} - \frac{15/16}{4!} (x-1)^{4} \\
   & = & 1 + \frac{1}{2}(x-1) - \frac{1}{8}(x-1)^2 + \frac{1}{16}(x-1)^3 - \frac{5}{128}(x-1)^4
%    & = & 1 + \frac{x-1}{2} + \frac{x^2 - 2x + 1}{8} + \frac{x^3 - 3x^2 + 3x - 1}{16} + \frac{5x^4 - 20x^3 + 30x^2 - 20x + 5}{128} \\
%    & = & \frac{128 + 64x - 64  + 16x^2 - 32x + 16  + 8x^3 - 24x^2 + 24x - 8 + 5x^4 - 20x^3 + 30x^2 - 20x + 5}{128} \\
%    & = & \frac{77 + 36x + 22x^2 - 12x^3 + 5x^4}{128} \\ 
%    & = & \frac{77}{128} + \frac{9x}{32} + \frac{11x^2}{64} - \frac{3x^3}{32} + \frac{5x^4}{128}
  \end{eqnarray*}

  
  \item Dado que 
  \begin{center}
   \begin{tabular}{rclcrcl}
    $f(x)$ & $=$ & $x^5 + 4x^2 + 3x + 1$ & \hspace{2cm} & $f(0)$ & $=$ & $1$ \\
    $f'(x)$ & $=$ & $5x^4 + 8x + 3$ & & $f'(0)$ & $=$ & $3$ \\
    $f''(x)$ & $=$ & $20x^3 + 8$ & & $f''(0)$ & $=$ & $8$ \\
    $f^{(3)}(x)$ & $=$ & $60x^2$ & & $f^{(3)}(0)$ & $=$ & $0$ \\
    $f^{(4)}(x)$ & $=$ & $120x$ & & $f^{(4)}(0)$ & $=$ & $0$
   \end{tabular}
  \end{center}
  Entonces el polinomio de Taylor, de grado $4$, para esta funci\'on alrededor de $x_0 = 0$ es
  \begin{eqnarray*}
   P_4(x) & = & \sum_{k=0}^{4} \frac{f^{(k)}(x_0)}{k!}(x-x_0)^k \\
   & = & \frac{1}{0!}(x-0)^0 + \frac{3}{1!}(x-0)^1 + \frac{8}{2!}(x-0)^2 + \frac{0}{3!}(x-0)^3 + \frac{0}{4!}(x-0)^4 \\
   & = & 1 + 3x + 4x^2
  \end{eqnarray*}
  
  \item Finalmente, dado que
  \begin{center}
   \begin{tabular}{rclcrcl}
    $f(x)$ & $=$ & $\cos(x)$ & \hspace{2cm} & $f(0)$ & $=$ & $1$ \\
    $f'(x)$ & $=$ & $-\sin(x)$ & & $f'(0)$ & $=$ & $0$ \\
    $f''(x)$ & $=$ & $-\cos(x)$ & & $f''(0)$ & $=$ & $-1$ \\
    $f^{(3)}(x)$ & $=$ & $\sin(x)$ & & $f^{(3)}(0)$ & $=$ & $0$ \\
    $f^{(4)}(x)$ & $=$ & $\cos(x)$ & & $f^{(4)}(0)$ & $=$ & $1$
   \end{tabular}
  \end{center}
  Entonces el polinomio de Taylor, de grado $4$, para esta funci\'on alrededor de $x_0 = 1$ es
  \begin{eqnarray*}
   P_4(x) & = & \sum_{k=0}^{4} \frac{f^{(k)}(x_0)}{k!}(x-x_0)^k \\
   & = & \frac{1}{0!}(x-0)^0 + \frac{0}{1!}(x-0)^1 + \frac{-1}{2!}(x-0)^2 + \frac{0}{3!}(x-0)^3 + \frac{1}{4!}(x-0)^4 \\
   & = & 1 - \frac{x^2}{2!} + \frac{x^4}{4!}
  \end{eqnarray*}
  que es a lo que se quer\'{\i}a llegar.${}_{\blacksquare}$
 \end{enumerate}
\end{solucion}

 \item \begin{enunciado}
 Sean $f(x) = \sin(x)$ y $P(x) = x - x^3/3! + x^5/5! - x^7/7! + x^9/9!$. Pruebe que $P^{(k)}(0) = f^{(k)}(0)$ para $k=1,2,\ldots, 9$.
\end{enunciado}

\begin{solucion}
 Dado que
 \begin{center}
  \begin{tabular}{rclcrcl}
   $P'(x)$ & $=$ & $1 - x^2/2! + x^4/4! - x^6/6! + x^8/8!$ & \hspace{1cm} $\Rightarrow$ \hspace{1cm} & $P'(0)$ & $=$ & $1$ \\
   $P''(x)$ & $=$ & $-x + x^3/3! - x^5/5! + x^7/7!$ & $\Rightarrow$ & $P''(0)$ & $=$ & $0$ \\
   $P^{(3)}(x)$ & $=$ & $-1 + x^2/2! - x^4/4! + x^6/6!$ & $\Rightarrow$ & $P^{(3)}(0)$ & $=$ & $-1$ \\
   $P^{(4)}(x)$ & $=$ & $x - x^3/3! + x^5/5!$ & $\Rightarrow$ & $P^{(4)}(0)$ & $=$ & $0$ \\
   $P^{(5)}(x)$ & $=$ & $1 - x^2/2! + x^4/4!$ & $\Rightarrow$ & $P^{(5)}(0)$ & $=$ & $1$ \\
   $P^{(6)}(x)$ & $=$ & $-x + x^3/3!$ & $\Rightarrow$ & $P^{(6)}(0)$ & $=$ & $0$ \\
   $P^{(7)}(x)$ & $=$ & $-1 + x^2/2!$ & $\Rightarrow$ & $P^{(7)}(0)$ & $=$ & $-1$ \\
   $P^{(8)}(x)$ & $=$ & $x$ & $\Rightarrow$ & $P^{(8)}(0)$ & $=$ & $0$ \\ 
   $P^{(9)}(x)$ & $=$ & $1$ & $\Rightarrow$ & $P^{(9)}(0)$ & $=$ & $1$
  \end{tabular}
 \end{center}
 Mientras que, por otro lado, se tiene que
 \begin{center}
  \begin{tabular}{rclcrcl}
   $f'(x)$ & $=$ & $\cos(x)$ & \hspace{1cm} $\Rightarrow$ \hspace{1cm} & $f'(0)$ & $=$ & $1$ \\
   $f''(x)$ & $=$ & $-\sin(x)$ & $\Rightarrow$ & $f''(0)$ & $=$ & $0$ \\
   $f^{(3)}(x)$ & $=$ & $-\cos(x)$ & $\Rightarrow$ & $f^{(3)}(0)$ & $=$ & $-1$ \\
   $f^{(4)}(x)$ & $=$ & $\sin(x)$ & $\Rightarrow$ & $f^{(4)}(0)$ & $=$ & $0$ \\
   $f^{(5)}(x)$ & $=$ & $\cos(x)$ & $\Rightarrow$ & $f^{(5)}(0)$ & $=$ & $1$ \\
   $f^{(6)}(x)$ & $=$ & $-\sin(x)$ & $\Rightarrow$ & $f^{(6)}(0)$ & $=$ & $0$ \\
   $f^{(7)}(x)$ & $=$ & $-\cos(x)$ & $\Rightarrow$ & $f^{(7)}(0)$ & $=$ & $-1$ \\
   $f^{(8)}(x)$ & $=$ & $\sin(x)$ & $\Rightarrow$ & $f^{(8)}(0)$ & $=$ & $0$ \\
   $f^{(9)}(x)$ & $=$ & $\cos(x)$ & \hspace{1cm} $\Rightarrow$ \hspace{1cm} & $f^{(9)}(0)$ & $=$ & $1$
  \end{tabular}
 \end{center}
 Por lo tanto, se tiene que $P^{(k)}(0) = f^{(k)}(0)$ para $k=1,2,\ldots, 9$, que es lo que se quer\'{\i}a probar.${}_{\blacksquare}$
\end{solucion}

 \newpage
 \item \begin{enunciado}
 Use divisi\'on sint\'etica (el m\'etodo de Horner) para hallar $P(c)$ en los siguientes casos.
 \begin{enumerate}[(a)]
  \item $P(x) = x^4 + x^3 - 13x^2 - x - 12$, $c=3$.
  
  \item $P(x) = 2x^7 + x^6 + x^5 - 2x^4 - x + 23$, $c = -1$.
 \end{enumerate}
\end{enunciado}

\begin{solucion}
 $\phantom{0}$
 \begin{enumerate}[(a)]
  \item Usando la tabla de Horner para el proceso de divisi\'on sint\'etica, resulta lo siguiente:
  \begin{center}
   \begin{tabular}{ccccccc}
    & $a_4$ & $a_3$ & $a_2$ & $a_1$ & $a_0$ \\
    \cline{1-1} 
    \multicolumn{1}{c|}{Dato} & $1$ & $1$ & $-13$ & $-1$ & $-12$ \\
    \multicolumn{1}{c|}{$c=3$} & & $3$ & $12$ & $-3$ & $-12$ \\
    \hline 
    & $1$ & $4$ & $-1$ & $-4$ & \multicolumn{2}{|l}{$-24 = P(3) = b_0$} \\
    & $b_4$ & $b_3$ & $b_2$ & $b_1$ & \multicolumn{2}{|c}{Resultado} \\
    \cline{6-7}
   \end{tabular}
  \end{center}
  Por lo tanto $P(3) = -24$.
  
  \item Usando la tabla de Horner para el proceso de divisi\'on sint\'etica, resulta lo siguiente:
  \begin{center}
   \begin{tabular}{cccccccccc}
    & $a_7$ & $a_6$ & $a_5$ & $a_4$ & $a_3$ & $a_2$ & $a_1$ & $a_0$ \\
    \cline{1-1} 
    \multicolumn{1}{c|}{Dato} & $2$ & $1$ & $1$ & $-2$ & $0$ & $0$ & $-1$ & $23$ \\
    \multicolumn{1}{c|}{$c=-1$} & & $-2$ & $1$ & $-2$ & $4$ & $-4$ & $4$ & $-3$ \\
    \hline 
    & $2$ & $-1$ & $2$ & $-4$ & $4$ & $-4$ & $3$ & \multicolumn{2}{|l}{$20 = P(-1) = b_0$} \\
    & $b_7$ & $b_6$ & $b_5$  & $b_4$ & $b_3$ & $b_2$ & $b_1$ & \multicolumn{2}{|c}{Resultado} \\
    \cline{9-10}
   \end{tabular}
  \end{center}
  Por lo tanto $P(-1) = 20$, que es a lo que se quer\'{\i}a llegar.${}_{\blacksquare}$
 \end{enumerate}
\end{solucion}

 \item \begin{enunciado}
 Halle el \'area media de todos los c\'{\i}rculos centrados en el origen cuyo radio est\'a comprendido entre $1$ y $3$.
\end{enunciado}

\begin{solucion}
 Sea $A(r)$ el \'area del c\'{\i}rculo centrado en el origen y cuyo radio, $r$, cumple que $1\leq r \leq 3$, entonces $A(r) = \pi r^2$. Luego entonces, el \'area media de todos los c\'{\i}rculos centrados en el origen cuyo radio est\'a comprendido entre $1$ y $3$ estar\'a dado por la altura media de la curva $A(r)$ en el intervalo $[1,3]$. Esto se calcula por medio de la integral como sigue:
 \begin{equation*}
  \frac{1}{3-1} \int_{1}^{3} A(r) \, dr = \frac{1}{2} \int_{1}^{3} \pi r^2 \, dr = \frac{\pi}{2} \left[ \frac{r^3}{3} \right]_{1}^{3} = \frac{\pi}{2}\left( \frac{(3)^3}{3} - \frac{(1)^3}{3}\right) = \frac{\pi(27-1)}{6} = \frac{26\pi}{6} = \frac{13\pi}{3}
 \end{equation*}
 Por lo tanto, $\displaystyle{ \frac{13\pi}{3} }$ es el valor medio de las \'areas mencionadas, que es a lo que se quer\'{\i}a llegar.${}_{\blacksquare}$
\end{solucion}

 \item \begin{enunciado}
 Supongamos que un polinomio $P(x)$ tiene $n$ ra\'{\i}ces reales en el intervalo $[a,b]$. Pruebe que $P^{(n-1)}(x)$ tiene al menos una ra\'{\i}z real en dicho intervalo.
\end{enunciado}

\begin{solucion}
 Dado que $P(x)$ es un polinomio, entonces existen sus primeras $n-1$ derivadas en todos los reales, en particular en $[a,b]$. N\'otese adem\'as que, como tiene $n$ ra\'{\i}ces reales (tambi\'en v\'alido si fuesen complejos), entonces $P(x)$ es de al menos grado $n$, por lo que la $(n-1)-$\'esima derivada es distinto la polinomio nulo. Luego, sean $x_0, x_1, \ldots, x_{n-1} \in [a,b]$ las $n$ ra\'{\i}ces reales, como $f(x_i) = 0$ para toda $j \in \mathbb{Z}\cap[0,n-1]$, entonces se puede aplicar el teorema de Rolle generalizado el cual asegura que existe alg\'un valor $c\in (a,b)$ tal que $P^{(n-1)}(c) = 0$, es decir, se garantiza que existe al menos una ra\'{\i}z, $c$, en el intervalo dado para el polinomio $P^{(n-1)}(x)$, que es a lo que se quer\'{\i}a llegar.${}_{\blacksquare}$
\end{solucion}

 \item \begin{enunciado}
 Supongamos que $f$, $f'$ y $f''$ est\'an definidas en un intervalo $[a,b]$, que $f(a) = f(b) = 0$ y que $f(c) > 0$ para todo $c \in (a,b)$. Pruebe que existe un n\'umero $d \in (a,b)$ tal que $f''(d) < 0$.
\end{enunciado}

\begin{solucion}
 Por teorema se sabe que si una funci\'on es derivable en un punto, entonces esta funci\'on es continua en dicho punto. Entonces, dado que $f$ es derivable en todo punto del intervalo $[a,b]$, entonces $f$ es continua en todo el intervalo $[a,b]$, an\'alogamente, por ser $f'$ derivable en $[a,b]$ se tiene que tambi\'en es continua $[a,b]$.
 \par 
 Por otro lado, sea $\displaystyle{m = \frac{a+b}{2}}$, entonces $m \in (a,b)$ y, por la suposiciones del enunciado, se deduce de ello que $f(m) > 0$. N\'otese que $m - a = \frac{a+b}{2} - \frac{2a}{2} = \frac{b-a}{2}$ y $m-b = \frac{a+b}{2} - \frac{2b}{2} = \frac{a-b}{2}$, es decir $m-a = -(m-b)$; adem\'as, como $b>a$, entonces $m-a = \frac{b-a}{2} > 0$ y $m-b = -\frac{b-a}{2} < 0$.
 \par 
 Entonces, como $f$ es continua en $[a,b]=[a,m]\cup[m,b]$ y $f'(x)$ existe para todo $x\in(a,m)$ y para todo $x\in(m,b)$, se puede usar el teorema del valor medio que garantiza la existencia de valores $c_1\in(a,m)$ y $c_2 \in (m,b)$ tales que $f'(c_1) = \frac{f(m)-f(a)}{m-a}$ y $f'(c_2) = \frac{f(b) - f(m)}{b-m}$, cuyas expresiones al simplificar resultan en que
 \begin{equation*}
  f'(c_1) = \frac{f(m) - f(a)}{m-a} = \frac{f(m)- 0}{m-a} = \frac{f(m)}{m-a} > 0
 \end{equation*}
 y
 \begin{equation*}
  f'(c_2) = \frac{f(b) - f(m)}{b-m} = \frac{0-f(m)}{-(m-b)} = -\frac{f(m)}{m-a} = - f'(c_1) < 0
 \end{equation*}
 Finalmente, como $f'$ es continua en $[a,b]$, en particular en $[c_1,c_2]$, y $f''(x)$ existe para todo $x\in(c_1,c_2)$, se puede aplicar una vez m\'as el teorema del valor medio y se tiene que existe un valor $d \in (c_1, c_2) \subset (a,b)$ tal que
 \begin{equation*}
  f''(d) = \frac{f(c_2) - f(c_1)}{c_2 - c_1} = \frac{2f(c_1)}{c_2 - c_1}
 \end{equation*}
 Como $f(c_1) < 0$ y $c_2 - c_1 > 0$, ya que $c_2 > c_1$, se tiene que $\displaystyle{ f''(d) = \frac{2f(c_1)}{c_2 - c_1} < 0 }$ y, por ello, el valor $d$ de este resultado cumple con todo lo pedido en el enunciado, es decir, se ha probado que existe un n\'umero $d \in (a,b)$ tal que $f''(d) < 0$. Q.E.D.${}_{\blacksquare}$
\end{solucion}

\end{enumerate}

\newpage

\section{N\'umeros binarios}

\begin{enumerate}
 \item \begin{enunciado}
 Use un computador para realizar las siguientes operaciones de forma acumulada; la intenci\'on es que el computador vaya haciendo las substracciones de forma repetida; sin emplear el atajo de la multiplicaci\'on.
 \begin{multicols}{2}
  \begin{enumerate}[(a)]
   \item $10\,000 - \sum_{k=1}^{100\,000} 0.1$
   \item $10\,000 - \sum_{k=1}^{80\,000} 0.125$
  \end{enumerate}
 \end{multicols}
\end{enunciado}

\begin{solucion}
 Las siguientes c\'alculos fueron realizadas con el programa Octave usando 64 cifras para representar n\'umeros reales, y maneja, aproximadamente entre 14 y 15 d\'{\i}gitos decimales de exactitud y que pueden ser observados hasta 14 de ellos ingresando la l\'{\i}nea de c\'odigo siguiente:
 \begin{verbatim}
> format long
 \end{verbatim}
  \vspace{-0.5cm}
 En cada inciso se muestra tambi\'en el c\'odigo en Octave con el que se hizo el c\'alculo.
 \begin{enumerate}[(a)]
  \item Usando las siguientes
  \begin{verbatim}
> a = 0;
> for i = 1:100000
>     a = a + 0.1;
> end
> 10000 - a
  \end{verbatim}
  \vspace{-0.5cm}
  de este modo se obtiene un resultado en pantalla como se muestra a continuaci\'on:
  \begin{verbatim}
-1.88483681995422e-08
  \end{verbatim}
  \vspace{-0.5cm}
  Es decir, seg\'un esto se tiene que
  \begin{equation*}
   10\,000 - \sum_{k=1}^{100\,000} 0.1 =
   -0.0000000188483681995422
  \end{equation*}
  
  \item Usando las siguientes
  \begin{verbatim}
> a = 0;
> for i = 1:80000
>     a = a + 0.125;
> end
> 10000 - a
  \end{verbatim}
  \vspace{-0.5cm}
  de este modo se obtiene un resultado en pantalla como se muestra a continuaci\'on:
  \begin{verbatim}
0
  \end{verbatim}
  \vspace{-0.5cm}
  Es decir, la resta dio exactamente $0$, que es  lo que se quer\'{\i}a llegar.${}_{\blacksquare}$
 \end{enumerate}
\end{solucion}

 \item \begin{enunciado}
 Use las relaciones (4) y (5) para convertir los siguientes n\'umeros binarios en su forma decimal (base 10).
 \begin{multicols}{2}
  \begin{enumerate}[(a)]
   \item $10101_{\text{dos}}$
   \item $111000_{\text{dos}}$
  \end{enumerate}
 \end{multicols}
 \begin{multicols}{2}
  \begin{enumerate}[(a)]
   \setcounter{enumii}{2}
   \item $11111110_{\text{dos}}$
   \item $1000000111_{\text{dos}}$
  \end{enumerate}
 \end{multicols}
\end{enunciado}

\begin{solucion}
 $\phantom{0}$
 \begin{enumerate}
  \item Desarrollando el n\'umero en base 2, se tiene que:
  \begin{eqnarray*}
   10101_{\text{dos}} 
   & = & 1 \times 2^{4} + 0 \times 2^{3} + 1 \times 2^{2} + 0\times 2^{1} + 1\times 2^{0} \\
   & = & 1\times 16 + 0\times 8 + 1\times 4 + 0 \times 2 + 1\times 1 \\
   & = & 16 + 4 + 1 \\
   & = & 21
  \end{eqnarray*}

  \item Desarrollando el n\'umero en base 2, se tiene que:
  \begin{eqnarray*}
   111000_{\text{dos}} 
   & = & 1\times 2^{5} + 1 \times 2^{4} + 1 \times 2^{3} + 0 \times 2^{2} + 0\times 2^{1} + 0 \times 2^{0} \\
   & = & 1\times 32 + 1\times 16 + 1\times 8 + 0 \times 4  + 0 \times 2 + 0\times 1 \\
   & = & 32 + 16 + 4 \\
   & = & 52
  \end{eqnarray*}

  \item Desarrollando el n\'umero en base 2, se tiene que:
  \begin{eqnarray*}
   11111110_{\text{dos}} 
   & = & 1\times 2^{7} + 1 \times 2^{6} + 1\times 2^{5} + 1 \times 2^{4} + 1 \times 2^{3} + 1 \times 2^{2} + 1 \times 2^{1} + 0 \times 2^{0} \\
   & = & 1 \times 128 + 1 \times 64 + 1\times 32 + 1\times 16 + 1\times 8 + 1 \times 4  + 1 \times 2 + 0\times 1 \\
   & = & 128 + 64 + 32 + 16 + 8 + 4 + 2 \\
   & = & 254
  \end{eqnarray*}

  \item Desarrollando el n\'umero en base 2, se tiene que:
  \begin{eqnarray*}
   1000000111_{\text{dos}} 
   & = & 1 \times 2^{9} + 0 \times 2^{8} + 0 \times 2^{7} + 0 \times 2^{6} + 0 \times 2^{5} + 0 \times 2^{4} \\ 
   & & + \; 0 \times 2^{3} + 1 \times 2^{2} + 1 \times 2^{1} + 0 \times 2^{0} \\
   & = & 1 \times 512 + 0 \times 256 + 0 \times 128 + 0 \times 64 + 0 \times 32 + 0 \times 16 \\
   & & + \; 0 \times 8 + 1 \times 4  + 1 \times 2 + 1 \times 1 \\
   & = & 512 + 4 + 2 + 1 \\
   & = & 519
  \end{eqnarray*}
  que es a lo que se quer\'{\i}a llegar.${}_{\blacksquare}$
 \end{enumerate}
\end{solucion}

 \item \begin{enunciado}
 Use las relaciones (16) y (17) para convertir las siguientes fracciones binarias en su forma decimal (base 10).
 \begin{multicols}{2}
  \begin{enumerate}[(a)]
   \item $0.11011_{\text{dos}}$
   \item $0.10101_{\text{dos}}$
  \end{enumerate}
 \end{multicols}
 \begin{multicols}{2}
  \begin{enumerate}[(a)]
   \setcounter{enumii}{2}
   \item $0.1010101_{\text{dos}}$
   \item $0.110110110_{\text{dos}}$
  \end{enumerate}
 \end{multicols}
\end{enunciado}

\begin{solucion}
 $\phantom{0}$
 \begin{enumerate}[(a)]
  \item Desarrollando, se obtiene que:
  \begin{eqnarray*}
   0.11011_{\text{dos}} & = & 1 \times 2^{-1} + 1 \times 2^{-2} + 0 \times 2^{-3} + 1 \times 2^{-4} + 1 \times 2^{-5} \\
   & = & 1 \times 0.5 + 1 \times 0.25 + 0 \times 0.125 + 1 \times 0.0625 + 1 \times 0.03125 \\
   & = & 0.5 + 0.25 + 0.0625 + 0.03125 \\
   & = & 0.84375 
  \end{eqnarray*}

  \item Desarrollando, se obtiene que:
  \begin{eqnarray*}
   0.10101_{\text{dos}} & = & 1 \times 2^{-1} + 0 \times 2^{-2} + 1 \times 2^{-3} + 0 \times 2^{-4} + 1 \times 2^{-5} \\
   & = & 1 \times 0.5 + 0 \times 0.25 + 1 \times 0.125 + 0 \times 0.0625 + 1 \times 0.03125 \\
   & = & 0.5 + 0.125 + 0.03125 \\
   & = & 0.65625
  \end{eqnarray*}

  \item Desarrollando, se obtiene que:
  \begin{eqnarray*}
   0.1010101_{\text{dos}} & = & 1 \times 2^{-1} + 0 \times 2^{-2} + 1 \times 2^{-3} + 0 \times 2^{-4} + 1 \times 2^{-5} + 0 \times 2^{-6} + 1 \times 2^{-7} \\
   & = & 1 \times 0.5 + 0 \times 0.25 + 1 \times 0.125 + 0 \times 0.0625 + 1 \times 0.03125 \\
   & & + \; 0 \times 0.015625 + 1 \times 0.0078125 \\
   & = & 0.5 + 0.125 + 0.03125 + 0.0078125 \\
   & = & 0.6640625
  \end{eqnarray*}

  \item Desarrollando, se obtiene que:
  \begin{eqnarray*}
   0.110110110_{\text{dos}} & = & 1 \times 2^{-1} + 1 \times 2^{-2} + 0 \times 2^{-3} + 1 \times 2^{-4} + 1 \times 2^{-5} + 0 \times 2^{-6} \\ 
   & & + \; 1 \times 2^{-7} + 1 \times 2^{-8} + 0 \times 2^{-9} \\
   & = & 1 \times 0.5 + 1 \times 0.25 + 0 \times 0.125 + 1 \times 0.0625 + 1 \times 0.03125 \\ 
   & & + \; 0 \times 0.015625 + 1 \times 0.0078125 + 1 \times 0.00390625 + 0 \times 0.001953125 \\
   & = & 0.5 + 0.25 + 0.0625 + 0.03125 + 0.0078125 + 0.00390625 \\
   & = & 0.85546875
  \end{eqnarray*}
  que es a lo que se quer\'{\i}a llegar.${}_{\blacksquare}$
 \end{enumerate}
\end{solucion}

 \item \begin{enunciado}
 Convierta los siguientes n\'umeros binarios en su forma decimal (base 10).
 \begin{multicols}{2}
  \begin{enumerate}[(a)]
   \item $1.0110101_{\text{dos}}$
   \item $11.0010010001_{\text{dos}}$
  \end{enumerate}
 \end{multicols}
\end{enunciado}

\begin{solucion}
 $\phantom{0}$
 \begin{enumerate}[(a)]
  \item Desarrollando, se obtiene que:
  \begin{eqnarray*}
   1.0110101_{\text{dos}} & = & 1 \times 2^{0} + 0 \times 2^{-1} + 1 \times 2^{-2} + 1 \times 2^{-3} + 0 \times 2^{-4} + 1 \times 2^{-5} \\ 
   & & + \; 0 \times 2^{-6} + 1 \times 2^{-7} \\
   & = & 1 \times 1 + 0 \times 0.5 + 1 \times 0.25 + 1 \times 0.125 + 0 \times 0.0625 + 1 \times 0.03125 \\ 
   & & + \; 0 \times 0.015625 + 1 \times 0.0078125 \\
   & = & 1 + 0.25 + 0.125 + 0.03125 + 0.0078125 \\ 
   & = & 1.4140625 
  \end{eqnarray*}

  \item Desarrollando, se obtiene que:
  \begin{eqnarray*}
   11.0010010001_{\text{dos}} & = & 1 \times 2^{1} + 1 \times 2^{0} + 0 \times 2^{-1} + 0 \times 2^{-2} + 1 \times 2^{-3} + 0 \times 2^{-4} + 0 \times 2^{-5} \\ 
   & & + \; 1 \times 2^{-6} + 0 \times 2^{-7} + 0 \times 2^{-8} + 0 \times 2^{-9} + 1 \times 2^{-10} \\
   & = & 1 \times 2 + 1 \times 1 + 0 \times 0.5 + 0 \times 0.25 + 1 \times 0.125 + 0 \times 0.0625 \\
   & & + \; 0 \times 0.03125 + 1 \times 0.015625 + 0 \times 0.0078125 + 0 \times 0.00390625 \\
   & & + \; 0 \times 0.001953125 + 1 \times 0.0009765625 \\
   & = & 2 + 1 + 0.125 + 0.015625 + 0.0009765625 \\ 
   & = & 3.1416015625
  \end{eqnarray*}
  que es a lo que se quer\'{\i}a llegar.${}_{\blacksquare}$
 \end{enumerate}
\end{solucion}

 \item \begin{enunciado}
 Los n\'umeros del Ejercicio 4 son aproximadamente $\sqrt{2}$ y $\pi$. Halle el error de dichas aproximaciones; es decir, halle
 \newline 
 \begin{tabular}{m{5cm}m{10cm}}
  \raggedleft $\phantom{0}$
  \begin{enumerate}[(a)]
   \item $\sqrt{2} - 1.0110101_{\text{dos}}$ 
  \end{enumerate}
  & (Use que $\sqrt{2} = 1.41421356237309\ldots$) \\ \vspace{-0.8cm} $\phantom{0}$
  \begin{enumerate}[(a)]
   \setcounter{enumii}{1}
   \item $\pi - 11.0010010001_{\text{dos}}$ 
  \end{enumerate}
  & \vspace{-0.8cm} (Use que $\pi = 3.14159265358979\ldots$)
 \end{tabular}
\end{enunciado}

\begin{solucion}
 Usando directamente los resultados del ejercicio anterior, se tiene que:
 \begin{enumerate}[(a)]
  \item Como $1.0110101_{\text{dos}} = 1.4140625$, entonces 
  \begin{eqnarray*}
   \sqrt{2} - 1.0110101_{\text{dos}} & = & \sqrt{2} - 1.4140625 \\
   & = & 1.41421356237309\ldots - 1.4140625 \\
   & = & 0.00015106237309\ldots
  \end{eqnarray*}

  \item Como $11.0010010001_{\text{dos}} = 3.1416015625$, entonces 
  \begin{eqnarray*}
   \pi - 11.0010010001_{\text{dos}} & = & \pi - 3.1416015625 \\
   & = & 3.14159265358979\ldots - 3.1416015625 \\
   & = & -0.00000890891021\ldots
  \end{eqnarray*}
  que es a lo que se quer\'{\i}a llegar.${}_{\blacksquare}$
 \end{enumerate}
\end{solucion}

 \item \begin{enunciado}
 Siga el Ejemplo 1.10 para convertir los siguientes n\'umeros en su forma binaria
 \begin{multicols}{4}
  \begin{enumerate}[(a)]
   \item $23$
   \item $87$
   \item $378$
   \item $2388$
  \end{enumerate}
 \end{multicols}
\end{enunciado}

\begin{solucion}
 Usando el algoritmo presentado, a trav\'es de divisiones iteradas entre dos y tomando los restos, se tienen las represetanciones como siguen.
 \begin{enumerate}[(a)]
  \item Empezando con $N=23$, se tiene que
  \begin{center}
   \begin{tabular}{rclrr}
    $23$ & $=$ & $2\times$ & $11+1$, & $b_0 = 1$ \\
    $11$ & $=$ & $2\times$ & $5 +1$, & $b_1 = 1$ \\
    $5$  & $=$ & $2\times$ & $2 +1$, & $b_2 = 1$ \\
    $2$  & $=$ & $2\times$ & $1 +0$, & $b_3 = 0$ \\
    $1$  & $=$ & $2\times$ & $0 +1$, & $b_4 = 1$
   \end{tabular}
  \end{center}
  As\'{\i} que la representaci\'on binaria de $23$ es
  \begin{equation*}
   23 = b_4 b_3 \ldots b_1b_{0}{}_{\text{dos}} = 10111_{\text{dos}}.
  \end{equation*}

  \item Empezando con $N=87$, se tiene que
  \begin{center}
   \begin{tabular}{rclrr}
    $87$ & $=$ & $2\times$ & $43+1$, & $b_0 = 1$ \\
    $43$ & $=$ & $2\times$ & $21 +1$, & $b_1 = 1$ \\
    $21$ & $=$ & $2\times$ & $10+1$, & $b_2 = 1$ \\
    $10$ & $=$ & $2\times$ & $5 +0$, & $b_3 = 0$ \\
    $5$  & $=$ & $2\times$ & $2 +1$, & $b_4 = 1$ \\
    $2$  & $=$ & $2\times$ & $1 +0$, & $b_5 = 0$ \\
    $1$  & $=$ & $2\times$ & $0 +1$, & $b_6 = 1$
   \end{tabular}
  \end{center}
  As\'{\i} que la representaci\'on binaria de $87$ es
  \begin{equation*}
   87 = b_6 b_5 b_4 \ldots b_1b_{0}{}_{\text{dos}} = 1010111_{\text{dos}}.
  \end{equation*}

  \item Empezando con $N=378$, se tiene que
  \begin{center}
   \begin{tabular}{rclrr}
    $378$ & $=$ & $2\times$ & $189+0$, & $b_0 = 0$ \\
    $189$ & $=$ & $2\times$ & $94 +1$, & $b_1 = 1$ \\
    $94$  & $=$ & $2\times$ & $47 +0$, & $b_2 = 0$ \\
    $47$  & $=$ & $2\times$ & $23 +1$, & $b_3 = 1$ \\
    $23$  & $=$ & $2\times$ & $11 +1$, & $b_4 = 1$ \\
    $11$  & $=$ & $2\times$ & $5  +1$, & $b_5 = 1$ \\
    $5$   & $=$ & $2\times$ & $2  +1$, & $b_6 = 1$ \\
    $2$   & $=$ & $2\times$ & $1  +0$, & $b_7 = 0$ \\
    $1$   & $=$ & $2\times$ & $0  +1$, & $b_8 = 1$
   \end{tabular}
  \end{center}
  As\'{\i} que la representaci\'on binaria de $378$ es
  \begin{equation*}
   378 = b_8 b_7 b_6 \ldots b_2b_1b_{0}{}_{\text{dos}} = 101111010_{\text{dos}}.
  \end{equation*}

  \item Empezando con $N=2388$, se tiene que
  \begin{center}
   \begin{tabular}{rclrr}
    $2388$ & $=$ & $2\times$ & $1194+0$, & $b_0    = 0$ \\
    $1194$ & $=$ & $2\times$ & $597 +0$, & $b_1    = 0$ \\
    $597$  & $=$ & $2\times$ & $298 +1$, & $b_2    = 1$ \\
    $298$  & $=$ & $2\times$ & $149 +0$, & $b_3    = 0$ \\
    $149$  & $=$ & $2\times$ & $74  +1$, & $b_4    = 1$ \\
    $74$   & $=$ & $2\times$ & $37  +0$, & $b_5    = 0$ \\
    $37$   & $=$ & $2\times$ & $18  +1$, & $b_6    = 1$ \\
    $18$   & $=$ & $2\times$ & $9   +0$, & $b_7    = 0$ \\
    $9$    & $=$ & $2\times$ & $4   +1$, & $b_8    = 1$ \\
    $4$    & $=$ & $2\times$ & $2   +0$, & $b_9    = 0$ \\
    $2$    & $=$ & $2\times$ & $1   +0$, & $b_{10} = 0$ \\
    $1$    & $=$ & $2\times$ & $0   +1$, & $b_{11} = 1$
   \end{tabular}
  \end{center}
  As\'{\i} que la representaci\'on binaria de $2388$ es
  \begin{equation*}
   2388 = b_{11} b_{10} b_9 \ldots b_2b_1b_0{}_{\text{dos}} = 100101010100_{\text{dos}}.
  \end{equation*}
  que es a lo que se quer\'{\i}a llegar.${}_{\blacksquare}$
 \end{enumerate}
\end{solucion}

 \item \begin{enunciado}
 Siga el Ejemplo 1.12 para convertir los siguientes n\'umeros en fracciones binarias de la forma $0.d_1d_2\cdots d_n{}_{\text{dos}}$.
 \begin{multicols}{4}
  \begin{enumerate}[(a)]
   \item $7/16$
   \item $13/16$
   \item $23/32$
   \item $75/128$
  \end{enumerate}
 \end{multicols}
\end{enunciado}

\begin{solucion}
 Para seguir el ejemplo mencionado, se hallar\'a primeramente una expresi\'on decimal de las fracciones, y, seguidamente, se realizar\'an las multiplicaciones por 2 iteradamente para obtener el resultado, como se muestra a continuaci\'on.
 \begin{enumerate}[(a)]
  \item Dado que $R = 7/16 = 0.4375$, entonces
  \begin{center}
   \begin{tabular}{rclrclrcl}
    & & & \hspace{1.5cm} & & \hspace{1.5cm} \\
    $2R$ & $=$ & $0.875$ & $d_1 =$ & $\text{ent}(0.875)$ & $=0$ & $F_1 =$ & $\text{frac}(0.875)$ & $=0.875$ \\
    $2F_1$ & $=$ & $1.75$ & $d_2 =$ & $\text{ent}(1.75)$ & $=1$ & $F_2 =$ & $\text{frac}(1.75)$ & $=0.75$ \\
    $2F_2$ & $=$ & $1.5$ & $d_3 =$ & $\text{ent}(1.5)$ & $=1$ & $F_3 =$ & $\text{frac}(1.5)$ & $=0.5$ \\
    $2F_3$ & $=$ & $1$ & $d_4 =$ & $\text{ent}(1)$ & $=1$ & $F_4 =$ & $\text{frac}(1)$ & $=0$ \\
   \end{tabular}
  \end{center}
  Por lo tanto
  \begin{equation*}
   \frac{7}{16} = 0.d_1d_2d_3d_4{}_{\text{dos}} = 0.0111_{\text{dos}}
  \end{equation*}

  \item Dado que $R = 13/16 = 0.8125$, entonces
  \begin{center}
   \begin{tabular}{rclrclrcl}
    & & & \hspace{1.5cm} & & \hspace{1.5cm} \\
    $2R$ & $=$ & $1.625$ & $d_1 =$ & $\text{ent}(1.625)$ & $=1$ & $F_1 =$ & $\text{frac}(1.625)$ & $=0.625$ \\
    $2F_1$ & $=$ & $1.25$ & $d_2 =$ & $\text{ent}(1.25)$ & $=1$ & $F_2 =$ & $\text{frac}(1.25)$ & $=0.25$ \\
    $2F_2$ & $=$ & $0.5$ & $d_3 =$ & $\text{ent}(0.5)$ & $=0$ & $F_3 =$ & $\text{frac}(0.5)$ & $=0.5$ \\
    $2F_3$ & $=$ & $1$ & $d_4 =$ & $\text{ent}(1)$ & $=1$ & $F_4 =$ & $\text{frac}(1)$ & $=0$ \\
   \end{tabular}
  \end{center}
  Por lo tanto
  \begin{equation*}
   \frac{13}{16} = 0.d_1d_2d_3d_4{}_{\text{dos}} = 0.1101_{\text{dos}}
  \end{equation*}

  \item Dado que $R = \frac{23}{32} = 0.71875$, entonces
  \begin{center}
   \begin{tabular}{rclrclrcl}
    & & & \hspace{1.5cm} & & \hspace{1.5cm} \\
    $2R$ & $=$ & $1.4375$ & $d_1 =$ & $\text{ent}(1.4375)$ & $=1$ & $F_1 =$ & $\text{frac}(1.4375)$ & $=0.4375$ \\
    $2F_1$ & $=$ & $0.875$ & $d_2 =$ & $\text{ent}(0.875)$ & $=0$ & $F_2 =$ & $\text{frac}(0.875)$ & $=0.875$ \\
    $2F_2$ & $=$ & $1.75$ & $d_3 =$ & $\text{ent}(1.75)$ & $=1$ & $F_3 =$ & $\text{frac}(1.75)$ & $=0.75$ \\
    $2F_3$ & $=$ & $1.5$ & $d_4 =$ & $\text{ent}(1.5)$ & $=1$ & $F_4 =$ & $\text{frac}(1.5)$ & $=0.5$ \\
    $2F_4$ & $=$ & $1$ & $d_5 =$ & $\text{ent}(1)$ & $=1$ & $F_5 =$ & $\text{frac}(1)$ & $=0$ \\
   \end{tabular}
  \end{center}
  Por lo tanto
  \begin{equation*}
   \frac{23}{32} = 0.d_1d_2d_3d_4d_5{}_{\text{dos}} = 0.10111_{\text{dos}}
  \end{equation*}

  \item Dado que $R = \frac{75}{128} = 0.5859375$, entonces
  \begin{center}
   \begin{tabular}{rclrclrcl}
    & & & \hspace{1.1cm} & & \hspace{1.1cm} \\
    $2R$ & $=$ & $1.171875$ & $d_1 =$ & $\text{ent}(1.171875)$ & $=1$ & $F_1 =$ & $\text{frac}(1.171875)$ & $=0.171875$ \\
    $2F_1$ & $=$ & $0.34375$ & $d_2 =$ & $\text{ent}(0.34375)$ & $=0$ & $F_2 =$ & $\text{frac}(0.34375)$ & $=0.34375$ \\
    $2F_2$ & $=$ & $0.6875$ & $d_3 =$ & $\text{ent}(0.6875)$ & $=0$ & $F_3 =$ & $\text{frac}(0.6875)$ & $=0.6875$ \\
    $2F_3$ & $=$ & $1.375$ & $d_4 =$ & $\text{ent}(1.375)$ & $=1$ & $F_4 =$ & $\text{frac}(1.375)$ & $=0.375$ \\
    $2F_4$ & $=$ & $0.75$ & $d_5 =$ & $\text{ent}(0.75)$ & $=0$ & $F_5 =$ & $\text{frac}(0.75)$ & $=0.75$ \\
    $2F_5$ & $=$ & $1.5$ & $d_6 =$ & $\text{ent}(1.5)$ & $=1$ & $F_6 =$ & $\text{frac}(1.5)$ & $=0.5$ \\
    $2F_6$ & $=$ & $1$ & $d_7 =$ & $\text{ent}(1)$ & $=1$ & $F_7 =$ & $\text{frac}(1)$ & $=0$ \\
   \end{tabular}
  \end{center}
  Por lo tanto
  \begin{equation*}
   \frac{75}{128} = 0.d_1d_2\ldots d_6d_7{}_{\text{dos}} = 0.1001011_{\text{dos}}
  \end{equation*}
  que es a lo que se quer\'{\i}a llegar.${}_{\blacksquare}$
 \end{enumerate}
\end{solucion}

 \item \begin{enunciado}
 Siga el Ejemplo 1.12 para convertir los siguientes n\'umeros en fracciones binarias peri\'odicas.
 \begin{multicols}{3}
  \begin{enumerate}[(a)]
   \item $1/10$
   \item $1/3$
   \item $1/7$
  \end{enumerate}
 \end{multicols}
\end{enunciado}

\begin{solucion}
 Para hallar la representaci\'on como en el ejemplo mencionado, se requirir\'a una representaci\'on decimal y, seguidamente, se realizar\'an las multiplicaciones por 2 iteradamente para obtener las partes enteras y las partes fraccionarias como se muestra a continuaci\'on.
 \begin{enumerate}[(a)]
  \item Dado que $R = \frac{1}{10} = 0.1$, entonces
  \begin{center}
   \begin{tabular}{rclrclrcl}
    & & & \hspace{1.5cm} & & \hspace{1.5cm} \\
    $2R$ & $=$ & $0.2$ & $d_1 =$ & $\text{ent}(0.2)$ & $=0$ & $F_1 =$ & $\text{frac}(0.2)$ & $=0.2$ \\
    $2F_1$ & $=$ & $0.4$ & $d_2 =$ & $\text{ent}(0.4)$ & $=0$ & $F_2 =$ & $\text{frac}(0.4)$ & $=0.4$ \\
    $2F_2$ & $=$ & $0.8$ & $d_3 =$ & $\text{ent}(0.8)$ & $=0$ & $F_3 =$ & $\text{frac}(0.8)$ & $=0.8$ \\
    $2F_3$ & $=$ & $1.6$ & $d_4 =$ & $\text{ent}(1.6)$ & $=1$ & $F_4 =$ & $\text{frac}(1.6)$ & $=0.6$ \\
    $2F_4$ & $=$ & $1.2$ & $d_5 =$ & $\text{ent}(1.2)$ & $=1$ & $F_5 =$ & $\text{frac}(1.2)$ & $=0.2$ \\
    $2F_5$ & $=$ & $0.4$ & $d_6 =$ & $\text{ent}(0.4)$ & $=0$ & $F_6 =$ & $\text{frac}(0.4)$ & $=0.4$ \\
   \end{tabular}
  \end{center}
  N\'otese que $2F_1 = 0.4 = 2F_5$, luego los patrones $d_k = d_{k+4}$ y $F_k = F_{k+4}$ se dar\'{\i}an para $k \in \mathbb{N}\backslash\{ 1 \}$. En consecuencia,
  \begin{equation*}
   \frac{1}{10} = 0.0\overline{0011}_{\text{dos}}
  \end{equation*}

  \item Dado que $R = \frac{1}{3} = 0.\overline{3}$, entonces 
  \begin{center}
   \begin{tabular}{rclrclrcl}
    & & & \hspace{1.5cm} & & \hspace{1.5cm} \\
    $2R$ & $=$ & $0.\overline{6}$ & $d_1 =$ & $\text{ent}\left(0.\overline{6}\right)$ & $=0$ & $F_1 =$ & $\text{frac}\left(0.\overline{6}\right)$ & $=0.\overline{6}$ \\
    $2F_1$ & $=$ & $1.\overline{3}$ & $d_2 =$ & $\text{ent}\left(1.\overline{3}\right)$ & $=1$ & $F_2 =$ & $\text{frac}\left(1.\overline{3}\right)$ & $=0.\overline{3}$ \\
    $2F_2$ & $=$ & $0.\overline{6}$ & $d_3 =$ & $\text{ent}\left(0.\overline{6}\right)$ & $=0$ & $F_3 =$ & $\text{frac}\left(0.\overline{6}\right)$ & $=0.\overline{6}$ \\
    $2F_3$ & $=$ & $1.\overline{3}$ & $d_4 =$ & $\text{ent}\left(1.\overline{3}\right)$ & $=1$ & $F_4 =$ & $\text{frac}\left(1.\overline{3}\right)$ & $=0.\overline{3}$ \\
   \end{tabular}
  \end{center}
  N\'otese que $2F_1 = 1.\overline{3} = 2F_3$, luego los patrones $d_k = d_{k+2}$ y $F_k = F_{k+2}$ se dar\'{\i}an para $k \in \mathbb{N}$. En consecuencia,
  \begin{equation*}
   \frac{1}{3} = 0.\overline{01}_{\text{dos}}
  \end{equation*}

  \item Dado que $R = \frac{1}{7} = 0.\overline{142857}$, entonces 
  \begin{center}
   \begin{tabular}{rcl}
    & \hspace{6cm} & \\
    $2R = 0.\overline{285714}$ & $d_1 = \text{ent}\left(0.\overline{285714}\right) = 0$ & $F_1 = \text{frac}\left(0.\overline{285714}\right) = 0.\overline{285714}$ \\
    $2F_1 = 0.\overline{571428}$ & $d_2 = \text{ent}\left(0.\overline{571428}\right) = 0$ & $F_2 = \text{frac}\left(0.\overline{571428}\right) = 0.\overline{571428}$ \\
    $2F_2 = 1.\overline{142857}$ & $d_3 = \text{ent}\left(1.\overline{142857}\right) = 1$ & $F_3 = \text{frac}\left(1.\overline{142857}\right) = 0.\overline{142857}$ \\
    $2F_3 = 0.\overline{285714}$ & $d_4 = \text{ent}\left(0.\overline{285714}\right) = 0$ & $F_4 = \text{frac}\left(0.\overline{285714}\right) = 0.\overline{285714}$ \\
    $2F_4 = 0.\overline{571428}$ & $d_5 = \text{ent}\left(0.\overline{571428}\right) = 0$ & $F_5 = \text{frac}\left(0.\overline{571428}\right) = 0.\overline{571428}$ \\
   \end{tabular}
  \end{center}
  N\'otese que $2F_1 = 0.\overline{571428} = F_4$, luego entonces los patrones $d_k = d_{k+3}$ y $F_k = F_{k+3}$ se dar\'{\i}an para $k \in \mathbb{N}$. En consecuencia
  \begin{equation*}
   \frac{1}{7} = 0.\overline{001}_{\text{dos}}
  \end{equation*}
  que es a lo que se quer\'{\i}a llegar${}_{\blacksquare}$
 \end{enumerate}
\end{solucion}

 \item \begin{enunciado}
 En las siguientes aproximaciones binarias con siete cifras significativas, halle el error de la aproximaci\'on $R - 0.d_1d_2d_3d_4d_5d_6d_7{}_{\text{dos}}$.
 \begin{multicols}{2}
  \begin{enumerate}[(a)]
   \item $R = 1/10 \approx 0.0001100_{\text{dos}}$
   
   \item $R = 1/7 \approx 0.0010010_{\text{dos}}$
  \end{enumerate}
 \end{multicols}
\end{enunciado}

\begin{solucion}
 $\phantom{0}$
 \begin{enumerate}[(a)]
  \item Ya que $1/10 = 0.1$ y $0.0001100_{\text{dos}} = 2^{-4} + 2^{-5} = 0.0625 + 0.03125 = 0.09375$, entonces
  \begin{equation*}
   R - 0.0001100_{\text{dos}} = 0.1 - 0.09375 = 0.00625
  \end{equation*}

  \item Ya que $0.0010010_{\text{dos}} = 2^{-3} + 2^{-6} = 0.125 + 0.015625 = 0.140625$ y $1/7 = 0.\overline{142857}$, entonces
  \begin{equation*}
   R - 0.0010010_{\text{dos}} = 0.\overline{142857} - 0.140625 = 0.002232\overline{142857}
  \end{equation*}
  que es a lo que se quer\'{\i}a llegar.${}_{\blacksquare}$
 \end{enumerate}
\end{solucion}

 \item \begin{enunciado}
 Pruebe que el desarrollo binario $1/7 = 0.\overline{001}_{\text{dos}}$ es equivalente a $\frac{1}{7} = \frac{1}{8} + \frac{1}{64} + \frac{1}{512} + \cdots$ y use el Teorema 1.14 para justificar dicho desarrollo.
\end{enunciado}

\begin{solucion}
 Desarrollando la expresi\'on $0.\overline{001}_{\text{dos}}$, se tiene lo siguiente
 \begin{eqnarray*}
  0.\overline{001}_{\text{dos}} & = & \left( 0\times 2^{-1} \right) + \left( 0\times 2^{-2} \right) + \left( 1\times 2^{-3} \right) + \left( 0\times 2^{-4} \right) + \left( 0\times 2^{-5} \right) + \left( 1\times 2^{-6} \right)  \cdots \\
  & = & 2^{-3} + 2^{-6} + 2^{-9} + \cdots \\
  & = & \displaystyle{ \sum_{k=1}^{\infty} \left( 2^{-3} \right)^k } \\
  \\
  & = & \displaystyle{ -1 + \sum_{k=0}^{\infty} \left( 2^{-3} \right)^k }
 \end{eqnarray*}
 Entonces, por el teorema 1.14, se sigue que $\displaystyle{ \sum_{k=0}^{\infty} \left( 2^{-3} \right)^k = \frac{1}{1 - 2^{-3}} }$, por lo que
 \begin{equation*}
  0.\overline{001}_{\text{dos}} = -1 + \frac{1}{1 - \frac{1}{8}} = -\frac{7}{7} + \frac{1}{\frac{7}{8}} = \frac{-7}{7} + \frac{8}{7} = \frac{1}{7}
 \end{equation*}
 Por lo tanto, como se vio que $0.\overline{001}_{\text{dos}} = 2^{-3} + 2^{-6} + 2^{-9} + \cdots = \frac{1}{8} + \frac{1}{64} + \frac{1}{512} + \cdots$, se concluye que es equivalente 
 \begin{equation*}
  \frac{1}{7} = \frac{1}{8} + \frac{1}{64} + \frac{1}{512} + \cdots
 \end{equation*}
 que es a lo que se quer\'{\i}a llegar.${}_{\blacksquare}$
\end{solucion}

 \item \begin{enunciado}
 Pruebe que el desarrollo binario $1/5 = 0.\overline{0011}_{\text{dos}}$ es equivalente a $\frac{1}{5} = \frac{3}{16} + \frac{3}{256} + \frac{3}{4096} + \cdots$ y use el Teorema 1.14 para justificar dicho desarrollo.
\end{enunciado}

\begin{solucion}
 Desarrollando la expresi\'on $0.\overline{0011}_{\text{dos}}$, se tiene lo siguiente
 \begin{eqnarray*}
  0.\overline{0011}_{\text{dos}} & = & \left( 0 \times 2^{-1} \right) + \left( 0 \times 2^{-2} \right) + \left( 1 \times 2^{-3} \right) + \left( 1 \times 2^{-4} \right) + \\
  & & + \; \left( 0 \times 2^{-5} \right) + \left( 0 \times 2^{-6} \right) + \left( 1 \times 2^{-7} \right) + \left( 1 \times 2^{-8} \right) + \cdots \\ 
  & = & 2^{-3} + 2^{-4} + 2^{-7} + 2^{-8} + 2^{-11} + 2^{-12} + \cdots \\ 
  & = & 2\times 2^{-4} + 2^{-4} + 2\times 2^{-8} + 2^{-8} + 2\times 2^{-12} + 2^{-12} + \cdots \\
  & = & 3\times 2^{-4} + 3\times 2^{-8} + 3\times 2^{-12} + \cdots \\
  & = & \frac{3}{16} + \frac{3}{256} + \frac{3}{4096} + \cdots \\
  & = & \displaystyle{ \sum_{k=1}^{\infty} \frac{3}{2^{4k}} = \sum_{k=1}^{\infty} 3\times \left( \frac{1}{2^{4}} \right)^k } \\
  \\
  & = & \displaystyle{ 3\sum_{k=1} \left( \frac{1}{2^{4}} \right)^k = -3 + 3 \sum_{k=0}^{\infty} \left( \frac{1}{2^4} \right)^k }
 \end{eqnarray*}
 queda entonces por probar que esta suma es, en efecto, $\frac{1}{5}$ para probar la equivalencia. Entonces, por el teorema 1.14, se sigue
 \begin{equation*}
  \sum_{k=0}^{\infty} \left( \frac{1}{2^4} \right)^k = \frac{1}{1-\frac{1}{2^4}} = \frac{1}{\frac{16}{16} - \frac{1}{16}} = \frac{1}{\frac{15}{16}} = \frac{16}{15}
 \end{equation*}
 por lo que se tiene que 
 \begin{equation*}
  0.\overline{0011}_{\text{dos}} = -3 + \cancel{3}\left( \frac{16}{\cancelto{5}{15}} \right)  = \frac{-15}{5} + \frac{16}{5} = \frac{1}{5}
 \end{equation*}
 que es a lo que se quer\'{\i}a llegar.${}_{\blacksquare}$
\end{solucion}

 \item \begin{enunciado}
 Pruebe que cualquier n\'umero $2^{-N}$, siendo $N$ un n\'umero natural, puede representarse como un n\'umero decimal con $N$ cifras significativas, es decir, $2^{-N} = 0.d_1d_2d_3\cdots d_N$. \textit{Indicaci\'on.} $1/2 = 0.5$, $1/4 = 0.25$, $\ldots$
\end{enunciado}

\begin{solucion}
 N\'otese que $2^{-N} = \left(  5^{N} \times 5^{-N} \right) \left( 2^{-N} \right) = 5^{N} \times \left( 2\times5 \right)^{-N} = 5^{N} \times 10^{-N}$. En el sistema decimal, $10^{-N}$ se representa como la unidad movido $N$ posiciones decimales, mientras que $5^N$ representa un n\'umero natural terminado en $5$, el cual, como es menor a $10^N$ que representa a la unidad movido $N$ posiciones a la izquierda, entonces $5^N$ tiene a lo sumo $N$ d\'{\i}gitos significativas, entonces se puede suponer que $5^N = d_1d_2d_3\cdots d_N$, posiblemente teniendo los primeros $k$ d\'{\i}gitos nulos, con $k\in \mathbb{N}\cap[1,N-1]$. Luego entonces el producto de $5^N$ por $10^{-N}$ representa el valor del n\'umero natural $5^N$ movido $N$ posiciones decimales, es decir
 \begin{equation*}
  2^{-N} = 5^N \times 10^{-N} = 0.d_1d_2d_3\cdots d_N
 \end{equation*}
 que es a lo que se quer\'{\i}a llegar.${}_{\blacksquare}$
\end{solucion}

 \item \begin{enunciado}
 Use la Tabla 1.3 para determinar qu\'e ocurre cuando un computador con una mantisa de cuatro cifras lleva a cabo los siguiente c\'alculos.
 \begin{multicols}{2}
  \begin{enumerate}[(a)]
   \item $\left( \frac{1}{3} + \frac{1}{5} \right) + \frac{1}{6}$
   \item $\left( \frac{1}{10} + \frac{1}{3} \right) + \frac{1}{5}$
  \end{enumerate}
 \end{multicols}
 \begin{multicols}{2}
  \begin{enumerate}[(a)]
   \setcounter{enumii}{2}
   \item $\left( \frac{3}{17} + \frac{1}{9} \right) + \frac{1}{7}$
   \item $\left( \frac{7}{10} + \frac{1}{9} \right) + \frac{1}{7}$
  \end{enumerate}
 \end{multicols}
\end{enunciado}

\begin{solucion}
 Para mayor comodidad, la tabla se anexa a continuaci\'on:
 \begin{center}
  \begin{tabular}{l|l|l|l|l|l|l|l|l}
   \hline 
    & \multicolumn{8}{c}{Exponente:} \\
   \cline{2-9}
   Mantisa & $n=-3$ & $n=-2$ & $n=-1$ & $n=0$ & $n=1$ & $n=2$ & $n=3$ & $n=4$ \\
   \hline 
   $0.1000_{\text{dos}}$ & $0.0625$ & $0.125$ & $0.25$ & $0.5$ & $1$ & $2$ & $4$ & $\phantom{1}8$ \\
   $0.1001_{\text{dos}}$ & $0.0703125$ & $0.140625$ & $0.28125$ & $0.5625$ & $1.125$ & $2.25$ & $4.5$ & $\phantom{1}9$ \\ 
   $0.1010_{\text{dos}}$ & $0.078125$ & $0.15625$ & $0.3125$ & $0.625$ & $1.25$ & $2.5$ & $5$ & $10$ \\
   $0.1011_{\text{dos}}$ & $0.0859375$ & $0.171875$ & $0.34375$ & $0.6875$ & $1.375$ & $2.75$ & $5.5$ & $11$ \\
   $0.1100_{\text{dos}}$ & $0.09375$ & $0.1875$ & $0.375$ & $0.75$ & $1.5$ & $3$ & $6$ & $12$ \\
   $0.1101_{\text{dos}}$ & $0.1015625$ & $0.203125$ & $0.40625$ & $0.8125$ & $1.625$ & $3.25$ & $6.5$ & $13$ \\
   $0.1110_{\text{dos}}$ & $0.109375$ & $0.21875$ & $0.4375$ & $0.875$ & $1.75$ & $3.5$ & $7$ & $14$ \\ 
   $0.1111_{\text{dos}}$ & $0.1171875$ & $0.234375$ & $0.46875$ & $0.9375$ & $1.875$ & $3.75$ & $7.5$ & $15$ \\
   \hline 
  \end{tabular}
 \end{center}
 Y las aproximaciones se har\'an redondeando al valor mayor o igual m\'as pr\'oximo en la tabla.
 \begin{enumerate}
  \item Dado que $1/3 = 0.\overline{3}$, $1/5 = 0.2$ y $1/6 = 0.1\overline{6}$, entonces los n\'umeros que m\'as se aproximan a estos en la tabla son $0.34375$, $0.203125$ y $0.171875$, que corresponden a los n\'umeros binarios $0.1011_{\text{dos}}\times 2^{-1}$, $0.1101_{\text{dos}}\times 2^{-2}$ y $0.1011_{\text{dos}}\times 2^{-2}$, respectivamente, por lo que se tiene lo siguiente
  \begin{center}
   \begin{tabular}{ccccc}
    $\frac{1}{3}$ & $\approx$ & $0.1011_{\text{dos}}\times 2^{-1}$ & $=$ & $0.1011_{\text{dos}}\phantom{0} \times 2^{-1}$ \\ 
    \vspace{-0.3cm} 
    \\
    $\frac{1}{5}$ & $\approx$ & $0.1101_{\text{dos}} \times 2^{-2}$ & $=$ & $0.01101_{\text{dos}} \times 2^{-1}$ \\
    \vspace{-0.4cm}
    \\
    \hhline{-~~~-}
    \vspace{-0.4cm}
    \\
    $\frac{8}{15}$ & & & & $1.00011_{\text{dos}} \times 2^{-1}$
   \end{tabular}
  \end{center}
  Entonces, suponiendo que el computador almacena el n\'umero $1.00011_{\text{dos}}\times 2^{-1}$ redondenado a $0.1001_{\text{dos}} \times 2^{0}$, el paso siguiente es
  \begin{center}
   \begin{tabular}{ccccc}
    $\frac{8}{15}$ & $\approx$ & $0.1001_{\text{dos}}\times 2^{0}$ & $=$ & $0.1001_{\text{dos}}\phantom{10} \times 2^{0}$ \\ 
    \vspace{-0.3cm} 
    \\
    $\frac{1}{6}$ & $\approx$ & $0.1011_{\text{dos}} \times 2^{-2}$ & $=$ & $0.001011_{\text{dos}} \times 2^{0}$ \\
    \vspace{-0.4cm}
    \\
    \hhline{-~~~-}
    \vspace{-0.4cm}
    \\
    $\frac{7}{10}$ & & & & $0.101111_{\text{dos}} \times 2^{0}$
   \end{tabular}
  \end{center}
  Entonces, volviendo a suponer que el n\'umero $0.110111_{\text{dos}} \times 2^{0}$ se almacena en el computador redondeando, en este caso a $0.1100_{\text{dos}} \times 2^{0}$, se obtiene que la soluci\'on del computador al problema de la suma es
  \begin{equation*}
   \frac{7}{10} \approx 0.1100_{\text{dos}} \times 2^{0} = 0.75
  \end{equation*}
  cuando en realidad $\frac{7}{10} = 0.7$. Por lo tanto, el error en el c\'alculo efectuado por el computador es
  \begin{equation*}
   0.7 - 0.75 = -0.05
  \end{equation*}
  que expresado como un porcentaje de $0.7$ es del $7.14\%$.

  \item Dado que $1/10 = 0.5$, $1/3 = 0.\overline{3}$ y $1/5 = 0.2$, entonces los n\'umeros que m\'as se aproximan a estos en la tabla son $0.1015625$, $0.34375$ y $0.203125$, que corresponden a los n\'umeros binarios $0.1101_{\text{dos}}\times 2^{-3}$, $0.1011_{\text{dos}}\times 2^{-1}$ y $0.1101_{\text{dos}}\times 2^{-2}$, respectivamente, por lo que se tiene lo siguiente
  \begin{center}
   \begin{tabular}{ccccc}
    $\frac{1}{10}$ & $\approx$ & $0.1101_{\text{dos}} \times 2^{-3}$ & $=$ & $0.001101_{\text{dos}} \times 2^{-1}$ \\
    \vspace{-0.3cm} 
    \\
    $\frac{1}{3}$ & $\approx$ & $0.1011_{\text{dos}}\times 2^{-1}$ & $=$ & $0.1011_{\text{dos}}\phantom{00} \times 2^{-1}$ \\ 
    \vspace{-0.4cm}
    \\
    \hhline{-~~~-}
    \vspace{-0.4cm}
    \\
    $\frac{13}{30}$ & & & & $0.111001_{\text{dos}} \times 2^{-1}$
   \end{tabular}
  \end{center}
  Entonces, suponiendo que el computador almacena el n\'umero $0.111001_{\text{dos}}\times 2^{-1}$ truncando a $0.1110_{\text{dos}} \times 2^{-1}$, el paso siguiente es
  \begin{center}
   \begin{tabular}{ccccc}
    $\frac{13}{30}$ & $\approx$ & $0.1110_{\text{dos}}\times 2^{-1}$ & $=$ & $0.1110_{\text{dos}}\phantom{0} \times 2^{-1}$ \\ 
    \vspace{-0.3cm} 
    \\
    $\frac{1}{5}$ & $\approx$ & $0.1101_{\text{dos}} \times 2^{-2}$ & $=$ & $0.01101_{\text{dos}} \times 2^{-1}$ \\
    \vspace{-0.4cm}
    \\
    \hhline{-~~~-}
    \vspace{-0.4cm}
    \\
    $\frac{19}{30}$ & & & & $1.01001_{\text{dos}} \times 2^{-1}$
   \end{tabular}
  \end{center}
  Entonces, volviendo a suponer que el n\'umero $1.01001_{\text{dos}} \times 2^{0}$ se almacena en el computador truncando, en este caso a $0.1010_{\text{dos}} \times 2^{0}$, se obtiene que la soluci\'on del computador al problema de la suma es
  \begin{equation*}
   \frac{19}{30} \approx 0.1010_{\text{dos}} \times 2^{0} = 0.625
  \end{equation*}
  cuando en realidad $\frac{19}{30} = 0.6\overline{3}$. Por lo tanto, el error en el c\'alculo efectuado por el computador es
  \begin{equation*}
   0.6\overline{3} - 0.625 = -0.008\overline{3}
  \end{equation*}
  que expresado como un porcentaje de $\frac{19}{30}$ es del $1.32\%$.

  \item Dado que $3/17 = 0.\overline{1764705882352941}$, $1/9 = 0.\overline{1}$ y $1/7 = 0.\overline{142857}$, entonces los n\'umeros que m\'as se aproximan a estos en la tabla, redondeando, son $0.1875$, $0.1171875$ y $0.15625$, que corresponden a los n\'umeros binarios $0.1100_{\text{dos}}\times 2^{-2}$, $0.1111_{\text{dos}}\times 2^{-3}$ y $0.1010_{\text{dos}}\times 2^{-2}$, respectivamente, por lo que se tiene lo siguiente
  \begin{center}
   \begin{tabular}{ccccc}
    $\frac{3}{17}$ & $\approx$ & $0.1100_{\text{dos}} \times 2^{-2}$ & $=$ & $0.1100_{\text{dos}}\phantom{1} \times 2^{-2}$ \\
    \vspace{-0.3cm} 
    \\
    $\frac{1}{9}$ & $\approx$ & $0.1111_{\text{dos}}\times 2^{-3}$ & $=$ & $0.01111_{\text{dos}} \times 2^{-2}$ \\ 
    \vspace{-0.4cm}
    \\
    \hhline{-~~~-}
    \vspace{-0.4cm}
    \\
    $\frac{44}{153}$ & & & & $1.00111_{\text{dos}} \times 2^{-2}$
   \end{tabular}
  \end{center}
  Entonces, suponiendo que el computador almacena el n\'umero $1.00111_{\text{dos}}\times 2^{-2}$ redondeando a $0.1010_{\text{dos}} \times 2^{-1}$, el paso siguiente es
  \begin{center}
   \begin{tabular}{ccccc}
    $\frac{44}{153}$ & $\approx$ & $0.1010_{\text{dos}}\times 2^{-1}$ & $=$ & $0.1010_{\text{dos}} \times 2^{-1}$ \\ 
    \vspace{-0.3cm} 
    \\
    $\frac{1}{7}$ & $\approx$ & $0.1010_{\text{dos}} \times 2^{-2}$ & $=$ & $0.0101_{\text{dos}} \times 2^{-1}$ \\
    \vspace{-0.4cm}
    \\
    \hhline{-~~~-}
    \vspace{-0.4cm}
    \\
    $\frac{461}{1071}$ & & & & $0.1111_{\text{dos}} \times 2^{-1}$
   \end{tabular}
  \end{center}
  El cual es un n\'umero que se puede almacenar en el computador, as\'{\i} que no hace falta redondear o truncar \'este. Por lo tanto se obtiene que la soluci\'on del computador al problema de la suma es
  \begin{equation*}
   \frac{461}{1071} \approx 0.1111_{\text{dos}} \times 2^{-1} = 0.46875
  \end{equation*}
  cuando en realidad $\frac{461}{1071}$ es un n\'umero con 48 d\'{\i}gitos en su periodo decimal y cuyos primeros d\'{\i}gitos son $0.4304388422\ldots$. Por lo tanto, el error en el c\'alculo efectuado por el computador es aproximadamente
  \begin{equation*}
   \frac{461}{1071} - 0.01111_{\text{dos}} \approx 0.4304388422 - 0.46875 = -0.0383111578.
  \end{equation*}
  que expresado como un porcentaje de $\frac{461}{1071}$ es del $8.9\%$.
  \par 
  N\'otese que las aproximaciones iniciales son mejores si se hacen aproximando por truncamiento al n\'umero m\'as pr\'oximo en la tabla, los cuales son, respectivamente para $3/17$, $1/9$ y $1/7$, $0.171875$, $0.109375$ y $0.140625$, que corresponden a los n\'umeros binarios $0.1011_{\text{dos}}\times 2^{-2}$, $0.1110_{\text{dos}}\times 2^{-3}$ y $0.1001_{\text{dos}} \times 2^{-2}$, respectivamente, por lo que ahora las cuentas cambian teniendo lo siguiente:
  \begin{center}
   \begin{tabular}{ccccc}
    $\frac{3}{17}$ & $\approx$ & $0.1011_{\text{dos}} \times 2^{-2}$ & $=$ & $0.1011_{\text{dos}} \times 2^{-2}$ \\
    \vspace{-0.3cm} 
    \\
    $\frac{1}{9}$ & $\approx$ & $0.1110_{\text{dos}}\times 2^{-3}$ & $=$ & $0.0111_{\text{dos}} \times 2^{-2}$ \\ 
    \vspace{-0.4cm}
    \\
    \hhline{-~~~-}
    \vspace{-0.4cm}
    \\
    $\frac{44}{153}$ & & & & $1.0010_{\text{dos}} \times 2^{-2}$
   \end{tabular}
  \end{center}
  el cual puede almacenar el computador como $0.1001_{\text{dos}} \times 2^{-1}$, y luego se tiene que
  \begin{center}
   \begin{tabular}{ccccc}
    $\frac{44}{153}$ & $\approx$ & $0.1001_{\text{dos}}\times 2^{-1}$ & $=$ & $0.1001_{\text{dos}} \phantom{0} \times 2^{-1}$ \\ 
    \vspace{-0.3cm} 
    \\
    $\frac{1}{7}$ & $\approx$ & $0.1001_{\text{dos}} \times 2^{-2}$ & $=$ & $0.01001_{\text{dos}} \times 2^{-1}$ \\
    \vspace{-0.4cm}
    \\
    \hhline{-~~~-}
    \vspace{-0.4cm}
    \\
    $\frac{461}{1071}$ & & & & $0.11011_{\text{dos}} \times 2^{-1}$
   \end{tabular}
  \end{center}
  el cual se puede suponer que la computadora lo almacena al n\'umero m\'as pr\'oximo que es redondeando para obtener $0.1110_{\text{dos}} \times 2^{-1}$. Por lo tanto, se obtiene que la soluci\'on del computador al problema de la suma es
  \begin{equation*}
   \frac{461}{1071} \approx 0.1110_{\text{dos}} \times 2^{-1} = 0.4375
  \end{equation*}
  Por lo tanto, en este caso, el error en el c\'alculo efectuador por el computador es aproximadamente
  \begin{equation*}
   \frac{461}{1071} - 0.0111_{\text{dos}} \approx 0.4304388422 - 0.4375 = -0.0070611578
  \end{equation*}
  que expresado como un porcentaje de $\frac{461}{1071}$ es del $1.64\%$, que es claramente menor a lo que se ten\'{\i}a previamente.

  \item Dado que $7/10 = 0.7$, $1/9 = 0.\overline{1}$ y $1/7 = 0.\overline{142857}$, entonces los n\'umeros que m\'as se aproximan a estos en la tabla, redondeando, son $0.75$, $0.1171875$ y $0.15625$, que corresponden a los n\'umeros binarios $0.1100_{\text{dos}}\times 2^{0}$, $0.1111_{\text{dos}}\times 2^{-3}$ y $0.1010_{\text{dos}}\times 2^{-2}$, respectivamente, por lo que se tiene lo siguiente
  \begin{center}
   \begin{tabular}{ccccc}
    $\frac{7}{10}$ & $\approx$ & $0.1100_{\text{dos}} \times 2^{0}\phantom{0}$ & $=$ & $0.1100_{\text{dos}}\phantom{011} \times 2^{0}$ \\ 
    \vspace{-0.3cm} 
    \\
    $\frac{1}{9}$ & $\approx$ & $0.1111_{\text{dos}} \times 2^{-3}$ & $=$ & $0.0001111_{\text{dos}} \times 2^{0}$ \\
    \vspace{-0.4cm}
    \\
    \hhline{-~~~-}
    \vspace{-0.4cm}
    \\
    $\frac{73}{90}$ & & & & $0.1101111_{\text{dos}} \times 2^{0}$
   \end{tabular}
  \end{center}
  Entonces, suponiendo que el computador almacena el n\'umero $0.1101111_{\text{dos}}\times 2^{0}$ redondenado a $0.1110_{\text{dos}} \times 2^{0}$, el paso siguiente es
  \begin{center}
   \begin{tabular}{ccccc}
    $\frac{73}{90}$ & $\approx$ & $0.1110_{\text{dos}}\times 2^{0}\phantom{1}$ & $=$ & $0.1110_{\text{dos}}\phantom{0} \times 2^{0}$ \\ 
    \vspace{-0.3cm} 
    \\
    $\frac{1}{7}$ & $\approx$ & $0.1010_{\text{dos}} \times 2^{-2}$ & $=$ & $0.00101_{\text{dos}} \times 2^{0}$ \\
    \vspace{-0.4cm}
    \\
    \hhline{-~~~-}
    \vspace{-0.4cm}
    \\
    $\frac{601}{630}$ & & & & $1.00001_{\text{dos}} \times 2^{0}$
   \end{tabular}
  \end{center}
  Entonces, volviendo a suponer que el n\'umero $1.00001_{\text{dos}} \times 2^{0}$ se almacena en el computador redondeando, en este caso a $1.0001_{\text{dos}} \times 2^{0}$, se obtiene que la soluci\'on del computador al problema de la suma es
  \begin{equation*}
   \frac{601}{630} \approx 1.0001_{\text{dos}} \times 2^{0} = 1.0625
  \end{equation*}
  cuando en realidad $\frac{601}{630} = 0.9\overline{539682}$. Por lo tanto, el error en el c\'alculo efectuado por el computador es
  \begin{equation*}
   0.9\overline{539682} - 1.0625 = -0.1085\overline{317460}
  \end{equation*}
  que expresado como un porcentaje de $\frac{601}{630}$ es del $10.85\%$.
  \par 
  N\'otese que las aproximaciones iniciales son mejores si se hacen aproximando por truncamiento al n\'umero m\'as pr\'oximo en la tabla, los cuales son, respectivamente para $7/10$, $1/9$ y $1/7$, $0.6875$, $0.109375$ y $0.140625$, que corresponden a los n\'umeros binarios $0.1011_{\text{dos}}\times 2^{0}$, $0.1110_{\text{dos}}\times 2^{-3}$ y $0.1001_{\text{dos}} \times 2^{-2}$, respectivamente, por lo que ahora las cuentas cambian teniendo lo siguiente:
  \begin{center}
   \begin{tabular}{ccccc}
    $\frac{7}{10}$ & $\approx$ & $0.1011_{\text{dos}} \times 2^{0}\phantom{1}$ & $=$ & $0.1011_{\text{dos}}\phantom{11} \times 2^{0}$ \\
    \vspace{-0.3cm} 
    \\
    $\frac{1}{9}$ & $\approx$ & $0.1110_{\text{dos}}\times 2^{-3}$ & $=$ & $0.000111_{\text{dos}} \times 2^{0}$ \\ 
    \vspace{-0.4cm}
    \\
    \hhline{-~~~-}
    \vspace{-0.4cm}
    \\
    $\frac{73}{90}$ & & & & $0.110011_{\text{dos}} \times 2^{0}$
   \end{tabular}
  \end{center}
  el cual se va suponer que el computador lo almacena redondeando como $0.1101_{\text{dos}} \times 2^{0}$, y luego se tiene que
  \begin{center}
   \begin{tabular}{ccccc}
    $\frac{73}{90}$ & $\approx$ & $0.1101_{\text{dos}}\times 2^{0}\phantom{1}$ & $=$ & $0.1101_{\text{dos}}\phantom{00} \times 2^{0}$ \\ 
    \vspace{-0.3cm} 
    \\
    $\frac{1}{7}$ & $\approx$ & $0.1001_{\text{dos}} \times 2^{-2}$ & $=$ & $0.001001_{\text{dos}} \times 2^{0}$ \\
    \vspace{-0.4cm}
    \\
    \hhline{-~~~-}
    \vspace{-0.4cm}
    \\
    $\frac{601}{630}$ & & & & $0.111101_{\text{dos}} \times 2^{0}$
   \end{tabular}
  \end{center}
  el cual se puede suponer que la computadora lo almacena al n\'umero m\'as pr\'oximo que es truncando para obtener $0.1111_{\text{dos}} \times 2^{0}$. Por lo tanto, se obtiene que la soluci\'on del computador al problema de la suma es
  \begin{equation*}
   \frac{601}{630} \approx 0.1111_{\text{dos}} \times 2^{0} = 0.9375
  \end{equation*}
  Por lo tanto, en este caso, el error en el c\'alculo efectuador por el computador es aproximadamente
  \begin{equation*}
   0.9\overline{539682} - 0.9375 = 0.0164\overline{682539}
  \end{equation*}
  que expresado como un porcentaje de $\frac{601}{630}$ es del $1.73\%$, que es claramente menor a lo que se ten\'{\i}a previamente.
  \par 
  En conclusi\'on, hay que buscar buenas aproximaciones iniciales para no acarrear demasiado error en la soluci\'on final, que es a lo que se quer\'{\i}a llegar.${}_{\blacksquare}$

 \end{enumerate}
\end{solucion}

 \item \begin{enunciado}
 Pruebe que si sustituimos $2$ por $3$ en todas las f\'ormulas de (8), el resultado es un m\'etodo para hallar la expresi\'on en base $3$ de un n\'umero natural. Utilice esto para expresar los siguientes n\'umeros en base $3$.
 \begin{multicols}{4}
  \begin{enumerate}[(a)]
   \item $10$
   \item $23$
   \item $421$
   \item $1784$
  \end{enumerate}
 \end{multicols}
\end{enunciado}

\begin{solucion}
 La demostraci\'on se realizar\'a intercambiando $2$ por cualquier n\'umero natural $n$ mayor a 1.
 \par 
 Primero que nada, se va a comprobar que el proceso es \'unico y que siempre termina. Esto es verdad ya que, por el teorema del algoritmo de la divisi\'on, para cualesquiera n\'umeros naturales $a$, $b$, con $b\neq 0$, se cumple que existen valores \'unico $q$ y $r$ tales que $a = qb + r$ con $0\leq r < b$, entonces se cumple que para cualesquiera $N$ y $n$ naturales, con $n > 1$, existen valores \'unicos $Q_0$ y $b_0$ tales que 
 \begin{equation*}
  N = Q_0 n + b_0, \qquad 0 \leq b_0 < n
 \end{equation*}
 adem\'as, $Q_0 < N$, pues si ocurriese que $Q_0 \geq N$, entonces, como $n > 1$, se tendr\'{\i}a que $(Q_0)(n) > (N)(1) = N$ y como $b_0 \geq 0$, entonces se tendr\'{\i}a que
 \begin{equation*}
  N = Q_0 n + b_0 \geq Q_0 n > N
 \end{equation*}
 lo cual es una contradicci\'on, por lo tanto $Q_0 < N$.
 \par 
 Ahora se probar\'a, por inducci\'on, que, en el proceso, $Q_k$ y $b_k$ son \'unicos, que $Q_{k-1} > Q_{k}$ y que $b_k \in \{ 0, 1, \ldots, n-1 \}$, para todo $k \in \mathbb{Z}\cap[0,J]$, donde $Q_{-1} = N$. 
 \par 
 Esto ya se demostr\'o para el primer caso, que en este contexto es cuando $k=0$. Suponiendo ahora que lo antes mencionado es cierto para un n\'umero natural cualquiera $0 < k < N$, se desea probar probar que es cierto para el caso $k+1$. Entonces, 
 \par 
 Entonces, por el teorema del algoritmo de la divisi\'on, dado dos n\'umeros naturales $Q_k$ y $n$, fijos, existen n\'umeros naturales \'unicos $Q_{k+1}$ y $b_{k+1}$ tales que $Q_k = Q_{k+1}n + b_{k+1}$ con $0\leq b_{k+1} < n$. Adem\'as $Q_{k} > Q_{k+1}$, ya que si no fuese as\'{\i}, entonces, como $b_{k+1} \geq 0$ y $n > 1$, se tendr\'{\i}a que
 \begin{equation*}
  Q_{k} = Q_{k+1}n + b_{k+1} \geq Q_{k+1}n > Q_{k}(1) = Q_{k}
 \end{equation*}
 lo cual es una contradicci\'on, por lo tanto, tambi\'en para el valor $k+1$ se cumple que $Q_{k+1}$ y $b_{k+1}$ son \'unicos, $Q_{k} > Q_{k+1}$ y $b_{k+1} \in \{ 0, 1, \ldots , n-1 \}$. Por lo tanto, queda demostrado por inducci\'on que esto es cierto para todo n\'umero entero no negativo $k$.
 \par 
 De el hecho de que $Q_k$ y $b_k$ sean \'unicos, se deduce entonces que el proceso generar\'a valores \'unicos en cada iteraci\'on. Queda por demostrar que el proceso termina, lo cual es cierto, porque en cada iteraci\'on se tiene un nuevo valor natural $Q_{k}$ de tal forma que la sucesi\'on $N = Q_{-1} > Q_{0} > Q_{1} > \cdots > Q_{k} > \cdots$ son todos n\'umeros naturales que decresen en cada iteraci\'on, por lo tanto, por el principio del buen orden, se tiene que este proceso tiene un valor natural m\'{\i}nimo, al que se le nombrar\'a $Q_{J-1}$, el cual debe de ser menor a $n$, ya que si fuese mayor o igual a $n$, por el teorema del algoritmo de la divisi\'on se llegar\'{\i}a a que existen enteros \'unicos $Q_J$ y $b_J$ tales que $Q_{J-1} = Q_J n + b_J$ con $0 \leq b_J < n$, entonces $QJn = Q_{J-1} - b_J \geq n - b_J > 0$, por lo que $Q_Jn > 0$ y, por ello, tanto $n$ como $Q_J$ son n\'umeros enteros positivos, pero como $Q_J$ es un n\'umero entero del que ya se demostr\'o que $Q_{J-1} > Q_J$, entonces el hecho de que $Q_J$ sea un natural contradice que $Q_{J-1}$ era el menor natural en la sucesi\'on $\{Q_{k}\}$, por lo tanto $Q_{J-1}$ debe cumplir que es menor a $n$ y, por lo tanto, los los valores $Q_J = 0$ y $n > b_J = Q_{J-1} > 0$ cumplen que $Q_{J-1} = Q_J n + b_J$, que, por el teorema del algoritmo de la divisi\'on son \'unicos, con lo cual acaba el proceso.
 \par 
 En conclusi\'on, las f\'ormulas de (8), al cambiar $2$ por un n\'umero natural $n > 1$ cualquiera, ofrece un algoritmo que genera n\'umeros enteros \'unicos $b_k$ tales que $0 \leq b_k < n$ para todo $k \in \mathbb{Z}\cap[0,J]$, para alg\'un $J$ entero no negativo. Ahora s\'olo queda por demostrar que $N = b_{J} \times n^{J} + b_{J-1}n^{J-1} + \cdots + b_1n + b_0$, el cual implica que la concatenaci\'on $b_J b_{J-1} b_{J-2} \ldots b_1 b_0$ es la representaci\'on en base $n$ de $N$.
 \par 
 Se demostrar\'a ahora que, usando la notaci\'on previa, $Q_{k-1} = b_{J}n^{J-k} + b_{J-1}n^{J-k-1} + \cdots + b_{k}$, para toda $k \in \mathbb{Z}\cap[0,J-1]$ a trav\'es de una inducci\'on desde $J-1$ hasta $0$.
 \par 
 El primer caso ya se prob\'o al final de la demostraci\'on anterior, esto es, que $Q_{J-1} = b_J > 0$. Suponiendo entonces que esto es cierto un valor entero arbitrario $k$ con $0 < k \leq J-1$, se desea probar que es cierto para el caso $k-1$.
 Dado que se tiene, por hip\'otesis de inducci\'on, que $Q_{k} = b_{J}n^{J-k-1} + b_{J-1}n^{J-k-2} + \cdots + b_{k+1}$ y que, por el proceso iterativo, $Q_{k-1} = Q_{k}n + b_{k}$, entonces 
 \begin{eqnarray*}
  Q_{k-1} & = & Q_{k}n + b_{k} \\
  & = & \left( b_{J}n^{J-k-1} + b_{J-1}n^{J-k-2} + \cdots + b_{k+1} \right) \times n + b_{k} \\
  & = & b_{J}n^{J-k} + b_{J-1}n^{J-k-1} + \cdots + b_{k+1}n + b_{k}
 \end{eqnarray*}
 que es justo a lo que se quer\'{\i}a llegar. Por lo tanto, esto es cierto para todo $k \in \mathbb{Z}\cap[0,J]$.
 \par 
 Por lo tanto, se cumple que
 \begin{equation*}
  Q_0 = b_Jn^{J-1} + b_{J-1}n^{J-2} + \cdots b_1
 \end{equation*}
 Por lo tanto, se tiene que
 \begin{equation*}
  Q_0n + b_0 = \left( b_Jn^{J-1} + b_{J-1}n^{J-2} + \cdots b_1 \right)\times n + b_0 = b_Jn^{J} + b_{J-1}n^{J-1} + \cdots b_1n + b_0 = N
 \end{equation*}
 que es la representaci\'on de $N$ a la que se quer\'{\i}a llegar. Es decir, al sustituir $2$ por $n$ en todas las f\'ormulas de (8), el resultado es un m\'etodo para hallar la expresi\'on en base $n$ de un n\'umero natural. Q.E.D.
 \begin{enumerate}[(a)]
  \item Dividiendo iteradamente $N=10$ entre $n=3$, se tiene que
  \begin{center}
   \begin{tabular}{rclrr}
    $10$ & $=$ & $3\times$ & $3+1$, & $b_0 = 1$ \\
    $3$ & $=$ & $3\times$ & $1 +0$, & $b_1 = 0$ \\
    $1$  & $=$ & $3\times$ & $0 +1$, & $b_2 = 1$
   \end{tabular}
  \end{center}
  As\'{\i} que la representaci\'on en base $3$ de $10$ es
  \begin{equation*}
   10 = b_2 b_1 b_{0}{}_{\text{tres}} = 101_{\text{tres}}.
  \end{equation*}

  \item Dividiendo iteradamente $N=23$ entre $n=3$, se tiene que
  \begin{center}
   \begin{tabular}{rclrr}
    $23$ & $=$ & $3\times$ & $7+2$, & $b_0 = 2$ \\
    $7$ & $=$ & $3\times$ & $2 +1$, & $b_1 = 1$ \\
    $2$  & $=$ & $3\times$ & $0 +2$, & $b_2 = 2$
   \end{tabular}
  \end{center}
  As\'{\i} que la representaci\'on en base $3$ de $23$ es
  \begin{equation*}
   23 = b_2 b_1 b_{0}{}_{\text{tres}} = 212_{\text{tres}}.
  \end{equation*}

  \item Dividiendo iteradamente $N=421$ entre $n=3$, se tiene que
  \begin{center}
   \begin{tabular}{rclrr}
    $421$ & $=$ & $3\times$ & $140 + 1$, & $b_0 = 1$ \\
    $140$ & $=$ & $3\times$ & $46 + 2$, & $b_1 = 2$ \\
    $46$  & $=$ & $3\times$ & $15 + 1$, & $b_2 = 1$ \\
    $15$ & $=$ & $3\times$ & $5 + 0$, & $b_3 = 0$ \\
    $5$  & $=$ & $3\times$ & $1 + 2$, & $b_4 = 2$ \\
    $1$  & $=$ & $3\times$ & $0 + 1$, & $b_5 = 1$
   \end{tabular}
  \end{center}
  As\'{\i} que la representaci\'on en base $3$ de $421$ es
  \begin{equation*}
   23 = b_5 b_4 \ldots  b_1 b_{0}{}_{\text{tres}} = 120121_{\text{tres}}.
  \end{equation*}

  \item Dividiendo iteradamente $N=1784$ entre $n=3$, se tiene que
  \begin{center}
   \begin{tabular}{rclrr}
    $1784$ & $=$ & $3\times$ & $594 + 2$, & $b_0 = 2$ \\
    $594$ & $=$ & $3\times$ & $198 + 0$, & $b_1 = 0$ \\
    $198$  & $=$ & $3\times$ & $66 + 0$, & $b_2 = 0$ \\
    $66$ & $=$ & $3\times$ & $22 + 0$, & $b_3 = 0$ \\
    $22$  & $=$ & $3\times$ & $7 + 1$, & $b_4 = 1$ \\
    $7$  & $=$ & $3\times$ & $2 + 1$, & $b_5 = 1$ \\
    $2$  & $=$ & $3\times$ & $0 + 2$, & $b_6 = 2$
   \end{tabular}
  \end{center}
  As\'{\i} que la representaci\'on en base $3$ de $1784$ es
  \begin{equation*}
   1784 = b_6 b_5 \ldots  b_1 b_{0}{}_{\text{tres}} = 2110002_{\text{tres}}.
  \end{equation*}
  que es a lo que se quer\'{\i}a llegar.${}_{\blacksquare}$
 \end{enumerate}
\end{solucion}

 \item \begin{enunciado}
 Pruebe que si sustituimos $2$ por $3$ en (22), el resultado es un m\'etodo para hallar la expresi\'on en base $3$ de un n\'umero positivo $R$ tal que $0 < R < 1$. Utilice esto para expresar los siguientes n\'umeros en base $3$.
 \begin{multicols}{4}
  \begin{enumerate}
   \item $1/3$
   \item $1/2$
   \item $1/10$
   \item $11/27$
  \end{enumerate}
 \end{multicols}
\end{enunciado}

\begin{solucion}
 La demostraci\'on se realizar\'a intercambiando $2$ por cualquier n\'umero natural $n$ mayor a $1$.
 \par 
 Primero que nada, se va a demostrar que existe al menos una expansi\'on decimal de un n\'umero en el intervalo $[0,1]$ en base $n$, con $n$ natural y mayor a $1$, es decir que para todo $x\in[0,1]$ y $n$ un entero mayor o igual a $2$, existen enteros $0\leq a_{k} \leq n-1$ tales que $x = \sum_{k=1}^{\infty} \frac{a_k}{n^k}$.
 Para ello se requeri\'a demostrar previamente el siguiente lema:
 \begin{lema}
  Considerando los intervalos $I_n = [c_n, d_n]$, $\forall n \in \mathbb{N}$. Si $I_n \supseteq I_{n+1}$ y $\lim (d_n - c_n) = 0$, entonces la intersecci\'on $\bigcap_{n=1}^{\infty} I_n$ consiste exactamente de un solo punto.
 \end{lema}
 \begin{demostracion}
  Para la demostraci\'on, primero se probar\'a que la intersecci\'on es no vac\'{\i}a y de ah\'{\i} se determinar\'a que consta de un \'unico punto. 
  \par 
  Sea $I_{n+1} = [c_{n+1}, d_{n+1}]$ e $I_{n} = [c_{n}, d_{n}]$, entonces $I_{n+1} = \{ x| c_{n+1} \leq x \leq d_{n+1} \}$ e $I_{n} = \{ x| c_n \leq x \leq d_n \}$. Por definici\'on, si $I_{n+1} \subseteq I_n$, entonces $\forall x \in I_{n+1}$ se cumple que $x\in I_n$, en otros t\'erminos esto significa que si $I_{n+1} \subseteq I_n$, entonces $\forall x$ tal que $c_{n+1} \leq x \leq d_{n+1}$ se cumple que $c_{n} \leq x \leq d_{n}$, por lo tanto $c_{n} \leq c_{n+1} \leq d_{n+1} \leq d_{n}$. Dado que $n$ fue elegido arbitrariamente, queda demostrado que
  \begin{equation} \label{resLema1}
   \forall n\in \mathbb{N}, \qquad c_{n} \leq c_{n+1} \leq d_{n+1} \leq d_{n}
  \end{equation}
  En particular, se cumple que
  \begin{equation} \label{resLema2}
   \forall k,n\in \mathbb{N} \text{ tal que } k<n, \qquad c_k \leq c_n \leq d_n \leq d_k
  \end{equation}
  Dado que $I_1 = [c_1, d_1]$, se cumple entonces lo siguiente: por \eqref{resLema2}, $c_1 \leq c_n \leq d_1$ y $c_1 \leq d_n \leq d_1$ y, por \eqref{resLema1}, la sucesi\'on $\{ c_k \}_{k\in\mathbb{N}}$ es creciente y la sucesi\'on $\{ d_k \}_{k\in\mathbb{N}}$ es decreciente. Entonces, por teorema de sucesiones, como $\{ c_k \}_{k\in\mathbb{N}}$ es creciente y acotado, se tiene que la sucesi\'on converge; an\'alogamente, como $\{ d_k \}_{k\in\mathbb{N}}$ es decreciente y acotado, entonces la sucesi\'on converge. Sean $a$ y $b$ los valores a los que convergen estas sucesiones, respectivamente, es decir $\lim_{n\to \infty} c_n = a$ y $\lim_{n\to \infty} d_n = b$, entonces se puede calcular $\lim_{n\to \infty} (d_n - c_n)$ como $\lim_{n\to \infty} (d_n - c_n) = \lim_{n\to\infty} d_n - \lim_{n\to\infty} c_n = b - a$. N\'otese que a\'un sin tener la suposici\'on de $\lim_{n\to\infty} (d_n - c_n) = 0$, se puede llegar a que $b \geq a$, ya que, como $d_n - c_n \geq 0$, $\forall n \in \mathbb{N}$, por teorema ocurre que $\lim_{n\to\infty}(d_n - c_n) \geq 0$, es decir $b-a \geq 0$.
  \par 
  De lo anterior se sigue que si $A = \{ x | a \leq x \leq b \}$, como $a \leq b$, entonces $A$ es no vac\'{\i}o. Por otro lado, ya que $c_n \leq a \leq b \leq d_n$ y, $\forall x \in A$, $a \leq x \leq b$, entonces $\forall x\in A$ se cumple que $c_n \leq x \leq d_n$, $\forall n\in\mathbb{N}$, es decir $x\in \bigcap_{n\in\mathbb{N}} I_n = \left\{ x | \forall n \in \mathbb{N}, \, c_n \leq x \leq d_n \right\}$. Por lo tanto $A \subseteq \bigcap_{n \in \mathbb{N}} I_n$, lo cual implica que $\bigcap_{n\in\mathbb{N}} I_n$ es no vac\'{\i}o.
  \par 
  Luego, a partir de la suposici\'on de que $\lim_{n\to \infty} (d_n - c_n) = 0$, como $\lim_{n\to \infty} (d_n - c_n) = b-a$, entonces $b-a = 0$, es decir $a = b$. Entonces, considerando un valor arbitrario fijo $x_0$, se tiene las siguientes equivalencias $x_0 \in \bigcap_{n\in \mathbb{N}} I_n$ si y s\'olo si $\forall n \in \mathbb{N}$ $ c_n \leq x_0 \leq d_n$ si y s\'olo si $\forall n \in\mathbb{N}$ $c_n \leq x_0$ y $x_0 \leq d_n$, pero como $\{c_k\}_{k\in\mathbb{N}}$ es una sucesi\'on creciente y $\{d_k\}_{k\in\mathbb{N}}$ es una sucesi\'on decreciente, entonces $c_n \leq x_0$, $\forall n\in\mathbb{N}$, es equivalente a que $\lim_{n\to\infty} c_n \leq \lim_{n\to\infty} x_0$; an\'alogamente, $x_0 \leq d_n$, $\forall n \in\mathbb{N}$, es equivalente a que $\lim_{n\to\infty} x_0 \leq \lim_{n\to\infty} b_n$, es decir $c_n \leq x_0 \leq d_n$, $\forall n\in\mathbb{N}$, es equivalente a $\lim_{n\to\infty} c_n \leq \lim_{n\to\infty} x_0 \leq \lim_{n\to\infty} d_n$ si y s\'olo si $a \leq x_0 \leq b$, pero como $a=b$, se concluye que $\forall x \in \bigcap_{n\in\mathbb{N}} I_n$, $x = a= b$, es decir $\bigcap_{n\in\mathbb{N}} I_n$ consta de un \'unico punto.$\square$
 \end{demostracion}
 Ahora se probar\'a lo que en principio se dijo, enunciado de la siguiente forma
 \begin{teorema}
  si $x\in[0,1]$ y $n$ es un entero mayor o igual a $2$, entonces existen enteros $0\leq a_k \leq n-1$ tales que
  \begin{equation*}
   x = \sum_{k=1}^{\infty} \frac{a_k}{n^k}
  \end{equation*}
  por lo que tiene sentido expresar cualquier valor como $0.a_1a_2a_3\ldots_{(n)}$, que es la expansi\'on decimal de $x$ en base $n$.
 \end{teorema}
 \begin{demostracion}
  Para realizar la demostraci\'on, se probar\'a por inducci\'on que, para todo $k \in \mathbb{N}$, existen valores fijos $0 \leq a_i \leq n-1$, $\forall i\in\mathbb{N}\cap[1,k]$, tales que
  \begin{equation*}
   x \in I_k = \left[ \sum_{i=1}^{k} \frac{a_i}{n^i}, \sum_{i=1}^{k} \frac{a_i + 1}{n^i} \right]
  \end{equation*}
  lo cual a su vez es equivalente a decir que $x = \sum_{i=}^{k} \frac{a_i}{k^i} + b_n$, donde $0 \leq b_n \leq \frac{1}{k^i}$.
  \par 
  As\'{\i} pues, para el caso base, sea $I_0 = [0,1]$, se parte este intervalo en $n$ subintervalos de longitud $\frac{1}{n}$: $\left[ 0, \frac{1}{n} \right], \left[ \frac{1}{n}, \frac{2}{n} \right], \left[ \frac{2}{n}, \frac{3}{n} \right], \ldots, \left[ \frac{n-1}{n}, 1 \right]$, entonces, como $[0,1] = \bigcup_{k=0}^{n-1} \left[ \frac{k}{n}, \frac{k+1}{n} \right]$, se tiene, a partir de que $x \in [0,1]$, que $x$ pertenece a alguno de estos subintervalos, sea $I_1$ alguno de estos intervalos en el que se encuentra $x$, entonces $I_1 = \left[ \frac{a_1}{n}, \frac{a_1 + 1}{n-1} \right]$, donde $a_1$ es un entero tal que $0 \leq a_1 \leq n-1$, esto significa que $\frac{a_1}{n} \leq x \leq \frac{a_1 + 1}{n}$, lo cual es equivalente a que $x = \frac{a_1}{n} + b_1$, donde $0 \leq b_1 \leq \frac{1}{n}$, lo cual prueba el caso base.
  \par 
  Ahora se probar\'a que $x \in I_{k+1} = \left[ \sum_{i=1}^{k+1} \frac{a_i}{n^i}, \sum_{i=1}^{k+1} \frac{a_i + 1}{n^i} \right]$, suponiendo que $x\in I_{k}$. Esto es as\'{\i} pues, si se divide $I_{k}$ en los $n$ subintervalos de longitud $\frac{1}{n^{k+1}}$:
  \begin{equation*}
   \left[ \sum_{i=1}^{k} \frac{a_i}{n^i}, \left( \sum_{i=1}^{k} \frac{a_i}{n^i} \right) + \frac{1}{n^{k+1}} \right], \left[ \left( \sum_{i=1}^{k} \frac{a_i}{n^i} \right) + \frac{1}{n^{k+1}}, \left( \sum_{i=1}^{k} \frac{a_i}{n^i} \right) + \frac{2}{n^{k+1}} \right], \ldots \hspace{2cm}
  \end{equation*}
  \begin{equation*}
   \hspace{6cm} \ldots, \left[ \left( \sum_{i=1}^{k} \frac{a_i}{n^i} \right) + \frac{n-1}{n^{k+1}}, \left( \sum_{i=1}^{k} \frac{a_i}{n^i} \right) + \frac{1}{n^{k}} \right]
  \end{equation*}
  entonces la uni\'on de estos subintervalos es igual a $I_k$, por lo tanto, como $x\in I_k$, se sigue que $x$ pertenece a alguno de estos intervalos, sea $I_{k+1}$ alguno de estos intervalos al que pertenece $x$, entonces $x\in I_{k+1} = \left[ \left( \sum_{i=1}^{k} \frac{a_i}{n^i} \right) + \frac{a_{k+1}}{n^{k+1}}, \left( \sum_{i=1}^{k} \frac{a_i}{n^i} \right) + \frac{a_{k+1}+1}{n^{k+1}} \right]$, donde $a_{k+1}$ es un emtero que cumple que $0 \leq a_{k+1} \leq n-1$, por lo tanto, existen enteros fijos $0 \leq a_i \leq n-1$, $\forall i \in\mathbb{N}\cap[1,k+1]$, tales que $x \in I_{k+1} = \left[ \sum_{i=1}^{k+1} \frac{a_i}{n^i}, \sum_{i=1}^{k+1} \frac{a_i + 1}{n^i} \right]$, lo cual concluye la demostraci\'on por inducci\'on.
  \par 
  N\'otese que $I_k \supseteq I_{k+1}$ y que el l\'{\i}mite de la cota mayor menos la cota menor de $I_{k}$, cuando $k$ tiende a infinito, es cero. Lo primero es claro, ya $I_{k+1}$ se construy\'o de modo que est\'e contenido en $I_k$, mientras que para lo segundo se puede probar como sigue:
  \begin{eqnarray*}
   \displaystyle{ \lim_{k\to\infty} \left\{ \left[ \left ( \sum_{i=1}^{k} \frac{a_i}{n^i} \right) + \frac{1}{n^k} \right] - \left( \sum_{i=1}^{k} \frac{a_i}{n^i} \right) \right\} }
   & = & \displaystyle{ \lim_{k\to\infty} \left[ \cancelto{0}{ \sum_{i=1}^{k} \left(\frac{a_i}{n^i} - \frac{a_i}{n^i}  \right)} + \frac{1}{n^k} \right] } \\
   \\
   & = & \lim_{k\to\infty} \frac{1}{n^k}
  \end{eqnarray*}
  donde este \'ultimo l\'{\i}mite es igual a $0$, ya que $n \geq 2$.
  Luego entonces, como $x \in I_{k}$, para todo $k\in \mathbb{N}$, se sigue que $x \in \bigcap_{k\in\mathbb{N}} I_k$, el cual, por el lema anteriormente demostrado, consta de un \'unico elemento y \'este es, por lo tanto, $x$. Finalmente, por la demostraci\'on anterior, este elemento es el l\'{\i}mite de la cota inferior (que es igual al l\'{\i}mite de la cota superior), es decir: 
  \begin{equation*}
   x = \lim_{k \to \infty} \sum_{i=1}^{k} \frac{a_i}{n^i} = \sum_{k=1}^{\infty} \frac{a_k}{n^k}
  \end{equation*}
  para ciertos enteros $0 \leq a_k \leq n-1$ y $n$ un entero mayor o igual a $2$. Q.E.D.${}_{\square}$
 \end{demostracion}
 Finalmente, sea $R$ un n\'umero tal que $0 < R < 1$, para cualquier entero $n \geq 2$ exsiten enteros $0 \leq  d_j \leq n-1$ tales que $R$ se puede expresar como
 \begin{equation} \label{eqRepresentacionDecimal1}
  R = \sum_{j=1}^{\infty} \frac{d_j}{n^j}
 \end{equation}
 Esto significa que, por definici\'on, los valores $d_j$, $\forall j \in \mathbb{N}$ son los d\'{\i}gitos de la expansi\'on decimal de $R$ en base $n$, es decir $R = 0.d_1d_2d_3\ldots_{(n)}$ donde el sub\'{\i}ndice $(n)$ indica que el n\'umero est\'a escrito en base $n$.
 \par 
 As\'{\i} pues, queda por demostrar que el m\'etodo para calcular los valores $d_j$ es v\'alido. Esto es, se proceder\'a por inducci\'on a demostrar que la sucesi\'on $\{ d_k \}_{k\in\mathbb{N}}$, de los d\'{\i}gitos en la representaci\'on en base $n$ de $R$, se pueden calcular como $d_k = \text{ent}(nF_{k-1})$, donde $F_k$ es el $k-$\'esimo elemento de la sucesi\'on $\{F_k\}_{k\in\mathbb{N}} = \left\{ \text{frac}(nF_{k-1}) \right\}_{k\in\mathbb{N}}$ y $F_0 = R$; adem\'as se probar\'a dentro de la inducci\'on que $F_{k} = n^{k} R - \left( d_1n^{k-1} + d_2n^{k-2} + \cdots + d_{k} \right)$, $\forall k \in \mathbb{N}$.
 \par 
 Como $R$ se representa seg\'un se indica en \eqref{eqRepresentacionDecimal1}, entonces, multiplicando por $n$ ambos miembros, el resultado es
 \begin{equation*}
  nR = \sum_{j=1}^{\infty} \frac{nd_j}{n^j} = d_1 +  \left( \sum_{j=1}^{\infty} \frac{d_{j+1}}{n^j} \right)
 \end{equation*}
 donde la cantidad entre par\'entesis de la expresi\'on anterior es un n\'umero positivo y menor que $1$; por lo tanto, $d_1 = \text{ent}(nR) = \text{ent}(nF_0)$ y $F_1 = \text{frac}(nR) = \text{frac}(nF_0) = nR - \text{ent}(nR) = nR - d_1$, lo cual prueba la base de inducci\'on.
 \par 
 Suponiendo ahora que los primeros $k$ valores $d_1, d_2, \ldots, d_k$ se pueden calcular como $d_{k} = \text{ent}(nF_{k-1})$ y $F_{k} = \text{frac}(nF_{k-1}) = n^{k}R - \left( d_1n^{k-1} + d_2n^{k-2} + \cdots + d_k \right)$, entonces se desea probar que $d_{k+1}$, el $(k+1)-$\'esimo d\'{\i}gito de la expansi\'on decimal en base $n$ de $R$, puede calcularse como $d_{k+1}=\text{ent}(nF_k)$, adem\'as $F_{k+1} = \text{frac}(nF_k) = n^{k+1} - \left( d_1n^{k} + d_2n^{k-1} + \cdots + d_kn + d_{k+1} \right)$.
 \par 
 Usando el resultado \eqref{eqRepresentacionDecimal1} y la base de inducci\'on, se tiene que
 \begin{equation*}
  F_k = n^k R - \left(  d_1n^{k-1} + d_2n^{k-2} + \cdots d_{k-1}n + d_k \right) = \sum_{j=1}^{\infty} \frac{d_{k+j}}{n^j}
 \end{equation*}
 entonces $nF_{k} = \sum_{j=1}^{\infty} \frac{nd_{k+j}}{n^j} = d_{k+1} + \left( \sum_{j=1}^{\infty} \frac{d_{k+1+j}}{n^j} \right)$, donde la cantidad entre par\'entesis es un n\'umero positivo y menor que $1$; por lo tanto, $d_{k+1} = \text{ent}(nF_k)$ y
 \begin{eqnarray*}
  F_{k+1} & = & \text{frac}(nF_k) \\
  & = & nF_k - \text{ent}(nF_k) \\
  & = & n\left[ n^kR - \left( d_1n^{k-1} + d_2n^{k-2} + \cdots d_{k-1}n + d_k \right) \right] - d_{k+1} \\
  & = & n^{k+1}R - \left( d_1n^{k} + d_2n^{k-1} + \cdots d_kn + d_{k+1} \right)
 \end{eqnarray*}
 lo cual concluye la demostraci\'on.${}_{\square}$
 \par 
 Usando lo ya demostrado, en el caso $n=3$, se procede a expresar los siguientes n\'umeros en base $3$ como sigue:
 \begin{enumerate}
  \item Dado que $R = 1/3 = 0.\overline{3}$ entonces 
  \begin{center}
   \begin{tabular}{rclcrclcrcl}
    $3R$ & $=$ & $1$ & \hspace{1.5cm} & $d_1 =$ & ent$(1)$ & $=1$ & \hspace{1.5cm} & $F_1=$ & frac$(1)$ & $=0$ 
   \end{tabular}
  \end{center}
  Como el valor de $F_1$ es cero, el proceso continua haciendo $d_k$ y $F_k$ iguales a $0$, por lo que se considera terminado. En consecuencia
  \begin{equation*}
   \frac{1}{3} = 0.1_\text{tres}
  \end{equation*}

  \item Dado que $R = 1/2 = 0.5$, entonces
  \begin{center}
   \begin{tabular}{rclcrclcrcl}
    $3R$ & $=$ & $1.5$ & \hspace{1.5cm} & $d_1 =$ & ent$(1.5)$ & $=1$ & \hspace{1.5cm} & $F_1=$ & frac$(1.5)$ & $=0.5$
   \end{tabular}
  \end{center}
  N\'otese que $F_0 = R = F_1$, luego los patrones $d_k = d_{k+1}$ y $F_k = F_{k+1}$ se dar\'{\i}an para $k\in\mathbb{N}$. En consecuencia
  \begin{equation*}
   \frac{1}{2} = 0.\overline{1}_{\text{tres}}
  \end{equation*}

  \item Dado que $R = 1/10 = 0.1$, entonces 
  \begin{center}
   \begin{tabular}{rclcrclcrcl}
    $3R$ & $=$ & $0.3$ & \hspace{1.5cm} & $d_1 =$ & ent$(0.3)$ & $=0$ & \hspace{1.5cm} & $F_1=$ & frac$(0.5)$ & $=0.3$ \\
    $3F_1$ & $=$ & $0.9$ & & $d_2 =$ & ent$(0.9)$ & $=0$ & & $F_2=$ & frac$(0.9)$ & $=0.9$ \\
    $3F_2$ & $=$ & $2.7$ & & $d_3 =$ & ent$(2.7)$ & $=2$ & & $F_3 =$ & frac$(2.7)$ & $=0.7$ \\
    $3F_3$ & $=$ & $2.1$ & & $d_4 =$ & ent$(2.1)$ & $=2$ & & $F_4=$ & frac$(2.1)$ & $=0.1$
   \end{tabular}
  \end{center}
  N\'otese que $F_0 = R = F_4$, luego los patrones $d_k = d_{k+4}$ y $F_k = F_{k+4}$ se dar\'{\i}an para $k\in\mathbb{N}$. En consecuencia
  \begin{equation*}
   \frac{1}{10} = 0.\overline{0022}_{\text{tres}}
  \end{equation*}

  \item Dado que $R = 11/27 = 0.\overline{407}$, entonces 
  \begin{center}
   \begin{tabular}{rclcrclcrcl}
    $3R$ & $=$ & $1.\overline{2}$ & \hspace{1.5cm} & $d_1 =$ & ent$\left( 1.\overline{2} \right)$ & $=1$ & \hspace{1.5cm} & $F_1=$ & frac$\left( 1.\overline{2} \right)$ & $=0.\overline{2}$ \\
    $3F_1$ & $=$ & $0.\overline{6}$ & & $d_2 =$ & ent$\left( 0.\overline{6} \right)$ & $=0$ & & $F_2=$ & frac$\left( 0.\overline{6} \right)$ & $=0.\overline{6}$ \\
    $3F_2$ & $=$ & $2$ & & $d_3 =$ & ent$(2)$ & $=2$ & & $F_3 =$ & frac$(2)$ & $=0$
   \end{tabular}
  \end{center}
  Como el valor de $F_3$ es cero, el proceso continua haciendo los siguientes valores $d_k$ y $F_k$ iguales a $0$, por lo que se considera terminado. En consecuencia
  \begin{equation*}
   \frac{11}{27} = 0.102_{\text{tres}}
  \end{equation*}
  que es a lo que se quer\'{\i}a llegar.${}_{\blacksquare}$
 \end{enumerate}
\end{solucion}

 \item \begin{enunciado}
 Pruebe que si sustituimos $2$ por $5$ en todas las f\'ormulas de (8), el resultado es un m\'etodo para hallar la expresi\'on en base $5$ de un n\'umero natural. Utilice esto para expresar los siguientes n\'umeros en base $5$.
 \begin{multicols}{4}
  \begin{enumerate}
   \item $10$
   \item $35$
   \item $721$
   \item $734$
  \end{enumerate}
 \end{multicols}
\end{enunciado}

\begin{solucion}
 La demostraci\'on ya se hizo en el enunciado 14 de esta secci\'on de ejercicios para el caso general $n\in\mathbb{N}\backslash\{ 1 \}$. Por lo que se procede hacer los c\'alculos cuando $n = 5$.
 \begin{enumerate}
  \item Dividiendo iteradamente $N=10$ entre $n=5$, se tiene que
  \begin{center}
   \begin{tabular}{rclrr}
    $10$ & $=$ & $5\times$ & $2 + 0$, & $b_0 = 0$ \\
    $2$ & $=$ & $5\times$ & $0 + 2$, & $b_1 = 2$
   \end{tabular}
  \end{center}
  As\'{\i} que la representaci\'on en base $5$ de $10$ es
  \begin{equation*}
   10 = b_1b_0{}_{\text{cinco}} = 20_{\text{cinco}}
  \end{equation*}

  \item Dividiendo iteradamente $N=35$ entre $n=5$, se tiene que
  \begin{center}
   \begin{tabular}{rclrr}
    $35$ & $=$ & $5\times$ & $7 + 0$, & $b_0 = 0$ \\
    $7$ & $=$ & $5\times$ & $1 + 2$, & $b_1 = 2$ \\
    $1$  & $=$ & $5\times$ & $0 + 1$, & $b_2 = 1$
   \end{tabular}
  \end{center}
  As\'{\i} que la representaci\'on en base $5$ de $35$ es
  \begin{equation*}
   35 = b_2b_1b_0{}_{\text{cinco}} = 120_{\text{cinco}}
  \end{equation*}

  \item Dividiendo iteradamente $N=721$ entre $n=5$, se tiene que
  \begin{center}
   \begin{tabular}{rclrr}
    $721$ & $=$ & $5\times$ & $144 + 1$, & $b_0 = 1$ \\
    $144$ & $=$ & $5\times$ & $28 + 4$, & $b_1 = 4$ \\
    $28$  & $=$ & $5\times$ & $5 + 3$, & $b_2 = 3$ \\
    $5$  & $=$ & $5\times$ & $1 + 0$, & $b_3 = 0$ \\
    $1$  & $=$ & $5\times$ & $0 + 1$, & $b_4 = 1$ 
   \end{tabular}
  \end{center}
  As\'{\i} que la representaci\'on en la base $5$ de $721$ es
  \begin{equation*}
   721 = b_4b_3b_2b_1b_0{}_{\text{cinco}} = 10341_{\text{cinco}}
  \end{equation*}

  \item Dividiendo iteradamente $N=734$ entre $n=5$, se tiene que
  \begin{center}
   \begin{tabular}{rclrr}
    $734$ & $=$ & $5\times$ & $146 + 4$, & $b_0 = 4$ \\
    $146$ & $=$ & $5\times$ & $29 + 1$, & $b_1 = 1$ \\
    $29$  & $=$ & $5\times$ & $5 + 4$, & $b_2 = 4$ \\
    $5$  & $=$ & $5\times$ & $1 + 0$, & $b_3 = 0$ \\
    $1$  & $=$ & $5\times$ & $0 + 1$, & $b_4 = 1$ 
   \end{tabular}
  \end{center}
  As\'{\i} que la representaci\'on en la base $5$ de $734$ es
  \begin{equation*}
   734 = b_4b_3b_2b_1b_0{}_{\text{cinco}} = 10414_{\text{cinco}}
  \end{equation*}
  que es a lo que se quer\'{\i}a llegar.${}_{\blacksquare}$
 \end{enumerate}
\end{solucion}

 \item \begin{enunciado}
 Pruebe que si sustituimos $2$ por $5$ en (22), el resultado es un m\'etodo para hallar la expresi\'on en base $5$ de un n\'umero positivo $R$ tal que $0 < R < 1$. Utilice esto para expresar los siguientes n\'umeros en base $5$.
 \begin{multicols}{4}
  \begin{enumerate}
   \item $1/3$
   \item $1/2$
   \item $1/10$
   \item $154/625$
  \end{enumerate}
 \end{multicols}
\end{enunciado}

\begin{solucion}
 La demostraci\'on ya se hizo en el enunciado 15 de esta secci\'on de ejercicios para el caso general $n \in \mathbb{N}\backslash\{1\}$. Por lo que se procede a hacer los c\'alculos cuando $n=5$.
 \begin{enumerate}
  \item Dado que $R = 1/3 = 0.\overline{3}$ entonces 
  \begin{center}
   \begin{tabular}{rclcrclcrcl}
    $5R$ & $=$ & $1.\overline{6}$ & \hspace{1.5cm} & $d_1 =$ & ent$(1.\overline{6})$ & $=1$ & \hspace{1.5cm} & $F_1=$ & frac$(1.\overline{6})$ & $=0.\overline{6}$ \\
    $5F_1$ & $=$ & $3.\overline{3}$ & & $d_2 =$ & ent$(3.\overline{3})$ & $=3$ & & $F_2 =$ & frac$(3.\overline{3})$ & $0.\overline{3}$
   \end{tabular}
  \end{center}
  N\'otese que $F_0 = R = F_2$, luego los patrones $d_k = d_{k+2}$ y $F_k = F_{k+2}$ se dar\'{\i}an para $k\in\mathbb{N}$. En consecuencia
  \begin{equation*}
   \frac{1}{3} = 0.\overline{13}_{\text{cinco}}
  \end{equation*}

  \item Dado que $R = 1/2 = 0.5$, entonces
  \begin{center}
   \begin{tabular}{rclcrclcrcl}
    $5R$ & $=$ & $2.5$ & \hspace{1.5cm} & $d_1 =$ & ent$(2.5)$ & $=2$ & \hspace{1.5cm} & $F_1=$ & frac$(2.5)$ & $=0.5$
   \end{tabular}
  \end{center}
  N\'otese que $F_0 = R = F_1$, luego los patrones $d_k = d_{k+1}$ y $F_k = F_{k+1}$ se dar\'{\i}an para $k\in\mathbb{N}$. En consecuencia
  \begin{equation*}
   \frac{1}{2} = 0.\overline{2}_{\text{cinco}}
  \end{equation*}

  \item Dado que $R = 1/10 = 0.1$, entonces 
  \begin{center}
   \begin{tabular}{rclcrclcrcl}
    $5R$ & $=$ & $0.5$ & \hspace{1.5cm} & $d_1 =$ & ent$(0.5)$ & $=0$ & \hspace{1.5cm} & $F_1=$ & frac$(0.5)$ & $=0.5$ \\
    $5F_1$ & $=$ & $2.5$ & & $d_2 =$ & ent$(2.5)$ & $=2$ & & $F_2=$ & frac$(2.5)$ & $=0.5$
   \end{tabular}
  \end{center}
  N\'otese que $F_1 = R = F_2$, luego los patrones $d_k = d_{k+1}$ y $F_k = F_{k+1}$ se dar\'{\i}an para $k\in\mathbb{N}\backslash\{ 1 \}$. En consecuencia
  \begin{equation*}
   \frac{1}{10} = 0.0\overline{2}_{\text{cinco}}
  \end{equation*}

  \item Dado que $R = 154/625 = 0.2464$, entonces 
  \begin{center}
   \begin{tabular}{rclcrclcrcl}
    $5R$ & $=$ & $1.232$ & \hspace{1.2cm} & $d_1 =$ & ent$\left( 1.232 \right)$ & $=1$ & \hspace{1cm} & $F_1=$ & frac$\left( 1.232 \right)$ & $=0.232$ \\
    $5F_1$ & $=$ & $1.16$ & & $d_2 =$ & ent$\left( 1.16 \right)$ & $=1$ & & $F_2=$ & frac$\left( 1.16 \right)$ & $=0.16$ \\
    $5F_2$ & $=$ & $0.8$ & & $d_3 =$ & ent$(0.8)$ & $=0$ & & $F_3 =$ & frac$(0.8)$ & $=0.8$ \\
    $5F_3$ & $=$ & $4$ & & $d_4 =$ & ent$(4)$ & $=4$ & & $F_4 =$ & frac$(4)$ & $=0$ \\
   \end{tabular}
  \end{center}
  Como el valor de $F_4$ es cero, el proceso continua haciendo los siguientes valores $d_k$ y $F_k$ iguales a $0$, por lo que se considera terminado. En consecuencia
  \begin{equation*}
   \frac{154}{625} = 0.1104_{\text{cinco}}
  \end{equation*}
  que es a lo que se quer\'{\i}a llegar.${}_{\blacksquare}$
 \end{enumerate}

\end{solucion}

\end{enumerate}

\newpage

\section{An\'alisis del error}

\begin{enumerate}
 \item \begin{enunciado}
 En cada uno de los casos siguientes, halle el error absoluto $E_x$ y el error relativo $R_x$ y determine el n\'umero de cifras significativas de la aproximaci\'on:
 \begin{enumerate}
  \item $x = 2.71828182$, $\widehat{x} = 2.7182$
  \item $y = 98\,350$, $\widehat{y} = 98\,000$
  \item $z = 0.000068$, $\widehat{z} = 0.00006$
 \end{enumerate}
\end{enunciado}

\begin{solucion}
 $\phantom{0}$
 \begin{enumerate}
  \item El error absoluto se calcula a continuaci\'on
  \begin{equation*}
   E_x = \left| x - \widehat{x} \right| = \left| 2.71828182 - 2.7182 \right| = 0.00008182
  \end{equation*}
  mientras que el error relativo se calcula como sigue
  \begin{equation*}
   R_x = \frac{\left| x - \widehat{x} \right|}{|x|} = \frac{\left| 2.71828182 - 2.7182 \right|}{|2.71828182|} = \frac{0.00008182}{2.71828182} \approx 0.00003009989597
  \end{equation*}
  Por lo tanto, como $\frac{10^{-5}}{2} = 0.000005$ y $\frac{10^{-4}}{2} = 0.00005$, entonces
  \begin{equation*}
   \frac{10^{-5}}{2} < R_x < \frac{10^{-4}}{2}
  \end{equation*}
  y, por tanto, $\widehat{x}$ es una aproximaci\'on a $x$ con cuatro cifras significativas.${}_{\square}$
  
  \item El error absoluto se calcula a continuaci\'on
  \begin{equation*}
   E_y = \left| y - \widehat{y} \right| = \left| 98\,350 - 98\,000 \right| = 350
  \end{equation*}
  mientras que el error relativo se calcula como sigue
  \begin{equation*}
   R_y = \frac{\left| y - \widehat{y} \right|}{|y|} = \frac{\left| 98\,350 - 98\,000 \right|}{|98\,350|} = \frac{350}{98\,350} \approx 0.0035587188612
  \end{equation*}
  Por lo tanto, como
  \begin{equation*}
   \frac{10^{-3}}{2} < R_y < \frac{10^{-2}}{2}
  \end{equation*}
  entonces $\widehat{y}$ es una aproximaci\'on a $y$ con dos cifras significativas.${}_{\square}$
  
  \item El error absoluto se calcula a continuaci\'on
  \begin{equation*}
   E_z = \left| z - \widehat{z} \right| = \left| 0.000068 - 0.00006 \right| = 0.000008
  \end{equation*}
  mientras que el error relativo se calcula como sigue
  \begin{equation*}
   R_z = \frac{\left| z - \widehat{z} \right|}{|z|} = \frac{\left| 0.000068 - 0.00006 \right|}{|0.000068|} = \frac{0.000008}{0.000068} \approx 0.117647
  \end{equation*}
  Por lo tanto, como
  \begin{equation*}
   R_z < \frac{10^{0}}{2}
  \end{equation*}
  entonces $\widehat{z}$ es una aproximaci\'on a $z$ sin cifras significativas, que es a lo que se quer\'{\i}a llegar.${}_{\blacksquare}$
 \end{enumerate}

\end{solucion}

 \item \begin{enunciado}
 Complete el siguiente c\'alculo
 \begin{equation*}
  \int_{0}^{1/4} e^{x^2} \, dx \approx \int_{0}^{1/4} \left( 1 + x^2 + \frac{x^4}{2!} + \frac{x^6}{3!} \right) \, dx = \widehat{p}
 \end{equation*}
 Determine qu\'e tipo de error se presenta en esta situaci\'on y compare su resultado con el valor exacto $p = 0.2553074606$.
\end{enunciado}

\begin{solucion}
 Integrando t\'ermino a t\'ermino el polinomio, se tiene que
 \begin{eqnarray*}
  \int_{0}^{1/4} e^{x^2} \, dx & \approx & \int_{0}^{1/4} \left( 1 + x^2 + \frac{x^4}{2!} + \frac{x^6}{3!} \right) \, dx = \left.\left( x + \frac{x^3}{3} + \frac{x^5}{5\left(2!\right)} + \frac{x^7}{7\left(3!\right)} \right) \right|_{x=0}^{x=1/4} \\
  & = & \frac{1}{4} + \frac{1}{192} + \frac{1}{10\,240} + \frac{1}{688\,128} = \frac{860\,160 + 17\,920 + 336 + 5}{3\,440\,640} = \frac{878\,421}{3\,440\,640} \\
  & = & \frac{292\,807}{1\,146\,880} = 0.25530744801339\overline{285714}.
 \end{eqnarray*}
 Dado que la expresi\'on matem\'atica original se ``reemplaz\'o'' por una f\'ormula m\'as sencilla, entonces se tiene, por definici\'on, un error de truncamiento.
 \par 
 Se tiene adem\'as lo siguiente:
 \begin{equation*}
  E_p = \left| p - \widehat{p} \right| = |0.2553074606 - 0.25530744801339\overline{285714}| = 0.000000017708660\overline{714285}
 \end{equation*}
 y
 \begin{equation*}
  R_p = \frac{\left| p - \widehat{p} \right|}{|p|} = \frac{0.000000017708660\overline{714285}}{0.2553074606} \approx 0.00000006971461
 \end{equation*}
 por lo que $\frac{10^{-7}}{2} < R_p < \frac{10^{-6}}{2}$, y, por tanto, $\widehat{p}$ es una aproximaci\'on a $p$ con seis cifras significativas, que es a lo que se quer\'{\i}a llegar.${}_{\blacksquare}$
\end{solucion}

 \item \begin{enunciado}
 $\phantom{0}$
 \begin{enumerate}
  \item Consideremos los datos $p_1 = 1.414$ y $p_2 = 0.09125$, que viene dados con una precisi\'on de cuatro cifras significativas. Determine el resultado adecuado en esta situaci\'on de la suma $p_1 + p_2$ y el producto $p_1p_2$.
  
  \item Consideremos los datos $p_1 = 31.415$ y $p_2 = 0.027182$, que viene dados con una precisi\'on de cinco cifras significatvias. Determine el resultado adecuado en esta situaci\'on de la suma $p_1 + p_2$ y el producto $p_1p_2$.
 \end{enumerate}
\end{enunciado}

\begin{solucion}
 Para resolver los siguientes ejercicios se encontrar\'a primero la suma y el producto real y luego se aplicar\'a la poda o el redondeo seg\'un si despu\'es las primeras cifras significativas (las primeras cifras sin contar los ceros a la izquierda), sigue una cifra menor a 5 o mayor o igual que 5, respectivamente, entonces, el resultado se vuelve f\'acil de ver que es de tantas cifras significativas como se pidi\'o.
 \begin{enumerate}
  \item El valor real de $p_1 + p_2$ es:
  \begin{equation*}
   p_1 + p_2 = 1.414 + 0.09125 = 1.50525
  \end{equation*}
  Entonces, con cuatro cifras significativas, esto vale
  \begin{equation*}
   \widehat{p_1 + p_2} = fl_{\text{pod}}(1.50525) = 1.505
  \end{equation*}
  Por otro lado, el producto real es:
  \begin{equation*}
   p_1p_2 = (1.414)(0.09125) = 0.1290275
  \end{equation*}
  Entonces, con cuatro cifras significativas, esto vale
  \begin{equation*}
   \widehat{p_1p_2} = fl_{\text{pod}}(0.1290275) = 0.1290
  \end{equation*}
  Por lo tanto, con cuatro cifras significativas, la suma es: $p_1 + p_2 = 1.505$ y el producto es: $p_1p_2 = 0.1290$.${}_{\square}$
  
  \item El valor real de $p_1 + p_2$ es:
  \begin{equation*}
   p_1 + p_2 = 31.415 + 0.027182 = 31.442182
  \end{equation*}
  Entonces, con cinco cifras significativas, esto vale
  \begin{equation*}
   \widehat{p_1 + p_2} = fl_{\text{pod}}(31.442182) = 31.442
  \end{equation*}
  Por otro lado, el producto real es:
  \begin{equation*}
   p_1p_2 = (31.415)(0.027182) = 0.85392253
  \end{equation*}
  Entonces, con cinco cifras significativas, esto vale
  \begin{equation*}
   \widehat{p_1p_2} = fl_{\text{pod}}(0.85392253) = 0.85392
  \end{equation*}
  Por lo tanto, con cinco cifras significativas, la suma es: $p_1 + p_2 = 31.442182$ y el producto es: $p_1p_2 = 0.85392$, que es a lo que se quer\'{\i}a llegar.${}_{\blacksquare}$
 \end{enumerate}
\end{solucion}

 \item \begin{enunciado}
 Complete los siguientes c\'alculos y diga qu\'e tipo de error se presenta en cada situaci\'on.
 \begin{enumerate}
  \item $\displaystyle{ \frac{\sin\left( \frac{\pi}{4} + 0.00001 \right) - \sin\left( \frac{\pi}{4} \right)}{0.00001} = \frac{0.70711385222 - 0.70710678119}{0.00001} = \cdots }$
  
  \item $\displaystyle{ \frac{\ln(2 + 0.00005) - \ln(2)}{0.00005} = \frac{0.69317218025 - 0.69314718056}{0.00005} = \cdots }$
 \end{enumerate}
\end{enunciado}

\begin{solucion}
 $\phantom{0}$
 \begin{enumerate}
  \item Siguiendo con los c\'alculos, se tiene que
  \begin{eqnarray*}
   \frac{\sin\left( \frac{\pi}{4} + 0.00001 \right) - \sin\left( \frac{\pi}{4} \right)}{0.00001} & = & \frac{0.70711385222 - 0.70710678119}{0.00001} \\
   & = & \frac{0.00000707103}{0.00001} = 0.707103
  \end{eqnarray*}
  El cual contiene un error por p\'erdida de cifras significativas al estar restando valores muy cercanos.
  
  \item Siguiendo con los c\'alculos, se tiene que
  \begin{eqnarray*}
   \frac{\ln(2 + 0.00005) - \ln(2)}{0.00005} & = & \frac{0.69317218025 - 0.69314718056}{0.00005} \\
   & = & \frac{0.00002499969}{0.00005} = 0.4999938
  \end{eqnarray*}
  Nuevamente, el error es por p\'erdida de cifras significativas debido a la resta de valores muy cercanos, que es a lo que se quer\'{\i}a llegar.${}_{\blacksquare}$
 \end{enumerate}
\end{solucion}

 \item \begin{enunciado}
 La p\'erdida de cifras significativas se puede evitar a veces reordenando los t\'erminos de la funci\'on usando una identidad conocida del \'algebra o la trigonometr\'{\i}a. Encuentre, en cada uno de los siguientes casos, una f\'ormula equivalente a la dada que evite la p\'erdida de cifras significativas.
 \begin{enumerate}
  \item $\ln(x+1) - \ln(x)$ para $x$ grande.
  
  \item $\sqrt{x^2+1} - x$ para $x$ grande.
  
  \item $\cos^2(x) - \sin^2(x)$ para $x \approx \pi/4$.
  
  \item $\displaystyle{\sqrt{\frac{1+\cos(x)}{2}}}$ para $x \approx \pi$.
 \end{enumerate}
\end{enunciado}

\begin{solucion}
 $\phantom{0}$
 \begin{enumerate}
  \item Por propiedades de logaritmos:
  \begin{equation*}
   \ln(x+1) - \ln(x) = \ln\left( \frac{x+1}{x} \right) = \ln\left(1 + \frac{1}{x} \right)._{\square}
  \end{equation*}
  
  \item Multiplicando y dividiendo por el conjugado, se tiene que:
  \begin{equation*}
   \sqrt{x^2 + 1} - x = \frac{x^2 + 1 - x^2}{\sqrt{x^2 + 1} + x} = \frac{1}{\sqrt{x^2 + 1} + x}._{\square}
  \end{equation*}

  \item Usando la propiedad de la suma de \'angulos en un coseno, $\cos(\alpha+\beta) = \cos(\alpha)\cos(\beta) - \sin(\alpha)\sin(\beta)$, en el que en este caso $\alpha=\beta=x$, se tiene que:
  \begin{equation*}
   \cos^2(x) - \sin^2(x) = \cos\left(2x\right)._{\square}
  \end{equation*}

  \item Usando de nuevo la propiedad de la suma de \'angulos en un coseno, pero en este caso haciendo $\alpha = \beta = \frac{x}{2}$, y la identidad $\sin^2(x) + \cos^2(x) = 1$, se tiene que:
  \begin{equation*}
   \cos(x) = \cos^2\left( \frac{x}{2} \right) - \sin^2 \left( \frac{x}{2} \right) = \cos^2\left( \frac{x}{2} \right) - \left[ 1 - \cos^2\left( \frac{x}{2} \right) \right] = 2\cos^2\left( \frac{x}{2} \right) - 1
  \end{equation*}
  por lo que, despejando $\cos^2\left( \frac{x}{2} \right)$, se tiene que
  \begin{equation*}
   \cos^2\left( \frac{x}{2} \right) = \frac{1+\cos(x)}{2}
  \end{equation*}
  Por lo tanto
  \begin{equation*}
   \sqrt{\frac{1+\cos(x)}{2}} = \cos\left( \frac{x}{2} \right)
  \end{equation*}
  que es a lo que se quer\'{\i}a llegar.${}_{\blacksquare}$
 \end{enumerate}

\end{solucion}

 \item \begin{enunciado}
 \textit{Evaluaci\'on Polinomial}. Sean
 \begin{equation*}
  P(x) = x^3 - 3x^2 + 3x - 1, \qquad Q(x) = \left((x-3)x+3\right)x-1, \qquad R(x) = (x-1)^3
 \end{equation*}
 \begin{enumerate}
  \item Usando aritm\'etica en coma flotante con cuatro cifras y redondeo, calcule $P(2.72)$, $Q(2.72)$ y $R(2.72)$. En el c\'alculo de $P(x)$, suponga que $(2.72)^3 = 20.12$ y $(2.72)^2 = 7.398$.
  
  \item Usando aritm\'etica en coma flotante con cuatro cifras y redondeo, calcule $P(0.975)$, $Q(0.975)$ y $R(0.975)$. En el c\'alculo de $P(x)$, suponga que $(0.975)^3 = 0.9268$, $(0.975)^2 = 0.9506$.
 \end{enumerate}
\end{enunciado}

\begin{solucion}
 Aunque no se explica antes, no se va a considerar el signo menos de los n\'umeros negativos como un espacio o cifra significativa. Adem\'as, al comparar que se est\'an dando para $P(x)$, se entiende que el redondeo es al n\'umero m\'as pr\'oximo y no se est\'a considerando al menor valor mayor que el valor real.
 \begin{enumerate}
  \item 
  \begin{eqnarray*}
   P(2.72) & = & 2.72^3 - 3\left(2.72^2\right) + 3(2.72) - 1 = 20.12 - 3(7.398) + 8.16 - 1 = 27.28 - 22.19 \\
   & = & 5.09 \\
   Q(2.72) & = & \left((2.72 - 3)(2.72) + 3 \right)(2.72) - 1 = \left( (-0.28)(2.72) + 3 \right)(2.72) - 1 \\
   & = & (-0.7616+3)(2.72) - 1 = (2.238)(2.72)-1 = 6.087-1 \\
   & = & 5.087 \\
   R(2.72) & = & (2.72-1)^3 = (1.72)^3 = 5.088
  \end{eqnarray*}
  
  \item 
  \begin{eqnarray*}
   P(0.975) & = & 0.975^3 - 3(0.975)^2 + 3(0.975) - 1 = 0.9268 - 3(0.9506) + 2.925 - 1 \\
   & = & 2.852 - 2.852 \\
   & = & 0 \\
   Q(0.975) & = & \left( (0.975-3)(0.975)+3 \right)(0.975)-1 = \left( (-2.025)(0.975) +3\right)(0.975) - 1 \\
   & = & (-1.974 + 3)(0.975)-1 = (1.026)(0.975)-1 = 1-1 \\
   & = & 0 \\
   R(0.975) & = & (0.975-1)^3 = (-0.025)^3 = -0.00001563
  \end{eqnarray*}
  que es a lo que se quer\'{\i}a llegar.${}_{\blacksquare}$
 \end{enumerate}
\end{solucion}

 \item \begin{enunciado}
 usando aritm\'etica en coma flotante con tres cifras y redondeo, calcule las siguientes sumas (sumando en el orden que se indica):
 \begin{multicols}{2}
  \begin{enumerate}[(a)]
   \item $\sum_{k=1}^6 \frac{1}{3^k}$
   
   \item $\sum_{k=1}^6 \frac{1}{3^{7-k}}$
  \end{enumerate}
 \end{multicols}
\end{enunciado}

\begin{solucion}
 Para la soluci\'on, se indicar\'a con par\'entesis el orden de los sumandos, para mayor claridad, realizando paso a paso cada suma.
 \begin{enumerate}[(a)]
  \item 
  \begin{eqnarray*}
   \sum_{k=1}^6 \frac{1}{3^k} & = & \frac{1}{3} + \frac{1}{9} + \frac{1}{27} + \frac{1}{81} + \frac{1}{243} + \frac{1}{729} \\
   & = & \left( \left( \left( (0.333 + 0.111) + 0.037\right) + 0.0123\right) + 0.00412\right) + 0.00137 \\
   & = & \left( \left( (0.444 + 0.037) + 0.0123\right) + 0.00412\right) + 0.00137 \\
   & = & \left( (0.481 + 0.0123) + 0.00412\right) + 0.00137 \\
   & = & (0.493 + 0.00412) + 0.00137 = 0.497 + 0.00137 \\
   & = & 0.498._{\square}
  \end{eqnarray*}

  \item 
  \begin{eqnarray*}
   \sum_{k=1}^6 \frac{1}{3^{7-k}} & = & \frac{1}{729} + \frac{1}{243} + \frac{1}{81} + \frac{1}{27} + \frac{1}{9} \frac{1}{3} \\
   & = & \left( \left( \left( (0.00137 + 0.00412) + 0.0123\right) + 0.037\right) + 0.111\right) + 0.333 \\
   & = & \left( \left( (0.00549 + 0.0123) + 0.037\right) + 0.111\right) + 0.333 \\
   & = & \left( (0.0178 + 0.037) + 0.111\right) + 0.333 \\
   & = & (0.0548 + 0.111) + 0.333 = 0.166 + 0.333 \\
   & = & 0.499
  \end{eqnarray*}
  que es a lo que se quer\'{\i}a llegar.${}_{\blacksquare}$
 \end{enumerate}
\end{solucion}

 \item \begin{enunciado}
 Discuta la propagaci\'on de los errores en las siguientes operaciones:
 \begin{enumerate}[(a)]
  \item La suma de tres n\'umeros:
  \begin{equation*}
   p + q +r = \left( \widehat{p} + \varepsilon_p \right) + \left( \widehat{q} + \varepsilon_q \right) + \left( \widehat{r} + \varepsilon_r \right).
  \end{equation*}

  \item El cociente de dos n\'umeros: $\displaystyle{ \frac{p}{q} = \frac{\widehat{p} + \varepsilon_p}{\widehat{q} + \varepsilon_q} }$.

  \item El producto de tres n\'umeros:
  \begin{equation*}
   pqr = \left( \widehat{p} + \varepsilon_p \right)\left( \widehat{q} + \varepsilon_q \right)\left( \widehat{r} + \varepsilon_r \right).
  \end{equation*}
 \end{enumerate}
\end{enunciado}

\begin{solucion}
 
\end{solucion}

\end{enumerate}

\end{document}
