\begin{enunciado}
 Siga el Ejemplo 1.12 para convertir los siguientes n\'umeros en fracciones binarias peri\'odicas.
 \begin{multicols}{3}
  \begin{enumerate}[(a)]
   \item $1/10$
   \item $1/3$
   \item $1/7$
  \end{enumerate}
 \end{multicols}
\end{enunciado}

\begin{solucion}
 Para hallar la representaci\'on como en el ejemplo mencionado, se requirir\'a una representaci\'on decimal y, seguidamente, se realizar\'an las multiplicaciones por 2 iteradamente para obtener las partes enteras y las partes fraccionarias como se muestra a continuaci\'on.
 \begin{enumerate}[(a)]
  \item Dado que $R = \frac{1}{10} = 0.1$, entonces
  \begin{center}
   \begin{tabular}{rclrclrcl}
    & & & \hspace{1.5cm} & & \hspace{1.5cm} \\
    $2R$ & $=$ & $0.2$ & $d_1 =$ & $\text{ent}(0.2)$ & $=0$ & $F_1 =$ & $\text{frac}(0.2)$ & $=0.2$ \\
    $2F_1$ & $=$ & $0.4$ & $d_2 =$ & $\text{ent}(0.4)$ & $=0$ & $F_2 =$ & $\text{frac}(0.4)$ & $=0.4$ \\
    $2F_2$ & $=$ & $0.8$ & $d_3 =$ & $\text{ent}(0.8)$ & $=0$ & $F_3 =$ & $\text{frac}(0.8)$ & $=0.8$ \\
    $2F_3$ & $=$ & $1.6$ & $d_4 =$ & $\text{ent}(1.6)$ & $=1$ & $F_4 =$ & $\text{frac}(1.6)$ & $=0.6$ \\
    $2F_4$ & $=$ & $1.2$ & $d_5 =$ & $\text{ent}(1.2)$ & $=1$ & $F_5 =$ & $\text{frac}(1.2)$ & $=0.2$ \\
    $2F_5$ & $=$ & $0.4$ & $d_6 =$ & $\text{ent}(0.4)$ & $=0$ & $F_6 =$ & $\text{frac}(0.4)$ & $=0.4$ \\
   \end{tabular}
  \end{center}
  N\'otese que $2F_1 = 0.4 = 2F_5$, luego los patrones $d_k = d_{k+4}$ y $F_k = F_{k+4}$ se dar\'{\i}an para $k \in \mathbb{N}\backslash\{ 1 \}$. En consecuencia,
  \begin{equation*}
   \frac{1}{10} = 0.0\overline{0011}_{\text{dos}}
  \end{equation*}

  \item Dado que $R = \frac{1}{3} = 0.\overline{3}$, entonces 
  \begin{center}
   \begin{tabular}{rclrclrcl}
    & & & \hspace{1.5cm} & & \hspace{1.5cm} \\
    $2R$ & $=$ & $0.\overline{6}$ & $d_1 =$ & $\text{ent}\left(0.\overline{6}\right)$ & $=0$ & $F_1 =$ & $\text{frac}\left(0.\overline{6}\right)$ & $=0.\overline{6}$ \\
    $2F_1$ & $=$ & $1.\overline{3}$ & $d_2 =$ & $\text{ent}\left(1.\overline{3}\right)$ & $=1$ & $F_2 =$ & $\text{frac}\left(1.\overline{3}\right)$ & $=0.\overline{3}$ \\
    $2F_2$ & $=$ & $0.\overline{6}$ & $d_3 =$ & $\text{ent}\left(0.\overline{6}\right)$ & $=0$ & $F_3 =$ & $\text{frac}\left(0.\overline{6}\right)$ & $=0.\overline{6}$ \\
    $2F_3$ & $=$ & $1.\overline{3}$ & $d_4 =$ & $\text{ent}\left(1.\overline{3}\right)$ & $=1$ & $F_4 =$ & $\text{frac}\left(1.\overline{3}\right)$ & $=0.\overline{3}$ \\
   \end{tabular}
  \end{center}
  N\'otese que $2F_1 = 1.\overline{3} = 2F_3$, luego los patrones $d_k = d_{k+2}$ y $F_k = F_{k+2}$ se dar\'{\i}an para $k \in \mathbb{N}$. En consecuencia,
  \begin{equation*}
   \frac{1}{3} = 0.\overline{01}_{\text{dos}}
  \end{equation*}

  \item Dado que $R = \frac{1}{7} = 0.\overline{142857}$, entonces 
  \begin{center}
   \begin{tabular}{rcl}
    & \hspace{6cm} & \\
    $2R = 0.\overline{285714}$ & $d_1 = \text{ent}\left(0.\overline{285714}\right) = 0$ & $F_1 = \text{frac}\left(0.\overline{285714}\right) = 0.\overline{285714}$ \\
    $2F_1 = 0.\overline{571428}$ & $d_2 = \text{ent}\left(0.\overline{571428}\right) = 0$ & $F_2 = \text{frac}\left(0.\overline{571428}\right) = 0.\overline{571428}$ \\
    $2F_2 = 1.\overline{142857}$ & $d_3 = \text{ent}\left(1.\overline{142857}\right) = 1$ & $F_3 = \text{frac}\left(1.\overline{142857}\right) = 0.\overline{142857}$ \\
    $2F_3 = 0.\overline{285714}$ & $d_4 = \text{ent}\left(0.\overline{285714}\right) = 0$ & $F_4 = \text{frac}\left(0.\overline{285714}\right) = 0.\overline{285714}$ \\
    $2F_4 = 0.\overline{571428}$ & $d_5 = \text{ent}\left(0.\overline{571428}\right) = 0$ & $F_5 = \text{frac}\left(0.\overline{571428}\right) = 0.\overline{571428}$ \\
   \end{tabular}
  \end{center}
  N\'otese que $2F_1 = 0.\overline{571428} = F_4$, luego entonces los patrones $d_k = d_{k+3}$ y $F_k = F_{k+3}$ se dar\'{\i}an para $k \in \mathbb{N}$. En consecuencia
  \begin{equation*}
   \frac{1}{7} = 0.\overline{001}_{\text{dos}}
  \end{equation*}
  que es a lo que se quer\'{\i}a llegar${}_{\blacksquare}$
 \end{enumerate}
\end{solucion}
