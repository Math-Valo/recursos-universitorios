\begin{enunciado}
 Pruebe que si sustituimos $2$ por $5$ en (22), el resultado es un m\'etodo para hallar la expresi\'on en base $5$ de un n\'umero positivo $R$ tal que $0 < R < 1$. Utilice esto para expresar los siguientes n\'umeros en base $5$.
 \begin{multicols}{4}
  \begin{enumerate}
   \item $1/3$
   \item $1/2$
   \item $1/10$
   \item $154/625$
  \end{enumerate}
 \end{multicols}
\end{enunciado}

\begin{solucion}
 La demostraci\'on ya se hizo en el enunciado 15 de esta secci\'on de ejercicios para el caso general $n \in \mathbb{N}\backslash\{1\}$. Por lo que se procede a hacer los c\'alculos cuando $n=5$.
 \begin{enumerate}
  \item Dado que $R = 1/3 = 0.\overline{3}$ entonces 
  \begin{center}
   \begin{tabular}{rclcrclcrcl}
    $5R$ & $=$ & $1.\overline{6}$ & \hspace{1.5cm} & $d_1 =$ & ent$(1.\overline{6})$ & $=1$ & \hspace{1.5cm} & $F_1=$ & frac$(1.\overline{6})$ & $=0.\overline{6}$ \\
    $5F_1$ & $=$ & $3.\overline{3}$ & & $d_2 =$ & ent$(3.\overline{3})$ & $=3$ & & $F_2 =$ & frac$(3.\overline{3})$ & $0.\overline{3}$
   \end{tabular}
  \end{center}
  N\'otese que $F_0 = R = F_2$, luego los patrones $d_k = d_{k+2}$ y $F_k = F_{k+2}$ se dar\'{\i}an para $k\in\mathbb{N}$. En consecuencia
  \begin{equation*}
   \frac{1}{3} = 0.\overline{13}_{\text{cinco}}
  \end{equation*}

  \item Dado que $R = 1/2 = 0.5$, entonces
  \begin{center}
   \begin{tabular}{rclcrclcrcl}
    $5R$ & $=$ & $2.5$ & \hspace{1.5cm} & $d_1 =$ & ent$(2.5)$ & $=2$ & \hspace{1.5cm} & $F_1=$ & frac$(2.5)$ & $=0.5$
   \end{tabular}
  \end{center}
  N\'otese que $F_0 = R = F_1$, luego los patrones $d_k = d_{k+1}$ y $F_k = F_{k+1}$ se dar\'{\i}an para $k\in\mathbb{N}$. En consecuencia
  \begin{equation*}
   \frac{1}{2} = 0.\overline{2}_{\text{cinco}}
  \end{equation*}

  \item Dado que $R = 1/10 = 0.1$, entonces 
  \begin{center}
   \begin{tabular}{rclcrclcrcl}
    $5R$ & $=$ & $0.5$ & \hspace{1.5cm} & $d_1 =$ & ent$(0.5)$ & $=0$ & \hspace{1.5cm} & $F_1=$ & frac$(0.5)$ & $=0.5$ \\
    $5F_1$ & $=$ & $2.5$ & & $d_2 =$ & ent$(2.5)$ & $=2$ & & $F_2=$ & frac$(2.5)$ & $=0.5$
   \end{tabular}
  \end{center}
  N\'otese que $F_1 = R = F_2$, luego los patrones $d_k = d_{k+1}$ y $F_k = F_{k+1}$ se dar\'{\i}an para $k\in\mathbb{N}\backslash\{ 1 \}$. En consecuencia
  \begin{equation*}
   \frac{1}{10} = 0.0\overline{2}_{\text{cinco}}
  \end{equation*}

  \item Dado que $R = 154/625 = 0.2464$, entonces 
  \begin{center}
   \begin{tabular}{rclcrclcrcl}
    $5R$ & $=$ & $1.232$ & \hspace{1.2cm} & $d_1 =$ & ent$\left( 1.232 \right)$ & $=1$ & \hspace{1cm} & $F_1=$ & frac$\left( 1.232 \right)$ & $=0.232$ \\
    $5F_1$ & $=$ & $1.16$ & & $d_2 =$ & ent$\left( 1.16 \right)$ & $=1$ & & $F_2=$ & frac$\left( 1.16 \right)$ & $=0.16$ \\
    $5F_2$ & $=$ & $0.8$ & & $d_3 =$ & ent$(0.8)$ & $=0$ & & $F_3 =$ & frac$(0.8)$ & $=0.8$ \\
    $5F_3$ & $=$ & $4$ & & $d_4 =$ & ent$(4)$ & $=4$ & & $F_4 =$ & frac$(4)$ & $=0$ \\
   \end{tabular}
  \end{center}
  Como el valor de $F_4$ es cero, el proceso continua haciendo los siguientes valores $d_k$ y $F_k$ iguales a $0$, por lo que se considera terminado. En consecuencia
  \begin{equation*}
   \frac{154}{625} = 0.1104_{\text{cinco}}
  \end{equation*}
  que es a lo que se quer\'{\i}a llegar.${}_{\blacksquare}$
 \end{enumerate}

\end{solucion}
